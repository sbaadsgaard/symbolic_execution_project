\documentclass{article}

% Imports
\usepackage[utf8]{inputenc}
\usepackage[margin=1in]{geometry}
\usepackage{lastpage}
\usepackage{fancyhdr}
\usepackage{graphicx}

% Header
\setlength{\headheight}{48.14pt} 
\fancyhf{}
\fancyhead[OL]{
\includegraphics[scale=0.3]{AUlogo}
\vspace{-0.5cm}
}
\fancyhead[OR]{
\begin{tabular}{l}
\textbf{\sc Bachelor Contract}       \\
Date: \today              \\
Page \thepage/\pageref{LastPage}\\
~~
\end{tabular}}

\newcommand{\timeest}[1]{$\mathbf{#1}$}% used to write time estimations without excessive math-mode

\begin{document}
\pagestyle{fancy}

% Meta-information about group/advisors/etc.
\bgroup\def\arraystretch{1.5}
\begin{table}[h]
\begin{tabular}{ll}
\textbf{Advisor}     & Anders Møller    \\
\textbf{Students}    & Søren Baadsgaard \\
\textbf{Languages}   & English / Danish \\
\textbf{Text tools}  & \LaTeX         \\
\textbf{Other tools} & Scala (Programming language)         
\end{tabular}
\end{table}
\egroup\vspace{-0.cm}

\subsection*{Project Description (at least 10-20 lines)}
%NOTE this is simply the text from the project proposal document. May be subject to change.
Testing programs thoroughly requires good selections of input values, to make sure that all corner 
cases of the programs are reached. Symbolic execution is a classic technique for automatically 
obtaining such input values. Instead of running programs using
concrete values, it uses symbolic 
expressions together with automated theorem proving tools. Within the last decade, a variant 
called directed automated random testing, which combines concrete and symbolic execution, has 
become widely used.
The aim of t
his project is to explore symbolic execution techniques. The work will consist of 
reading and summarizing research papers, and in implementing a small prototype tool that 
performs symbolic execution for a simple programming language.

\subsection*{Provisional Table of Contents}
\begin{itemize}
    \item Abstract (10-20 lines)
    \item Section 1: Introduction (1-2 pages)
    \item Section 2: Review of literature (4-8 pages)
    \item Section 3: Description of Task A (4-8 pages)
    \item Section 4: Description of Task B (4-8 pages)
    \item Section 5: Description of Task C (4-8 pages)
    \item Section 6: Comparison to other work and ideas for future work (2-4 pages)
    \item Section 7: Conclusion (1-2 pages)
    \item Acknowledgements (3-5 lines)
    \item References ($\frac{1}{2}$-1 page)
    \item Appendix with programming code, tables, full proofs, etc. (5-20 pages)
\end{itemize}

\subsection*{Provisional Time Plan}

%Make a simple programming language (imperative) start med bools/int - Lav en interpreter til dette. 

\paragraph{First week of February (15 hours)}~\\\noindent
Planning of activities, including the production of the Bachelor's contract.

\paragraph{Rest of February and first half of March (\timeest{3\times 15} hours)}~\\\noindent
Read literature (one or more scientific papers) and make draft of Section 2 in Bachelor's report.

\paragraph{Rest of March and first week of April (\timeest{2\times 15+2\times 30} hours)}~\\\noindent
Completion of task A and make draft of Section 3 in Bachelor's report.

\paragraph{Rest of April (\timeest{3\times 30} hours)}~\\\noindent
Completion of task B and make draft of Section 4 in Bachelor's report.

\paragraph{First three weeks of May (\timeest{3\times 30} hours)}~\\\noindent
Completion of task C and make draft of Section 2 in Bachelor's report.

\paragraph{Last week of April of first half of June (\timeest{3\times 30} hours)}~\\\noindent
Write the missing parts, put drafts together, make things consistent, proof reading.

\end{document}
