%https://tex.stackexchange.com/questions/9057/best-practice-for-control-flow-charts

\newcommand{\motexample}{
		\begin{figure}[!h]
		\begin{center}
		\begin{tikzpicture}[%
		->,
		shorten >=2pt,
		>=stealth,
		node distance=1cm,
		noname/.style={%
			minimum width=5em,
			minimum height=2em,
			draw,
			align=left
		}
		]
		\node[noname] (1)                                             {$revenue := units \cdot 2$};
		\node[noname] (2) [below=of 1]                                {$ revenue \geq 16$};
		\node[noname] (3) [below=of 2]						  {$revenue := revenue - 10$};
		\node[noname] (4) [below= of 3]							      {$ revenue \geq cost$};
		\node[noname] (5) [below right= of 4]						  {\textsl{Assertion Error}};
		\node[noname] (6) [below left= of 4]						  {return $revenue$};
		
		
		\path (1) edge             									node {} (2)
		(2) edge [bend right=70pt] 							node[right] {$False$} (6)
		(2) edge 											node[right] {$True$} (3)
		(3) edge											node {} (4)
		(4) edge 											node[right, below] {$True$} (5)
		(4) edge											node[right, below] {$False$} (6);
		\end{tikzpicture}
	\end{center}
	\caption{Control-flow graph for \textsc{ComputeRevenue}}
	\end{figure}
}

\newcommand{\sumprogram}{
	\begin{figure}[!h]
	\begin{center}
		\begin{tikzpicture}[%
		->,
		shorten >=2pt,
		>=stealth,
		node distance=1cm,
		noname/.style={%
			minimum width=5em,
			minimum height=3em,
			draw
		}
		]
		\node[noname] (1)                                             {x := a + b};
		\node[noname] (2) [below=of 1]                                {y := b + c};
		\node[noname] (3) [below=of 2] 								  {z := x + y - b};
		\node[noname] (4) [below=of 3]                                {return z};
		
		\path (1) edge                   node {} (2)
		(2) edge                   node {} (3)
		(3) edge                   node {} (4);
		\end{tikzpicture}
	\end{center}
	\caption{Control-flow graph for ComputeSum}
	\end{figure}
}

\newcommand{\pow}{
	\begin{figure}[!h]
	\begin{center}
		\begin{tikzpicture}[%
		->,
		shorten >=2pt,
		>=stealth,
		node distance=1cm,
		noname/.style={%
			minimum width=5em,
			minimum height=3em,
			draw
		}
		]
		\node[noname] (1)                                             {$r := 1$};
		\node[noname] (2) [below=of 1]                                {$i := 0$};
		\node[noname] (3) [below=of 2] 								  {$i < b$};
		\node[noname] (4) [below=of 3]								  {$r := r\cdot a$};
		\node[noname] (5) [below=of 4]								  {$i := i + 1$};
		\node[noname] (6) [below=of 5]								  {return $r$};
		
		
		\path (1) edge             									node {} (2)
		(2) edge                  								    node {} (3)
		(3) edge [bend left = 50pt] 								node[right] {$False$} (6)
		(3) edge                 								    node[right] {$True$} (4)
		(4) edge 												    node {} (5)
		(5) edge [bend left = 50pt]								node {} (3);
		\end{tikzpicture}
	\end{center}
	\caption{Control-flow graph for \textsc{ComputePow}}
	\end{figure}
}

\newcommand{\ifstm}{
	\begin{figure}[!h]
		\begin{center}
			\begin{tikzpicture}[%
			font=\scriptsize,
			->,
			shorten >=2pt,
			>=stealth,
			node distance=1cm,
			noname/.style={%
				minimum width=5em,
				minimum height=3em,
				draw
			}
			]
			\node[noname] (1)    {\begin{tabular}{c}
				\textbf{If} $q_k$ \textbf{then} \textsl{statement 1} \textbf{else} \textsl{statement 2}\\
				Current PC: $[q_1, q_2, \ldots, q_{k-1}]$
				\end{tabular}};
			\node[noname] (2) [below left= of 1] {\begin{tabular}{c}
				\textsl{statement 1}\\
				New PC: $[q_1, q_2, \ldots, q_k]$
				\end{tabular}};
			\node[noname] (3) [below right= of 1] {\begin{tabular}{c}
				\textsl{statement 2}\\
				New PC: $[q_1, q_2, \ldots, \neg q_k]$
				\end{tabular}};
			
			\path (1) edge node[below right, align=center] {$q_1\land\ldots \land q_k$ \\ satisfiable} (2) 
			(1) edge node[below left, align=center] {$q_1\land \ldots \land \neg q_k$\\satisfiable} (3);	
 			\end{tikzpicture}
		\end{center}
		\caption{Abstract overview of the symbolic execution of an \textsl{if}-statement, which potentially leads to two new execution paths, each with a new \pc.}
	\end{figure}
}
\iffalse
\newcommand{\exectree}{
	\begin{figure}[!h]
		\begin{center}
			\begin{tikzpicture}[%
			font=\scriptsize,
			->,
			shorten >=2pt,
			>=stealth,
			node distance=1cm,
			noname/.style={%
				minimum width=5em,
				minimum height=3em,
				draw
			}
			]
			\node[noname] (1) {$revenue := \alpha \cdot 2$};
			\node[noname] (2) [below= of 1] {\textbf{if} $\alpha \geq 16$};
			\node[noname] ()
			\end{tikzpicture}
		\end{center}
	\end{figure}

}

\fi