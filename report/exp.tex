In this chapter we will introduce \explanguage which is a small programming language which consists of top-level functions, expressions and statements.

\section{Syntax of \explanguage}

\explanguage consists of expressions and statements. Expressions evaluates to values and does not change the control flow of the program. A statement evaluates to a value and a possibly updated variable environment. They may also change the control flow of the program through conditional statements. 

\subsection{Expressions}

Expressions consists of integers$\langle I, \rangle$, booleans$\langle B \rangle$,  and variables that are referenced by identifiers$\langle Id \rangle$. Furthermore they consist of arithmetic operations and comparisons of integers. 
\begin{grammar}
	<I> ::= 0 | 1 | -1 | 2 | -2 | $\ldots$
	
	<B> ::= True | False 
	
	<Id> ::= a | b | c | $\ldots$ 
	
	<E> ::= <I> 
	\alt <B>	
	\alt <Id>
	\alt <E> + <E> | <E> - <E> | <E> * <E> | <E> / <E>
	\alt <E> \textless <E> | <E> \textgreater <E> | <E> $\leq$ <E> | <E> $\geq$ <E> | <E> $==$ <E>
\end{grammar}

\subsection{Statements}


\subsubsection{variable declaration and assignment}
Variables implicitly declared, so variable declaration and assignment are contained in the same expression:
\begin{grammar}
	<S> ::= <Id> = <Exp>
\end{grammar} 
The value of an \textsl{assignment}-statement is the value of the expression on the right-hand side.  
\subsubsection{Conditional statements}
\explanguage supports three different conditional statements, namely \textsl{if-then-else} statements, \textsl{while} statements and \textsl{assert} statements:

\begin{grammar}
	<S> ::= if <E> then <S> else <S>
	\alt while <E> do <S>
	\alt assert <E>
\end{grammar}
Where the condition must be an expression that evaluates to a boolean value. 
The value of an \textsl{if-then-else} statement is the value of the statement that ends up being evaluated, depending on the condition. In a \textsl{while} statement, we are not guaranteed that the second statement is evaluated, so we introduce a special \textsl{unit} value which will be the value of any \textsl{while} statement. An assert statement will have the \textsl{unit} value if the condition evaluates to \textsl{true}. If the condition evaluates to \textsl{false}, the execution ends with an error.

\bigskip

Finally a statement may simply be an expression, or one statement followed by another:

\begin{grammar}
	<S> ::= E
	\alt <S> <S>
\end{grammar}
 
 

\subsubsection{Functions}
\explanguage supports top-level functions that must be defined at the beginning of the program. A function declaration$\langle F \rangle$ consists of an identifier followed by a parameter list with zero or more identifiers and finally a function body which is one or more statements.

\begin{grammar}
	<F> ::= <Id> (<Id>$^{*}$) \{ <E> \}
\end{grammar}

A function call then consists of an identifier, referencing a function declaration, followed by a list of expressions which is the function arguments:

\begin{grammar}
	<E> ::= <Id> (<E>$^{*}$) 
\end{grammar}

The length of the argument list and the parameter list in the declaration must be equal. Furthermore the expressions in the argument list must evaluate to either integers or  boolean values.
The value of a function call is the value of the final statement evaluated in the function body. Since expressions only return values, functions does not have any side effects.


\subsubsection{Programs}
We finally define the syntax of a \explanguage program, which is one or more function declarations, followed by a function call. 

\begin{grammar}
	<P> ::= <F> <F>$^{*}$  <Id> (<E>$^{*}$) 
\end{grammar}









	