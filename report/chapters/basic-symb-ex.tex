\section{description}

In this chapter we will describe the process of implementing symbolic execution for a simple imperative language called \simpl.

\section{Introducing the \simpl language}

\simpl\textbf{(Simple Imperative Programming Language)} is a small imperative programming language, designed to highlight the interesting use cases of symbolic execution. The language supports only one type, namely the set integers $\mathbb{N}$.
Furthermore we will interpret 0 as \emph{false} and any other integer as \emph{true}.
 \simpl supports basic
 variables that can be assigned the value of any expression, as well as basic branching functionality through an \textbf{If} - \textbf{Then} - \textbf{Else} statement. Furthermore it allows for looping through a \textbf{While} - \textbf{Do} statement.

We will describe the language formally, by the following Context Free Grammar:

%TODO find a better way to illustrate the grammar 
\newpage
\begin{grammar}
	<int> ::= 0 | 1 | -1 | 2 | -2 | $\ldots$
	
	<var> ::= a | b | c | $\ldots$ 
	
	<exp> ::= <int>
	\alt <var>
	\alt <exp> $+$ <exp> | <exp> $-$ <exp> | <exp> $*$ <exp> | <exp> $/$ <exp>
	\alt <exp> $>$ <exp> | <exp> $==$ <exp> 
	\alt ( <exp> )	
	
	<stm> ::= <exp>
	\alt <var> = <exp>
	\alt <stm> <stm>
	\alt if <exp> then <stm> else <stm>
	\alt while <exp> do <stm>
	\alt Print E
	
\end{grammar}

where $+, *, -, /$ denotes the usual arithmatic operators on integers, and $>, ==$ denotes the comparison-operators of \emph{greater-than} and \emph{equal-to} respectively. When interpreting a comparison-operator we will return 0 for \emph{false} and 1 for \emph{true}. 