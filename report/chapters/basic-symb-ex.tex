\section{description}

In this chapter we will describe the process of implementing symbolic execution for a simple imperative language called \simpl.

\section{Introducing the \simpl language}

\simpl\textbf{(Simple Imperative Programming Language)} is a small imperative programming language, designed to highlight the interesting use cases of symbolic execution. The language supports two types, namely the set integers $\mathbb{N}$ and the boolean values $true$ and $false$.
 \simpl supports basic
 variables that can be assigned integer values, as well as basic branching functionality through an \textbf{If} - \textbf{Then} - \textbf{Else} statement. Furthermore it allows for looping through a \textbf{While} - \textbf{Do} statement. It also supports top-level functions and the use of recursion.

We will describe the language formally, by the following Context Free Grammar:

%TODO find a better way to illustrate the grammar 
\newpage
\begin{grammar}
	<int> ::= 0 | 1 | -1 | 2 | -2 | $\ldots$
	
	<Id> ::= a | b | c | $\ldots$ 
	
	<exp> ::= <aexp> | <bexp> | <nil>
	
	<nil> ::= ()
	
	<bexp> ::= True | False
	\alt <aexp> $>$ <aexp>
	\alt <aexp> $==$ <aexp>
	
	<aexp> ::= <int> | <id>
	\alt <aexp> + <aexp> | <aexp> - <aexp> 
	\alt <aexp> $\cdot$ <aexp> | <aexp> / <aexp>
	\alt <cexp>
	
	<cexp> ::= <Id> (<aexp>*) \text{ \# Call expression}
	
	<stm> ::= <exp>
	\alt <Id> = <aexp>
	\alt <stm> <stm>
	\alt if <exp> then <stm> else <stm>
	\alt while <exp> do <stm>
	
	<fdecl> ::= <Id> (<Id>*) {<fbody>}
	
	<fbody> ::= <stm>
	\alt <fdecl> <fbody>
	
	<prog> ::= <fdecl>* <stm>
	
\end{grammar}

\subsubsection{Expressions}
expressions comes in three different types, arithmetic, boolean and a $nil$ expressions.

\textbf{arithmetic expressions} consists of integers, variables referencing integers, or the usual binary operations on these two. We also consider function calls an arithmetic expression, but assigning anything other than integers to a variable will result in a runtime error. 
\textbf{boolean expressions} consists of the boolean values $true$ and $false$, as well as comparisons of arithmetic expressions. 

\textbf{nil}
  nil expression is a special case which is only returned from \emph{while} statements. 

\subsubsection{Statements}
Statements consists of assigning integer values to variables, \emph{if-then-else} statements for branching and a \emph{while-do} statement for looping. Finally a statement can simply an expression, as well as a compound statement to allow for more than one statement to be executed.

\subsubsection{Function declarations}

Function declarations consists of an identifier, followed by a list of zero or more identifiers for parameters, and finally a function body which is simply a statement. Functions does not have any side effects, so any variables declared in the function body will be considered local. Furthermore, any mutations of globally defined variables will only exist in the scope of that particular function.

\iffalse

where $+, *, -, /$ denotes the usual arithmatic operators on integers, and $>, ==$ denotes the comparison-operators of \emph{greater-than} and \emph{equal-to} respectively. We consider program to simply be a collection of function declarations. A function  Each program will start with a function \emph{main()} whose body will be the program to be executed. Furthermore function calls will not have any side effects, therefore any variable declarations inside a function will be considered local to that scope, and any mutations of variables in a outer scope, will only exist in the scope of that function. Statements will always return the value of the final operation performed, where \emph{assign-statements} return the value of the expression on the right hand side. There are two special cases, namely \emph{while} and-\emph{fdecl}-statements, which will return the value nil value which is simply the empty value. 

\fi

\subsubsection{programs}

We consider a program to be zero of more top-level function declarations, as well a statement. The statement will act as the starting point when executing a a program. 
\subsection{Interpreting \simpl}

In order to work with \simpl, we have build a simple interpreter using the \emph{Scala} programming language.

 We represent our environment as a map 
\begin{equation}
	env: \langle Id \rangle   \rightarrow \mathbb{N}
\end{equation}
and each time we interpret a statement we return an instance of this map that reflects the current state of the environment. Note that we restrict both variables and function arguments to integer values. 

A program is represented as an object $Prog$ which carries a map

\begin{equation*}
	funcs: \langle Id \rangle \rightarrow \langle fdecl \rangle
\end{equation*}

of top-level function declarations, as well as a root statement. To interpret the program we simply traverse the tree starting with the root statement. Interpreting function calls consists of looking up the function in the $Prog$ object, add the function parameters to the environment and then interpreting the statement in the function body. In order to keep functions from having side effects, we simply return the original environment as it was before adding function parameters and interpreting the body. 

\subsection{Symbolic interpreter for \simpl}

To be able to do symbolic execution of a program written in \simpl, we must extend our implementation to allow for symbolic values to exists. To do this we 
will add an extra non-terminal to our grammar which will represent symbolic values. Our grammar will then look like 

\begin{grammar}
	<sym> ::= $\alpha$ | $\beta$ | $\gamma$ | $\ldots$
	
	<int> ::= 0 | 1 | $-1$ | $\ldots$
	
	\vdots
	
	<aexp> ::= <sym> | <int> | <id> | $\ldots$
	
	\vdots
\end{grammar}

Furthermore, we must extend the capabilities of our environment, so that it does not only map to Integer values, but instead to arbitrary expressions over integers and symbolic values. Furthermore we must extend the return type of functions as well. These too, will be allowed to return arbitrary expressions over integers and symbolic values. 

\subsubsection{Representing the path constraint}

Since the path constraint is simply a conjunction of expression on the form 
$(\langle aexp \rangle op \langle aexp \rangle)$  with  $op \in \{ >, == \}$, we will simply represent the $PC$ as a list of such expressions. So if $PC = true \land e1 \land e1 \land \ldots \land e_n$, we will represent it as the list $L_{pc} = [true, e1, e2, \ldots, e_n]$