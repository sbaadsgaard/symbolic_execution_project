\section{description}

In this chapter we will describe the process of implementing symbolic execution for a simple imperative language called \simpl.

\section{Introducing the \simpl language}

\simpl\textbf{(Simple Imperative Programming Language)} is a small imperative programming language, designed to highlight the interesting use cases of symbolic execution. The language supports two types, namely the set integers $\mathbb{N}$ and the boolean values $true$ and $false$.
 \simpl supports basic
 variables that can be assigned integer values, as well as basic branching functionality through an \textbf{If} - \textbf{Then} - \textbf{Else} statement. Furthermore it allows for looping through a \textbf{While} - \textbf{Do} statement. It also supports top-level functions and the use of recursion.

We will describe the language formally, by the following Context Free Grammar:

%TODO find a better way to illustrate the grammar 
\newpage
\begin{grammar}
	<int> ::= 0 | 1 | -1 | 2 | -2 | $\ldots$
	
	<Id> ::= a | b | c | $\ldots$ 
	
	<exp> ::= <aexp> | <bexp> | <nil>
	
	<nil> ::= ()
	
	<bexp> ::= True | False
	\alt <aexp> $>$ <aexp>
	\alt <aexp> $==$ <aexp>
	
	<aexp> ::= <int> | <id>
	\alt <aexp> + <aexp> | <aexp> - <aexp> 
	\alt <aexp> $\cdot$ <aexp> | <aexp> / <aexp>
	\alt <cexp>
	
	<cexp> ::= <Id> (<aexp>*) \text{ \# Call expression}
	
	<stm> ::= <exp>
	\alt <Id> = <aexp>
	\alt <stm> <stm>
	\alt if <exp> then <stm> else <stm>
	\alt while <exp> do <stm>
	
	<fdecl> ::= <Id> (<Id>*) {<fbody>}
	
	<fbody> ::= <stm>
	\alt <fdecl> <fbody>
	
	<prog> ::= <fdecl>* <stm>
	
\end{grammar}

\subsubsection{Expressions}
\simpl supports two different types of expressions, arithmetic expressions and boolean expressions.
\textbf{arithmetic expressions} consists of integers, variables referencing integers, or the usual binary operations on these two. We also consider function calls an arithmetic expression, and therefore functions must return integer values.
\textbf{boolean expressions} consists of the boolean values $true$ and $false$, as well as comparisons of arithmetic expressions. 

\subsubsection{Statements}
Statements consists of assigning integer values to variables, \emph{if-then-else} statements for branching and a \emph{while-do} statement for looping. Finally a statement can simply an expression, as well as a compound statement to allow for more than one statement to be executed.

\subsubsection{Function declarations}

Function declarations consists of an identifier, followed by a list of zero or more identifiers for parameters, and finally a function body which is simply a statement. Functions does not have any side effects, so any variables declared in the function body will be considered local. Furthermore, any mutations of globally defined variables will only exist in the scope of that particular function.

\iffalse

where $+, *, -, /$ denotes the usual arithmatic operators on integers, and $>, ==$ denotes the comparison-operators of \emph{greater-than} and \emph{equal-to} respectively. We consider program to simply be a collection of function declarations. A function  Each program will start with a function \emph{main()} whose body will be the program to be executed. Furthermore function calls will not have any side effects, therefore any variable declarations inside a function will be considered local to that scope, and any mutations of variables in a outer scope, will only exist in the scope of that function. Statements will always return the value of the final operation performed, where \emph{assign-statements} return the value of the expression on the right hand side. There are two special cases, namely \emph{while} and-\emph{fdecl}-statements, which will return the value nil value which is simply the empty value. 

\fi

\subsubsection{programs}

We consider a program to be zero of more top-level function declarations, as well a statement. The statement will act as the starting point when executing a a program. 

\subsection{Interpreting \simpl}

In order to work with \simpl, we have build a simple interpreter using the \emph{Scala} programming language. To keep track of our program state, we define a map
\begin{equation}
	env: \langle Id \rangle   \rightarrow \mathbb{N}
\end{equation}
that maps variable names to integer values. When interpreting a statement or an expression, this map will be passed along. Whenever we interpret a function call, we make a copy of the current environment to which we add the call-parameters. This copy is then passed to the interpretation of the function body, in order to ensure that functions are side-effect free. 

A program is represented as an object $Prog$ which carries a map

\begin{equation*}
	funcs: \langle Id \rangle \rightarrow \langle fdecl \rangle
\end{equation*}
from function names to top-level function declarations, as well as a root statement. To interpret the program we simply traverse the AST starting with the root statement. 


\subsubsection{return values}
In order to handle return types, we extend the implementation of the grammar with a non-terminal 

\begin{grammar}
	<value> ::= IntValue | BoolValue | Unit.
\end{grammar}

Arithmetic expressions will always return an $IntValue$ and Boolean expressions will always return a $BoolValue$. $Unit$ is a special return value which is reserved for \emph{while}-statements.


\subsection{Symbolic interpreter for \simpl}

To be able to do symbolic execution of a program written in \simpl, we must extend our implementation to allow for symbolic values to exists. To do this we 
will add an extra non-terminal to our grammar which will represent symbolic values. Our grammar will then look like 

\begin{grammar}
	<sym> ::= $\alpha$ | $\beta$ | $\gamma$ | $\ldots$
	
	<int> ::= 0 | 1 | $-1$ | $\ldots$
	
	\vdots
	
	<aexp> ::= <sym> | <int> | <id> | $\ldots$
	
	\vdots
\end{grammar}

\subsubsection{Determining feasible paths}
In order to determine which execution paths are feasible, we use the \textbf{Java} implementation of the \emph{SMT-solver} \textbf{Z3}. 
\subsubsection{Return values}
We extend our return values with the terminal $SymValue$ which contains expressions of the type $Expr$ from \textbf{Z3}, over integers and symbolic values.

\subsubsection{Representing the path constraint}
To represent the \emph{path-constraint} we implement a class $PathConstraint$ which contains a boolean formula $f = BoolExpr \land BoolExpr \land ...$ where each expression of type $BoolExpr$ is a condition that the input values must satisfy. 
\\

\subsubsection{Representing the program state}
We must extend the capabilities of our environment, so that it does not only map to Integer values, but instead to arbitrary expressions over integers and symbolic values. Therefore we define environment as a map
\begin{equation*}
	m: \langle Id \rangle \rightarrow SymValue
\end{equation*}
from variable identifiers to values of type $SymValue$.

\subsubsection{Execution strategy}
The first execution strategy that we implement is a naive approach, where all feasible paths will be explored in \emph{Depth-first} order, starting with the \emph{else}-branch. Note that our definition of \emph{feasible} is any path that we can determine to be satisfiable. This means that \emph{path-constraints} that \textbf{Z3} cannot determine the satisfiability of, will be regarded as infeasible, and ignored.
This strategy is sufficient for small \emph{finite} programs, but scales badly to programs with large recursion trees, and it runs forever on programs with infinite recursion trees. 
