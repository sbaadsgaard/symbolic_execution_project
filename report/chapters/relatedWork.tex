Classical symbolic execution is introduced in \citep{King76}. The paper establishes the concept of symbolically executing a program by replacing concrete input values by symbols, and letting variables reference integer polynomials over these symbols. The concept of a \pc is also introduced as a conjunction of boolean expressions over the input symbols. These expressions are constraints placed on the input symbols that must be satisfied to execute a long the given path. Whenever the execution reaches a conditional statement, e.g an \textsl{if}-statement, the satisfiability of both execution path is checked. If both paths are satisfiable, the execution is forked into two independent executions. One execution follows the path corresponding to the condition evaluating to true, while the other follows the path corresponding to the condition evaluating to false. The paper also touches on the difficulty of deciding the satisfiability of paths, and that for most practical programs the total number of execution paths are too large to systematically explore. \citep{Godefroid:2005:DDA:1064978.1065036} presents a tool called DART(Directed Automated Random Testing), which is an instance of concolic execution. The program is executed with initially random input values. During the execution, both a concrete and a symbolic environment is maintained, and whenever the execution branches, the choice of branch is registered in a path-constraint , using the symbolic environment. At the end of an execution, the path-constraint is used to generate a new set of concrete that follow a different execution path. One of the main advantages of the DART approach is the ability to avoid getting stuck trying to decide the satisfiability of a \pc. This is achieved by substituting in concrete values whenever such undecidable contraints arises. 

