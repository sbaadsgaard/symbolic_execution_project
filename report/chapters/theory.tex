In this chapter we will cover the theory behind symbolic execution. We will start by describing what it means to \emph{symbolically execute} a program and how we deal branching. We will explain the connection between a symbolic execution of a program, and a concrete one. 
\iffalse 	
\section{Principles of symbolic execution }
	
	In \cite{king76}, symbolic execution is described as a practical approach between simple program testing and program proving. At one extreme, program testing allows the programmer to get some level of confidence in the program, by running it on a well-selected sample of input values, but this sample size will be but a fraction of the possible input values. At the other extreme, program proving will give complete confidence in the programs correctness. To achieve this, one must provide a precise specification of correct behavior, as well as be able to perform perform formal proof steps to conclude that the program satisfies this specification. This is a challenging task, even for relatively simple programs. \emph{Symbolic execution} will serve as a practical middle ground between these two, in which we try to extend simple program testing to cover more general classes of inputs.  
\fi

\section{Symbolically executing a program}
	
	
	
	\iffalse
	\cite{CadarSen13} describes symbolic execution as follows:
	
	Any input to the program will be replaced with \emph{symbolic} values instead of concrete ones. Any operations on the symbolic values, will result in expressions over these, so our program state can be describes as a map from variable names to expressions over the input values. This also means that any return value from the execution will be such an expression. In \cite{king76} an example is given of a program written in a simple language with only signed integer values and arithmetic operations on these. In a symbolic context, this will translate to a program that operates on polynomials with integer coefficients.
	
	\fi   
	
	When we execute a program symbolically, we use symbolic values as input data to the program, instead of concrete values. Instead of referencing concrete values, variables will reference expressions over the symbolic values, and therefore the return value of a program will also be such expressions \cite{CadarSen13}.
	
	
	
	To illustrate this, we consider the following program, that takes parameters $a, b, c$ and computes the sum:
	
	
	\Tree[.{x = a + b} [.{y = b + c} [.{z = x + y - b} [.{return z} ] ] ] ]

	
	If we run this program on with concrete values, say $a = 2, b = 3, c = 4$, we would get the following execution:
	First we assign $a+b = 5$ to the variable $x$. Then we assign $b + c = 7$ to the variable $y$· Next we assign $x + y - b = 9$ to variable $z$ and finally we return $z = 9$, which is indeed the sum of 2, 3 and 4. 
	\\
	Let us now run the program which symbolic input values $\alpha, \beta$ and $\gamma$ for $a, b$ and $c$ respectively. 
	
	We would then get the following execution: We assign $\alpha + \beta$ to $x$. We then assign $\beta + \gamma$ to $y$. Finally we assign $(\alpha + \beta) + (\beta + \gamma) - \beta$ to $z$ and return $z = \alpha + \beta + \gamma$. We can conclude that the program correctly computes the sum of $a, b$ and $c$, for any possible value of these.
	
\section{Path constraints and constraint solving.}
		In the previous section we gave an example of a symbolic execution of a program that computes the sum of three integers. This program contains no conditional statements, so it will follow the same execution path no matter what input we run it with, and we do not place any constraints on the input. In general, a program will follow different execution paths, depending on the outcome of any conditional statements along the path. 
		
		To encapsulate this, we introduce a \emph{path-constraint} which is a list of boolean expressions $\{q_1, q_2, \ldots, q_n \}$ over the symbolic values. To follow a given path, $q_1 \land \ldots \land q_n$ must be \emph{satisfiable}. To be \emph{satisfiable}, there must exist an assignment of concrete values, to the symbolic ones so that the conjunction of the expressions evaluates to true. For example, $q = (2\cdot \alpha > \beta) \land (\alpha < \beta)$ is satisfiable, because we can choose $\alpha = 10$ and $\beta = 15$ in which case $q$ evaluates to \emph{true}.
		\\ 
		Whenever we reach a conditional statement with condition $q$, we consider the two following expressions:
		

		\begin{enumerate}
			\item $ pc \land q$
			\item $ pc \land \neg q$
		\end{enumerate}	
		where $pc$ is the conjunction of all the expressions currently contained in the \emph{path-constraint}.
		\\
		This gives a number of possible scenarios:	
		\begin{itemize}
			\item \textbf{Only the first expression is satisfiable}: 
			We add $q$ to the \emph{path-constraint}, and we continue the execution along the same path.
			\item \textbf{If only the second expression is satisfiable}:
			We add $\not q$ to the \emph{path-constraint}, and we continue along the same path. 
			\item \textbf{Both expressions are satisfiable}: In this case, the execution can follow two paths; one where we assume $q$ and one where we assume $\not q$. At this point we \emph{fork} we fork the execution by considering two different executions of remaining program. Both executions start with the same shared variable state, and \emph{path-constraints} that is equal up to the final element. One execution will have $q$ as the final element and the other will have $\not q$. 
			These two executions will now follow two different execution paths that differ from this conditional statement and onward.
		\end{itemize}
		
		We illustrate this with the following example:
		
	\begin{lstlisting}
	Fun pow(a, b) 
		var r = 1
		var i = 0
		while (i < b) 
			r = r*a
			i = i + 1 
		return r
		
		\end{lstlisting}
		
		If we assign $a = \alpha$ and $b = \beta$, we get the following execution:
		
		\begin{itemize}
			\item PC is initialized to $true$
			\item $r \gets 1$
			\item $i \gets 0$
			\item We hit a branching point, so we check if $true \land (0 < \beta)$ and $true \land \neg (0 < \beta)$ are satisfiable. Since they both are, we must fork:
			\item \textbf{Case} $ \neg (0 < \beta) $: $PC' \gets true \land \neg (0 < \beta) $.
				The program returns $1$. So we can conclude that the program returns 1 when $\beta \leq 0$.
			\item \textbf{Case} $ (0 < \beta)$: $PC \gets true \land (0 < \beta)$.
			\subitem  $ r \gets 1 \cdot a$
			\subitem  $ i \gets 0 + 1$
			\item We hit a branching point again, so we check if $true \land (0 < \beta) \land (1 < \beta)$ and $true \land (0 < \beta) \land \neg (1 < \beta)$ are satisfiable. Since both they are, we fork again:
			\item \textbf{Case} $ \neg(1 < \beta)$: $PC' \gets true \land (0 < \beta) \land \neg (1 < \beta)$. The program returns $\alpha$. So we can conclude that the program returns $\alpha$ when $ \beta = 1$.
			\item \textbf{Case} $ 1 < \beta$: $PC \gets true \land (0 < \beta) \land (1 < \beta)$.
			\subitem $r \gets a*a$
			\subitem $i \gets 1 + 1$
			\item We hit a branching point $\ldots$	
		\end{itemize}
		
		An important property of the \emph{path-constraint} is that it can never become identically false. To see why this is the case, we have to look at the possible updates of $PC$. At the start of an execution, it will be initialized with the value $true$. At any branching point, it will be updated with exactly one of the expressions $PC \land q$ and $PC \land \neg q$, and only if the given expression is satisfiable. So $PC$ will never end up looking like $\ldots \land q \land \ldots \land \neg q \land \ldots $ for some condition $q$. What this means is that when the program terminates at the end of some execution path, $PC$ will be a satisfiable formula over the symbolic values, which means that we can solve the constraints and derive a set of concrete values which will follow the exact same path if we execute the program normally.   
		
		%TODO Find better titles for sections 
\section{Limitations of symbolic execution}
		
		\subsection{Infinite execution trees}
			As demonstrated in the last example in the previous sections, a symbolic execution can easily become infinite as soon as we introduce branching and some looping structure. This is further illustrated if we consider an execution tree for a program. In \cite{king76}, they are described by enumerating each statement, and let each node in the tree, correspond to an execution of one of the command. The edges going out from a node corresponds to the transition from one statement to the next. From this we can see that non forking statements will only have a single edge outgoing, while forking statements will have two. The forking edges will then correspond to splitting up the execution and following a different path, with a different \emph{path-constraint}. As an example, we can look at the execution tree for the program \emph{pow}, that computes $a^b$. 
			
			\Tree[.1 [.2 [.3 [.4 [.5 [.6 [.3 [.4 [.5 [.6 [.3 [.$\vdots$ ] [.{7: return $\alpha^2$ when $\beta = 2$  }
			] ] ]  ] ] [.{7: return $\alpha$ when $\beta = 1$} ]  ]  ] ] ] [.{7: return 1 when $\beta <= 0$ } ] ]  ]  ] 
							
		
			As we can see, this tree is infinite, since $b = \beta$ and $\beta$ can be arbitrarily large. In this case, our symbolic execution would run forever if we insisted on exhaustively exploring all possible paths. This illustrates one of the limitations of symbolic execution. For programs with infinite execution trees, we simply cannot exhaust all possible inputs, so we have to restrict our testing to exploring only a finite number of paths. 
			
			(Maybe write something about induction over trees, and finite trees allowing for exhaustive search)
			
			\subsection{the ability(or inability) to decide whether a given path is feasible}
			
			(Write something about restrictions on SMT-solvers e.g SAT being NP Complete and some theories may be undecidable)