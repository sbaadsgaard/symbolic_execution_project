In this chapter we will cover the theory behind symbolic execution. We will start by describing what it means to \emph{symbolically execute} a program and how we deal branching. We will also explain the connection between a symbolic execution of a program, and a concrete execution. We shall restrict our focus to programs that takes integer values as input and allows us to do arithmetic operations on such values. In the end we will cover the challenges and limitations of symbolic execution that arises when these restrictions are lifted. 


\section{Symbolic executing of a program}
	
	During a normal execution of a program, input values consists of integers. During a symbolic execution we replace concrete values by symbols e.g $\alpha$ and $ \beta$, that acts as placeholders for actual integers. We will refer to symbols and arithmetic expressions over these as \emph{symbolic values}.
	 The program environment consists of variables that can reference both concrete and symbolic values. \cite{CadarSen13}.
	\\
	To illustrate this, we consider the following program that takes parameters $a, b, c$ and computes the sum:
	%https://tex.stackexchange.com/questions/9057/best-practice-for-control-flow-charts
	\sumprogram{}
	Lets consider running the program with concrete values $a = 2, b = 3, c = 4$. we then get the following execution:
	First we assign $a+b = 5$ to the variable $x$. Then we assign $b + c = 7$ to the variable $y$. Next we assign $x + y - b = 9$ to variable $z$ and finally we return $z = 9$, which is indeed the sum of 2, 3 and 4. 
	\\
	Let us now run the program with symbolic input values $\alpha, \beta$ and $\gamma$ for $a, b$ and $c$ respectively. 

	
	We would then get the following execution: First we assign $\alpha + \beta$ to $x$. We then assign $\beta + \gamma$ to $y$. Next we assign $(\alpha + \beta) + (\beta + \gamma) - \beta$ to $z$.Finally we return $z = \alpha + \beta + \gamma$. We can conclude that the program correctly computes the sum of $a, b$ and $c$, for any possible value of these.
	
\section{Execution paths and path constraints}
		The program that we considered in the previous section contains no conditional statements, which means it only has a single possible execution path. In general, a program with conditional statements $s_1, s_2, \ldots, s_n$ with conditions $q_1, q_2, \ldots, q_n$, will have several execution paths that are uniquely determined by the value of these conditions. In symbolic execution, we model this by introducing a \emph{path-constraint} for each execution path. The \emph{path-constraint} is a list of boolean expressions $\lbrack q_1, q_2, \ldots, q_n \rbrack$ over the symbolic values, corresponding to conditions from the conditional statements along the path. At the start of an execution, the \emph{path-constraint} only contains the expression $true$, since we have not encountered any conditional statements. to continue execution along a path, $q_1 \land \ldots \land q_n$ must \emph{satisfiable}. To be \emph{satisfiable}, there must exist an assignment of integers to the symbols, such that the conjunction of the expressions evaluates to true. For example, $q = (2\cdot \alpha > \beta) \land (\alpha < \beta)$ is satisfiable, because we can choose $\alpha = 10$ and $\beta = 15$ in which case $q$ evaluates to \emph{true}.
		\\ 
		Whenever we reach a conditional statement with condition $q_k$, we consider the two following expressions:
		

		\begin{enumerate}
			\item $ pc \land q_k$
			\item $ pc \land \neg q_k$
		\end{enumerate}	
		where $pc$ is the conjunction of all the expressions currently contained in the \emph{path-constraint}.
		\\
		This gives a number of possible scenarios:	
		\begin{itemize}
			\item \textbf{Only the first expression is satisfiable}: Execution continues with a new \emph{path-constraint} $\lbrack q_1, q_2, \ldots, q_k \rbrack$, along the path corresponding to $q_k$ evaluating to to \emph{true}.
			\item \textbf{Only the second expression is satisfiable}:  Execution continues with a new \emph{path-constraint} $\lbrack q_1, q_2, \ldots, \neg q_k \rbrack$, along the path corresponding to $q_k$ evaluating to to \emph{false}.
			
			\item \textbf{Both expressions are satisfiable}: In this case, the execution can continue along two paths; one corresponding to the condition being \emph{false} and one being \emph{true}. At this point we \emph{fork} the execution by considering two different executions of the remaining part of the program. Both executions start with the same variable state and \emph{path-constraints} that are the same
			 up to the final element. One will have $q_k$ as the final element and the other will have $\neg q_k$. 
			These two executions will continue along different execution paths that differ from this conditional statement and onward.
		\end{itemize}
		
		To illustrate this, we consider the program from the motivating example, that takes input parameters $units$ and $costs$:
		
		\motexample{}
		
		We assign symbolic values $\alpha$ and $\beta$ to $units$ and $cost$ respectively, and get the following symbolic execution:
		
		First we assign $2\cdot \alpha$ to $revenue$. We then reach a conditional statement with condition $q_1 = \alpha \cdot 2 \geq 16$. To proceed, we need to check the satisfiability of the following two expressions:
		\begin{enumerate}
			\item $true \land (\alpha \cdot 2 \geq 16)$
			\item $true \land \neg (\alpha \cdot 2 \geq 16)$.
		\end{enumerate}
		Since both these expressions are satisfiable, we need to fork. We continue execution with a new  \emph{path-constraint} $\lbrack true, (\alpha \cdot 2 \geq 16) \rbrack$, along the \emph{T} path. We also start a new execution with the same variable bindings and a \emph{path-constraint} equal to $\lbrack true, \neg (\alpha \cdot 2 \geq 16) \rbrack$. This execution will continue along the \emph{F} path, and it reaches the return statement and returns $\alpha \cdot 2$.
		The first execution assigns $2\cdot \alpha - 10$ to $revenue$ and then reach another conditional statement with condition $2\cdot \alpha - 10 < \beta$. We consider the following expressions:
		\begin{enumerate}
			\item $true \land (\alpha \cdot 2 \geq 16) \land (((2\cdot \alpha) - 10) < \beta)$
			\item $true \land (\alpha \cdot 2 \geq 16) \land \neg (((2\cdot \alpha) - 10) < \beta)$
		\end{enumerate}
		Both of these expressions are satisfiable, so we fork again. In the end we have discovered all three possible execution paths:
		\begin{enumerate}
			\item $true \land \neg (\alpha \cdot 2 \geq 16)$
			\item $true \land (\alpha \cdot 2 \geq 16) \land \neg (((2\cdot \alpha) - 10) < \beta)$
			\item $true \land (\alpha \cdot 2 \geq 16) \land (((2\cdot \alpha) - 10) < \beta)$.
		\end{enumerate}
		
		The first two \emph{path-constraints} represents the two different paths that leads to the return statement, where the first one returns $2\cdot \alpha$ and the second one returns $2\cdot \alpha - 10$. Inputs that satisfy these, does not result in a crash.
		The final \emph{path-constraint} represents the path that leads to the $Error$ statement, so we can conclude that all input values that satisfy these constraints, will result in a program crash.
	
		
\section{Constraint solving}
	
	In the previous section we described how to handle programs with multiple execution paths by introducing a \emph{path-constraint} for each path, which a list of constraints on the input symbols. This system of constraints defines a class of integers that will cause the program to execute along this path. By solving the system from each \emph{path-constraint}, we obtain a member from each of class which forms a set of concrete inputs that cover all possible paths.    
	
	If we consider the motivating example again, we found three different paths, represented by the following \emph{path-constraints}:
	\begin{enumerate}
		\item $true \land \neg (\alpha \cdot 2 \geq 16)$
		\item $true \land (\alpha \cdot 2 \geq 16) \land \neg (2\cdot \alpha - 10 < \beta)$
		\item $true \land (\alpha \cdot 2 \geq 16) \land (2\cdot \alpha - 10 < \beta)$.
	\end{enumerate}
	
	By solving for $\alpha$ and $\beta$, we obtain the set of inputs $\{(7, \beta), (8,6), (8, 7)\}$, that covers all possible execution paths. Note that we have excluded a concrete value for $\beta$ in the first test case, because the \emph{path-constraint} does not depend on the value of $\beta$. 
	 
\section{Limitations and challenges of symbolic execution}
	So far we have only considered symbolic execution of programs with a small number of execution paths. Furthermore, the constraints placed on the input symbols have all been linear.
	In this section we will cover the challenges that arise when we consider more general programs.
	
	\subsection{The number of possible execution paths} 
		Since each conditional statement in a given program can result in two different execution paths, the total number of paths to be explored is potentially exponential in the number of conditional statements. 
		For this reason, the running time of the symbolic execution quickly gets out of hands if we explore all paths. 
		 The challenge gets even greater if the program contains a looping statement. We illustrate this by considering the following program that implements the power-function for integers $a$ and $b$, with symbolic values $\alpha$ and $\beta$ for $a$ and $b$:
		\pow{}
		
	This program contains a $While$-statement with condition $q = i < b$. The $k'th$ time we reach this statement we will consider the following two expressions:
	\begin{enumerate}
		\item $true \land (0 < \beta) \land (1 < \beta) \land \ldots \land (k-1 < \beta) $
		\item $true \land (0 < \beta) \land (1 < \beta) \land \ldots \land \neg (k-1 < \beta) $.
	\end{enumerate}
	Both of these expressions are satisfiable, so we fork the execution. This is the case for any $k > 0$, which means that the number of possible execution paths is infinite. If we insist on exploring all paths, the symbolic execution will simply continue for ever. 
	
	\subsection{deciding satisfiability of \emph{path-constraints}}
	A key component of symbolic execution, is deciding whether or not a \emph{path-constraint} is satisfiable, in which case the corresponding path is eligible for exploration. Consider the following \emph{path-constraint} from the motivating example:
	
	\begin{equation}	
		true \land (\alpha \cdot 2 \geq 16) \land \neg (2\cdot \alpha - 10 < \beta).
	\end{equation}
	
	To decide if this is satisfiable or not, we must determine if there exist an assignment of integer values to $\alpha$ and $\beta$ such that the formula evaluates to \emph{true}. We notice that the formula is a conjunction of linear inequalities. We can assign these to variables $q_1$ and $q_2$ and get
	\begin{align}
		q_1 & = (\alpha \cdot 2 \geq 16) \\
		q_2 & = (2\cdot \alpha - 10 < \beta)
	\end{align}
	The formula would then be $true\land q_1 \land \neg q_2$,
	where $q_1$ and $q_2$ can have values \emph{true} or \emph{false} depending on whether or not the linear inequality holds for some integer values of $\alpha$ and $\beta$. The question then becomes twofold: Does there exist an assignment of \emph{true} or \emph{false} to $q_1$ and $q_2$ such that the formula evaluates to true? And if so, does this assignment lead to a system of linear inequalities that is satisfiable?
	In this example, we can assign \emph{true} to $q_1$ and \emph{false} $q_2$, which gives the following system of linear inequalities:
	\begin{align}
		& \alpha \cdot 2 \geq 16 \\
		2  \cdot & \alpha - 10 - \beta \geq 0 
	\end{align}
	where we moved $\beta$ to the left hand side. This system is satisfied by $\alpha = 8$ and $\beta = 6$. So we can conclude that the \emph{path-constraint} is indeed satisfiable.
	
	\subsubsection{The SMT problem}
	
	The example we just gave, is an instance of the \emph{Satisfiability Modulo Theories(\textbf{SMT}) probem}. In this problem we are given a logical formula which consists of the conjunction or disjunction of boolean variables $q_1, q_2, \ldots, q_n$, or their negation. The task is then to decide if there exist an assignment of boolean values $true$ and $false$ to this variables, so that the formula evaluates to \emph{true}. Furthermore, each of these boolean variables represent some formula belonging to a theory. Such a theory could be the \emph{theory of Linear Integer Arithmetic(\textbf{LIA})} which we will explain shortly. If there exist an assignment that satisfies the original formula, this assignment must also be valid w.r.t the given theory. Note that the first part of this problem is simply the \emph{boolean SAT problem}, which is known to be \emph{NP-complete}, so solving this part alone takes worst-case exponential time. % TODO INSERT CITATION OF LEC NOTES FROM OPT.
	
	
	\subsubsection{The theory of linear arithmetic}
	
	
	
	