In this chapter we will introduce \explanguage which is a small programming language which consists of top-level functions, expressions and statements. We start by describing the syntax of the language, and then we describe a concrete interpreter for the language.

\section{Syntax of \explanguage}

\explanguage consists of expressions and statements. Expressions evaluates to values and does not change the control flow of the program. A statement evaluates to a value and a possibly updated variable environment. They may also change the control flow of the program through conditional statements.
 

\subsection{Expressions}


Expressions consists of concrete values which can be integers$\langle I, \rangle$, booleans$\langle B \rangle$, and a special \textsl{unit} value. Furthermore they consist of variables that are referenced by identifiers$\langle Id \rangle$. Finally they consist of arithmetic operations and comparisons of integers. 
\begin{grammar}
	<I> ::= 0 | 1 | -1 | 2 | -2 | $\ldots$
	
	<B> ::= True | False
	
	<CV> ::= <I> 
	\alt <B>
	\alt unit  
	
	<Id> ::= a | b | c | $\ldots$ 
	
	<E> ::= <CV>
	\alt <E> + <E> | <E> - <E> | <E> * <E> | <E> / <E>
	\alt <E> \textless <E> | <E> \textgreater <E> | <E> $\leq$ <E> | <E> $\geq$ <E> | <E> $==$ <E>
\end{grammar}

\subsection{Statements}


\subsubsection{variable declaration and assignment}
Variables implicitly declared, so variable declaration and assignment are contained in the same expression:
\begin{grammar}
	<S> ::= <Id> = <Exp>
\end{grammar} 
The value of an \textsl{assignment}-statement is the value of the expression on the right-hand side.  
\subsubsection{Conditional statements}
\explanguage supports three different conditional statements, namely \textsl{if-then-else} statements, \textsl{while} statements and \textsl{assert} statements:

\begin{grammar}
	<S> ::= if <E> then <S> else <S>
	\alt while <E> do <S>
	\alt assert <E>
\end{grammar}
Where the condition must be an expression that evaluates to a boolean value. 
The value of an \textsl{if-then-else} statement is the value of the statement that ends up being evaluated, depending on the condition. In a \textsl{while} statement, we are not guaranteed that the second statement is evaluated, so we introduce a special \textsl{unit} value which will be the value of any \textsl{while} statement. An assert statement will have the \textsl{unit} value if the condition evaluates to \textsl{true}. If the condition evaluates to \textsl{false}, the execution ends with an error.

\bigskip

Finally a statement may simply be an expression, or one statement followed by another:

\begin{grammar}
	<S> ::= <E>
	\alt <S> <S>
\end{grammar}
 The value of an \textsl{expression} statement is simply the value of the expression, and the value of a \textsl{sequence} statement is the value of evaluating the second statement, using the environment from the result of evaluating the first statement. 
 

\subsubsection{Functions}
\explanguage supports top-level functions that must be defined at the beginning of the program. A function declaration$\langle F \rangle$ consists of an identifier followed by a parameter list with zero or more identifiers and finally a function body which is one or more statements.

\begin{grammar}
	<F> ::= <Id> (<Id>$^{*}$) \{ <E> \}
\end{grammar}

A function call then consists of an identifier, referencing a function declaration, followed by a list of expressions which is the function arguments:

\begin{grammar}
	<E> ::= <Id> (<E>$^{*}$) 
\end{grammar}

The length of the argument list and the parameter list in the declaration must be equal. Furthermore the expressions in the argument list must evaluate to either integers or  boolean values.
The value of a function call is the value of the final statement evaluated in the function body. Since expressions only return values, functions does not have any side effects.


\subsubsection{Programs}
We finally define the syntax of a \explanguage program, which is one or more function declarations, followed by a function call. 

\begin{grammar}
	<P> ::= <F> <F>$^{*}$  <Id> (<E>$^{*}$) 
\end{grammar}

\section{Concrete interpreter for \explanguage}
To interpret the language, we have implemented an interpreter in the programming language \textsl{Scala}. The implementation is purely functional, so we avoid any state and side-effects. We translate the grammar we just described into the following object:

\begin{figure}[!h]
	\begin{lstlisting}[style=simple]
		object ConcreteGrammar {
			sealed trait ConcreteValue
			object ConcreteValue {\\
				case class True() extends ConcreteValue
				case class False() extends ConcreteValue
				case class IntValue(v: Int) extends ConcreteValue
				case class UnitValue() extends ConcreteValue
			}
			sealed trait Exp
			sealed trait Stm
			case class Id(s: String)
			case class FDecl(name: Id, params: List[Id], stm: Stm)
			case class Prog(funcs: HashMap[String, FDecl], fCall: CallExp)	
		}
	\end{lstlisting}
	\caption{High level overview of grammar implementation in scala.}
\end{figure}

The interpreter consists of the following functions:

\begin{figure}[!h]
	\begin{lstlisting}[style=simple]
	class ConcreteInterpreter {
		def interpProg(p: Prog): Result[ConcreteValue, String]
			
		def interpExp(p: Prog,
			e: Exp, 
			env: HashMap[Id, ConcreteValue])
			:Result[ConcreteValue, String]
						  
		def interpStm(p: Prog,
			s: Stm,
			env: HashMap[Id, ConcreteValue]):
			Result[(ConcreteValue, HashMap[Id, ConcreteValue]), String]
	}
	\end{lstlisting}
	\caption{High level overview of the interpreter implementation in Scala. For the full implementation, see apendix A.}
\end{figure}
\newpage
We use an immutable map of type \textsl{HashMap[Id, ConcreteValue]} to represent our program environment.

The interpretation is started by a call to \textsl{interpProg} with a program \textsl{p}, which then call calls \textsl{interpExp(p, p.fCall, env)} where \textsl{env} is a fresh environment. This means that the function referenced by \textsl{p.fCall} acts as a main-function. 

\subsection{Error handling}
We wish to keep our implementation purely functional, so we need to avoid throwing exceptions whenever we encounter an error. Instead we define a trait \textsl{Result[+V, +E]}, that can either be \textsl{Ok(v: V)},  or \textsl{Error(e: E)}, for some types \textsl{V} and \textsl{E}.

\begin{figure}[!h]
	\begin{lstlisting}[style=simple]
		trait Result[+V, +E]
		case class Ok[V](v: V) extends Result[V, Nothing]
		case class Error[E](e: E) extends Result[Nothing, E]
	\end{lstlisting}
\end{figure}

We let \textsl{InterpExp} have return value \textsl{Result[ConcreteValue, String]}, meaning we return \textsl{Ok(v: ConcreteValue)} if we do not encounter any errors, and \textsl{Error(e: String)} if we do. In this case \textsl{e} will be a message that describes what sort of error we encountered. \textsl{InterpStm} has return value 
 \textsl{Result[(ConcreteValue, HashMap[Id, String]), String]} since we also return a potentially updated environment. 

\subsubsection{Map and flatMap}

We define two functions, \textsl{map} and \textsl{flatMap} for \textsl{Result}:

\begin{lstlisting}[style=simple]
	def map[T](f: V => T): Result[T, E] = this match {
	case Ok(v) => Ok(f(v))
	case Error(e) => Error(e)
	}
	
	def flatMap[EE >: E, T](f: V => Result[T, EE]): Result[T, EE] 
	= this match {
		case Ok(v) => f(v)
		case Error(e) => Error(e)
	}
	
	def traverse[V, W, E](vs: List[V])(f: V => Result[W, E]):
	 Result[List[W], E] = vs match {
		case Nil => Ok(Nil)
		case hd::tl => f(hd).flatMap( w => traverse(tl)(f)
	.map(w :: _))
	}
\end{lstlisting}
For some result \textsl{r}, \textsl{map} allows us to apply a function $f: V \rightarrow T$ to \textsl{v} if \textsl{r = Ok(v)}. Otherwise we simply return \textsl{r}. \textsl{flatMap} has the same functionality except that we apply a function that a \textsl{Result}. This way we avoid nesting like \textsl{Ok(Ok(v))}. These two functions allows us to handle errors seamlessly. If, for example we wish to interpret an arithmetic expression \textsl{AExp(e1, e2, Add())}, we simply do 

\begin{lstlisting}[style=simple]
	interpExp(p, e1, env).flatMap(
		v1 => interpExp(p, e2, env).map(v2 => v1.v + v2.v)
	)
\end{lstlisting}
If we encounter an error during the interpretation of \textsl{e1} this error is returned immediately. If not we continue by interpreting \textsl{e2}. If we encounter an error here, we return this error, otherwise we continue and return the sum of the two expressions. Here we assume that \textsl{v1} and \textsl{v2} are of type \textsl{IntValue(v: Int)}. The full implementation also includes checking that both expressions evaluate to the proper types.
\\
Finally we define a function \textsl{traverse}, that takes a list of type \textsl{V}, a function $f: V \rightarrow Result[W, E]$ and returns a \textsl{Result[List[W]], E]}. We use this function to interpret functions calls \textsl{CallExp(Id, List[Exp])}. To do this, we need to evaluate the argument list, and if we do not encounter any errors, construct a local environment and evaluate the function body. We could simply map \textsl{interpExp} on to the argument list, and check for each element in the resulting list if it is an error. This would require two passes over the list. Instead, \textsl{traverse} applies $f$ to each element in the list, and if it results in an error, we immediately return this error. Otherwise we prepend the resulting value onto the result of traversing the remaining list. We only pass over the list once, and we immediately return the first error encountered.
