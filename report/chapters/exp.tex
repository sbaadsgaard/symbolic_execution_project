
In chapter 5 and 6, we describe an implementation of a concrete and symbolic interpreter for a small toy language, written in the programming language \textsl{Scala}. We do this to: 
\begin{itemize}
	\item Demonstrate the principles that we discussed in chapter 3, by describing an actual implementation of symbolic execution for a small toy language.
	\item Be able to clearly see the differences between a symbolic and a concrete execution by comparing the two interpreters, and looking at the necessary changes to the source language. 
\end{itemize} 

In this chapter we introduce the toy language, called \explanguage. We start by giving a formal definition through a grammar, and then we describe a purely functional implementation of a concrete interpreter for the language. We have chosen to implement both interpreters in a purely functional way to avoid dealing with states and side effects. This will allow us to easily showcase the important differences by simply looking at the relevant function signatures. In order to do this, we also describe a technique to avoid throwing exception whenever we encounter errors, such as type errors, assertion errors or referencing undefined variables or functions. Doing this gives a much more elegant implementation, and it greatly simplifies the implementation of the symbolic interpreter as we do not wish to interrupt the execution whenever we encounter such an error.

\section{Syntax of \explanguage}

\explanguage consists of expressions and statements. Expressions evaluates to values and does not change the control flow of the program. A statement evaluates to a value and a possibly updated variable environment. They may also change the control flow of the program through conditional statements.
 

\subsection{Expressions}


Expressions consists of concrete values which can be integers$\langle I, \rangle$, booleans$\langle B \rangle$, and a special \textsl{unit} value. Furthermore they consist of variables referenced by identifiers. Finally they consist of arithmetic operations on integers and comparison of integers 

\begin{alignat*}{3}
	int & \in \textsc{Integer} && := \quad 0 \alt 1 \alt -1 \alt \ldots\\
	bool & \in \textsc{Boolean} && := \quad True \alt False\\
	val & \in \textsc{ConcreteValue} && := \quad int \alt bool \alt unit\\
	id & \in \textsc{Identifier} && := \quad a \alt b \alt c \alt \ldots \\
	exp & \in \textsc{Expression} && := \quad val \alt id\\ & && \alt exp + exp \alt exp - exp \alt exp * exp \alt exp / exp\\
	& && \alt exp < exp \alt exp > exp \alt exp \leq exp\\
	& &&  \alt exp \geq exp \alt exp == exp
\end{alignat*}


\subsection{Statements}
Statements in \explanguage consists of assignments to values. Variables are implicitly declared, so assigning to a variable that does not already exist, will create the variable. 

\begin{alignat*}{3}
	smt & \in \textsc{Statement} && := \quad id  = exp 
\end{alignat*} 
The value of an assignment statement is the value of the expression on the right hand side. This expression must evaluate to either an integer value or a boolean value. \explanguage supports three different conditional statements;  \textsl{if}-statements, \textsl{while}-statements and \textsl{while}-statements. 

\begin{alignat*}{3}
	stm & \in \textsc{Statement} && := \quad \textbf{if }  exp \textbf{ then } stm \textbf{ else } stm\\
	& && \alt \textbf{while } exp \textbf{ do } stm\\
	& && \alt \textbf{assert } exp
\end{alignat*}

The condition expression must evaluate to a boolean value. The value of an \textsl{if}-statement is the value of the statement that gets evaluated depending on the value of the condition. The value of a \textsl{while}-statement is $unit$. The value of an \textsl{assert}-statement is $unit$ if the condition evaluates to $True$, otherwise it is an assertion error. Finally a statement can simply be an expression or a sequence of statements. 

\begin{alignat*}{3}
	stm & \in \textsc{Statement} && := \quad exp \alt stm \ stm
\end{alignat*}


\subsection{Functions and programs}
\explanguage supports top-level functions that must be defined at the beginning of the program. A function definition consists of an identifier, followed by a parameter list with zero or more identifiers. The function body consists of one or more statements. A function call is an expression consisting of an identifier matching one of the defined functions followed by a argument list with zero or more expresssions. The value of a function call is the value of the final statement in the function body. 

\begin{alignat*}{3}
	f & \in \textsc{Function} && := \quad id \ (id^{*}) \ \{stm\}\\
	call & \in \textsc{Expression} && := \quad id \ (exp^{*})
\end{alignat*}

Since expressions only evaluates to a value, it follows that function calls does not have any side effects.

Finally we define the syntax of a \explanguage program to be one or more function declarations, followed by a call to one of these functions. 

\begin{alignat*}{3}
	prog & \in \textsc{Program} && := \quad f \ f^{*} \ call
\end{alignat*}

\section{Concrete interpreter for \explanguage}


\iffalse
\begin{figure}[!h]
	\begin{lstlisting}[style=simple]
		object ConcreteGrammar {
			sealed trait ConcreteValue
			object ConcreteValue {\\
				case class True() extends ConcreteValue
				case class False() extends ConcreteValue
				case class IntValue(v: Int) extends ConcreteValue
				case class UnitValue() extends ConcreteValue
			}
			sealed trait Exp
			sealed trait Stm
			case class Id(s: String)
			case class FDecl(name: Id, params: List[Id], stm: Stm)
			case class Prog(funcs: HashMap[String, FDecl], fCall: CallExp)	
		}
	\end{lstlisting}
	\caption{High level overview of grammar implementation in \textsl{Scala}.}
\end{figure}
\fi

To interpret \explanguage we define the three following functions

\begin{figure}[!h]
	\begin{lstlisting}[style=simple]
	class ConcreteInterpreter {
		def interpProg(p: Prog): Result[ConcreteValue, String]
			
		def interpExp(p: Prog, e: Exp, env: HashMap[Id, ConcreteValue]): Result[ConcreteValue, String]
						  
		def interpStm(p: Prog, s: Stm, env: HashMap[Id, ConcreteValue]): Result[(ConcreteValue, HashMap[Id, ConcreteValue]), String]
	}
	\end{lstlisting}
	\caption{High level overview of the interpreter implementation in \textsl{Scala}.}
\end{figure}

 \noindent where \textsl{interpProg} acts as the entry function. It takes a program \textsl{p} of type \textsl{Prog(funcs: Map[Id, FDecl], fCall: CallExp)} and calls \textsl{interpExp} with \textsl{p}, \textsl{fCall} and a fresh environment. We use an immutable map of type \mbox{\textsl{Map[Id, ConcreteValue]}} to represent our environment. From the signature of \textsl{interpExp} we see that the interpretation of an expression depends on the current environment and the program which holds the top level functions, and that it results in a concrete value. The final function is \textsl{interpStm} whose signature tells us that the interpretation of a statement also depends on the current environment and the program, and that results in both a concrete value and a potentially updated environment. 

\subsection{Error handling without side effects}
 As previously mentioned we wish to avoid side effects in our implementation. We must therefore avoid throwing exceptions whenever we encounter errors.
 Instead we define a trait \textsl{Result[+V, +E]}, that can either be \textsl{Ok(v: V)},  or \textsl{Error(e: E)}, for some types \textsl{V} and \textsl{E} \cite{Chiusano:2014:FPS:2688794}.

\begin{figure}[!h]
	\begin{lstlisting}[style=simple]
		trait Result[+V, +E]
		case class Ok[V](v: V) extends Result[V, Nothing]
		case class Error[E](e: E) extends Result[Nothing, E]
	\end{lstlisting}
\end{figure}

We let \textsl{InterpExp} have return value \textsl{Result[ConcreteValue, String]}, meaning we return \textsl{Ok(v: ConcreteValue)} if we do not encounter any errors, and \textsl{Error(e: String)} if we do. In this case \textsl{e} will be a message that describes what sort of error we encountered. \textsl{InterpStm} has return value 
 \textsl{Result[(ConcreteValue, HashMap[Id, String]), String]} since we also return a potentially updated environment. 

\subsubsection{Map and flatMap}

We define three functions \textsl{map} and \textsl{flatMap} and traverse:

\begin{lstlisting}[style=simple]
	def map[T](f: V => T): Result[T, E] = this match {
	case Ok(v) => Ok(f(v))
	case Error(e) => Error(e)
	}
	
	def flatMap[EE >: E, T](f: V => Result[T, EE]): Result[T, EE] 
	= this match {
		case Ok(v) => f(v)
		case Error(e) => Error(e)
	}
	
	def traverse[V, W, E](vs: List[V])(f: V => Result[W, E]):
	 Result[List[W], E] = vs match {
		case Nil => Ok(Nil)
		case hd::tl => f(hd).flatMap( w => traverse(tl)(f)
	.map(w :: _))
	}
\end{lstlisting}
For some result \textsl{r}, \textsl{map} allows us to apply a function $f: V \rightarrow T$ to \textsl{v} if \textsl{r = Ok(v)}. Otherwise we simply return \textsl{r}. \textsl{flatMap} has the same functionality except that we apply a function that also returns a \textsl{Result}. This way we avoid nesting like \textsl{Ok(Ok(v))}. These two functions allows us to handle errors seamlessly. If, for example we wish to interpret an arithmetic expression \textsl{AExp(e1, e2, Add())}, we simply do 

\begin{lstlisting}[style=simple]
	interpExp(p, e1, env).flatMap(
		v1 => interpExp(p, e2, env).map(v2 => v1.v + v2.v)
	)
\end{lstlisting}
If we encounter an error during the interpretation of \textsl{e1} this error is returned immediately. If not we continue by interpreting \textsl{e2}. If we encounter an error here, we return this error, otherwise we continue and return the sum of the two expressions. Here we assume that \textsl{v1} and \textsl{v2} are of type \textsl{IntValue(v: Int)}. The full implementation also includes checking that both expressions evaluate to the proper types.
\\
Finally we define a function \textsl{traverse}, that takes a list of type \textsl{V}, a function $f: V \rightarrow Result[W, E]$ and returns a \textsl{Result[List[W], E]}. We use this function to interpret functions calls \textsl{CallExp(Id, List[Exp])}. To do this, we need to evaluate the argument list, and if we do not encounter any errors, construct a local environment and evaluate the function body. We could simply map \textsl{interpExp} on to the argument list, and check for each element in the resulting list if it is an error. This would require two passes over the list. Instead, \textsl{traverse} applies $f$ to each element in the list, and if it results in an error, we immediately return this error. Otherwise we prepend the resulting value onto the result of traversing the remaining list. We only pass over the list once, and we immediately return the first error encountered.
