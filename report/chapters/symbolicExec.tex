In this chapter we will cover the theory behind symbolic execution. We will start by describing what it means to \emph{symbolically} execute a program and how we deal branching. We will also explain the connection between a symbolic execution of a program, and a concrete execution. We shall restrict our focus to programs that takes integer values as input and allows us to do arithmetic operations on such values. In the end we will cover the challenges and limitations of symbolic execution that arises when these restrictions are lifted. 


\section{Symbolic executing of a program}
	
	During a normal execution of a program, input values consists of integers. During a symbolic execution we replace concrete values by symbols e.g $\alpha$ and $ \beta$, that acts as placeholders for actual integers. We will refer to symbols and arithmetic expressions over these as \emph{symbolic values}.
	 The program environment consists of variables that can reference both concrete and symbolic values \citep{CadarSen13} \citep{King76}.
	\\
	To illustrate this, we consider the following program that takes parameters $a, b$ and $ c$ and computes their sum. The computation may seem unnecessarily complicated, but we do it this way to clearly what we mean by symbolic values, and how they relate to concrete values.
	\begin{figure}[!h]
		\begin{algorithmic}
			\Procedure{ComputeSum}{$a, b, c$}
			\State $ x := a + b$
			\State $ y := b + c$
			\State $ z := x + y - b$
			\State \Return{$z$}
			\EndProcedure
		\end{algorithmic}
	\end{figure}

	\sumprogram{}
	\newpage
	Lets consider running the program with concrete values $a = 2, b = 3$ and $c = 4$. We then get the following execution:
	First we assign $a+b = 5$ to the variable $x$. Then we assign $b + c = 7$ to the variable $y$. Next we assign $x + y - b = 9$ to variable $z$ and finally we return $z = 9$, which is indeed the sum of 2, 3 and 4. 
	\\
	Let us now run the program with symbolic input values $\alpha, \beta$ and $\gamma$ for $a, b$ and $c$ respectively. 

	
	We then get the following execution: First we assign $\alpha + \beta$ to $x$. We then assign $\beta + \gamma$ to $y$. Next we assign $(\alpha + \beta) + (\beta + \gamma) - \beta$ to $z$.Finally we return $z = \alpha + \beta + \gamma$. We can conclude that the program correctly computes the sum of $a, b$ and $c$, for any possible value of these.
	
\section{Execution paths and path constraints}
		The program that we considered in the previous section contains no conditional statements, which means it only has a single possible execution path. In general, a program with conditional statements $s_1, s_2, \ldots, s_n$ with conditions $q_1, q_2, \ldots, q_n$, will have several execution paths that are uniquely determined by the value of these conditions. In symbolic execution, we model this by introducing a \emph{path-constraint} for each execution path. The \emph{path-constraint} is a list of boolean expressions $\lbrack q_1, q_2, \ldots, q_k \rbrack$ over the symbolic values, corresponding to conditions from the conditional statements along the path. At the start of an execution, the \emph{path-constraint} only contains the expression $true$, since we have not encountered any conditional statements. to continue execution along a path, $q_1 \land \ldots \land q_k$ must be \emph{satisfiable}. To be \emph{satisfiable}, there must exist an assignment of integers to the symbols, such that the conjunction of the conditions evaluates to true. For example, $q = (2\cdot \alpha > \beta) \land (\alpha < \beta)$ is satisfiable, because we can choose $\alpha = 10$ and $\beta = 15$ in which case $q$ evaluates to \emph{true}. On the other hand $q' = (2 \cdot \alpha < 4) \land (\alpha > 4)$ is clearly not satisfiable since the first condition stipulates that $\alpha < 2$ and the second condition stipulates that $\alpha > 4$.
		\\
		
		\iffalse
		 This situation can for example occur if we are executing two consecutive \textsl{if}-statements, where the first one has condition $2\cdot \alpha < 4$ and the second has condition $\alpha > 4$. 
		\fi
	
		\noindent Whenever we reach a conditional statement with condition $q_k$, we consider the two following expressions:
	
		\begin{enumerate}
			\item $ q_1 \land q_2 \land \ldots \land q_k$
			\item $ q_2 \land q_2 \land \ldots \neg q_k$
		\end{enumerate}	
		This gives a number of possible scenarios:	
		\ifstm	
		\begin{itemize}
			\item \textbf{Only the first expression is satisfiable}: Execution continues with a new \emph{path-constraint} $\lbrack q_1, q_2, \ldots, q_k \rbrack$, along the path corresponding to $q_k$ evaluating to to \emph{true}.
			\item \textbf{Only the second expression is satisfiable}:  Execution continues with a new \emph{path-constraint} $\lbrack q_1, q_2, \ldots, \neg q_k \rbrack$, along the path corresponding to $q_k$ evaluating to to \emph{false}.
			
			\item \textbf{Both expressions are satisfiable}: In this case, the execution can continue along two paths; one corresponding to the condition being \emph{false} and one being \emph{true}. At this point we \emph{fork} the execution by considering two different executions of the remaining part of the program. Both executions start with the same environment and \emph{path-constraints} that are equal up to the final condition. One will have $q_k$ as the final condition and the other will have $\neg q_k$. 
			These two executions will continue along different execution paths that differ from this conditional statement and onward \citep{King76}.
		\end{itemize} 
		
		To illustrate this, we consider the program from the motivating example, that takes input parameters $units$ and $minimum$:
		
		\motexample{}
		\newpage
		We assign symbolic values $\alpha$ and $\beta$ to $units$ and $minimum$ respectively, and get the following symbolic execution:
		
		First we assign $2\cdot \alpha$ to $total$. We then reach an \textsl{if}-statement with condition $\alpha \cdot 2 \geq 16$. To proceed, we need to check the satisfiability of the following two expressions:
		\begin{enumerate}
			\item $true \land (\alpha \cdot 2 \geq 16)$
			\item $true \land \neg (\alpha \cdot 2 \geq 16)$.
		\end{enumerate}
		Since both these expressions are satisfiable, we need to fork. We continue execution with a new  \emph{path-constraint} $\lbrack true, (\alpha \cdot 2 \geq 16) \rbrack$, along the path corresponding to the condition evaluating to \emph{true}. We also start a new execution with the same environment and a \emph{path-constraint} equal to $\lbrack true, \neg (\alpha \cdot 2 \geq 16) \rbrack$. This execution will continue along the path corresponding to the condition evaluating to \emph{false}, and it immediately reaches the return statement and returns $\alpha \cdot 2$.
		The first execution assigns $2\cdot \alpha - 10$ to $total$ and then reach an \textsl{assert}-statement with condition $2\cdot \alpha - 10 \geq \beta$. We consider the following expressions:
		\begin{enumerate}
			\item $true \land (\alpha \cdot 2 \geq 16) \land (((2\cdot \alpha) - 10) \geq \beta)$
			\item $true \land (\alpha \cdot 2 \geq 16) \land \neg (((2\cdot \alpha) - 10) \geq \beta)$
		\end{enumerate}
		Both of these expressions are satisfiable, so we fork again. In the end we have discovered all three possible execution paths with the following \emph{path-constraints}:
		\begin{enumerate}
			\item $true \land \neg (\alpha \cdot 2 \geq 16)$
			\item $true \land (\alpha \cdot 2 \geq 16) \land (((2\cdot \alpha) - 10) \geq \beta)$
			\item $true \land (\alpha \cdot 2 \geq 16) 
			\land \neg (((2\cdot \alpha) - 10) \geq \beta)$.
		\end{enumerate}
		
		%\exectree
		The first two \emph{path-constraints} corresponds to the two different paths that leads to the return  ment, where the first one returns $2\cdot \alpha$ and the second one returns $2\cdot \alpha - 10$. Inputs that satisfy these, does not result in a crash.
		The final \emph{path-constraint} corresponds to the path that leads to the \textsl{Assertion Error}, so we can conclude that all input values that satisfy these constraints, will result in a program crash.
	
		
\section{Constraint solving}
	As we just described, a symbolic execution of a program results in one or more \emph{path-constraints} corresponding to each possible execution path. We know that each of these \emph{path-constraints} are satisfiable, so we can solve the \pc by finding an assignment of concrete values to the symbols, that causes it to evaluate to true. Consider for example 
	
	\begin{equation*}
		true \land (\alpha \cdot 2 \geq 16) \land (2\cdot \alpha - 10 \geq \beta)
	\end{equation*}
	
	which is one of the three resulting \emph{path-constraints} from the motivating example. From the first condition we get that $\alpha \geq 8$, so we can choose $\alpha = 8$. The second condition then gives us that $6 \geq \beta$, so we can choose $\beta = 6$. If we do a concrete execution of the program with $units = 8$ and $minimum = 6$, it will follow the execution path that correspond to this \pc. We can do the same for the remaining two \emph{path-constraints}, and in the end we will have a pair of concrete input values for each possible execution path. So symbolic execution not only allows us to explore all possible execution paths, it also allows to generate a small set of concrete input values that cover all these paths.  

	
	\iffalse
	In the previous section we described how to handle programs with multiple execution paths by introducing a \emph{path-constraint} for each path, which a list of constraints on the input symbols. This system of constraints defines a class of integers that will cause the program to execute along this path. By solving the system from each \emph{path-constraint}, we obtain a member from each of class which forms a set of concrete inputs that cover all possible paths.    
	
	If we consider the motivating example again, we found three different paths, represented by the following \emph{path-constraints}:
	\begin{enumerate}
		\item $true \land \neg (\alpha \cdot 2 \geq 16)$
		\item $true \land (\alpha \cdot 2 \geq 16) \land (2\cdot \alpha - 10 \geq \beta)$
		\item $true \land (\alpha \cdot 2 \geq 16) \land \neg (2\cdot \alpha - 10 \geq \beta)$.
	\end{enumerate}
	
	By solving for $\alpha$ and $\beta$, we obtain the set of inputs $\{(7, \beta), (8,6), (8, 7)\}$, that covers all possible execution paths. Note that we have excluded a concrete value for $\beta$ in the first test case, because the \emph{path-constraint} does not depend on the value of $\beta$. 
	 
	\fi 
\section{Limitations and challenges of symbolic execution}
	So far we have only considered symbolic execution of programs with a small number of execution paths. Furthermore, the constraints placed on the input symbols have all been linear expressions.
	In this section we will cover the challenges that arise when we consider more general programs.
	
	\subsection{The number of possible execution paths} 
		Since each conditional statement in a given program can result in two different execution paths, the total number of paths to be explored is potentially exponential in the number of conditional statements. 
		For this reason, the running time of the symbolic execution quickly gets out of hands if we explore all paths. 
		 The challenge gets even greater if the program contains a looping statement. In this case, the number of execution paths is potentially infinite \citep{CadarSen13}.
		  We illustrate this by considering the following program that computes $a^b$ for integers $a$ and $b$, with symbolic values $\alpha$ and $\beta$ for $a$ and $b$:
		\begin{figure}[!h]
			\begin{algorithmic}
				\Procedure{ComputePow}{$a,b$}
				\State $r := 1$
				\State $i := 1$
				\While{$i \leq b$}
					\State $ r := r\cdot a$
					\State $ i := i + 1$
				\EndWhile
				\State \Return{$r$}
				\EndProcedure
			\end{algorithmic}
		\end{figure}
		\pow{}
		
	This program contains a \textsl{while}-statement with condition $i \leq b$. The $k'th$ time we reach this statement we will consider the following two expressions:
	\begin{enumerate}
		\item $true \land (1 \leq \beta) \land (2 \leq \beta) \land \ldots \land (k-1 \leq \beta) $
		\item $true \land (1 \leq \beta) \land (1 \leq \beta) \land \ldots \land \neg (k-1 \leq \beta) $.
	\end{enumerate}
	Both of these expressions are satisfiable, so we fork the execution. This is the case for any $k > 0$, which means that the number of possible execution paths is infinite. If we insist on exploring all paths, the symbolic execution will simply continue for ever. To avoid this, we can include some other termination criteria. As an example, we could have limit on the number of times we allow the execution to fork, and as soon as this limit is reached we simply ignore any further execution paths.
	
	\subsection{Deciding satisfiability of \emph{path-constraints}}
	A key component of symbolic execution, is deciding if a \emph{path-constraint} is satisfiable, in which case the corresponding execution path is eligible for exploration. Consider the following \emph{path-constraint} from the motivating example:
	
	\begin{equation*}	
		true \land (\alpha \cdot 2 \geq 16) \land \neg (2\cdot \alpha - 10 < \beta).
	\end{equation*}
	
	To decide if this is satisfiable or not, we must determine if there exist an assignment of integer values to $\alpha$ and $\beta$ such that the formula evaluates to \emph{true}. We notice that the formula is a conjunction of linear inequalities. We can assign these to variables $q_1$ and $q_2$ and get
	\begin{align*}
		q_1 & = (\alpha \cdot 2 \geq 16) \\
		q_2 & = (2\cdot \alpha - 10 < \beta)
	\end{align*}
	The formula would then be $true\land q_1 \land \neg q_2$,
	where $q_1$ and $q_2$ can have values \emph{true} or \emph{false} depending on whether or not the linear inequality holds for some integer values of $\alpha$ and $\beta$. The question then becomes twofold: Does there exist an assignment of \emph{true} and \emph{false} to $q_1$ and $q_2$ such that the formula evaluates to true? And if so, does this assignment lead to a system of linear inequalities that is satisfiable?
	In this example, we can assign \emph{true} to $q_1$ and \emph{false} to $q_2$, which gives the following system of linear inequalities:
	\begin{align*}
		& \alpha \cdot 2 \geq 16 \\
		2  \cdot & \alpha - \beta \geq 10 
	\end{align*}
	
	where we gathered the constant terms on the left hand side, and the symbols the right hand side. From the first equation we get that $\alpha \geq 8$ so we choose $\alpha = 8$. From the second equation we then get that $\beta \leq 6$, so we choose $\beta = 6$ and this gives us a satisfying assignment for the path constraint. Since deciding satsifiability of a \pc is such an critical part of symbolic execution, it is important to know if we can always correctly \emph{yes} or \emph{no}. In the next section we will study this question more closely, and we will see that it depends on what sort of constraints we place on the input values. 
	\subsubsection{The SMT problem}

	The example we just gave, is an instance of the \emph{Satisfiability Modulo Theories}(SMT) problem. To understand SMT, we first consider the \emph{Boolean Satisfiability}(SAT) problem. In this problem we are given a CNF formula , which is the logical conjunction 1 or more clauses, where a clause is the logical disjunction of one or more boolean variables or their negation.  We want to decide if there exists an assignment of truth values to each variable such that the formula evaluates to true. for example, $(q_1 \lor q_2 \lor \neg q3) \land (q_1 \lor \neg q_2 \lor q_3)$ is a \emph{yes}-instance of this problem, since $q_1 = true$, $q_2 = false$ and $q_3 = true$ causes the formula to evaluate to \emph{true}. On the other hand $(q_1 \lor q_2 \lor q_3) \land (\neg q_1 \lor \neg q_2 \lor \neg q_3)$ is clearly a \emph{no}-instance. This problem is \emph{decidable}, meaning that we can always correctly answer \emph{yes} or \emph{no} to whether or not a given formula is satisfiable. However it is also NP-complete, which means that we do not know any method of solving this problem with a better worst-case running time than simply trying all possible assignments which is exponential in the number of clauses \citep{Miltersen15}. SMT is then an extention of this problem. We now let the boolean variables $q_1, q_2, \ldots, q_n$ represent expressions from some theory such as the \emph{theory of integer linear arithmetic(LIA)} \citep{DeMoura2011}. In LIA, an expression e is defined as 
		\begin{alignat*}{3}
			e & \in \textsc{Expression} && := \quad p \bowtie p\\
			\bowtie & \in \textsc{Constraint} && := \quad \leq \alt \geq \alt =\\
			p & \in \textsc{Polynomial} && := \quad a \alt a\cdot x \alt p \circ p\\
			\circ & \in \textsc{Operator} && := \quad + \alt - \\
			a & \in \textsc{Integer} && := \quad 0 \alt 1 \alt -1 \alt \ldots\\ 
			x & \in \textsc{Variable} && := \quad a \alt b \alt c \alt \ldots		
		\end{alignat*} 
	We now want to decide if the formula is satisfiable with respect to the theory. To be satisfiable w.r.t LIA, we require that there exist an assignment of truth values to the boolean variables such that the formula evaluates to \emph{true}, and that for such an assignment, there exists an assignment of integer values to the variables in the underlying expressions such that all the constraints are satisfied. In the previous section we showed that 
	
	\begin{equation*}	
	true \land (\alpha \cdot 2 \geq 16) \land \neg (2\cdot \alpha - 10 < \beta).
	\end{equation*}

	is a \emph{yes}-instance of the SMT problem, since we could let $q_1$ represent $(\alpha \cdot 2 \geq 16)$ and $q_2$ represent $(2\cdot \alpha - 10 < \beta)$. We could then assign $q_1 = true$ and $q_2 = false$. For this assignment of truth variables, we could choose $\alpha = 8$ and $\beta = 6$ in which case all the constraints hold. On the other hand 
	
	\begin{equation*}
		(2\cdot \alpha < 4) \land (\alpha > 4)
	\end{equation*}
	
	is clearly a \emph{no}-instance. If we let $q_1$ represent $(2\cdot \alpha < 4)$ and $q_2$ represent $(\alpha > 4)$, the formula becomes $q_1\land q_2$. The only assignment of truth variables that satisfy this is $q_1 = true$ and $q_2 = true$. But it is clear that there does not exist an integer value for $\alpha$ such that both the constraints hold. LIA is a \emph{decidable} theory, meaning that we can always correctly answer \emph{yes} or \emph{no} to an instance of the SMT problem, when it is with respect to LIA. Given an SMT formula, we can construct an Integer Linear Program with the expressions from the formula as constraints. This program will we feasible if and only if the SMT formula is satsifiable w.r.t LIA. We can check the feasibility of the Integer linear Program by using the \emph{branch-and-bound} algorithm. If the program is feasible, this will also give us a satisfying assignment \citep{Vanderbei01linearprogramming:}. This means that as long as we only consider programs with linear expressions as conditions, we can always decide the satisfiability of a \pc.
	
	\subsubsection{Undecidable theories}
	Let us consider the following extension of the definition of expressions in LIA:
	\begin{alignat*}{3}
		e & \in \textsc{Expression} && := \quad p \bowtie p\\
		\bowtie & \in \textsc{Constraint} && := \quad \leq \alt \geq \alt =\\
		p & \in \textsc{Polynomial} && := \quad a \alt a\cdot x \alt p \circ p\\
		\circ & \in \textsc{Operator} && := \text{\hl{$\quad + \alt - \alt \cdot \quad$}} \\
		a & \in \textsc{Integer} && := \quad 0 \alt 1 \alt -1 \alt \ldots\\ 
		x & \in \textsc{Variable} && := \quad a \alt b \alt c \alt \ldots
	\end{alignat*}
	
	This definition allows for expressions with nonlinear terms such as $ 5\cdot x - 10 \leq 3 \cdot y^3$. Such expressions belong to the \emph{Theory of Nonlinear Integer Arithmetic}(NLIA). This theory is \emph{undecidable}, meaning that there does not exist an algorithm that always halt with a correct answer to the question of whether or not there exists an assignment of integer values to the variables such that all constraints are satisfied. 
	
	\subsubsection{Consequences of undecidable theories for symbolic execution.}
	If we do symbolic execution on a program that has conditional statements with conditions that belong to an undecidable theory, we might get stuck trying to decide the satisfiability of a path constraint forever. To avoid this, we can include some other termination criteria, such as a time limit, at which point we output \emph{unknown} instead of \emph{yes} or \emph{no}. This leaves us with two options. We can continue the symbolic execution and assume that the \pc is in fact satisfiable. In this case, we might end up exploring execution paths that are infeasible, meaning that no concrete input values will cause the program to execute along this path. The other option is to assume that the \pc is unsatisfiable. In this case we do not explore infeasible execution paths, but we can no longer be sure that we explore all feasible paths.