Consider the following program that takes integer inputs $units$ and $cost$ 
\motexample{}
We would like to know if this program ever fails, so we have to figure out if there exist integer inputs for which the program reaches the $Error$ statement. We might try to run the program on different input values, e.g $(units = 8, cost = 5)$, $(units = 7, cost = 10)$. Running the program with these inputs, does not crash the program, but we are still not convinced that it wont crash for some other input values.
By observing the program long enough, we realize that the input must satisfy the following two constraints to crash:

\begin{align*}
	 units \cdot 2 & \geq 16\\
	 units \cdot 2 & < cost
\end{align*}

which is the case for $(units = 8, cost = 7)$. This realization was not immediately obvious, and for more complex programs, answering the same question is even more difficult. The key insight is that the conditional statements dictates which execution path the program will follow. In this report we will present \emph{symbolic execution}, which is a technique to systematically explore different execution paths and generate concrete input values that will follow these same paths. 