% \begin{meta-comment}
%
% $Id: mdwtab.dtx,v 1.8 1996/12/09 23:20:42 mdw Exp $
%
% Another rewrite of the tabular environment, and maths alignments
%
% (c) 1996 Mark Wooding
%
%----- Revision history -----------------------------------------------------
%
% $Log: mdwtab.dtx,v $
% Revision 1.8  1996/12/09 23:20:42  mdw
% (\tab@setstrut): Fixed so that it uses \dimen\tw@ for the strut depth,
% as advertised.
%
% Revision 1.7  1996/11/29 21:59:16  mdw
% Fixed a little formatting mistake in a syntax diagram, and switched over
% to the new syntax diagram commands on the grounds that they're slightly
% less messy.  Maybe.
%
% Revision 1.6  1996/11/19 20:54:33  mdw
% Entered into RCS
%
%
% \end{meta-comment}
%
% \begin{meta-comment} <general public licence>
%%
%% mdwtab package -- another rewrite of the tabular environment, etc.
%% Copyright (c) 1996 Mark Wooding
%%
%% This program is free software; you can redistribute it and/or modify
%% it under the terms of the GNU General Public License as published by
%% the Free Software Foundation; either version 2 of the License, or
%% (at your option) any later version.
%%
%% This program is distributed in the hope that it will be useful,
%% but WITHOUT ANY WARRANTY; without even the implied warranty of
%% MERCHANTABILITY or FITNESS FOR A PARTICULAR PURPOSE.  See the
%% GNU General Public License for more details.
%%
%% You should have received a copy of the GNU General Public License
%% along with this program; if not, write to the Free Software
%% Foundation, Inc., 675 Mass Ave, Cambridge, MA 02139, USA.
%%
% \end{meta-comment}
%
% \begin{meta-comment} <Package preambles>
%<+mdwtab>\NeedsTeXFormat{LaTeX2e}
%<+mdwtab>\ProvidesPackage{mdwtab}
%<+mdwtab>                [1998/04/28 1.9 Table typesetting with style]
%<+mathenv>\NeedsTeXFormat{LaTeX2e}
%<+mathenv>\ProvidesPackage{mathenv}
%<+mathenv>                [1998/04/28 1.9 Various maths environments]
% \end{meta-comment}
%
% \CheckSum{2758}
%% \CharacterTable
%%  {Upper-case    \A\B\C\D\E\F\G\H\I\J\K\L\M\N\O\P\Q\R\S\T\U\V\W\X\Y\Z
%%   Lower-case    \a\b\c\d\e\f\g\h\i\j\k\l\m\n\o\p\q\r\s\t\u\v\w\x\y\z
%%   Digits        \0\1\2\3\4\5\6\7\8\9
%%   Exclamation   \!     Double quote  \"     Hash (number) \#
%%   Dollar        \$     Percent       \%     Ampersand     \&
%%   Acute accent  \'     Left paren    \(     Right paren   \)
%%   Asterisk      \*     Plus          \+     Comma         \,
%%   Minus         \-     Point         \.     Solidus       \/
%%   Colon         \:     Semicolon     \;     Less than     \<
%%   Equals        \=     Greater than  \>     Question mark \?
%%   Commercial at \@     Left bracket  \[     Backslash     \\
%%   Right bracket \]     Circumflex    \^     Underscore    \_
%%   Grave accent  \`     Left brace    \{     Vertical bar  \|
%%   Right brace   \}     Tilde         \~}
%%
%
% \begin{meta-comment}
%
%<*driver>
% \begin{meta-comment}
%
% $Id: mdwtools.tex,v 1.4 1996/11/19 20:55:55 mdw Exp $
%
% Common declarations for mdwtools.dtx files
%
% (c) 1996 Mark Wooding
%
%----- Revision history -----------------------------------------------------
%
% $Log: mdwtools.tex,v $
% Revision 1.4  1996/11/19 20:55:55  mdw
% Entered into RCS
%
%
% \end{meta-comment}
%
% \begin{meta-comment} <general public licence>
%%
%% mdwtools common declarations
%% Copyright (c) 1996 Mark Wooding
%%
%% This program is free software; you can redistribute it and/or modify
%% it under the terms of the GNU General Public License as published by
%% the Free Software Foundation; either version 2 of the License, or
%% (at your option) any later version.
%%
%% This program is distributed in the hope that it will be useful,
%% but WITHOUT ANY WARRANTY; without even the implied warranty of
%% MERCHANTABILITY or FITNESS FOR A PARTICULAR PURPOSE.  See the
%% GNU General Public License for more details.
%%
%% You should have received a copy of the GNU General Public License
%% along with this program; if not, write to the Free Software
%% Foundation, Inc., 675 Mass Ave, Cambridge, MA 02139, USA.
%%
% \end{meta-comment}
%
% \begin{meta-comment} <file preamble>
%<*mdwtools>
\ProvidesFile{mdwtools.tex}
             [1996/05/10 1.4 Shared definitions for mdwtools .dtx files]
%</mdwtools>
% \end{meta-comment}
%
% \CheckSum{668}
%% \CharacterTable
%%  {Upper-case    \A\B\C\D\E\F\G\H\I\J\K\L\M\N\O\P\Q\R\S\T\U\V\W\X\Y\Z
%%   Lower-case    \a\b\c\d\e\f\g\h\i\j\k\l\m\n\o\p\q\r\s\t\u\v\w\x\y\z
%%   Digits        \0\1\2\3\4\5\6\7\8\9
%%   Exclamation   \!     Double quote  \"     Hash (number) \#
%%   Dollar        \$     Percent       \%     Ampersand     \&
%%   Acute accent  \'     Left paren    \(     Right paren   \)
%%   Asterisk      \*     Plus          \+     Comma         \,
%%   Minus         \-     Point         \.     Solidus       \/
%%   Colon         \:     Semicolon     \;     Less than     \<
%%   Equals        \=     Greater than  \>     Question mark \?
%%   Commercial at \@     Left bracket  \[     Backslash     \\
%%   Right bracket \]     Circumflex    \^     Underscore    \_
%%   Grave accent  \`     Left brace    \{     Vertical bar  \|
%%   Right brace   \}     Tilde         \~}
%%
%
% \section{Introduction and user guide}
%
% This file is really rather strange; it gets |\input| by other package
% documentation files to set up most of the environmental gubbins for them.
% It handles almost everything, like loading a document class, finding any
% packages, and building and formatting the title.
%
% It also offers an opportunity for users to customise my nice documentation,
% by using a |mdwtools.cfg| file (not included).
%
%
% \subsection{Declarations}
%
% A typical documentation file contains something like
% \begin{listinglist} \listingsize \obeylines
% |% \begin{meta-comment}
%
% $Id: mdwtools.tex,v 1.4 1996/11/19 20:55:55 mdw Exp $
%
% Common declarations for mdwtools.dtx files
%
% (c) 1996 Mark Wooding
%
%----- Revision history -----------------------------------------------------
%
% $Log: mdwtools.tex,v $
% Revision 1.4  1996/11/19 20:55:55  mdw
% Entered into RCS
%
%
% \end{meta-comment}
%
% \begin{meta-comment} <general public licence>
%%
%% mdwtools common declarations
%% Copyright (c) 1996 Mark Wooding
%%
%% This program is free software; you can redistribute it and/or modify
%% it under the terms of the GNU General Public License as published by
%% the Free Software Foundation; either version 2 of the License, or
%% (at your option) any later version.
%%
%% This program is distributed in the hope that it will be useful,
%% but WITHOUT ANY WARRANTY; without even the implied warranty of
%% MERCHANTABILITY or FITNESS FOR A PARTICULAR PURPOSE.  See the
%% GNU General Public License for more details.
%%
%% You should have received a copy of the GNU General Public License
%% along with this program; if not, write to the Free Software
%% Foundation, Inc., 675 Mass Ave, Cambridge, MA 02139, USA.
%%
% \end{meta-comment}
%
% \begin{meta-comment} <file preamble>
%<*mdwtools>
\ProvidesFile{mdwtools.tex}
             [1996/05/10 1.4 Shared definitions for mdwtools .dtx files]
%</mdwtools>
% \end{meta-comment}
%
% \CheckSum{668}
%% \CharacterTable
%%  {Upper-case    \A\B\C\D\E\F\G\H\I\J\K\L\M\N\O\P\Q\R\S\T\U\V\W\X\Y\Z
%%   Lower-case    \a\b\c\d\e\f\g\h\i\j\k\l\m\n\o\p\q\r\s\t\u\v\w\x\y\z
%%   Digits        \0\1\2\3\4\5\6\7\8\9
%%   Exclamation   \!     Double quote  \"     Hash (number) \#
%%   Dollar        \$     Percent       \%     Ampersand     \&
%%   Acute accent  \'     Left paren    \(     Right paren   \)
%%   Asterisk      \*     Plus          \+     Comma         \,
%%   Minus         \-     Point         \.     Solidus       \/
%%   Colon         \:     Semicolon     \;     Less than     \<
%%   Equals        \=     Greater than  \>     Question mark \?
%%   Commercial at \@     Left bracket  \[     Backslash     \\
%%   Right bracket \]     Circumflex    \^     Underscore    \_
%%   Grave accent  \`     Left brace    \{     Vertical bar  \|
%%   Right brace   \}     Tilde         \~}
%%
%
% \section{Introduction and user guide}
%
% This file is really rather strange; it gets |\input| by other package
% documentation files to set up most of the environmental gubbins for them.
% It handles almost everything, like loading a document class, finding any
% packages, and building and formatting the title.
%
% It also offers an opportunity for users to customise my nice documentation,
% by using a |mdwtools.cfg| file (not included).
%
%
% \subsection{Declarations}
%
% A typical documentation file contains something like
% \begin{listinglist} \listingsize \obeylines
% |% \begin{meta-comment}
%
% $Id: mdwtools.tex,v 1.4 1996/11/19 20:55:55 mdw Exp $
%
% Common declarations for mdwtools.dtx files
%
% (c) 1996 Mark Wooding
%
%----- Revision history -----------------------------------------------------
%
% $Log: mdwtools.tex,v $
% Revision 1.4  1996/11/19 20:55:55  mdw
% Entered into RCS
%
%
% \end{meta-comment}
%
% \begin{meta-comment} <general public licence>
%%
%% mdwtools common declarations
%% Copyright (c) 1996 Mark Wooding
%%
%% This program is free software; you can redistribute it and/or modify
%% it under the terms of the GNU General Public License as published by
%% the Free Software Foundation; either version 2 of the License, or
%% (at your option) any later version.
%%
%% This program is distributed in the hope that it will be useful,
%% but WITHOUT ANY WARRANTY; without even the implied warranty of
%% MERCHANTABILITY or FITNESS FOR A PARTICULAR PURPOSE.  See the
%% GNU General Public License for more details.
%%
%% You should have received a copy of the GNU General Public License
%% along with this program; if not, write to the Free Software
%% Foundation, Inc., 675 Mass Ave, Cambridge, MA 02139, USA.
%%
% \end{meta-comment}
%
% \begin{meta-comment} <file preamble>
%<*mdwtools>
\ProvidesFile{mdwtools.tex}
             [1996/05/10 1.4 Shared definitions for mdwtools .dtx files]
%</mdwtools>
% \end{meta-comment}
%
% \CheckSum{668}
%% \CharacterTable
%%  {Upper-case    \A\B\C\D\E\F\G\H\I\J\K\L\M\N\O\P\Q\R\S\T\U\V\W\X\Y\Z
%%   Lower-case    \a\b\c\d\e\f\g\h\i\j\k\l\m\n\o\p\q\r\s\t\u\v\w\x\y\z
%%   Digits        \0\1\2\3\4\5\6\7\8\9
%%   Exclamation   \!     Double quote  \"     Hash (number) \#
%%   Dollar        \$     Percent       \%     Ampersand     \&
%%   Acute accent  \'     Left paren    \(     Right paren   \)
%%   Asterisk      \*     Plus          \+     Comma         \,
%%   Minus         \-     Point         \.     Solidus       \/
%%   Colon         \:     Semicolon     \;     Less than     \<
%%   Equals        \=     Greater than  \>     Question mark \?
%%   Commercial at \@     Left bracket  \[     Backslash     \\
%%   Right bracket \]     Circumflex    \^     Underscore    \_
%%   Grave accent  \`     Left brace    \{     Vertical bar  \|
%%   Right brace   \}     Tilde         \~}
%%
%
% \section{Introduction and user guide}
%
% This file is really rather strange; it gets |\input| by other package
% documentation files to set up most of the environmental gubbins for them.
% It handles almost everything, like loading a document class, finding any
% packages, and building and formatting the title.
%
% It also offers an opportunity for users to customise my nice documentation,
% by using a |mdwtools.cfg| file (not included).
%
%
% \subsection{Declarations}
%
% A typical documentation file contains something like
% \begin{listinglist} \listingsize \obeylines
% |\input{mdwtools}|
% \<declarations>
% |\mdwdoc|
% \end{listinglist}
% The initial |\input| reads in this file and sets up the various commands
% which may be needed.  The final |\mdwdoc| actually starts the document,
% inserting a title (which is automatically generated), a table of
% contents etc., and reads the documentation file in (using the |\DocInput|
% command from the \package{doc} package.
%
% \subsubsection{Describing packages}
%
% \DescribeMacro{\describespackage}
% \DescribeMacro{\describesclass}
% \DescribeMacro{\describesfile}
% \DescribeMacro{\describesfile*}
% The most important declarations are those which declare what the
% documentation describes.  Saying \syntax{"\\describespackage{<package>}"}
% loads the \<package> (if necessary) and adds it to the auto-generated
% title, along with a footnote containing version information.  Similarly,
% |\describesclass| adds a document class name to the title (without loading
% it -- the document itself must do this, with the |\documentclass| command).
% For files which aren't packages or classes, use the |\describesfile| or
% |\describesfile*| command (the $*$-version won't |\input| the file, which
% is handy for files like |mdwtools.tex|, which are already input).
%
% \DescribeMacro{\author}
% \DescribeMacro{\date}
% \DescribeMacro{\title}
% The |\author|, |\date| and |\title| declarations work slightly differently
% to normal -- they ensure that only the \emph{first} declaration has an
% effect.  (Don't you play with |\author|, please, unless you're using this
% program to document your own packages.)  Using |\title| suppresses the
% automatic title generation.
%
% \DescribeMacro{\docdate}
% The default date is worked out from the version string of the package or
% document class whose name is the same as that of the documentation file.
% You can choose a different `main' file by saying
% \syntax{"\\docdate{"<file>"}"}.
%
% \subsubsection{Contents handling}
%
% \DescribeMacro{\addcontents}
% A documentation file always has a table of contents.  Other
% contents-like lists can be added by saying
% \syntax{"\\addcontents{"<extension>"}{"<command>"}"}.  The \<extension>
% is the file extension of the contents file (e.g., \lit{lot} for the
% list of tables); the \<command> is the command to actually typeset the
% contents file (e.g., |\listoftables|).
%
% \subsubsection{Other declarations}
%
% \DescribeMacro{\implementation}
% The \package{doc} package wants you to say
% \syntax{"\\StopEventually{"<stuff>"}"}' before describing the package
% implementation.  Using |mdwtools.tex|, you just say |\implementation|, and
% everything works.  It will automatically read in the licence text (from
% |gpl.tex|, and wraps some other things up.
%
% 
% \subsection{Other commands}
%
% The |mdwtools.tex| file includes the \package{syntax} and \package{sverb}
% packages so that they can be used in documentation files.  It also defines
% some trivial commands of its own.
%
% \DescribeMacro{\<}
% Saying \syntax{"\\<"<text>">" is the same as "\\synt{"<text>"}"}; this
% is a simple abbreviation.
%
% \DescribeMacro{\smallf}
% Saying \syntax{"\\smallf" <number>"/"<number>} typesets a little fraction,
% like this: \smallf 3/4.  It's useful when you want to say that the default
% value of a length is 2 \smallf 1/2\,pt, or something like that.
%
%
% \subsection{Customisation}
%
% You can customise the way that the package documentation looks by writing
% a file called |mdwtools.cfg|.  You can redefine various commands (before
% they're defined here, even; |mdwtools.tex| checks most of the commands that
% it defines to make sure they haven't been defined already.
%
% \DescribeMacro{\indexing}
% If you don't want the prompt about whether to generate index files, you
% can define the |\indexing| command to either \lit{y} or \lit{n}.  I'd
% recommend that you use |\providecommand| for this, to allow further
% customisation from the command line.
%
% \DescribeMacro{\mdwdateformat}
% If you don't like my date format (maybe you're American or something),
% you can redefine the |\mdwdateformat| command.  It takes three arguments:
% the year, month and date, as numbers; it should expand to something which
% typesets the date nicely.  The default format gives something like
% `10 May 1996'.  You can produce something rather more exotic, like
% `10\textsuperscript{th} May \textsc{\romannumeral 1996}' by saying
%\begin{listing}
%\newcommand{\mdwdateformat}[3]{%
%  \number#3\textsuperscript{\numsuffix{#3}}\ %
%  \monthname{#2}\ %
%  \textsc{\romannumeral #1}%
%}
%\end{listing}
% \DescribeMacro{\monthname}
% \DescribeMacro{\numsuffix}
% Saying \syntax{"\\monthname{"<number>"}"} expands to the name of the
% numbered month (which can be useful when doing date formats).  Saying
% \syntax{"\\numsuffix{"<number>"}"} will expand to the appropriate suffix
% (`th' or `rd' or whatever) for the \<number>.  You'll have to superscript
% it yourself, if this is what you want to do.  Putting the year number
% in roman numerals is just pretentious |;-)|.
%
% \DescribeMacro{\mdwhook}
% After all the declarations in |mdwtools.tex|, the command |\mdwhook| is
% executed, if it exists.  This can be set up by the configuration file
% to do whatever you want.
%
% There are lots of other things you can play with; you should look at the
% implementation section to see what's possible.
%
% \implementation
%
% \section{Implementation}
%
%    \begin{macrocode}
%<*mdwtools>
%    \end{macrocode}
%
% The first thing is that I'm not a \LaTeX\ package or anything official
% like that, so I must enable `|@|' as a letter by hand.
%
%    \begin{macrocode}
\makeatletter
%    \end{macrocode}
%
% Now input the user's configuration file, if it exists.  This is fairly
% simple stuff.
%
%    \begin{macrocode}
\@input{mdwtools.cfg}
%    \end{macrocode}
%
% Well, that's the easy bit done.
%
%
% \subsection{Initialisation}
%
% Obviously the first thing to do is to obtain a document class.  Obviously,
% it would be silly to do this if a document class has already been loaded,
% either by the package documentation or by the configuration file.
%
% The only way I can think of for finding out if a document class is already
% loaded is by seeing if the |\documentclass| command has been redefined
% to raise an error.  This isn't too hard, really.
%
%    \begin{macrocode}
\ifx\documentclass\@twoclasseserror\else
  \documentclass[a4paper]{ltxdoc}
  \ifx\doneclasses\mdw@undefined\else\doneclasses\fi
\fi
%    \end{macrocode}
%
% As part of my standard environment, I'll load some of my more useful
% packages.  If they're already loaded (possibly with different options),
% I'll not try to load them again.
%
%    \begin{macrocode}
\@ifpackageloaded{doc}{}{\usepackage{doc}}
\@ifpackageloaded{syntax}{}{\usepackage[rounded]{syntax}}
\@ifpackageloaded{sverb}{}{\usepackage{sverb}}
%    \end{macrocode}
%
%
% \subsection{Some macros for interaction}
%
% I like the \LaTeX\ star-boxes, although it's a pain having to cope with
% \TeX's space-handling rules.  I'll define a new typing-out macro which
% makes spaces more significant, and has a $*$-version which doesn't put
% a newline on the end, and interacts prettily with |\read|.
%
% First of all, I need to make spaces active, so I can define things about
% active spaces.
%
%    \begin{macrocode}
\begingroup\obeyspaces
%    \end{macrocode}
%
% Now to define the main macro.  This is easy stuff.  Spaces must be
% carefully rationed here, though.
%
% I'll start a group, make spaces active, and make spaces expand to ordinary
% space-like spaces.  Then I'll look for a star, and pass either |\message|
% (which doesn't start a newline, and interacts with |\read| well) or
% |\immediate\write 16| which does a normal write well.
%
%    \begin{macrocode}
\gdef\mdwtype{%
\begingroup\catcode`\ \active\let \space%
\@ifstar{\mdwtype@i{\message}}{\mdwtype@i{\immediate\write\sixt@@n}}%
}
\endgroup
%    \end{macrocode}
%
% Now for the easy bit.  I have the thing to do, and the thing to do it to,
% so do that and end the group.
%
%    \begin{macrocode}
\def\mdwtype@i#1#2{#1{#2}\endgroup}
%    \end{macrocode}
%
%
% \subsection{Decide on indexing}
%
% A configuration file can decide on indexing by defining the |\indexing|
% macro to either \lit{y} or \lit{n}.  If it's not set, then I'll prompt
% the user.
%
% First of all, I want a switch to say whether I'm indexing.
%
%    \begin{macrocode}
\newif\ifcreateindex
%    \end{macrocode}
%
% Right: now I need to decide how to make progress.  If the macro's not set,
% then I want to set it, and start a row of stars.
%
%    \begin{macrocode}
\ifx\indexing\@@undefined
  \mdwtype{*****************************}
  \def\indexing{?}
\fi
%    \end{macrocode}
%
% Now enter a loop, asking the user whether to do indexing, until I get
% a sensible answer.
%
%    \begin{macrocode}
\loop
  \@tempswafalse
  \if y\indexing\@tempswatrue\createindextrue\fi
  \if Y\indexing\@tempswatrue\createindextrue\fi
  \if n\indexing\@tempswatrue\createindexfalse\fi
  \if N\indexing\@tempswatrue\createindexfalse\fi
  \if@tempswa\else
  \mdwtype*{* Create index files? (y/n) *}
  \read\sixt@@n to\indexing%
\repeat
%    \end{macrocode}
%
% Now, based on the results of that, display a message about the indexing.
%
%    \begin{macrocode}
\mdwtype{*****************************}
\ifcreateindex
  \mdwtype{* Creating index files      *}
  \mdwtype{* This may take some time   *}
\else
  \mdwtype{* Not creating index files  *}
\fi
\mdwtype{*****************************}
%    \end{macrocode}
%
% Now I can play with the indexing commands of the \package{doc} package
% to do whatever it is that the user wants.
%
%    \begin{macrocode}
\ifcreateindex
  \CodelineIndex
  \EnableCrossrefs
\else
  \CodelineNumbered
  \DisableCrossrefs
\fi
%    \end{macrocode}
%
% And register lots of plain \TeX\ things which shouldn't be indexed.
% This contains lots of |\if|\dots\ things which don't fit nicely in
% conditionals, which is a shame.  Still, it doesn't matter that much,
% really.
%
%    \begin{macrocode}
\DoNotIndex{\def,\long,\edef,\xdef,\gdef,\let,\global}
\DoNotIndex{\if,\ifnum,\ifdim,\ifcat,\ifmmode,\ifvmode,\ifhmode,%
            \iftrue,\iffalse,\ifvoid,\ifx,\ifeof,\ifcase,\else,\or,\fi}
\DoNotIndex{\box,\copy,\setbox,\unvbox,\unhbox,\hbox,%
            \vbox,\vtop,\vcenter}
\DoNotIndex{\@empty,\immediate,\write}
\DoNotIndex{\egroup,\bgroup,\expandafter,\begingroup,\endgroup}
\DoNotIndex{\divide,\advance,\multiply,\count,\dimen}
\DoNotIndex{\relax,\space,\string}
\DoNotIndex{\csname,\endcsname,\@spaces,\openin,\openout,%
            \closein,\closeout}
\DoNotIndex{\catcode,\endinput}
\DoNotIndex{\jobname,\message,\read,\the,\m@ne,\noexpand}
\DoNotIndex{\hsize,\vsize,\hskip,\vskip,\kern,\hfil,\hfill,\hss}
\DoNotIndex{\m@ne,\z@,\z@skip,\@ne,\tw@,\p@}
\DoNotIndex{\dp,\wd,\ht,\vss,\unskip}
%    \end{macrocode}
%
% Last bit of indexing stuff, for now: I'll typeset the index in two columns
% (the default is three, which makes them too narrow for my tastes).
%
%    \begin{macrocode}
\setcounter{IndexColumns}{2}
%    \end{macrocode}
%
%
% \subsection{Selectively defining things}
%
% I don't want to tread on anyone's toes if they redefine any of these
% commands and things in a configuration file.  The following definitions
% are fairly evil, but should do the job OK.
%
% \begin{macro}{\@gobbledef}
%
% This macro eats the following |\def|inition, leaving not a trace behind.
%
%    \begin{macrocode}
\def\@gobbledef#1#{\@gobble}
%    \end{macrocode}
%
% \end{macro}
%
% \begin{macro}{\tdef}
% \begin{macro}{\tlet}
%
% The |\tdef| command is a sort of `tentative' definition -- it's like
% |\def| if the control sequence named doesn't already have a definition.
% |\tlet| does the same thing with |\let|.
%
%    \begin{macrocode}
\def\tdef#1{
  \ifx#1\@@undefined%
    \expandafter\def\expandafter#1%
  \else%
    \expandafter\@gobbledef%
  \fi%
}
\def\tlet#1#2{\ifx#1\@@undefined\let#1=#2\fi}
%    \end{macrocode}
%
% \end{macro}
% \end{macro}
%
%
% \subsection{General markup things}
%
% Now for some really simple things.  I'll define how to typeset package
% names and environment names (both in the sans serif font, for now).
%
%    \begin{macrocode}
\tlet\package\textsf
\tlet\env\textsf
%    \end{macrocode}
%
% I'll define the |\<|\dots|>| shortcut for syntax items suggested in the
% \package{syntax} package.
%
%    \begin{macrocode}
\tdef\<#1>{\synt{#1}}
%    \end{macrocode}
%
% And because it's used in a few places (mainly for typesetting lengths),
% here's a command for typesetting fractions in text.
%
%    \begin{macrocode}
\tdef\smallf#1/#2{\ensuremath{^{#1}\!/\!_{#2}}}
%    \end{macrocode}
%
%
% \subsection{A table environment}
%
% \begin{environment}{tab}
%
% Most of the packages don't use the (obviously perfect) \package{mdwtab}
% package, because it's big, and takes a while to load.  Here's an
% environment for typesetting centred tables.  The first (optional) argument
% is some declarations to perform.  The mandatory argument is the table
% preamble (obviously).
%
%    \begin{macrocode}
\@ifundefined{tab}{%
  \newenvironment{tab}[2][\relax]{%
    \par\vskip2ex%
    \centering%
    #1%
    \begin{tabular}{#2}%
  }{%
    \end{tabular}%
    \par\vskip2ex%
  }
}{}
%    \end{macrocode}
%
% \end{environment}
%
%
% \subsection{Commenting out of stuff}
%
% \begin{environment}{meta-comment}
%
% Using |\iffalse|\dots|\fi| isn't much fun.  I'll define a gobbling
% environment using the \package{sverb} stuff.
%
%    \begin{macrocode}
\ignoreenv{meta-comment}
%    \end{macrocode}
%
% \end{environment}
%
%
% \subsection{Float handling}
%
% This gubbins will try to avoid float pages as much as possible, and (with
% any luck) encourage floats to be put on the same pages as text.
%
%    \begin{macrocode}
\def\textfraction{0.1}
\def\topfraction{0.9}
\def\bottomfraction{0.9}
\def\floatpagefraction{0.7}
%    \end{macrocode}
%
% Now redefine the default float-placement parameters to allow `here' floats.
%
%    \begin{macrocode}
\def\fps@figure{htbp}
\def\fps@table{htbp}
%    \end{macrocode}
%
%
% \subsection{Other bits of parameter tweaking}
%
% Make \env{grammar} environments look pretty, by indenting the left hand
% sides by a large amount.
%
%    \begin{macrocode}
\grammarindent1in
%    \end{macrocode}
%
% I don't like being told by \TeX\ that my paragraphs are hard to linebreak:
% I know this already.  This lot should shut \TeX\ up about most problems.
%
%    \begin{macrocode}
\sloppy
\hbadness\@M
\hfuzz10\p@
%    \end{macrocode}
%
% Also make \TeX\ shut up in the index.  The \package{multicol} package
% irritatingly plays with |\hbadness|.  This is the best hook I could find
% for playing with this setting.
%
%    \begin{macrocode}
\expandafter\def\expandafter\IndexParms\expandafter{%
  \IndexParms%
  \hbadness\@M%
}
%    \end{macrocode}
%
% The other thing I really don't like is `Marginpar moved' warnings.  This
% will get rid of them, and lots of other \LaTeX\ warnings at the same time.
%
%    \begin{macrocode}
\let\@latex@warning@no@line\@gobble
%    \end{macrocode}
%
% Put some extra space between table rows, please.
%
%    \begin{macrocode}
\def\arraystretch{1.2}
%    \end{macrocode}
%
% Most of the code is at guard level one, so typeset that in upright text.
%
%    \begin{macrocode}
\setcounter{StandardModuleDepth}{1}
%    \end{macrocode}
%
%
% \subsection{Contents handling}
%
% I use at least one contents file (the main table of contents) although
% I may want more.  I'll keep a list of contents files which I need to
% handle.
%
% There are two things I need to do to contents files here:
% \begin{itemize}
% \item I must typeset the table of contents at the beginning of the
%       document; and
% \item I want to typeset tables of contents in two columns (using the
%       \package{multicol} package).
% \end{itemize}
%
% The list consists of items of the form
% \syntax{"\\do{"<extension>"}{"<command>"}"}, where \<extension> is the
% file extension of the contents file, and \<command> is the command to
% typeset it.
%
% \begin{macro}{\docontents}
%
% This is where I keep the list of contents files.  I'll initialise it to
% just do the standard contents table.
%
%    \begin{macrocode}
\def\docontents{\do{toc}{\tableofcontents}}
%    \end{macrocode}
%
% \end{macro}
%
% \begin{macro}{\addcontents}
%
% By saying \syntax{"\\addcontents{"<extension>"}{"<command>"}"}, a document
% can register a new table of contents which gets given the two-column
% treatment properly.  This is really easy to implement.
%
%    \begin{macrocode}
\def\addcontents#1#2{%
  \toks@\expandafter{\docontents\do{#1}{#2}}%
  \edef\docontents{\the\toks@}%
}
%    \end{macrocode}
%
% \end{macro}
%
%
% \subsection{Finishing it all off}
%
% \begin{macro}{\finalstuff}
%
% The |\finalstuff| macro is a hook for doing things at the end of the
% document.  Currently, it inputs the licence agreement as an appendix.
%
%    \begin{macrocode}
\tdef\finalstuff{\appendix\part*{Appendix}\input{gpl}}
%    \end{macrocode}
%
% \end{macro}
%
% \begin{macro}{\implementation}
%
% The |\implementation| macro starts typesetting the implementation of
% the package(s).  If we're not doing the implementation, it just does
% this lot and ends the input file.
%
% I define a macro with arguments inside the |\StopEventually|, which causes
% problems, since the code gets put through an extra level of |\def|fing
% depending on whether the implementation stuff gets typeset or not.  I'll
% store the code I want to do in a separate macro.
%
%    \begin{macrocode}
\def\implementation{\StopEventually{\attheend}}
%    \end{macrocode}
%
% Now for the actual activity.  First, I'll do the |\finalstuff|.  Then, if
% \package{doc}'s managed to find the \package{multicol} package, I'll add
% the end of the environment to the end of each contents file in the list.
% Finally, I'll read the index in from its formatted |.ind| file.
%
%    \begin{macrocode}
\tdef\attheend{%
  \finalstuff%
  \ifhave@multicol%
    \def\do##1##2{\addtocontents{##1}{\protect\end{multicols}}}%
    \docontents%
  \fi%
  \PrintIndex%
}
%    \end{macrocode}
%
% \end{macro}
%
%
% \subsection{File version information}
%
% \begin{macro}{\mdwpkginfo}
%
% For setting up the automatic titles, I'll need to be able to work out
% file versions and things.  This macro will, given a file name, extract
% from \LaTeX\ the version information and format it into a sensible string.
%
% First of all, I'll put the original string (direct from the
% |\Provides|\dots\ command).  Then I'll pass it to another macro which can
% parse up the string into its various bits, along with the original
% filename.
%
%    \begin{macrocode}
\def\mdwpkginfo#1{%
  \edef\@tempa{\csname ver@#1\endcsname}%
  \expandafter\mdwpkginfo@i\@tempa\@@#1\@@%
}
%    \end{macrocode}
%
% Now for the real business.  I'll store the string I build in macros called
% \syntax{"\\"<filename>"date", "\\"<filename>"version" and
% "\\"<filename>"info"}, which store the file's date, version and
% `information string' respectively.  (Note that the file extension isn't
% included in the name.)
%
% This is mainly just tedious playing with |\expandafter|.  The date format
% is defined by a separate macro, which can be modified from the
% configuration file.
%
%    \begin{macrocode}
\def\mdwpkginfo@i#1/#2/#3 #4 #5\@@#6.#7\@@{%
  \expandafter\def\csname #6date\endcsname%
    {\protect\mdwdateformat{#1}{#2}{#3}}%
  \expandafter\def\csname #6version\endcsname{#4}%
  \expandafter\def\csname #6info\endcsname{#5}%
}
%    \end{macrocode}
%
% \end{macro}
%
% \begin{macro}{\mdwdateformat}
%
% Given three arguments, a year, a month and a date (all numeric), build a
% pretty date string.  This is fairly simple really.
%
%    \begin{macrocode}
\tdef\mdwdateformat#1#2#3{\number#3\ \monthname{#2}\ \number#1}
\def\monthname#1{%
  \ifcase#1\or%
     January\or February\or March\or April\or May\or June\or%
     July\or August\or September\or October\or November\or December%
  \fi%
}
\def\numsuffix#1{%
  \ifnum#1=1 st\else%
  \ifnum#1=2 nd\else%
  \ifnum#1=3 rd\else%
  \ifnum#1=21 st\else%
  \ifnum#1=22 nd\else%
  \ifnum#1=23 rd\else%
  \ifnum#1=31 st\else%
  th%
  \fi\fi\fi\fi\fi\fi\fi%
}
%    \end{macrocode}
%
% \end{macro}
%
% \begin{macro}{\mdwfileinfo}
%
% Saying \syntax{"\\mdwfileinfo{"<file-name>"}{"<info>"}"} extracts the
% wanted item of \<info> from the version information for file \<file-name>.
%
%    \begin{macrocode}
\def\mdwfileinfo#1#2{\mdwfileinfo@i{#2}#1.\@@}
\def\mdwfileinfo@i#1#2.#3\@@{\csname#2#1\endcsname}
%    \end{macrocode}
%
% \end{macro}
%
%
% \subsection{List handling}
%
% There are several other lists I need to build.  These macros will do
% the necessary stuff.
%
% \begin{macro}{\mdw@ifitem}
%
% The macro \syntax{"\\mdw@ifitem"<item>"\\in"<list>"{"<true-text>"}"^^A
% "{"<false-text>"}"} does \<true-text> if the \<item> matches any item in
% the \<list>; otherwise it does \<false-text>.
%
%    \begin{macrocode}
\def\mdw@ifitem#1\in#2{%
  \@tempswafalse%
  \def\@tempa{#1}%
  \def\do##1{\def\@tempb{##1}\ifx\@tempa\@tempb\@tempswatrue\fi}%
  #2%
  \if@tempswa\expandafter\@firstoftwo\else\expandafter\@secondoftwo\fi%
}
%    \end{macrocode}
%
% \end{macro}
%
% \begin{macro}{\mdw@append}
%
% Saying \syntax{"\\mdw@append"<item>"\\to"<list>} adds the given \<item>
% to the end of the given \<list>.
%
%    \begin{macrocode}
\def\mdw@append#1\to#2{%
  \toks@{\do{#1}}%
  \toks\tw@\expandafter{#2}%
  \edef#2{\the\toks\tw@\the\toks@}%
}
%    \end{macrocode}
%
% \end{macro}
%
% \begin{macro}{\mdw@prepend}
%
% Saying \syntax{"\\mdw@prepend"<item>"\\to"<list>} adds the \<item> to the
% beginning of the \<list>.
%
%    \begin{macrocode}
\def\mdw@prepend#1\to#2{%
  \toks@{\do{#1}}%
  \toks\tw@\expandafter{#2}%
  \edef#2{\the\toks@\the\toks\tw@}%
}
%    \end{macrocode}
%
% \end{macro}
%
% \begin{macro}{\mdw@add}
%
% Finally, saying \syntax{"\\mdw@add"<item>"\\to"<list>} adds the \<item>
% to the list only if it isn't there already.
%
%    \begin{macrocode}
\def\mdw@add#1\to#2{\mdw@ifitem#1\in#2{}{\mdw@append#1\to#2}}
%    \end{macrocode}
%
% \end{macro}
%
%
% \subsection{Described file handling}
%
% I'l maintain lists of packages, document classes, and other files
% described by the current documentation file.
%
% First of all, I'll declare the various list macros.
%
%    \begin{macrocode}
\def\dopackages{}
\def\doclasses{}
\def\dootherfiles{}
%    \end{macrocode}
%
% \begin{macro}{\describespackage}
%
% A document file can declare that it describes a package by saying
% \syntax{"\\describespackage{"<package-name>"}"}.  I add the package to
% my list, read the package into memory (so that the documentation can
% offer demonstrations of it) and read the version information.
%
%    \begin{macrocode}
\def\describespackage#1{%
  \mdw@ifitem#1\in\dopackages{}{%
    \mdw@append#1\to\dopackages%
    \usepackage{#1}%
    \mdwpkginfo{#1.sty}%
  }%
}
%    \end{macrocode}
%
% \end{macro}
%
% \begin{macro}{\describesclass}
%
% By saying \syntax{"\\describesclass{"<class-name>"}"}, a document file
% can declare that it describes a document class.  I'll assume that the
% document class is already loaded, because it's much too late to load
% it now.
%
%    \begin{macrocode}
\def\describesclass#1{\mdw@add#1\to\doclasses\mdwpkginfo{#1.cls}}
%    \end{macrocode}
%
% \end{macro}
%
% \begin{macro}{\describesfile}
%
% Finally, other `random' files, which don't have the status of real \LaTeX\
% packages or document classes, can be described by saying \syntax{^^A
% "\\describesfile{"<file-name>"}" or "\\describesfile*{"<file-name>"}"}.
% The difference is that the starred version will not |\input| the file.
%
%    \begin{macrocode}
\def\describesfile{%
  \@ifstar{\describesfile@i\@gobble}{\describesfile@i\input}%
}
\def\describesfile@i#1#2{%
  \mdw@ifitem#2\in\dootherfiles{}{%
    \mdw@add#2\to\dootherfiles%
    #1{#2}%
    \mdwpkginfo{#2}%
  }%
}
%    \end{macrocode}
%
% \end{macro}
%
%
% \subsection{Author and title handling}
%
% I'll redefine the |\author| and |\title| commands so that I get told
% whether I need to do it myself.
%
% \begin{macro}{\author}
%
% This is easy: I'll save the old meaning, and then redefine |\author| to
% do the old thing and redefine itself to then do nothing.
%
%    \begin{macrocode}
\let\mdw@author\author
\def\author{\let\author\@gobble\mdw@author}
%    \end{macrocode}
%
% \end{macro}
%
% \begin{macro}{\title}
%
% And oddly enough, I'll do exactly the same thing for the title, except
% that I'll also disable the |\mdw@buildtitle| command, which constructs
% the title automatically.
%
%    \begin{macrocode}
\let\mdw@title\title
\def\title{\let\title\@gobble\let\mdw@buildtitle\relax\mdw@title}
%    \end{macrocode}
%
% \end{macro}
%
% \begin{macro}{\date}
%
% This works in a very similar sort of way.
%
%    \begin{macrocode}
\def\date#1{\let\date\@gobble\def\today{#1}}
%    \end{macrocode}
%
% \end{macro}
%
% \begin{macro}{\datefrom}
%
% Saying \syntax{"\\datefrom{"<file-name>"}"} sets the document date from
% the given filename.
%
%    \begin{macrocode}
\def\datefrom#1{%
  \protected@edef\@tempa{\noexpand\date{\csname #1date\endcsname}}%
  \@tempa%
}
%    \end{macrocode}
%
% \end{macro}
%
% \begin{macro}{\docfile}
%
% Saying \syntax{"\\docfile{"<file-name>"}"} sets up the file name from which
% documentation will be read.
%
%    \begin{macrocode}
\def\docfile#1{%
  \def\@tempa##1.##2\@@{\def\@basefile{##1.##2}\def\@basename{##1}}%
  \edef\@tempb{\noexpand\@tempa#1\noexpand\@@}%
  \@tempb%
}
%    \end{macrocode}
%
% I'll set up a default value as well.
%
%    \begin{macrocode}
\docfile{\jobname.dtx}
%    \end{macrocode}
%
% \end{macro}
%
%
% \subsection{Building title strings}
%
% This is rather tricky.  For each list, I need to build a legible looking
% string.
%
% \begin{macro}{\mdw@addtotitle}
%
% By saying
%\syntax{"\\mdw@addtotitle{"<list>"}{"<command>"}{"<singular>"}{"<plural>"}"}
% I can add the contents of a list to the current title string in the
% |\mdw@title| macro.
%
%    \begin{macrocode}
\tdef\mdw@addtotitle#1#2#3#4{%
%    \end{macrocode}
%
% Now to get to work.  I need to keep one `lookahead' list item, and a count
% of the number of items read so far.  I'll keep the lookahead item in
% |\@nextitem| and the counter in |\count@|.
%
%    \begin{macrocode}
  \count@\z@%
%    \end{macrocode}
%
% Now I'll define what to do for each list item.  The |\protect| command is
% already set up appropriately for playing with |\edef| commands.
%
%    \begin{macrocode}
  \def\do##1{%
%    \end{macrocode}
%
% The first job is to add the previous item to the title string.  If this
% is the first item, though, I'll just add the appropriate \lit{The } or
% \lit{ and the } string to the title (this is stored in the |\@prefix|
% macro).
%
%    \begin{macrocode}
    \edef\mdw@title{%
      \mdw@title%
      \ifcase\count@\@prefix%
      \or\@nextitem%
      \else, \@nextitem%
      \fi%
    }%
%    \end{macrocode}
%
% That was rather easy.  Now I'll set up the |\@nextitem| macro for the
% next time around the loop.
%
%    \begin{macrocode}
    \edef\@nextitem{%
      \protect#2{##1}%
      \protect\footnote{%
        The \protect#2{##1} #3 is currently at version %
        \mdwfileinfo{##1}{version}, dated \mdwfileinfo{##1}{date}.%
      }\space%
    }%
%    \end{macrocode}
%
% Finally, I need to increment the counter.
%
%    \begin{macrocode}
    \advance\count@\@ne%
  }%
%    \end{macrocode}
%
% Now execute the list.
%
%    \begin{macrocode}
  #1%
%    \end{macrocode}
%
% I still have one item left over, unless the list was empty.  I'll add
% that now.
%
%    \begin{macrocode}
  \edef\mdw@title{%
    \mdw@title%
    \ifcase\count@%
    \or\@nextitem\space#3%
    \or\ and \@nextitem\space#4%
    \fi%
  }%
%    \end{macrocode}
%
% Finally, if $|\count@| \ne 0$, I must set |\@prefix| to \lit{ and the }.
%
%    \begin{macrocode}
  \ifnum\count@>\z@\def\@prefix{ and the }\fi%
}
%    \end{macrocode}
%
% \end{macro}
%
% \begin{macro}{\mdw@buildtitle}
%
% This macro will actually do the job of building the title string.
%
%    \begin{macrocode}
\tdef\mdw@buildtitle{%
%    \end{macrocode}
%
% First of all, I'll open a group to avoid polluting the namespace with
% my gubbins (although the code is now much tidier than it has been in
% earlier releases).
%
%    \begin{macrocode}
  \begingroup%
%    \end{macrocode}
%
% The title building stuff makes extensive use of |\edef|.  I'll set
% |\protect| appropriately.  (For those not in the know,
% |\@unexpandable@protect| expands to `|\noexpand\protect\noexpand|',
% which prevents expansion of the following macro, and inserts a |\protect|
% in front of it ready for the next |\edef|.)
%
%    \begin{macrocode}
  \let\@@protect\protect\let\protect\@unexpandable@protect%
%    \end{macrocode}
%
% Set up some simple macros ready for the main code.
%
%    \begin{macrocode}
  \def\mdw@title{}%
  \def\@prefix{The }%
%    \end{macrocode}
%
% Now build the title.  This is fun.
%
%    \begin{macrocode}
  \mdw@addtotitle\dopackages\package{package}{packages}%
  \mdw@addtotitle\doclasses\package{document class}{document classes}%
  \mdw@addtotitle\dootherfiles\texttt{file}{files}%
%    \end{macrocode}
%
% Now I want to end the group and set the title from my string.  The
% following hacking will do this.
%
%    \begin{macrocode}
  \edef\next{\endgroup\noexpand\title{\mdw@title}}%
  \next%
}
%    \end{macrocode}
%
% \end{macro}
%
%
% \subsection{Starting the main document}
%
% \begin{macro}{\mdwdoc}
%
% Once the document preamble has done all of its stuff, it calls the
% |\mdwdoc| command, which takes over and really starts the documentation
% going.
%
%    \begin{macrocode}
\def\mdwdoc{%
%    \end{macrocode}
%
% First, I'll construct the title string.
%
%    \begin{macrocode}
  \mdw@buildtitle%
  \author{Mark Wooding}%
%    \end{macrocode}
%
% Set up the date string based on the date of the package which shares
% the same name as the current file.
%
%    \begin{macrocode}
  \datefrom\@basename%
%    \end{macrocode}
%
% Set up verbatim characters after all the packages have started.
%
%    \begin{macrocode}
  \shortverb\|%
  \shortverb\"%
%    \end{macrocode}
%
% Start the document, and put the title in.
%
%    \begin{macrocode}
  \begin{document}
  \maketitle%
%    \end{macrocode}
%
% This is nasty.  It makes maths displays work properly in demo environments.
% \emph{The \LaTeX\ Companion} exhibits the bug which this hack fixes.  So
% ner.
%
%    \begin{macrocode}
  \abovedisplayskip\z@%
%    \end{macrocode}
%
% Now start the contents tables.  After starting each one, I'll make it
% be multicolumnar.
%
%    \begin{macrocode}
  \def\do##1##2{%
    ##2%
    \ifhave@multicol\addtocontents{##1}{%
      \protect\begin{multicols}{2}%
      \hbadness\@M%
    }\fi%
  }%
  \docontents%
%    \end{macrocode}
%
% Input the main file now.
%
%    \begin{macrocode}
  \DocInput{\@basefile}%
%    \end{macrocode}
%
% That's it.  I'm done.
%
%    \begin{macrocode}
  \end{document}
}
%    \end{macrocode}
%
% \end{macro}
%
%
% \subsection{And finally\dots}
%
% Right at the end I'll put a hook for the configuration file.
%
%    \begin{macrocode}
\ifx\mdwhook\@@undefined\else\expandafter\mdwhook\fi
%    \end{macrocode}
%
% That's all the code done now.  I'll change back to `user' mode, where
% all the magic control sequences aren't allowed any more.
%
%    \begin{macrocode}
\makeatother
%</mdwtools>
%    \end{macrocode}
%
% Oh, wait!  What if I want to typeset this documentation?  Aha.  I'll cope
% with that by comparing |\jobname| with my filename |mdwtools|.  However,
% there's some fun here, because |\jobname| contains category-12 letters,
% while my letters are category-11.  Time to play with |\string| in a messy
% way.
%
%    \begin{macrocode}
%<*driver>
\makeatletter
\edef\@tempa{\expandafter\@gobble\string\mdwtools}
\edef\@tempb{\jobname}
\ifx\@tempa\@tempb
  \describesfile*{mdwtools.tex}
  \docfile{mdwtools.tex}
  \makeatother
  \expandafter\mdwdoc
\fi
\makeatother
%</driver>
%    \end{macrocode}
%
% That's it.  Done!
%
% \hfill Mark Wooding, \today
%
% \Finale
%
\endinput
|
% \<declarations>
% |\mdwdoc|
% \end{listinglist}
% The initial |\input| reads in this file and sets up the various commands
% which may be needed.  The final |\mdwdoc| actually starts the document,
% inserting a title (which is automatically generated), a table of
% contents etc., and reads the documentation file in (using the |\DocInput|
% command from the \package{doc} package.
%
% \subsubsection{Describing packages}
%
% \DescribeMacro{\describespackage}
% \DescribeMacro{\describesclass}
% \DescribeMacro{\describesfile}
% \DescribeMacro{\describesfile*}
% The most important declarations are those which declare what the
% documentation describes.  Saying \syntax{"\\describespackage{<package>}"}
% loads the \<package> (if necessary) and adds it to the auto-generated
% title, along with a footnote containing version information.  Similarly,
% |\describesclass| adds a document class name to the title (without loading
% it -- the document itself must do this, with the |\documentclass| command).
% For files which aren't packages or classes, use the |\describesfile| or
% |\describesfile*| command (the $*$-version won't |\input| the file, which
% is handy for files like |mdwtools.tex|, which are already input).
%
% \DescribeMacro{\author}
% \DescribeMacro{\date}
% \DescribeMacro{\title}
% The |\author|, |\date| and |\title| declarations work slightly differently
% to normal -- they ensure that only the \emph{first} declaration has an
% effect.  (Don't you play with |\author|, please, unless you're using this
% program to document your own packages.)  Using |\title| suppresses the
% automatic title generation.
%
% \DescribeMacro{\docdate}
% The default date is worked out from the version string of the package or
% document class whose name is the same as that of the documentation file.
% You can choose a different `main' file by saying
% \syntax{"\\docdate{"<file>"}"}.
%
% \subsubsection{Contents handling}
%
% \DescribeMacro{\addcontents}
% A documentation file always has a table of contents.  Other
% contents-like lists can be added by saying
% \syntax{"\\addcontents{"<extension>"}{"<command>"}"}.  The \<extension>
% is the file extension of the contents file (e.g., \lit{lot} for the
% list of tables); the \<command> is the command to actually typeset the
% contents file (e.g., |\listoftables|).
%
% \subsubsection{Other declarations}
%
% \DescribeMacro{\implementation}
% The \package{doc} package wants you to say
% \syntax{"\\StopEventually{"<stuff>"}"}' before describing the package
% implementation.  Using |mdwtools.tex|, you just say |\implementation|, and
% everything works.  It will automatically read in the licence text (from
% |gpl.tex|, and wraps some other things up.
%
% 
% \subsection{Other commands}
%
% The |mdwtools.tex| file includes the \package{syntax} and \package{sverb}
% packages so that they can be used in documentation files.  It also defines
% some trivial commands of its own.
%
% \DescribeMacro{\<}
% Saying \syntax{"\\<"<text>">" is the same as "\\synt{"<text>"}"}; this
% is a simple abbreviation.
%
% \DescribeMacro{\smallf}
% Saying \syntax{"\\smallf" <number>"/"<number>} typesets a little fraction,
% like this: \smallf 3/4.  It's useful when you want to say that the default
% value of a length is 2 \smallf 1/2\,pt, or something like that.
%
%
% \subsection{Customisation}
%
% You can customise the way that the package documentation looks by writing
% a file called |mdwtools.cfg|.  You can redefine various commands (before
% they're defined here, even; |mdwtools.tex| checks most of the commands that
% it defines to make sure they haven't been defined already.
%
% \DescribeMacro{\indexing}
% If you don't want the prompt about whether to generate index files, you
% can define the |\indexing| command to either \lit{y} or \lit{n}.  I'd
% recommend that you use |\providecommand| for this, to allow further
% customisation from the command line.
%
% \DescribeMacro{\mdwdateformat}
% If you don't like my date format (maybe you're American or something),
% you can redefine the |\mdwdateformat| command.  It takes three arguments:
% the year, month and date, as numbers; it should expand to something which
% typesets the date nicely.  The default format gives something like
% `10 May 1996'.  You can produce something rather more exotic, like
% `10\textsuperscript{th} May \textsc{\romannumeral 1996}' by saying
%\begin{listing}
%\newcommand{\mdwdateformat}[3]{%
%  \number#3\textsuperscript{\numsuffix{#3}}\ %
%  \monthname{#2}\ %
%  \textsc{\romannumeral #1}%
%}
%\end{listing}
% \DescribeMacro{\monthname}
% \DescribeMacro{\numsuffix}
% Saying \syntax{"\\monthname{"<number>"}"} expands to the name of the
% numbered month (which can be useful when doing date formats).  Saying
% \syntax{"\\numsuffix{"<number>"}"} will expand to the appropriate suffix
% (`th' or `rd' or whatever) for the \<number>.  You'll have to superscript
% it yourself, if this is what you want to do.  Putting the year number
% in roman numerals is just pretentious |;-)|.
%
% \DescribeMacro{\mdwhook}
% After all the declarations in |mdwtools.tex|, the command |\mdwhook| is
% executed, if it exists.  This can be set up by the configuration file
% to do whatever you want.
%
% There are lots of other things you can play with; you should look at the
% implementation section to see what's possible.
%
% \implementation
%
% \section{Implementation}
%
%    \begin{macrocode}
%<*mdwtools>
%    \end{macrocode}
%
% The first thing is that I'm not a \LaTeX\ package or anything official
% like that, so I must enable `|@|' as a letter by hand.
%
%    \begin{macrocode}
\makeatletter
%    \end{macrocode}
%
% Now input the user's configuration file, if it exists.  This is fairly
% simple stuff.
%
%    \begin{macrocode}
\@input{mdwtools.cfg}
%    \end{macrocode}
%
% Well, that's the easy bit done.
%
%
% \subsection{Initialisation}
%
% Obviously the first thing to do is to obtain a document class.  Obviously,
% it would be silly to do this if a document class has already been loaded,
% either by the package documentation or by the configuration file.
%
% The only way I can think of for finding out if a document class is already
% loaded is by seeing if the |\documentclass| command has been redefined
% to raise an error.  This isn't too hard, really.
%
%    \begin{macrocode}
\ifx\documentclass\@twoclasseserror\else
  \documentclass[a4paper]{ltxdoc}
  \ifx\doneclasses\mdw@undefined\else\doneclasses\fi
\fi
%    \end{macrocode}
%
% As part of my standard environment, I'll load some of my more useful
% packages.  If they're already loaded (possibly with different options),
% I'll not try to load them again.
%
%    \begin{macrocode}
\@ifpackageloaded{doc}{}{\usepackage{doc}}
\@ifpackageloaded{syntax}{}{\usepackage[rounded]{syntax}}
\@ifpackageloaded{sverb}{}{\usepackage{sverb}}
%    \end{macrocode}
%
%
% \subsection{Some macros for interaction}
%
% I like the \LaTeX\ star-boxes, although it's a pain having to cope with
% \TeX's space-handling rules.  I'll define a new typing-out macro which
% makes spaces more significant, and has a $*$-version which doesn't put
% a newline on the end, and interacts prettily with |\read|.
%
% First of all, I need to make spaces active, so I can define things about
% active spaces.
%
%    \begin{macrocode}
\begingroup\obeyspaces
%    \end{macrocode}
%
% Now to define the main macro.  This is easy stuff.  Spaces must be
% carefully rationed here, though.
%
% I'll start a group, make spaces active, and make spaces expand to ordinary
% space-like spaces.  Then I'll look for a star, and pass either |\message|
% (which doesn't start a newline, and interacts with |\read| well) or
% |\immediate\write 16| which does a normal write well.
%
%    \begin{macrocode}
\gdef\mdwtype{%
\begingroup\catcode`\ \active\let \space%
\@ifstar{\mdwtype@i{\message}}{\mdwtype@i{\immediate\write\sixt@@n}}%
}
\endgroup
%    \end{macrocode}
%
% Now for the easy bit.  I have the thing to do, and the thing to do it to,
% so do that and end the group.
%
%    \begin{macrocode}
\def\mdwtype@i#1#2{#1{#2}\endgroup}
%    \end{macrocode}
%
%
% \subsection{Decide on indexing}
%
% A configuration file can decide on indexing by defining the |\indexing|
% macro to either \lit{y} or \lit{n}.  If it's not set, then I'll prompt
% the user.
%
% First of all, I want a switch to say whether I'm indexing.
%
%    \begin{macrocode}
\newif\ifcreateindex
%    \end{macrocode}
%
% Right: now I need to decide how to make progress.  If the macro's not set,
% then I want to set it, and start a row of stars.
%
%    \begin{macrocode}
\ifx\indexing\@@undefined
  \mdwtype{*****************************}
  \def\indexing{?}
\fi
%    \end{macrocode}
%
% Now enter a loop, asking the user whether to do indexing, until I get
% a sensible answer.
%
%    \begin{macrocode}
\loop
  \@tempswafalse
  \if y\indexing\@tempswatrue\createindextrue\fi
  \if Y\indexing\@tempswatrue\createindextrue\fi
  \if n\indexing\@tempswatrue\createindexfalse\fi
  \if N\indexing\@tempswatrue\createindexfalse\fi
  \if@tempswa\else
  \mdwtype*{* Create index files? (y/n) *}
  \read\sixt@@n to\indexing%
\repeat
%    \end{macrocode}
%
% Now, based on the results of that, display a message about the indexing.
%
%    \begin{macrocode}
\mdwtype{*****************************}
\ifcreateindex
  \mdwtype{* Creating index files      *}
  \mdwtype{* This may take some time   *}
\else
  \mdwtype{* Not creating index files  *}
\fi
\mdwtype{*****************************}
%    \end{macrocode}
%
% Now I can play with the indexing commands of the \package{doc} package
% to do whatever it is that the user wants.
%
%    \begin{macrocode}
\ifcreateindex
  \CodelineIndex
  \EnableCrossrefs
\else
  \CodelineNumbered
  \DisableCrossrefs
\fi
%    \end{macrocode}
%
% And register lots of plain \TeX\ things which shouldn't be indexed.
% This contains lots of |\if|\dots\ things which don't fit nicely in
% conditionals, which is a shame.  Still, it doesn't matter that much,
% really.
%
%    \begin{macrocode}
\DoNotIndex{\def,\long,\edef,\xdef,\gdef,\let,\global}
\DoNotIndex{\if,\ifnum,\ifdim,\ifcat,\ifmmode,\ifvmode,\ifhmode,%
            \iftrue,\iffalse,\ifvoid,\ifx,\ifeof,\ifcase,\else,\or,\fi}
\DoNotIndex{\box,\copy,\setbox,\unvbox,\unhbox,\hbox,%
            \vbox,\vtop,\vcenter}
\DoNotIndex{\@empty,\immediate,\write}
\DoNotIndex{\egroup,\bgroup,\expandafter,\begingroup,\endgroup}
\DoNotIndex{\divide,\advance,\multiply,\count,\dimen}
\DoNotIndex{\relax,\space,\string}
\DoNotIndex{\csname,\endcsname,\@spaces,\openin,\openout,%
            \closein,\closeout}
\DoNotIndex{\catcode,\endinput}
\DoNotIndex{\jobname,\message,\read,\the,\m@ne,\noexpand}
\DoNotIndex{\hsize,\vsize,\hskip,\vskip,\kern,\hfil,\hfill,\hss}
\DoNotIndex{\m@ne,\z@,\z@skip,\@ne,\tw@,\p@}
\DoNotIndex{\dp,\wd,\ht,\vss,\unskip}
%    \end{macrocode}
%
% Last bit of indexing stuff, for now: I'll typeset the index in two columns
% (the default is three, which makes them too narrow for my tastes).
%
%    \begin{macrocode}
\setcounter{IndexColumns}{2}
%    \end{macrocode}
%
%
% \subsection{Selectively defining things}
%
% I don't want to tread on anyone's toes if they redefine any of these
% commands and things in a configuration file.  The following definitions
% are fairly evil, but should do the job OK.
%
% \begin{macro}{\@gobbledef}
%
% This macro eats the following |\def|inition, leaving not a trace behind.
%
%    \begin{macrocode}
\def\@gobbledef#1#{\@gobble}
%    \end{macrocode}
%
% \end{macro}
%
% \begin{macro}{\tdef}
% \begin{macro}{\tlet}
%
% The |\tdef| command is a sort of `tentative' definition -- it's like
% |\def| if the control sequence named doesn't already have a definition.
% |\tlet| does the same thing with |\let|.
%
%    \begin{macrocode}
\def\tdef#1{
  \ifx#1\@@undefined%
    \expandafter\def\expandafter#1%
  \else%
    \expandafter\@gobbledef%
  \fi%
}
\def\tlet#1#2{\ifx#1\@@undefined\let#1=#2\fi}
%    \end{macrocode}
%
% \end{macro}
% \end{macro}
%
%
% \subsection{General markup things}
%
% Now for some really simple things.  I'll define how to typeset package
% names and environment names (both in the sans serif font, for now).
%
%    \begin{macrocode}
\tlet\package\textsf
\tlet\env\textsf
%    \end{macrocode}
%
% I'll define the |\<|\dots|>| shortcut for syntax items suggested in the
% \package{syntax} package.
%
%    \begin{macrocode}
\tdef\<#1>{\synt{#1}}
%    \end{macrocode}
%
% And because it's used in a few places (mainly for typesetting lengths),
% here's a command for typesetting fractions in text.
%
%    \begin{macrocode}
\tdef\smallf#1/#2{\ensuremath{^{#1}\!/\!_{#2}}}
%    \end{macrocode}
%
%
% \subsection{A table environment}
%
% \begin{environment}{tab}
%
% Most of the packages don't use the (obviously perfect) \package{mdwtab}
% package, because it's big, and takes a while to load.  Here's an
% environment for typesetting centred tables.  The first (optional) argument
% is some declarations to perform.  The mandatory argument is the table
% preamble (obviously).
%
%    \begin{macrocode}
\@ifundefined{tab}{%
  \newenvironment{tab}[2][\relax]{%
    \par\vskip2ex%
    \centering%
    #1%
    \begin{tabular}{#2}%
  }{%
    \end{tabular}%
    \par\vskip2ex%
  }
}{}
%    \end{macrocode}
%
% \end{environment}
%
%
% \subsection{Commenting out of stuff}
%
% \begin{environment}{meta-comment}
%
% Using |\iffalse|\dots|\fi| isn't much fun.  I'll define a gobbling
% environment using the \package{sverb} stuff.
%
%    \begin{macrocode}
\ignoreenv{meta-comment}
%    \end{macrocode}
%
% \end{environment}
%
%
% \subsection{Float handling}
%
% This gubbins will try to avoid float pages as much as possible, and (with
% any luck) encourage floats to be put on the same pages as text.
%
%    \begin{macrocode}
\def\textfraction{0.1}
\def\topfraction{0.9}
\def\bottomfraction{0.9}
\def\floatpagefraction{0.7}
%    \end{macrocode}
%
% Now redefine the default float-placement parameters to allow `here' floats.
%
%    \begin{macrocode}
\def\fps@figure{htbp}
\def\fps@table{htbp}
%    \end{macrocode}
%
%
% \subsection{Other bits of parameter tweaking}
%
% Make \env{grammar} environments look pretty, by indenting the left hand
% sides by a large amount.
%
%    \begin{macrocode}
\grammarindent1in
%    \end{macrocode}
%
% I don't like being told by \TeX\ that my paragraphs are hard to linebreak:
% I know this already.  This lot should shut \TeX\ up about most problems.
%
%    \begin{macrocode}
\sloppy
\hbadness\@M
\hfuzz10\p@
%    \end{macrocode}
%
% Also make \TeX\ shut up in the index.  The \package{multicol} package
% irritatingly plays with |\hbadness|.  This is the best hook I could find
% for playing with this setting.
%
%    \begin{macrocode}
\expandafter\def\expandafter\IndexParms\expandafter{%
  \IndexParms%
  \hbadness\@M%
}
%    \end{macrocode}
%
% The other thing I really don't like is `Marginpar moved' warnings.  This
% will get rid of them, and lots of other \LaTeX\ warnings at the same time.
%
%    \begin{macrocode}
\let\@latex@warning@no@line\@gobble
%    \end{macrocode}
%
% Put some extra space between table rows, please.
%
%    \begin{macrocode}
\def\arraystretch{1.2}
%    \end{macrocode}
%
% Most of the code is at guard level one, so typeset that in upright text.
%
%    \begin{macrocode}
\setcounter{StandardModuleDepth}{1}
%    \end{macrocode}
%
%
% \subsection{Contents handling}
%
% I use at least one contents file (the main table of contents) although
% I may want more.  I'll keep a list of contents files which I need to
% handle.
%
% There are two things I need to do to contents files here:
% \begin{itemize}
% \item I must typeset the table of contents at the beginning of the
%       document; and
% \item I want to typeset tables of contents in two columns (using the
%       \package{multicol} package).
% \end{itemize}
%
% The list consists of items of the form
% \syntax{"\\do{"<extension>"}{"<command>"}"}, where \<extension> is the
% file extension of the contents file, and \<command> is the command to
% typeset it.
%
% \begin{macro}{\docontents}
%
% This is where I keep the list of contents files.  I'll initialise it to
% just do the standard contents table.
%
%    \begin{macrocode}
\def\docontents{\do{toc}{\tableofcontents}}
%    \end{macrocode}
%
% \end{macro}
%
% \begin{macro}{\addcontents}
%
% By saying \syntax{"\\addcontents{"<extension>"}{"<command>"}"}, a document
% can register a new table of contents which gets given the two-column
% treatment properly.  This is really easy to implement.
%
%    \begin{macrocode}
\def\addcontents#1#2{%
  \toks@\expandafter{\docontents\do{#1}{#2}}%
  \edef\docontents{\the\toks@}%
}
%    \end{macrocode}
%
% \end{macro}
%
%
% \subsection{Finishing it all off}
%
% \begin{macro}{\finalstuff}
%
% The |\finalstuff| macro is a hook for doing things at the end of the
% document.  Currently, it inputs the licence agreement as an appendix.
%
%    \begin{macrocode}
\tdef\finalstuff{\appendix\part*{Appendix}% \iffalse <meta-comment>
%
% $Id: gpl.tex,v 1.1 1996/11/19 20:51:14 mdw Exp $
%
% The GNU General Public Licence as a LaTeX section
%
% (c) 1989, 1991 Free Software Foundation, Inc.
%   LaTeX markup and minor formatting changes by Mark Wooding
%

%----- Revision history -----------------------------------------------------
%
% $Log: gpl.tex,v $
% Revision 1.1  1996/11/19 20:51:14  mdw
% Initial revision
%

% --- Chapter heading ---
%
% We don't know whether this ought to be a section or a chapter.  Easy.
% We'll see if chapters are possible.
%
% \fi

\begingroup
\makeatletter

\edef\next#1#2#3{\relax
  \ifx\chapter\@@undefined
    \ifx\documentclass\@notprerr#2\else#3\fi
  \else#1\fi
}

\expandafter\endgroup\next
{
  \let\gpltoplevel\chapter
  \let\gplsec\section
  \let\gplend\endinput
}{
  \let\gpltoplevel\section
  \let\gplsec\subsection
  \let\gplend\endinput
}{
  \documentclass[a4paper]{article}
  \def\gpltoplevel#1{%
    \vspace*{1in}%
    \hbox to\hsize{\hfil\LARGE\bfseries#1\hfil}%
    \vspace{1in}%
  }
  \let\gplsec\section
  \def\gplend{\end{document}}
  \advance\textwidth1in
  \advance\oddsidemargin-.5in
  \sloppy
  \begin{document}
}

%^^A-------------------------------------------------------------------------
\gpltoplevel{The GNU General Public Licence}


The following is the text of the GNU General Public Licence, under the terms
of which this software is distrubuted.

\vspace{12pt}

\begin{center}
\textbf{GNU GENERAL PUBLIC LICENSE} \\
Version 2, June 1991
\end{center}

\begin{center}
Copyright (C) 1989, 1991 Free Software Foundation, Inc. \\
675 Mass Ave, Cambridge, MA 02139, USA

Everyone is permitted to copy and distribute verbatim copies \\
of this license document, but changing it is not allowed.
\end{center}


\gplsec{Preamble}

The licenses for most software are designed to take away your freedom to
share and change it.  By contrast, the GNU General Public License is intended
to guarantee your freedom to share and change free software---to make sure
the software is free for all its users.  This General Public License applies
to most of the Free Software Foundation's software and to any other program
whose authors commit to using it.  (Some other Free Software Foundation
software is covered by the GNU Library General Public License instead.)  You
can apply it to your programs, too.

When we speak of free software, we are referring to freedom, not price.  Our
General Public Licenses are designed to make sure that you have the freedom
to distribute copies of free software (and charge for this service if you
wish), that you receive source code or can get it if you want it, that you
can change the software or use pieces of it in new free programs; and that
you know you can do these things.

To protect your rights, we need to make restrictions that forbid anyone to
deny you these rights or to ask you to surrender the rights.  These
restrictions translate to certain responsibilities for you if you distribute
copies of the software, or if you modify it.

For example, if you distribute copies of such a program, whether gratis or
for a fee, you must give the recipients all the rights that you have.  You
must make sure that they, too, receive or can get the source code.  And you
must show them these terms so they know their rights.

We protect your rights with two steps: (1) copyright the software, and (2)
offer you this license which gives you legal permission to copy, distribute
and/or modify the software.

Also, for each author's protection and ours, we want to make certain that
everyone understands that there is no warranty for this free software.  If
the software is modified by someone else and passed on, we want its
recipients to know that what they have is not the original, so that any
problems introduced by others will not reflect on the original authors'
reputations.

Finally, any free program is threatened constantly by software patents.  We
wish to avoid the danger that redistributors of a free program will
individually obtain patent licenses, in effect making the program
proprietary.  To prevent this, we have made it clear that any patent must be
licensed for everyone's free use or not licensed at all.

The precise terms and conditions for copying, distribution and modification
follow.


\gplsec{Terms and conditions for copying, distribution and modification}

\begin{enumerate}

\makeatletter \setcounter{\@listctr}{-1} \makeatother

\item [0.] This License applies to any program or other work which contains a
      notice placed by the copyright holder saying it may be distributed
      under the terms of this General Public License.  The ``Program'',
      below, refers to any such program or work, and a ``work based on the
      Program'' means either the Program or any derivative work under
      copyright law: that is to say, a work containing the Program or a
      portion of it, either verbatim or with modifications and/or translated
      into another language.  (Hereinafter, translation is included without
      limitation in the term ``modification''.)  Each licensee is addressed
      as ``you''.

      Activities other than copying, distribution and modification are not
      covered by this License; they are outside its scope.  The act of
      running the Program is not restricted, and the output from the Program
      is covered only if its contents constitute a work based on the Program
      (independent of having been made by running the Program).  Whether that
      is true depends on what the Program does.

\item [1.] You may copy and distribute verbatim copies of the Program's
      source code as you receive it, in any medium, provided that you
      conspicuously and appropriately publish on each copy an appropriate
      copyright notice and disclaimer of warranty; keep intact all the
      notices that refer to this License and to the absence of any warranty;
      and give any other recipients of the Program a copy of this License
      along with the Program.

      You may charge a fee for the physical act of transferring a copy, and
      you may at your option offer warranty protection in exchange for a fee.

\item [2.] You may modify your copy or copies of the Program or any portion
      of it, thus forming a work based on the Program, and copy and
      distribute such modifications or work under the terms of Section 1
      above, provided that you also meet all of these conditions:

      \begin{enumerate}

      \item [(a)] You must cause the modified files to carry prominent
            notices stating that you changed the files and the date of any
            change.

      \item [(b)] You must cause any work that you distribute or publish,
            that in whole or in part contains or is derived from the Program
            or any part thereof, to be licensed as a whole at no charge to
            all third parties under the terms of this License.

      \item [(c)] If the modified program normally reads commands
            interactively when run, you must cause it, when started running
            for such interactive use in the most ordinary way, to print or
            display an announcement including an appropriate copyright notice
            and a notice that there is no warranty (or else, saying that you
            provide a warranty) and that users may redistribute the program
            under these conditions, and telling the user how to view a copy
            of this License.  (Exception: if the Program itself is
            interactive but does not normally print such an announcement,
            your work based on the Program is not required to print an
            announcement.)

      \end{enumerate}

      These requirements apply to the modified work as a whole.  If
      identifiable sections of that work are not derived from the Program,
      and can be reasonably considered independent and separate works in
      themselves, then this License, and its terms, do not apply to those
      sections when you distribute them as separate works.  But when you
      distribute the same sections as part of a whole which is a work based
      on the Program, the distribution of the whole must be on the terms of
      this License, whose permissions for other licensees extend to the
      entire whole, and thus to each and every part regardless of who wrote
      it.

      Thus, it is not the intent of this section to claim rights or contest
      your rights to work written entirely by you; rather, the intent is to
      exercise the right to control the distribution of derivative or
      collective works based on the Program.

      In addition, mere aggregation of another work not based on the Program
      with the Program (or with a work based on the Program) on a volume of a
      storage or distribution medium does not bring the other work under the
      scope of this License.

\item [3.] You may copy and distribute the Program (or a work based on it,
      under Section 2) in object code or executable form under the terms of
      Sections 1 and 2 above provided that you also do one of the following:

      \begin{enumerate}

      \item [(a)] Accompany it with the complete corresponding
            machine-readable source code, which must be distributed under the
            terms of Sections 1 and 2 above on a medium customarily used for
            software interchange; or,

      \item [(b)] Accompany it with a written offer, valid for at least three
            years, to give any third party, for a charge no more than your
            cost of physically performing source distribution, a complete
            machine-readable copy of the corresponding source code, to be
            distributed under the terms of Sections 1 and 2 above on a medium
            customarily used for software interchange; or,

      \item [(c)] Accompany it with the information you received as to the
            offer to distribute corresponding source code.  (This alternative
            is allowed only for noncommercial distribution and only if you
            received the program in object code or executable form with such
            an offer, in accord with Subsection b above.)

      \end{enumerate}

      The source code for a work means the preferred form of the work for
      making modifications to it.  For an executable work, complete source
      code means all the source code for all modules it contains, plus any
      associated interface definition files, plus the scripts used to control
      compilation and installation of the executable.  However, as a special
      exception, the source code distributed need not include anything that
      is normally distributed (in either source or binary form) with the
      major components (compiler, kernel, and so on) of the operating system
      on which the executable runs, unless that component itself accompanies
      the executable.

      If distribution of executable or object code is made by offering access
      to copy from a designated place, then offering equivalent access to
      copy the source code from the same place counts as distribution of the
      source code, even though third parties are not compelled to copy the
      source along with the object code.

\item [4.] You may not copy, modify, sublicense, or distribute the Program
      except as expressly provided under this License.  Any attempt otherwise
      to copy, modify, sublicense or distribute the Program is void, and will
      automatically terminate your rights under this License.  However,
      parties who have received copies, or rights, from you under this
      License will not have their licenses terminated so long as such parties
      remain in full compliance.

\item [5.] You are not required to accept this License, since you have not
      signed it.  However, nothing else grants you permission to modify or
      distribute the Program or its derivative works.  These actions are
      prohibited by law if you do not accept this License.  Therefore, by
      modifying or distributing the Program (or any work based on the
      Program), you indicate your acceptance of this License to do so, and
      all its terms and conditions for copying, distributing or modifying the
      Program or works based on it.

\item [6.] Each time you redistribute the Program (or any work based on the
      Program), the recipient automatically receives a license from the
      original licensor to copy, distribute or modify the Program subject to
      these terms and conditions.  You may not impose any further
      restrictions on the recipients' exercise of the rights granted herein.
      You are not responsible for enforcing compliance by third parties to
      this License.

\item [7.] If, as a consequence of a court judgment or allegation of patent
      infringement or for any other reason (not limited to patent issues),
      conditions are imposed on you (whether by court order, agreement or
      otherwise) that contradict the conditions of this License, they do not
      excuse you from the conditions of this License.  If you cannot
      distribute so as to satisfy simultaneously your obligations under this
      License and any other pertinent obligations, then as a consequence you
      may not distribute the Program at all.  For example, if a patent
      license would not permit royalty-free redistribution of the Program by
      all those who receive copies directly or indirectly through you, then
      the only way you could satisfy both it and this License would be to
      refrain entirely from distribution of the Program.

      If any portion of this section is held invalid or unenforceable under
      any particular circumstance, the balance of the section is intended to
      apply and the section as a whole is intended to apply in other
      circumstances.

      It is not the purpose of this section to induce you to infringe any
      patents or other property right claims or to contest validity of any
      such claims; this section has the sole purpose of protecting the
      integrity of the free software distribution system, which is
      implemented by public license practices.  Many people have made
      generous contributions to the wide range of software distributed
      through that system in reliance on consistent application of that
      system; it is up to the author/donor to decide if he or she is willing
      to distribute software through any other system and a licensee cannot
      impose that choice.

      This section is intended to make thoroughly clear what is believed to
      be a consequence of the rest of this License.

\item [8.] If the distribution and/or use of the Program is restricted in
      certain countries either by patents or by copyrighted interfaces, the
      original copyright holder who places the Program under this License may
      add an explicit geographical distribution limitation excluding those
      countries, so that distribution is permitted only in or among countries
      not thus excluded.  In such case, this License incorporates the
      limitation as if written in the body of this License.

\item [9.] The Free Software Foundation may publish revised and/or new
      versions of the General Public License from time to time.  Such new
      versions will be similar in spirit to the present version, but may
      differ in detail to address new problems or concerns.

      Each version is given a distinguishing version number.  If the Program
      specifies a version number of this License which applies to it and
      ``any later version'', you have the option of following the terms and
      conditions either of that version or of any later version published by
      the Free Software Foundation.  If the Program does not specify a
      version number of this License, you may choose any version ever
      published by the Free Software Foundation.

\item [10.] If you wish to incorporate parts of the Program into other free
      programs whose distribution conditions are different, write to the
      author to ask for permission.  For software which is copyrighted by the
      Free Software Foundation, write to the Free Software Foundation; we
      sometimes make exceptions for this.  Our decision will be guided by the
      two goals of preserving the free status of all derivatives of our free
      software and of promoting the sharing and reuse of software generally.

\begin{center}
NO WARRANTY
\end{center}

\bfseries

\item [11.] Because the Program is licensed free of charge, there is no
      warranty for the Program, to the extent permitted by applicable law.
      except when otherwise stated in writing the copyright holders and/or
      other parties provide the program ``as is'' without warranty of any
      kind, either expressed or implied, including, but not limited to, the
      implied warranties of merchantability and fitness for a particular
      purpose.  The entire risk as to the quality and performance of the
      Program is with you.  Should the Program prove defective, you assume
      the cost of all necessary servicing, repair or correction.

\item [12.] In no event unless required by applicable law or agreed to in
      writing will any copyright holder, or any other party who may modify
      and/or redistribute the program as permitted above, be liable to you
      for damages, including any general, special, incidental or
      consequential damages arising out of the use or inability to use the
      program (including but not limited to loss of data or data being
      rendered inaccurate or losses sustained by you or third parties or a
      failure of the Program to operate with any other programs), even if
      such holder or other party has been advised of the possibility of such
      damages.

\end{enumerate}

\begin{center}
\textbf{END OF TERMS AND CONDITIONS}
\end{center}


\gplsec{Appendix: How to Apply These Terms to Your New Programs}

If you develop a new program, and you want it to be of the greatest possible
use to the public, the best way to achieve this is to make it free software
which everyone can redistribute and change under these terms.

To do so, attach the following notices to the program.  It is safest to
attach them to the start of each source file to most effectively convey the
exclusion of warranty; and each file should have at least the ``copyright''
line and a pointer to where the full notice is found.

\begin{verbatim}
<one line to give the program's name and a brief idea of what it does.>
Copyright (C) 19yy  <name of author>

This program is free software; you can redistribute it and/or modify
it under the terms of the GNU General Public License as published by
the Free Software Foundation; either version 2 of the License, or
(at your option) any later version.

This program is distributed in the hope that it will be useful,
but WITHOUT ANY WARRANTY; without even the implied warranty of
MERCHANTABILITY or FITNESS FOR A PARTICULAR PURPOSE.  See the
GNU General Public License for more details.

You should have received a copy of the GNU General Public License
along with this program; if not, write to the Free Software
Foundation, Inc., 675 Mass Ave, Cambridge, MA 02139, USA.
\end{verbatim}

Also add information on how to contact you by electronic and paper mail.

If the program is interactive, make it output a short notice like this when
it starts in an interactive mode:

\begin{verbatim}
Gnomovision version 69, Copyright (C) 19yy name of author
Gnomovision comes with ABSOLUTELY NO WARRANTY; for details type `show w'.
This is free software, and you are welcome to redistribute it
under certain conditions; type `show c' for details.
\end{verbatim}

The hypothetical commands `show w' and `show c' should show the appropriate
parts of the General Public License.  Of course, the commands you use may be
called something other than `show w' and `show c'; they could even be
mouse-clicks or menu items--whatever suits your program.

You should also get your employer (if you work as a programmer) or your
school, if any, to sign a ``copyright disclaimer'' for the program, if
necessary.  Here is a sample; alter the names:

\begin{verbatim}
Yoyodyne, Inc., hereby disclaims all copyright interest in the program
`Gnomovision' (which makes passes at compilers) written by James Hacker.

<signature of Ty Coon>, 1 April 1989
Ty Coon, President of Vice
\end{verbatim}

This General Public License does not permit incorporating your program into
proprietary programs.  If your program is a subroutine library, you may
consider it more useful to permit linking proprietary applications with the
library.  If this is what you want to do, use the GNU Library General Public
License instead of this License.

\gplend
}
%    \end{macrocode}
%
% \end{macro}
%
% \begin{macro}{\implementation}
%
% The |\implementation| macro starts typesetting the implementation of
% the package(s).  If we're not doing the implementation, it just does
% this lot and ends the input file.
%
% I define a macro with arguments inside the |\StopEventually|, which causes
% problems, since the code gets put through an extra level of |\def|fing
% depending on whether the implementation stuff gets typeset or not.  I'll
% store the code I want to do in a separate macro.
%
%    \begin{macrocode}
\def\implementation{\StopEventually{\attheend}}
%    \end{macrocode}
%
% Now for the actual activity.  First, I'll do the |\finalstuff|.  Then, if
% \package{doc}'s managed to find the \package{multicol} package, I'll add
% the end of the environment to the end of each contents file in the list.
% Finally, I'll read the index in from its formatted |.ind| file.
%
%    \begin{macrocode}
\tdef\attheend{%
  \finalstuff%
  \ifhave@multicol%
    \def\do##1##2{\addtocontents{##1}{\protect\end{multicols}}}%
    \docontents%
  \fi%
  \PrintIndex%
}
%    \end{macrocode}
%
% \end{macro}
%
%
% \subsection{File version information}
%
% \begin{macro}{\mdwpkginfo}
%
% For setting up the automatic titles, I'll need to be able to work out
% file versions and things.  This macro will, given a file name, extract
% from \LaTeX\ the version information and format it into a sensible string.
%
% First of all, I'll put the original string (direct from the
% |\Provides|\dots\ command).  Then I'll pass it to another macro which can
% parse up the string into its various bits, along with the original
% filename.
%
%    \begin{macrocode}
\def\mdwpkginfo#1{%
  \edef\@tempa{\csname ver@#1\endcsname}%
  \expandafter\mdwpkginfo@i\@tempa\@@#1\@@%
}
%    \end{macrocode}
%
% Now for the real business.  I'll store the string I build in macros called
% \syntax{"\\"<filename>"date", "\\"<filename>"version" and
% "\\"<filename>"info"}, which store the file's date, version and
% `information string' respectively.  (Note that the file extension isn't
% included in the name.)
%
% This is mainly just tedious playing with |\expandafter|.  The date format
% is defined by a separate macro, which can be modified from the
% configuration file.
%
%    \begin{macrocode}
\def\mdwpkginfo@i#1/#2/#3 #4 #5\@@#6.#7\@@{%
  \expandafter\def\csname #6date\endcsname%
    {\protect\mdwdateformat{#1}{#2}{#3}}%
  \expandafter\def\csname #6version\endcsname{#4}%
  \expandafter\def\csname #6info\endcsname{#5}%
}
%    \end{macrocode}
%
% \end{macro}
%
% \begin{macro}{\mdwdateformat}
%
% Given three arguments, a year, a month and a date (all numeric), build a
% pretty date string.  This is fairly simple really.
%
%    \begin{macrocode}
\tdef\mdwdateformat#1#2#3{\number#3\ \monthname{#2}\ \number#1}
\def\monthname#1{%
  \ifcase#1\or%
     January\or February\or March\or April\or May\or June\or%
     July\or August\or September\or October\or November\or December%
  \fi%
}
\def\numsuffix#1{%
  \ifnum#1=1 st\else%
  \ifnum#1=2 nd\else%
  \ifnum#1=3 rd\else%
  \ifnum#1=21 st\else%
  \ifnum#1=22 nd\else%
  \ifnum#1=23 rd\else%
  \ifnum#1=31 st\else%
  th%
  \fi\fi\fi\fi\fi\fi\fi%
}
%    \end{macrocode}
%
% \end{macro}
%
% \begin{macro}{\mdwfileinfo}
%
% Saying \syntax{"\\mdwfileinfo{"<file-name>"}{"<info>"}"} extracts the
% wanted item of \<info> from the version information for file \<file-name>.
%
%    \begin{macrocode}
\def\mdwfileinfo#1#2{\mdwfileinfo@i{#2}#1.\@@}
\def\mdwfileinfo@i#1#2.#3\@@{\csname#2#1\endcsname}
%    \end{macrocode}
%
% \end{macro}
%
%
% \subsection{List handling}
%
% There are several other lists I need to build.  These macros will do
% the necessary stuff.
%
% \begin{macro}{\mdw@ifitem}
%
% The macro \syntax{"\\mdw@ifitem"<item>"\\in"<list>"{"<true-text>"}"^^A
% "{"<false-text>"}"} does \<true-text> if the \<item> matches any item in
% the \<list>; otherwise it does \<false-text>.
%
%    \begin{macrocode}
\def\mdw@ifitem#1\in#2{%
  \@tempswafalse%
  \def\@tempa{#1}%
  \def\do##1{\def\@tempb{##1}\ifx\@tempa\@tempb\@tempswatrue\fi}%
  #2%
  \if@tempswa\expandafter\@firstoftwo\else\expandafter\@secondoftwo\fi%
}
%    \end{macrocode}
%
% \end{macro}
%
% \begin{macro}{\mdw@append}
%
% Saying \syntax{"\\mdw@append"<item>"\\to"<list>} adds the given \<item>
% to the end of the given \<list>.
%
%    \begin{macrocode}
\def\mdw@append#1\to#2{%
  \toks@{\do{#1}}%
  \toks\tw@\expandafter{#2}%
  \edef#2{\the\toks\tw@\the\toks@}%
}
%    \end{macrocode}
%
% \end{macro}
%
% \begin{macro}{\mdw@prepend}
%
% Saying \syntax{"\\mdw@prepend"<item>"\\to"<list>} adds the \<item> to the
% beginning of the \<list>.
%
%    \begin{macrocode}
\def\mdw@prepend#1\to#2{%
  \toks@{\do{#1}}%
  \toks\tw@\expandafter{#2}%
  \edef#2{\the\toks@\the\toks\tw@}%
}
%    \end{macrocode}
%
% \end{macro}
%
% \begin{macro}{\mdw@add}
%
% Finally, saying \syntax{"\\mdw@add"<item>"\\to"<list>} adds the \<item>
% to the list only if it isn't there already.
%
%    \begin{macrocode}
\def\mdw@add#1\to#2{\mdw@ifitem#1\in#2{}{\mdw@append#1\to#2}}
%    \end{macrocode}
%
% \end{macro}
%
%
% \subsection{Described file handling}
%
% I'l maintain lists of packages, document classes, and other files
% described by the current documentation file.
%
% First of all, I'll declare the various list macros.
%
%    \begin{macrocode}
\def\dopackages{}
\def\doclasses{}
\def\dootherfiles{}
%    \end{macrocode}
%
% \begin{macro}{\describespackage}
%
% A document file can declare that it describes a package by saying
% \syntax{"\\describespackage{"<package-name>"}"}.  I add the package to
% my list, read the package into memory (so that the documentation can
% offer demonstrations of it) and read the version information.
%
%    \begin{macrocode}
\def\describespackage#1{%
  \mdw@ifitem#1\in\dopackages{}{%
    \mdw@append#1\to\dopackages%
    \usepackage{#1}%
    \mdwpkginfo{#1.sty}%
  }%
}
%    \end{macrocode}
%
% \end{macro}
%
% \begin{macro}{\describesclass}
%
% By saying \syntax{"\\describesclass{"<class-name>"}"}, a document file
% can declare that it describes a document class.  I'll assume that the
% document class is already loaded, because it's much too late to load
% it now.
%
%    \begin{macrocode}
\def\describesclass#1{\mdw@add#1\to\doclasses\mdwpkginfo{#1.cls}}
%    \end{macrocode}
%
% \end{macro}
%
% \begin{macro}{\describesfile}
%
% Finally, other `random' files, which don't have the status of real \LaTeX\
% packages or document classes, can be described by saying \syntax{^^A
% "\\describesfile{"<file-name>"}" or "\\describesfile*{"<file-name>"}"}.
% The difference is that the starred version will not |\input| the file.
%
%    \begin{macrocode}
\def\describesfile{%
  \@ifstar{\describesfile@i\@gobble}{\describesfile@i\input}%
}
\def\describesfile@i#1#2{%
  \mdw@ifitem#2\in\dootherfiles{}{%
    \mdw@add#2\to\dootherfiles%
    #1{#2}%
    \mdwpkginfo{#2}%
  }%
}
%    \end{macrocode}
%
% \end{macro}
%
%
% \subsection{Author and title handling}
%
% I'll redefine the |\author| and |\title| commands so that I get told
% whether I need to do it myself.
%
% \begin{macro}{\author}
%
% This is easy: I'll save the old meaning, and then redefine |\author| to
% do the old thing and redefine itself to then do nothing.
%
%    \begin{macrocode}
\let\mdw@author\author
\def\author{\let\author\@gobble\mdw@author}
%    \end{macrocode}
%
% \end{macro}
%
% \begin{macro}{\title}
%
% And oddly enough, I'll do exactly the same thing for the title, except
% that I'll also disable the |\mdw@buildtitle| command, which constructs
% the title automatically.
%
%    \begin{macrocode}
\let\mdw@title\title
\def\title{\let\title\@gobble\let\mdw@buildtitle\relax\mdw@title}
%    \end{macrocode}
%
% \end{macro}
%
% \begin{macro}{\date}
%
% This works in a very similar sort of way.
%
%    \begin{macrocode}
\def\date#1{\let\date\@gobble\def\today{#1}}
%    \end{macrocode}
%
% \end{macro}
%
% \begin{macro}{\datefrom}
%
% Saying \syntax{"\\datefrom{"<file-name>"}"} sets the document date from
% the given filename.
%
%    \begin{macrocode}
\def\datefrom#1{%
  \protected@edef\@tempa{\noexpand\date{\csname #1date\endcsname}}%
  \@tempa%
}
%    \end{macrocode}
%
% \end{macro}
%
% \begin{macro}{\docfile}
%
% Saying \syntax{"\\docfile{"<file-name>"}"} sets up the file name from which
% documentation will be read.
%
%    \begin{macrocode}
\def\docfile#1{%
  \def\@tempa##1.##2\@@{\def\@basefile{##1.##2}\def\@basename{##1}}%
  \edef\@tempb{\noexpand\@tempa#1\noexpand\@@}%
  \@tempb%
}
%    \end{macrocode}
%
% I'll set up a default value as well.
%
%    \begin{macrocode}
\docfile{\jobname.dtx}
%    \end{macrocode}
%
% \end{macro}
%
%
% \subsection{Building title strings}
%
% This is rather tricky.  For each list, I need to build a legible looking
% string.
%
% \begin{macro}{\mdw@addtotitle}
%
% By saying
%\syntax{"\\mdw@addtotitle{"<list>"}{"<command>"}{"<singular>"}{"<plural>"}"}
% I can add the contents of a list to the current title string in the
% |\mdw@title| macro.
%
%    \begin{macrocode}
\tdef\mdw@addtotitle#1#2#3#4{%
%    \end{macrocode}
%
% Now to get to work.  I need to keep one `lookahead' list item, and a count
% of the number of items read so far.  I'll keep the lookahead item in
% |\@nextitem| and the counter in |\count@|.
%
%    \begin{macrocode}
  \count@\z@%
%    \end{macrocode}
%
% Now I'll define what to do for each list item.  The |\protect| command is
% already set up appropriately for playing with |\edef| commands.
%
%    \begin{macrocode}
  \def\do##1{%
%    \end{macrocode}
%
% The first job is to add the previous item to the title string.  If this
% is the first item, though, I'll just add the appropriate \lit{The } or
% \lit{ and the } string to the title (this is stored in the |\@prefix|
% macro).
%
%    \begin{macrocode}
    \edef\mdw@title{%
      \mdw@title%
      \ifcase\count@\@prefix%
      \or\@nextitem%
      \else, \@nextitem%
      \fi%
    }%
%    \end{macrocode}
%
% That was rather easy.  Now I'll set up the |\@nextitem| macro for the
% next time around the loop.
%
%    \begin{macrocode}
    \edef\@nextitem{%
      \protect#2{##1}%
      \protect\footnote{%
        The \protect#2{##1} #3 is currently at version %
        \mdwfileinfo{##1}{version}, dated \mdwfileinfo{##1}{date}.%
      }\space%
    }%
%    \end{macrocode}
%
% Finally, I need to increment the counter.
%
%    \begin{macrocode}
    \advance\count@\@ne%
  }%
%    \end{macrocode}
%
% Now execute the list.
%
%    \begin{macrocode}
  #1%
%    \end{macrocode}
%
% I still have one item left over, unless the list was empty.  I'll add
% that now.
%
%    \begin{macrocode}
  \edef\mdw@title{%
    \mdw@title%
    \ifcase\count@%
    \or\@nextitem\space#3%
    \or\ and \@nextitem\space#4%
    \fi%
  }%
%    \end{macrocode}
%
% Finally, if $|\count@| \ne 0$, I must set |\@prefix| to \lit{ and the }.
%
%    \begin{macrocode}
  \ifnum\count@>\z@\def\@prefix{ and the }\fi%
}
%    \end{macrocode}
%
% \end{macro}
%
% \begin{macro}{\mdw@buildtitle}
%
% This macro will actually do the job of building the title string.
%
%    \begin{macrocode}
\tdef\mdw@buildtitle{%
%    \end{macrocode}
%
% First of all, I'll open a group to avoid polluting the namespace with
% my gubbins (although the code is now much tidier than it has been in
% earlier releases).
%
%    \begin{macrocode}
  \begingroup%
%    \end{macrocode}
%
% The title building stuff makes extensive use of |\edef|.  I'll set
% |\protect| appropriately.  (For those not in the know,
% |\@unexpandable@protect| expands to `|\noexpand\protect\noexpand|',
% which prevents expansion of the following macro, and inserts a |\protect|
% in front of it ready for the next |\edef|.)
%
%    \begin{macrocode}
  \let\@@protect\protect\let\protect\@unexpandable@protect%
%    \end{macrocode}
%
% Set up some simple macros ready for the main code.
%
%    \begin{macrocode}
  \def\mdw@title{}%
  \def\@prefix{The }%
%    \end{macrocode}
%
% Now build the title.  This is fun.
%
%    \begin{macrocode}
  \mdw@addtotitle\dopackages\package{package}{packages}%
  \mdw@addtotitle\doclasses\package{document class}{document classes}%
  \mdw@addtotitle\dootherfiles\texttt{file}{files}%
%    \end{macrocode}
%
% Now I want to end the group and set the title from my string.  The
% following hacking will do this.
%
%    \begin{macrocode}
  \edef\next{\endgroup\noexpand\title{\mdw@title}}%
  \next%
}
%    \end{macrocode}
%
% \end{macro}
%
%
% \subsection{Starting the main document}
%
% \begin{macro}{\mdwdoc}
%
% Once the document preamble has done all of its stuff, it calls the
% |\mdwdoc| command, which takes over and really starts the documentation
% going.
%
%    \begin{macrocode}
\def\mdwdoc{%
%    \end{macrocode}
%
% First, I'll construct the title string.
%
%    \begin{macrocode}
  \mdw@buildtitle%
  \author{Mark Wooding}%
%    \end{macrocode}
%
% Set up the date string based on the date of the package which shares
% the same name as the current file.
%
%    \begin{macrocode}
  \datefrom\@basename%
%    \end{macrocode}
%
% Set up verbatim characters after all the packages have started.
%
%    \begin{macrocode}
  \shortverb\|%
  \shortverb\"%
%    \end{macrocode}
%
% Start the document, and put the title in.
%
%    \begin{macrocode}
  \begin{document}
  \maketitle%
%    \end{macrocode}
%
% This is nasty.  It makes maths displays work properly in demo environments.
% \emph{The \LaTeX\ Companion} exhibits the bug which this hack fixes.  So
% ner.
%
%    \begin{macrocode}
  \abovedisplayskip\z@%
%    \end{macrocode}
%
% Now start the contents tables.  After starting each one, I'll make it
% be multicolumnar.
%
%    \begin{macrocode}
  \def\do##1##2{%
    ##2%
    \ifhave@multicol\addtocontents{##1}{%
      \protect\begin{multicols}{2}%
      \hbadness\@M%
    }\fi%
  }%
  \docontents%
%    \end{macrocode}
%
% Input the main file now.
%
%    \begin{macrocode}
  \DocInput{\@basefile}%
%    \end{macrocode}
%
% That's it.  I'm done.
%
%    \begin{macrocode}
  \end{document}
}
%    \end{macrocode}
%
% \end{macro}
%
%
% \subsection{And finally\dots}
%
% Right at the end I'll put a hook for the configuration file.
%
%    \begin{macrocode}
\ifx\mdwhook\@@undefined\else\expandafter\mdwhook\fi
%    \end{macrocode}
%
% That's all the code done now.  I'll change back to `user' mode, where
% all the magic control sequences aren't allowed any more.
%
%    \begin{macrocode}
\makeatother
%</mdwtools>
%    \end{macrocode}
%
% Oh, wait!  What if I want to typeset this documentation?  Aha.  I'll cope
% with that by comparing |\jobname| with my filename |mdwtools|.  However,
% there's some fun here, because |\jobname| contains category-12 letters,
% while my letters are category-11.  Time to play with |\string| in a messy
% way.
%
%    \begin{macrocode}
%<*driver>
\makeatletter
\edef\@tempa{\expandafter\@gobble\string\mdwtools}
\edef\@tempb{\jobname}
\ifx\@tempa\@tempb
  \describesfile*{mdwtools.tex}
  \docfile{mdwtools.tex}
  \makeatother
  \expandafter\mdwdoc
\fi
\makeatother
%</driver>
%    \end{macrocode}
%
% That's it.  Done!
%
% \hfill Mark Wooding, \today
%
% \Finale
%
\endinput
|
% \<declarations>
% |\mdwdoc|
% \end{listinglist}
% The initial |\input| reads in this file and sets up the various commands
% which may be needed.  The final |\mdwdoc| actually starts the document,
% inserting a title (which is automatically generated), a table of
% contents etc., and reads the documentation file in (using the |\DocInput|
% command from the \package{doc} package.
%
% \subsubsection{Describing packages}
%
% \DescribeMacro{\describespackage}
% \DescribeMacro{\describesclass}
% \DescribeMacro{\describesfile}
% \DescribeMacro{\describesfile*}
% The most important declarations are those which declare what the
% documentation describes.  Saying \syntax{"\\describespackage{<package>}"}
% loads the \<package> (if necessary) and adds it to the auto-generated
% title, along with a footnote containing version information.  Similarly,
% |\describesclass| adds a document class name to the title (without loading
% it -- the document itself must do this, with the |\documentclass| command).
% For files which aren't packages or classes, use the |\describesfile| or
% |\describesfile*| command (the $*$-version won't |\input| the file, which
% is handy for files like |mdwtools.tex|, which are already input).
%
% \DescribeMacro{\author}
% \DescribeMacro{\date}
% \DescribeMacro{\title}
% The |\author|, |\date| and |\title| declarations work slightly differently
% to normal -- they ensure that only the \emph{first} declaration has an
% effect.  (Don't you play with |\author|, please, unless you're using this
% program to document your own packages.)  Using |\title| suppresses the
% automatic title generation.
%
% \DescribeMacro{\docdate}
% The default date is worked out from the version string of the package or
% document class whose name is the same as that of the documentation file.
% You can choose a different `main' file by saying
% \syntax{"\\docdate{"<file>"}"}.
%
% \subsubsection{Contents handling}
%
% \DescribeMacro{\addcontents}
% A documentation file always has a table of contents.  Other
% contents-like lists can be added by saying
% \syntax{"\\addcontents{"<extension>"}{"<command>"}"}.  The \<extension>
% is the file extension of the contents file (e.g., \lit{lot} for the
% list of tables); the \<command> is the command to actually typeset the
% contents file (e.g., |\listoftables|).
%
% \subsubsection{Other declarations}
%
% \DescribeMacro{\implementation}
% The \package{doc} package wants you to say
% \syntax{"\\StopEventually{"<stuff>"}"}' before describing the package
% implementation.  Using |mdwtools.tex|, you just say |\implementation|, and
% everything works.  It will automatically read in the licence text (from
% |gpl.tex|, and wraps some other things up.
%
% 
% \subsection{Other commands}
%
% The |mdwtools.tex| file includes the \package{syntax} and \package{sverb}
% packages so that they can be used in documentation files.  It also defines
% some trivial commands of its own.
%
% \DescribeMacro{\<}
% Saying \syntax{"\\<"<text>">" is the same as "\\synt{"<text>"}"}; this
% is a simple abbreviation.
%
% \DescribeMacro{\smallf}
% Saying \syntax{"\\smallf" <number>"/"<number>} typesets a little fraction,
% like this: \smallf 3/4.  It's useful when you want to say that the default
% value of a length is 2 \smallf 1/2\,pt, or something like that.
%
%
% \subsection{Customisation}
%
% You can customise the way that the package documentation looks by writing
% a file called |mdwtools.cfg|.  You can redefine various commands (before
% they're defined here, even; |mdwtools.tex| checks most of the commands that
% it defines to make sure they haven't been defined already.
%
% \DescribeMacro{\indexing}
% If you don't want the prompt about whether to generate index files, you
% can define the |\indexing| command to either \lit{y} or \lit{n}.  I'd
% recommend that you use |\providecommand| for this, to allow further
% customisation from the command line.
%
% \DescribeMacro{\mdwdateformat}
% If you don't like my date format (maybe you're American or something),
% you can redefine the |\mdwdateformat| command.  It takes three arguments:
% the year, month and date, as numbers; it should expand to something which
% typesets the date nicely.  The default format gives something like
% `10 May 1996'.  You can produce something rather more exotic, like
% `10\textsuperscript{th} May \textsc{\romannumeral 1996}' by saying
%\begin{listing}
%\newcommand{\mdwdateformat}[3]{%
%  \number#3\textsuperscript{\numsuffix{#3}}\ %
%  \monthname{#2}\ %
%  \textsc{\romannumeral #1}%
%}
%\end{listing}
% \DescribeMacro{\monthname}
% \DescribeMacro{\numsuffix}
% Saying \syntax{"\\monthname{"<number>"}"} expands to the name of the
% numbered month (which can be useful when doing date formats).  Saying
% \syntax{"\\numsuffix{"<number>"}"} will expand to the appropriate suffix
% (`th' or `rd' or whatever) for the \<number>.  You'll have to superscript
% it yourself, if this is what you want to do.  Putting the year number
% in roman numerals is just pretentious |;-)|.
%
% \DescribeMacro{\mdwhook}
% After all the declarations in |mdwtools.tex|, the command |\mdwhook| is
% executed, if it exists.  This can be set up by the configuration file
% to do whatever you want.
%
% There are lots of other things you can play with; you should look at the
% implementation section to see what's possible.
%
% \implementation
%
% \section{Implementation}
%
%    \begin{macrocode}
%<*mdwtools>
%    \end{macrocode}
%
% The first thing is that I'm not a \LaTeX\ package or anything official
% like that, so I must enable `|@|' as a letter by hand.
%
%    \begin{macrocode}
\makeatletter
%    \end{macrocode}
%
% Now input the user's configuration file, if it exists.  This is fairly
% simple stuff.
%
%    \begin{macrocode}
\@input{mdwtools.cfg}
%    \end{macrocode}
%
% Well, that's the easy bit done.
%
%
% \subsection{Initialisation}
%
% Obviously the first thing to do is to obtain a document class.  Obviously,
% it would be silly to do this if a document class has already been loaded,
% either by the package documentation or by the configuration file.
%
% The only way I can think of for finding out if a document class is already
% loaded is by seeing if the |\documentclass| command has been redefined
% to raise an error.  This isn't too hard, really.
%
%    \begin{macrocode}
\ifx\documentclass\@twoclasseserror\else
  \documentclass[a4paper]{ltxdoc}
  \ifx\doneclasses\mdw@undefined\else\doneclasses\fi
\fi
%    \end{macrocode}
%
% As part of my standard environment, I'll load some of my more useful
% packages.  If they're already loaded (possibly with different options),
% I'll not try to load them again.
%
%    \begin{macrocode}
\@ifpackageloaded{doc}{}{\usepackage{doc}}
\@ifpackageloaded{syntax}{}{\usepackage[rounded]{syntax}}
\@ifpackageloaded{sverb}{}{\usepackage{sverb}}
%    \end{macrocode}
%
%
% \subsection{Some macros for interaction}
%
% I like the \LaTeX\ star-boxes, although it's a pain having to cope with
% \TeX's space-handling rules.  I'll define a new typing-out macro which
% makes spaces more significant, and has a $*$-version which doesn't put
% a newline on the end, and interacts prettily with |\read|.
%
% First of all, I need to make spaces active, so I can define things about
% active spaces.
%
%    \begin{macrocode}
\begingroup\obeyspaces
%    \end{macrocode}
%
% Now to define the main macro.  This is easy stuff.  Spaces must be
% carefully rationed here, though.
%
% I'll start a group, make spaces active, and make spaces expand to ordinary
% space-like spaces.  Then I'll look for a star, and pass either |\message|
% (which doesn't start a newline, and interacts with |\read| well) or
% |\immediate\write 16| which does a normal write well.
%
%    \begin{macrocode}
\gdef\mdwtype{%
\begingroup\catcode`\ \active\let \space%
\@ifstar{\mdwtype@i{\message}}{\mdwtype@i{\immediate\write\sixt@@n}}%
}
\endgroup
%    \end{macrocode}
%
% Now for the easy bit.  I have the thing to do, and the thing to do it to,
% so do that and end the group.
%
%    \begin{macrocode}
\def\mdwtype@i#1#2{#1{#2}\endgroup}
%    \end{macrocode}
%
%
% \subsection{Decide on indexing}
%
% A configuration file can decide on indexing by defining the |\indexing|
% macro to either \lit{y} or \lit{n}.  If it's not set, then I'll prompt
% the user.
%
% First of all, I want a switch to say whether I'm indexing.
%
%    \begin{macrocode}
\newif\ifcreateindex
%    \end{macrocode}
%
% Right: now I need to decide how to make progress.  If the macro's not set,
% then I want to set it, and start a row of stars.
%
%    \begin{macrocode}
\ifx\indexing\@@undefined
  \mdwtype{*****************************}
  \def\indexing{?}
\fi
%    \end{macrocode}
%
% Now enter a loop, asking the user whether to do indexing, until I get
% a sensible answer.
%
%    \begin{macrocode}
\loop
  \@tempswafalse
  \if y\indexing\@tempswatrue\createindextrue\fi
  \if Y\indexing\@tempswatrue\createindextrue\fi
  \if n\indexing\@tempswatrue\createindexfalse\fi
  \if N\indexing\@tempswatrue\createindexfalse\fi
  \if@tempswa\else
  \mdwtype*{* Create index files? (y/n) *}
  \read\sixt@@n to\indexing%
\repeat
%    \end{macrocode}
%
% Now, based on the results of that, display a message about the indexing.
%
%    \begin{macrocode}
\mdwtype{*****************************}
\ifcreateindex
  \mdwtype{* Creating index files      *}
  \mdwtype{* This may take some time   *}
\else
  \mdwtype{* Not creating index files  *}
\fi
\mdwtype{*****************************}
%    \end{macrocode}
%
% Now I can play with the indexing commands of the \package{doc} package
% to do whatever it is that the user wants.
%
%    \begin{macrocode}
\ifcreateindex
  \CodelineIndex
  \EnableCrossrefs
\else
  \CodelineNumbered
  \DisableCrossrefs
\fi
%    \end{macrocode}
%
% And register lots of plain \TeX\ things which shouldn't be indexed.
% This contains lots of |\if|\dots\ things which don't fit nicely in
% conditionals, which is a shame.  Still, it doesn't matter that much,
% really.
%
%    \begin{macrocode}
\DoNotIndex{\def,\long,\edef,\xdef,\gdef,\let,\global}
\DoNotIndex{\if,\ifnum,\ifdim,\ifcat,\ifmmode,\ifvmode,\ifhmode,%
            \iftrue,\iffalse,\ifvoid,\ifx,\ifeof,\ifcase,\else,\or,\fi}
\DoNotIndex{\box,\copy,\setbox,\unvbox,\unhbox,\hbox,%
            \vbox,\vtop,\vcenter}
\DoNotIndex{\@empty,\immediate,\write}
\DoNotIndex{\egroup,\bgroup,\expandafter,\begingroup,\endgroup}
\DoNotIndex{\divide,\advance,\multiply,\count,\dimen}
\DoNotIndex{\relax,\space,\string}
\DoNotIndex{\csname,\endcsname,\@spaces,\openin,\openout,%
            \closein,\closeout}
\DoNotIndex{\catcode,\endinput}
\DoNotIndex{\jobname,\message,\read,\the,\m@ne,\noexpand}
\DoNotIndex{\hsize,\vsize,\hskip,\vskip,\kern,\hfil,\hfill,\hss}
\DoNotIndex{\m@ne,\z@,\z@skip,\@ne,\tw@,\p@}
\DoNotIndex{\dp,\wd,\ht,\vss,\unskip}
%    \end{macrocode}
%
% Last bit of indexing stuff, for now: I'll typeset the index in two columns
% (the default is three, which makes them too narrow for my tastes).
%
%    \begin{macrocode}
\setcounter{IndexColumns}{2}
%    \end{macrocode}
%
%
% \subsection{Selectively defining things}
%
% I don't want to tread on anyone's toes if they redefine any of these
% commands and things in a configuration file.  The following definitions
% are fairly evil, but should do the job OK.
%
% \begin{macro}{\@gobbledef}
%
% This macro eats the following |\def|inition, leaving not a trace behind.
%
%    \begin{macrocode}
\def\@gobbledef#1#{\@gobble}
%    \end{macrocode}
%
% \end{macro}
%
% \begin{macro}{\tdef}
% \begin{macro}{\tlet}
%
% The |\tdef| command is a sort of `tentative' definition -- it's like
% |\def| if the control sequence named doesn't already have a definition.
% |\tlet| does the same thing with |\let|.
%
%    \begin{macrocode}
\def\tdef#1{
  \ifx#1\@@undefined%
    \expandafter\def\expandafter#1%
  \else%
    \expandafter\@gobbledef%
  \fi%
}
\def\tlet#1#2{\ifx#1\@@undefined\let#1=#2\fi}
%    \end{macrocode}
%
% \end{macro}
% \end{macro}
%
%
% \subsection{General markup things}
%
% Now for some really simple things.  I'll define how to typeset package
% names and environment names (both in the sans serif font, for now).
%
%    \begin{macrocode}
\tlet\package\textsf
\tlet\env\textsf
%    \end{macrocode}
%
% I'll define the |\<|\dots|>| shortcut for syntax items suggested in the
% \package{syntax} package.
%
%    \begin{macrocode}
\tdef\<#1>{\synt{#1}}
%    \end{macrocode}
%
% And because it's used in a few places (mainly for typesetting lengths),
% here's a command for typesetting fractions in text.
%
%    \begin{macrocode}
\tdef\smallf#1/#2{\ensuremath{^{#1}\!/\!_{#2}}}
%    \end{macrocode}
%
%
% \subsection{A table environment}
%
% \begin{environment}{tab}
%
% Most of the packages don't use the (obviously perfect) \package{mdwtab}
% package, because it's big, and takes a while to load.  Here's an
% environment for typesetting centred tables.  The first (optional) argument
% is some declarations to perform.  The mandatory argument is the table
% preamble (obviously).
%
%    \begin{macrocode}
\@ifundefined{tab}{%
  \newenvironment{tab}[2][\relax]{%
    \par\vskip2ex%
    \centering%
    #1%
    \begin{tabular}{#2}%
  }{%
    \end{tabular}%
    \par\vskip2ex%
  }
}{}
%    \end{macrocode}
%
% \end{environment}
%
%
% \subsection{Commenting out of stuff}
%
% \begin{environment}{meta-comment}
%
% Using |\iffalse|\dots|\fi| isn't much fun.  I'll define a gobbling
% environment using the \package{sverb} stuff.
%
%    \begin{macrocode}
\ignoreenv{meta-comment}
%    \end{macrocode}
%
% \end{environment}
%
%
% \subsection{Float handling}
%
% This gubbins will try to avoid float pages as much as possible, and (with
% any luck) encourage floats to be put on the same pages as text.
%
%    \begin{macrocode}
\def\textfraction{0.1}
\def\topfraction{0.9}
\def\bottomfraction{0.9}
\def\floatpagefraction{0.7}
%    \end{macrocode}
%
% Now redefine the default float-placement parameters to allow `here' floats.
%
%    \begin{macrocode}
\def\fps@figure{htbp}
\def\fps@table{htbp}
%    \end{macrocode}
%
%
% \subsection{Other bits of parameter tweaking}
%
% Make \env{grammar} environments look pretty, by indenting the left hand
% sides by a large amount.
%
%    \begin{macrocode}
\grammarindent1in
%    \end{macrocode}
%
% I don't like being told by \TeX\ that my paragraphs are hard to linebreak:
% I know this already.  This lot should shut \TeX\ up about most problems.
%
%    \begin{macrocode}
\sloppy
\hbadness\@M
\hfuzz10\p@
%    \end{macrocode}
%
% Also make \TeX\ shut up in the index.  The \package{multicol} package
% irritatingly plays with |\hbadness|.  This is the best hook I could find
% for playing with this setting.
%
%    \begin{macrocode}
\expandafter\def\expandafter\IndexParms\expandafter{%
  \IndexParms%
  \hbadness\@M%
}
%    \end{macrocode}
%
% The other thing I really don't like is `Marginpar moved' warnings.  This
% will get rid of them, and lots of other \LaTeX\ warnings at the same time.
%
%    \begin{macrocode}
\let\@latex@warning@no@line\@gobble
%    \end{macrocode}
%
% Put some extra space between table rows, please.
%
%    \begin{macrocode}
\def\arraystretch{1.2}
%    \end{macrocode}
%
% Most of the code is at guard level one, so typeset that in upright text.
%
%    \begin{macrocode}
\setcounter{StandardModuleDepth}{1}
%    \end{macrocode}
%
%
% \subsection{Contents handling}
%
% I use at least one contents file (the main table of contents) although
% I may want more.  I'll keep a list of contents files which I need to
% handle.
%
% There are two things I need to do to contents files here:
% \begin{itemize}
% \item I must typeset the table of contents at the beginning of the
%       document; and
% \item I want to typeset tables of contents in two columns (using the
%       \package{multicol} package).
% \end{itemize}
%
% The list consists of items of the form
% \syntax{"\\do{"<extension>"}{"<command>"}"}, where \<extension> is the
% file extension of the contents file, and \<command> is the command to
% typeset it.
%
% \begin{macro}{\docontents}
%
% This is where I keep the list of contents files.  I'll initialise it to
% just do the standard contents table.
%
%    \begin{macrocode}
\def\docontents{\do{toc}{\tableofcontents}}
%    \end{macrocode}
%
% \end{macro}
%
% \begin{macro}{\addcontents}
%
% By saying \syntax{"\\addcontents{"<extension>"}{"<command>"}"}, a document
% can register a new table of contents which gets given the two-column
% treatment properly.  This is really easy to implement.
%
%    \begin{macrocode}
\def\addcontents#1#2{%
  \toks@\expandafter{\docontents\do{#1}{#2}}%
  \edef\docontents{\the\toks@}%
}
%    \end{macrocode}
%
% \end{macro}
%
%
% \subsection{Finishing it all off}
%
% \begin{macro}{\finalstuff}
%
% The |\finalstuff| macro is a hook for doing things at the end of the
% document.  Currently, it inputs the licence agreement as an appendix.
%
%    \begin{macrocode}
\tdef\finalstuff{\appendix\part*{Appendix}% \iffalse <meta-comment>
%
% $Id: gpl.tex,v 1.1 1996/11/19 20:51:14 mdw Exp $
%
% The GNU General Public Licence as a LaTeX section
%
% (c) 1989, 1991 Free Software Foundation, Inc.
%   LaTeX markup and minor formatting changes by Mark Wooding
%

%----- Revision history -----------------------------------------------------
%
% $Log: gpl.tex,v $
% Revision 1.1  1996/11/19 20:51:14  mdw
% Initial revision
%

% --- Chapter heading ---
%
% We don't know whether this ought to be a section or a chapter.  Easy.
% We'll see if chapters are possible.
%
% \fi

\begingroup
\makeatletter

\edef\next#1#2#3{\relax
  \ifx\chapter\@@undefined
    \ifx\documentclass\@notprerr#2\else#3\fi
  \else#1\fi
}

\expandafter\endgroup\next
{
  \let\gpltoplevel\chapter
  \let\gplsec\section
  \let\gplend\endinput
}{
  \let\gpltoplevel\section
  \let\gplsec\subsection
  \let\gplend\endinput
}{
  \documentclass[a4paper]{article}
  \def\gpltoplevel#1{%
    \vspace*{1in}%
    \hbox to\hsize{\hfil\LARGE\bfseries#1\hfil}%
    \vspace{1in}%
  }
  \let\gplsec\section
  \def\gplend{\end{document}}
  \advance\textwidth1in
  \advance\oddsidemargin-.5in
  \sloppy
  \begin{document}
}

%^^A-------------------------------------------------------------------------
\gpltoplevel{The GNU General Public Licence}


The following is the text of the GNU General Public Licence, under the terms
of which this software is distrubuted.

\vspace{12pt}

\begin{center}
\textbf{GNU GENERAL PUBLIC LICENSE} \\
Version 2, June 1991
\end{center}

\begin{center}
Copyright (C) 1989, 1991 Free Software Foundation, Inc. \\
675 Mass Ave, Cambridge, MA 02139, USA

Everyone is permitted to copy and distribute verbatim copies \\
of this license document, but changing it is not allowed.
\end{center}


\gplsec{Preamble}

The licenses for most software are designed to take away your freedom to
share and change it.  By contrast, the GNU General Public License is intended
to guarantee your freedom to share and change free software---to make sure
the software is free for all its users.  This General Public License applies
to most of the Free Software Foundation's software and to any other program
whose authors commit to using it.  (Some other Free Software Foundation
software is covered by the GNU Library General Public License instead.)  You
can apply it to your programs, too.

When we speak of free software, we are referring to freedom, not price.  Our
General Public Licenses are designed to make sure that you have the freedom
to distribute copies of free software (and charge for this service if you
wish), that you receive source code or can get it if you want it, that you
can change the software or use pieces of it in new free programs; and that
you know you can do these things.

To protect your rights, we need to make restrictions that forbid anyone to
deny you these rights or to ask you to surrender the rights.  These
restrictions translate to certain responsibilities for you if you distribute
copies of the software, or if you modify it.

For example, if you distribute copies of such a program, whether gratis or
for a fee, you must give the recipients all the rights that you have.  You
must make sure that they, too, receive or can get the source code.  And you
must show them these terms so they know their rights.

We protect your rights with two steps: (1) copyright the software, and (2)
offer you this license which gives you legal permission to copy, distribute
and/or modify the software.

Also, for each author's protection and ours, we want to make certain that
everyone understands that there is no warranty for this free software.  If
the software is modified by someone else and passed on, we want its
recipients to know that what they have is not the original, so that any
problems introduced by others will not reflect on the original authors'
reputations.

Finally, any free program is threatened constantly by software patents.  We
wish to avoid the danger that redistributors of a free program will
individually obtain patent licenses, in effect making the program
proprietary.  To prevent this, we have made it clear that any patent must be
licensed for everyone's free use or not licensed at all.

The precise terms and conditions for copying, distribution and modification
follow.


\gplsec{Terms and conditions for copying, distribution and modification}

\begin{enumerate}

\makeatletter \setcounter{\@listctr}{-1} \makeatother

\item [0.] This License applies to any program or other work which contains a
      notice placed by the copyright holder saying it may be distributed
      under the terms of this General Public License.  The ``Program'',
      below, refers to any such program or work, and a ``work based on the
      Program'' means either the Program or any derivative work under
      copyright law: that is to say, a work containing the Program or a
      portion of it, either verbatim or with modifications and/or translated
      into another language.  (Hereinafter, translation is included without
      limitation in the term ``modification''.)  Each licensee is addressed
      as ``you''.

      Activities other than copying, distribution and modification are not
      covered by this License; they are outside its scope.  The act of
      running the Program is not restricted, and the output from the Program
      is covered only if its contents constitute a work based on the Program
      (independent of having been made by running the Program).  Whether that
      is true depends on what the Program does.

\item [1.] You may copy and distribute verbatim copies of the Program's
      source code as you receive it, in any medium, provided that you
      conspicuously and appropriately publish on each copy an appropriate
      copyright notice and disclaimer of warranty; keep intact all the
      notices that refer to this License and to the absence of any warranty;
      and give any other recipients of the Program a copy of this License
      along with the Program.

      You may charge a fee for the physical act of transferring a copy, and
      you may at your option offer warranty protection in exchange for a fee.

\item [2.] You may modify your copy or copies of the Program or any portion
      of it, thus forming a work based on the Program, and copy and
      distribute such modifications or work under the terms of Section 1
      above, provided that you also meet all of these conditions:

      \begin{enumerate}

      \item [(a)] You must cause the modified files to carry prominent
            notices stating that you changed the files and the date of any
            change.

      \item [(b)] You must cause any work that you distribute or publish,
            that in whole or in part contains or is derived from the Program
            or any part thereof, to be licensed as a whole at no charge to
            all third parties under the terms of this License.

      \item [(c)] If the modified program normally reads commands
            interactively when run, you must cause it, when started running
            for such interactive use in the most ordinary way, to print or
            display an announcement including an appropriate copyright notice
            and a notice that there is no warranty (or else, saying that you
            provide a warranty) and that users may redistribute the program
            under these conditions, and telling the user how to view a copy
            of this License.  (Exception: if the Program itself is
            interactive but does not normally print such an announcement,
            your work based on the Program is not required to print an
            announcement.)

      \end{enumerate}

      These requirements apply to the modified work as a whole.  If
      identifiable sections of that work are not derived from the Program,
      and can be reasonably considered independent and separate works in
      themselves, then this License, and its terms, do not apply to those
      sections when you distribute them as separate works.  But when you
      distribute the same sections as part of a whole which is a work based
      on the Program, the distribution of the whole must be on the terms of
      this License, whose permissions for other licensees extend to the
      entire whole, and thus to each and every part regardless of who wrote
      it.

      Thus, it is not the intent of this section to claim rights or contest
      your rights to work written entirely by you; rather, the intent is to
      exercise the right to control the distribution of derivative or
      collective works based on the Program.

      In addition, mere aggregation of another work not based on the Program
      with the Program (or with a work based on the Program) on a volume of a
      storage or distribution medium does not bring the other work under the
      scope of this License.

\item [3.] You may copy and distribute the Program (or a work based on it,
      under Section 2) in object code or executable form under the terms of
      Sections 1 and 2 above provided that you also do one of the following:

      \begin{enumerate}

      \item [(a)] Accompany it with the complete corresponding
            machine-readable source code, which must be distributed under the
            terms of Sections 1 and 2 above on a medium customarily used for
            software interchange; or,

      \item [(b)] Accompany it with a written offer, valid for at least three
            years, to give any third party, for a charge no more than your
            cost of physically performing source distribution, a complete
            machine-readable copy of the corresponding source code, to be
            distributed under the terms of Sections 1 and 2 above on a medium
            customarily used for software interchange; or,

      \item [(c)] Accompany it with the information you received as to the
            offer to distribute corresponding source code.  (This alternative
            is allowed only for noncommercial distribution and only if you
            received the program in object code or executable form with such
            an offer, in accord with Subsection b above.)

      \end{enumerate}

      The source code for a work means the preferred form of the work for
      making modifications to it.  For an executable work, complete source
      code means all the source code for all modules it contains, plus any
      associated interface definition files, plus the scripts used to control
      compilation and installation of the executable.  However, as a special
      exception, the source code distributed need not include anything that
      is normally distributed (in either source or binary form) with the
      major components (compiler, kernel, and so on) of the operating system
      on which the executable runs, unless that component itself accompanies
      the executable.

      If distribution of executable or object code is made by offering access
      to copy from a designated place, then offering equivalent access to
      copy the source code from the same place counts as distribution of the
      source code, even though third parties are not compelled to copy the
      source along with the object code.

\item [4.] You may not copy, modify, sublicense, or distribute the Program
      except as expressly provided under this License.  Any attempt otherwise
      to copy, modify, sublicense or distribute the Program is void, and will
      automatically terminate your rights under this License.  However,
      parties who have received copies, or rights, from you under this
      License will not have their licenses terminated so long as such parties
      remain in full compliance.

\item [5.] You are not required to accept this License, since you have not
      signed it.  However, nothing else grants you permission to modify or
      distribute the Program or its derivative works.  These actions are
      prohibited by law if you do not accept this License.  Therefore, by
      modifying or distributing the Program (or any work based on the
      Program), you indicate your acceptance of this License to do so, and
      all its terms and conditions for copying, distributing or modifying the
      Program or works based on it.

\item [6.] Each time you redistribute the Program (or any work based on the
      Program), the recipient automatically receives a license from the
      original licensor to copy, distribute or modify the Program subject to
      these terms and conditions.  You may not impose any further
      restrictions on the recipients' exercise of the rights granted herein.
      You are not responsible for enforcing compliance by third parties to
      this License.

\item [7.] If, as a consequence of a court judgment or allegation of patent
      infringement or for any other reason (not limited to patent issues),
      conditions are imposed on you (whether by court order, agreement or
      otherwise) that contradict the conditions of this License, they do not
      excuse you from the conditions of this License.  If you cannot
      distribute so as to satisfy simultaneously your obligations under this
      License and any other pertinent obligations, then as a consequence you
      may not distribute the Program at all.  For example, if a patent
      license would not permit royalty-free redistribution of the Program by
      all those who receive copies directly or indirectly through you, then
      the only way you could satisfy both it and this License would be to
      refrain entirely from distribution of the Program.

      If any portion of this section is held invalid or unenforceable under
      any particular circumstance, the balance of the section is intended to
      apply and the section as a whole is intended to apply in other
      circumstances.

      It is not the purpose of this section to induce you to infringe any
      patents or other property right claims or to contest validity of any
      such claims; this section has the sole purpose of protecting the
      integrity of the free software distribution system, which is
      implemented by public license practices.  Many people have made
      generous contributions to the wide range of software distributed
      through that system in reliance on consistent application of that
      system; it is up to the author/donor to decide if he or she is willing
      to distribute software through any other system and a licensee cannot
      impose that choice.

      This section is intended to make thoroughly clear what is believed to
      be a consequence of the rest of this License.

\item [8.] If the distribution and/or use of the Program is restricted in
      certain countries either by patents or by copyrighted interfaces, the
      original copyright holder who places the Program under this License may
      add an explicit geographical distribution limitation excluding those
      countries, so that distribution is permitted only in or among countries
      not thus excluded.  In such case, this License incorporates the
      limitation as if written in the body of this License.

\item [9.] The Free Software Foundation may publish revised and/or new
      versions of the General Public License from time to time.  Such new
      versions will be similar in spirit to the present version, but may
      differ in detail to address new problems or concerns.

      Each version is given a distinguishing version number.  If the Program
      specifies a version number of this License which applies to it and
      ``any later version'', you have the option of following the terms and
      conditions either of that version or of any later version published by
      the Free Software Foundation.  If the Program does not specify a
      version number of this License, you may choose any version ever
      published by the Free Software Foundation.

\item [10.] If you wish to incorporate parts of the Program into other free
      programs whose distribution conditions are different, write to the
      author to ask for permission.  For software which is copyrighted by the
      Free Software Foundation, write to the Free Software Foundation; we
      sometimes make exceptions for this.  Our decision will be guided by the
      two goals of preserving the free status of all derivatives of our free
      software and of promoting the sharing and reuse of software generally.

\begin{center}
NO WARRANTY
\end{center}

\bfseries

\item [11.] Because the Program is licensed free of charge, there is no
      warranty for the Program, to the extent permitted by applicable law.
      except when otherwise stated in writing the copyright holders and/or
      other parties provide the program ``as is'' without warranty of any
      kind, either expressed or implied, including, but not limited to, the
      implied warranties of merchantability and fitness for a particular
      purpose.  The entire risk as to the quality and performance of the
      Program is with you.  Should the Program prove defective, you assume
      the cost of all necessary servicing, repair or correction.

\item [12.] In no event unless required by applicable law or agreed to in
      writing will any copyright holder, or any other party who may modify
      and/or redistribute the program as permitted above, be liable to you
      for damages, including any general, special, incidental or
      consequential damages arising out of the use or inability to use the
      program (including but not limited to loss of data or data being
      rendered inaccurate or losses sustained by you or third parties or a
      failure of the Program to operate with any other programs), even if
      such holder or other party has been advised of the possibility of such
      damages.

\end{enumerate}

\begin{center}
\textbf{END OF TERMS AND CONDITIONS}
\end{center}


\gplsec{Appendix: How to Apply These Terms to Your New Programs}

If you develop a new program, and you want it to be of the greatest possible
use to the public, the best way to achieve this is to make it free software
which everyone can redistribute and change under these terms.

To do so, attach the following notices to the program.  It is safest to
attach them to the start of each source file to most effectively convey the
exclusion of warranty; and each file should have at least the ``copyright''
line and a pointer to where the full notice is found.

\begin{verbatim}
<one line to give the program's name and a brief idea of what it does.>
Copyright (C) 19yy  <name of author>

This program is free software; you can redistribute it and/or modify
it under the terms of the GNU General Public License as published by
the Free Software Foundation; either version 2 of the License, or
(at your option) any later version.

This program is distributed in the hope that it will be useful,
but WITHOUT ANY WARRANTY; without even the implied warranty of
MERCHANTABILITY or FITNESS FOR A PARTICULAR PURPOSE.  See the
GNU General Public License for more details.

You should have received a copy of the GNU General Public License
along with this program; if not, write to the Free Software
Foundation, Inc., 675 Mass Ave, Cambridge, MA 02139, USA.
\end{verbatim}

Also add information on how to contact you by electronic and paper mail.

If the program is interactive, make it output a short notice like this when
it starts in an interactive mode:

\begin{verbatim}
Gnomovision version 69, Copyright (C) 19yy name of author
Gnomovision comes with ABSOLUTELY NO WARRANTY; for details type `show w'.
This is free software, and you are welcome to redistribute it
under certain conditions; type `show c' for details.
\end{verbatim}

The hypothetical commands `show w' and `show c' should show the appropriate
parts of the General Public License.  Of course, the commands you use may be
called something other than `show w' and `show c'; they could even be
mouse-clicks or menu items--whatever suits your program.

You should also get your employer (if you work as a programmer) or your
school, if any, to sign a ``copyright disclaimer'' for the program, if
necessary.  Here is a sample; alter the names:

\begin{verbatim}
Yoyodyne, Inc., hereby disclaims all copyright interest in the program
`Gnomovision' (which makes passes at compilers) written by James Hacker.

<signature of Ty Coon>, 1 April 1989
Ty Coon, President of Vice
\end{verbatim}

This General Public License does not permit incorporating your program into
proprietary programs.  If your program is a subroutine library, you may
consider it more useful to permit linking proprietary applications with the
library.  If this is what you want to do, use the GNU Library General Public
License instead of this License.

\gplend
}
%    \end{macrocode}
%
% \end{macro}
%
% \begin{macro}{\implementation}
%
% The |\implementation| macro starts typesetting the implementation of
% the package(s).  If we're not doing the implementation, it just does
% this lot and ends the input file.
%
% I define a macro with arguments inside the |\StopEventually|, which causes
% problems, since the code gets put through an extra level of |\def|fing
% depending on whether the implementation stuff gets typeset or not.  I'll
% store the code I want to do in a separate macro.
%
%    \begin{macrocode}
\def\implementation{\StopEventually{\attheend}}
%    \end{macrocode}
%
% Now for the actual activity.  First, I'll do the |\finalstuff|.  Then, if
% \package{doc}'s managed to find the \package{multicol} package, I'll add
% the end of the environment to the end of each contents file in the list.
% Finally, I'll read the index in from its formatted |.ind| file.
%
%    \begin{macrocode}
\tdef\attheend{%
  \finalstuff%
  \ifhave@multicol%
    \def\do##1##2{\addtocontents{##1}{\protect\end{multicols}}}%
    \docontents%
  \fi%
  \PrintIndex%
}
%    \end{macrocode}
%
% \end{macro}
%
%
% \subsection{File version information}
%
% \begin{macro}{\mdwpkginfo}
%
% For setting up the automatic titles, I'll need to be able to work out
% file versions and things.  This macro will, given a file name, extract
% from \LaTeX\ the version information and format it into a sensible string.
%
% First of all, I'll put the original string (direct from the
% |\Provides|\dots\ command).  Then I'll pass it to another macro which can
% parse up the string into its various bits, along with the original
% filename.
%
%    \begin{macrocode}
\def\mdwpkginfo#1{%
  \edef\@tempa{\csname ver@#1\endcsname}%
  \expandafter\mdwpkginfo@i\@tempa\@@#1\@@%
}
%    \end{macrocode}
%
% Now for the real business.  I'll store the string I build in macros called
% \syntax{"\\"<filename>"date", "\\"<filename>"version" and
% "\\"<filename>"info"}, which store the file's date, version and
% `information string' respectively.  (Note that the file extension isn't
% included in the name.)
%
% This is mainly just tedious playing with |\expandafter|.  The date format
% is defined by a separate macro, which can be modified from the
% configuration file.
%
%    \begin{macrocode}
\def\mdwpkginfo@i#1/#2/#3 #4 #5\@@#6.#7\@@{%
  \expandafter\def\csname #6date\endcsname%
    {\protect\mdwdateformat{#1}{#2}{#3}}%
  \expandafter\def\csname #6version\endcsname{#4}%
  \expandafter\def\csname #6info\endcsname{#5}%
}
%    \end{macrocode}
%
% \end{macro}
%
% \begin{macro}{\mdwdateformat}
%
% Given three arguments, a year, a month and a date (all numeric), build a
% pretty date string.  This is fairly simple really.
%
%    \begin{macrocode}
\tdef\mdwdateformat#1#2#3{\number#3\ \monthname{#2}\ \number#1}
\def\monthname#1{%
  \ifcase#1\or%
     January\or February\or March\or April\or May\or June\or%
     July\or August\or September\or October\or November\or December%
  \fi%
}
\def\numsuffix#1{%
  \ifnum#1=1 st\else%
  \ifnum#1=2 nd\else%
  \ifnum#1=3 rd\else%
  \ifnum#1=21 st\else%
  \ifnum#1=22 nd\else%
  \ifnum#1=23 rd\else%
  \ifnum#1=31 st\else%
  th%
  \fi\fi\fi\fi\fi\fi\fi%
}
%    \end{macrocode}
%
% \end{macro}
%
% \begin{macro}{\mdwfileinfo}
%
% Saying \syntax{"\\mdwfileinfo{"<file-name>"}{"<info>"}"} extracts the
% wanted item of \<info> from the version information for file \<file-name>.
%
%    \begin{macrocode}
\def\mdwfileinfo#1#2{\mdwfileinfo@i{#2}#1.\@@}
\def\mdwfileinfo@i#1#2.#3\@@{\csname#2#1\endcsname}
%    \end{macrocode}
%
% \end{macro}
%
%
% \subsection{List handling}
%
% There are several other lists I need to build.  These macros will do
% the necessary stuff.
%
% \begin{macro}{\mdw@ifitem}
%
% The macro \syntax{"\\mdw@ifitem"<item>"\\in"<list>"{"<true-text>"}"^^A
% "{"<false-text>"}"} does \<true-text> if the \<item> matches any item in
% the \<list>; otherwise it does \<false-text>.
%
%    \begin{macrocode}
\def\mdw@ifitem#1\in#2{%
  \@tempswafalse%
  \def\@tempa{#1}%
  \def\do##1{\def\@tempb{##1}\ifx\@tempa\@tempb\@tempswatrue\fi}%
  #2%
  \if@tempswa\expandafter\@firstoftwo\else\expandafter\@secondoftwo\fi%
}
%    \end{macrocode}
%
% \end{macro}
%
% \begin{macro}{\mdw@append}
%
% Saying \syntax{"\\mdw@append"<item>"\\to"<list>} adds the given \<item>
% to the end of the given \<list>.
%
%    \begin{macrocode}
\def\mdw@append#1\to#2{%
  \toks@{\do{#1}}%
  \toks\tw@\expandafter{#2}%
  \edef#2{\the\toks\tw@\the\toks@}%
}
%    \end{macrocode}
%
% \end{macro}
%
% \begin{macro}{\mdw@prepend}
%
% Saying \syntax{"\\mdw@prepend"<item>"\\to"<list>} adds the \<item> to the
% beginning of the \<list>.
%
%    \begin{macrocode}
\def\mdw@prepend#1\to#2{%
  \toks@{\do{#1}}%
  \toks\tw@\expandafter{#2}%
  \edef#2{\the\toks@\the\toks\tw@}%
}
%    \end{macrocode}
%
% \end{macro}
%
% \begin{macro}{\mdw@add}
%
% Finally, saying \syntax{"\\mdw@add"<item>"\\to"<list>} adds the \<item>
% to the list only if it isn't there already.
%
%    \begin{macrocode}
\def\mdw@add#1\to#2{\mdw@ifitem#1\in#2{}{\mdw@append#1\to#2}}
%    \end{macrocode}
%
% \end{macro}
%
%
% \subsection{Described file handling}
%
% I'l maintain lists of packages, document classes, and other files
% described by the current documentation file.
%
% First of all, I'll declare the various list macros.
%
%    \begin{macrocode}
\def\dopackages{}
\def\doclasses{}
\def\dootherfiles{}
%    \end{macrocode}
%
% \begin{macro}{\describespackage}
%
% A document file can declare that it describes a package by saying
% \syntax{"\\describespackage{"<package-name>"}"}.  I add the package to
% my list, read the package into memory (so that the documentation can
% offer demonstrations of it) and read the version information.
%
%    \begin{macrocode}
\def\describespackage#1{%
  \mdw@ifitem#1\in\dopackages{}{%
    \mdw@append#1\to\dopackages%
    \usepackage{#1}%
    \mdwpkginfo{#1.sty}%
  }%
}
%    \end{macrocode}
%
% \end{macro}
%
% \begin{macro}{\describesclass}
%
% By saying \syntax{"\\describesclass{"<class-name>"}"}, a document file
% can declare that it describes a document class.  I'll assume that the
% document class is already loaded, because it's much too late to load
% it now.
%
%    \begin{macrocode}
\def\describesclass#1{\mdw@add#1\to\doclasses\mdwpkginfo{#1.cls}}
%    \end{macrocode}
%
% \end{macro}
%
% \begin{macro}{\describesfile}
%
% Finally, other `random' files, which don't have the status of real \LaTeX\
% packages or document classes, can be described by saying \syntax{^^A
% "\\describesfile{"<file-name>"}" or "\\describesfile*{"<file-name>"}"}.
% The difference is that the starred version will not |\input| the file.
%
%    \begin{macrocode}
\def\describesfile{%
  \@ifstar{\describesfile@i\@gobble}{\describesfile@i\input}%
}
\def\describesfile@i#1#2{%
  \mdw@ifitem#2\in\dootherfiles{}{%
    \mdw@add#2\to\dootherfiles%
    #1{#2}%
    \mdwpkginfo{#2}%
  }%
}
%    \end{macrocode}
%
% \end{macro}
%
%
% \subsection{Author and title handling}
%
% I'll redefine the |\author| and |\title| commands so that I get told
% whether I need to do it myself.
%
% \begin{macro}{\author}
%
% This is easy: I'll save the old meaning, and then redefine |\author| to
% do the old thing and redefine itself to then do nothing.
%
%    \begin{macrocode}
\let\mdw@author\author
\def\author{\let\author\@gobble\mdw@author}
%    \end{macrocode}
%
% \end{macro}
%
% \begin{macro}{\title}
%
% And oddly enough, I'll do exactly the same thing for the title, except
% that I'll also disable the |\mdw@buildtitle| command, which constructs
% the title automatically.
%
%    \begin{macrocode}
\let\mdw@title\title
\def\title{\let\title\@gobble\let\mdw@buildtitle\relax\mdw@title}
%    \end{macrocode}
%
% \end{macro}
%
% \begin{macro}{\date}
%
% This works in a very similar sort of way.
%
%    \begin{macrocode}
\def\date#1{\let\date\@gobble\def\today{#1}}
%    \end{macrocode}
%
% \end{macro}
%
% \begin{macro}{\datefrom}
%
% Saying \syntax{"\\datefrom{"<file-name>"}"} sets the document date from
% the given filename.
%
%    \begin{macrocode}
\def\datefrom#1{%
  \protected@edef\@tempa{\noexpand\date{\csname #1date\endcsname}}%
  \@tempa%
}
%    \end{macrocode}
%
% \end{macro}
%
% \begin{macro}{\docfile}
%
% Saying \syntax{"\\docfile{"<file-name>"}"} sets up the file name from which
% documentation will be read.
%
%    \begin{macrocode}
\def\docfile#1{%
  \def\@tempa##1.##2\@@{\def\@basefile{##1.##2}\def\@basename{##1}}%
  \edef\@tempb{\noexpand\@tempa#1\noexpand\@@}%
  \@tempb%
}
%    \end{macrocode}
%
% I'll set up a default value as well.
%
%    \begin{macrocode}
\docfile{\jobname.dtx}
%    \end{macrocode}
%
% \end{macro}
%
%
% \subsection{Building title strings}
%
% This is rather tricky.  For each list, I need to build a legible looking
% string.
%
% \begin{macro}{\mdw@addtotitle}
%
% By saying
%\syntax{"\\mdw@addtotitle{"<list>"}{"<command>"}{"<singular>"}{"<plural>"}"}
% I can add the contents of a list to the current title string in the
% |\mdw@title| macro.
%
%    \begin{macrocode}
\tdef\mdw@addtotitle#1#2#3#4{%
%    \end{macrocode}
%
% Now to get to work.  I need to keep one `lookahead' list item, and a count
% of the number of items read so far.  I'll keep the lookahead item in
% |\@nextitem| and the counter in |\count@|.
%
%    \begin{macrocode}
  \count@\z@%
%    \end{macrocode}
%
% Now I'll define what to do for each list item.  The |\protect| command is
% already set up appropriately for playing with |\edef| commands.
%
%    \begin{macrocode}
  \def\do##1{%
%    \end{macrocode}
%
% The first job is to add the previous item to the title string.  If this
% is the first item, though, I'll just add the appropriate \lit{The } or
% \lit{ and the } string to the title (this is stored in the |\@prefix|
% macro).
%
%    \begin{macrocode}
    \edef\mdw@title{%
      \mdw@title%
      \ifcase\count@\@prefix%
      \or\@nextitem%
      \else, \@nextitem%
      \fi%
    }%
%    \end{macrocode}
%
% That was rather easy.  Now I'll set up the |\@nextitem| macro for the
% next time around the loop.
%
%    \begin{macrocode}
    \edef\@nextitem{%
      \protect#2{##1}%
      \protect\footnote{%
        The \protect#2{##1} #3 is currently at version %
        \mdwfileinfo{##1}{version}, dated \mdwfileinfo{##1}{date}.%
      }\space%
    }%
%    \end{macrocode}
%
% Finally, I need to increment the counter.
%
%    \begin{macrocode}
    \advance\count@\@ne%
  }%
%    \end{macrocode}
%
% Now execute the list.
%
%    \begin{macrocode}
  #1%
%    \end{macrocode}
%
% I still have one item left over, unless the list was empty.  I'll add
% that now.
%
%    \begin{macrocode}
  \edef\mdw@title{%
    \mdw@title%
    \ifcase\count@%
    \or\@nextitem\space#3%
    \or\ and \@nextitem\space#4%
    \fi%
  }%
%    \end{macrocode}
%
% Finally, if $|\count@| \ne 0$, I must set |\@prefix| to \lit{ and the }.
%
%    \begin{macrocode}
  \ifnum\count@>\z@\def\@prefix{ and the }\fi%
}
%    \end{macrocode}
%
% \end{macro}
%
% \begin{macro}{\mdw@buildtitle}
%
% This macro will actually do the job of building the title string.
%
%    \begin{macrocode}
\tdef\mdw@buildtitle{%
%    \end{macrocode}
%
% First of all, I'll open a group to avoid polluting the namespace with
% my gubbins (although the code is now much tidier than it has been in
% earlier releases).
%
%    \begin{macrocode}
  \begingroup%
%    \end{macrocode}
%
% The title building stuff makes extensive use of |\edef|.  I'll set
% |\protect| appropriately.  (For those not in the know,
% |\@unexpandable@protect| expands to `|\noexpand\protect\noexpand|',
% which prevents expansion of the following macro, and inserts a |\protect|
% in front of it ready for the next |\edef|.)
%
%    \begin{macrocode}
  \let\@@protect\protect\let\protect\@unexpandable@protect%
%    \end{macrocode}
%
% Set up some simple macros ready for the main code.
%
%    \begin{macrocode}
  \def\mdw@title{}%
  \def\@prefix{The }%
%    \end{macrocode}
%
% Now build the title.  This is fun.
%
%    \begin{macrocode}
  \mdw@addtotitle\dopackages\package{package}{packages}%
  \mdw@addtotitle\doclasses\package{document class}{document classes}%
  \mdw@addtotitle\dootherfiles\texttt{file}{files}%
%    \end{macrocode}
%
% Now I want to end the group and set the title from my string.  The
% following hacking will do this.
%
%    \begin{macrocode}
  \edef\next{\endgroup\noexpand\title{\mdw@title}}%
  \next%
}
%    \end{macrocode}
%
% \end{macro}
%
%
% \subsection{Starting the main document}
%
% \begin{macro}{\mdwdoc}
%
% Once the document preamble has done all of its stuff, it calls the
% |\mdwdoc| command, which takes over and really starts the documentation
% going.
%
%    \begin{macrocode}
\def\mdwdoc{%
%    \end{macrocode}
%
% First, I'll construct the title string.
%
%    \begin{macrocode}
  \mdw@buildtitle%
  \author{Mark Wooding}%
%    \end{macrocode}
%
% Set up the date string based on the date of the package which shares
% the same name as the current file.
%
%    \begin{macrocode}
  \datefrom\@basename%
%    \end{macrocode}
%
% Set up verbatim characters after all the packages have started.
%
%    \begin{macrocode}
  \shortverb\|%
  \shortverb\"%
%    \end{macrocode}
%
% Start the document, and put the title in.
%
%    \begin{macrocode}
  \begin{document}
  \maketitle%
%    \end{macrocode}
%
% This is nasty.  It makes maths displays work properly in demo environments.
% \emph{The \LaTeX\ Companion} exhibits the bug which this hack fixes.  So
% ner.
%
%    \begin{macrocode}
  \abovedisplayskip\z@%
%    \end{macrocode}
%
% Now start the contents tables.  After starting each one, I'll make it
% be multicolumnar.
%
%    \begin{macrocode}
  \def\do##1##2{%
    ##2%
    \ifhave@multicol\addtocontents{##1}{%
      \protect\begin{multicols}{2}%
      \hbadness\@M%
    }\fi%
  }%
  \docontents%
%    \end{macrocode}
%
% Input the main file now.
%
%    \begin{macrocode}
  \DocInput{\@basefile}%
%    \end{macrocode}
%
% That's it.  I'm done.
%
%    \begin{macrocode}
  \end{document}
}
%    \end{macrocode}
%
% \end{macro}
%
%
% \subsection{And finally\dots}
%
% Right at the end I'll put a hook for the configuration file.
%
%    \begin{macrocode}
\ifx\mdwhook\@@undefined\else\expandafter\mdwhook\fi
%    \end{macrocode}
%
% That's all the code done now.  I'll change back to `user' mode, where
% all the magic control sequences aren't allowed any more.
%
%    \begin{macrocode}
\makeatother
%</mdwtools>
%    \end{macrocode}
%
% Oh, wait!  What if I want to typeset this documentation?  Aha.  I'll cope
% with that by comparing |\jobname| with my filename |mdwtools|.  However,
% there's some fun here, because |\jobname| contains category-12 letters,
% while my letters are category-11.  Time to play with |\string| in a messy
% way.
%
%    \begin{macrocode}
%<*driver>
\makeatletter
\edef\@tempa{\expandafter\@gobble\string\mdwtools}
\edef\@tempb{\jobname}
\ifx\@tempa\@tempb
  \describesfile*{mdwtools.tex}
  \docfile{mdwtools.tex}
  \makeatother
  \expandafter\mdwdoc
\fi
\makeatother
%</driver>
%    \end{macrocode}
%
% That's it.  Done!
%
% \hfill Mark Wooding, \today
%
% \Finale
%
\endinput

\describespackage{mdwtab}
\describespackage{mathenv}
\addcontents{lot}{\listoftables}
\mdwdoc
%</driver>
%
% \end{meta-comment}
%
%^^A-------------------------------------------------------------------------
% \renewcommand{\tabstyle}{\small}
%
% \section{User guide}
%
%
% The \package{mdwtab} package contains a reimplementation of the standard
% \LaTeX\ \env{tabular} and \env{array} environments.  This is not just an
% upgraded version: it's a complete rewrite.  It has several advantages over
% the official \package{array} package (not raw \LaTeX's, which is even less
% nice), and it's more-or-less compatible.  Most of these are rather
% technical, I'll admit.
%
% \begin{itemize}
%
% \item The newcolumn system is properly and perfectly integrated into the
%       system.  There are now \emph{no} `primitive' column types -- all the
%       standard types are created as user-defined columns.
%
% \item You can define entirely different table-like environments using the
%       equipment here.  It's still hard work, although less so than before.
%       I'll do an example of this some time.
%
% \item Construction of the preamble is generally much tidier.  I've used
%       token registers rather than |\edef|, and it's all done very nicely.
%
% \item Fine spacing before and after rules (described by DEK as `a mark of
%       quality') is now utterly trivial, since the preamble-generator will
%       store the appropriate information.
%
% \item You can use \env{array} in LR and paragraph modes without having
%       to surround it with `|$|' signs.
%
% \item Usually you don't want tables in the middle of paragraphs.  For these
%       cases, I've provided a simpler way to position the table
%       horizontally.
%
% \item Footnotes work properly inside \env{tabular} environments (hoorah!).
%       You can `catch' footnotes using the \env{minipage} environment if
%       you like.  (It uses an internal version of the \package{footnote}
%       package to handle footnotes, which doesn't provide extra goodies like
%       the \env{footnote} environment; you'll need to load the full package
%       explicitly to get them.)
%
% \item Standard \LaTeX\ tabular environments have a problem with lining up
%       ruled tables.  The |\firsthline| command given in the \textit{\LaTeX\
%       Companion} helps a bit, but it's not really good enough, and besides,
%       it doesn't \emph{actually} line the text up right after all.  The
%       \package{mdwtab} package does the job properly to begin with, so you
%       don't need to worry.
%
% \end{itemize}
%
% I've tested the following packages with \package{mdwtab}, and they all
% work.  Some of the contortions required to make them work weren't pleasant,
% but you don't need to know about them.  By a strange coincidence, all the
% packages were written by David Carlisle.  Anyway, here's the list:
% \begin{itemize}
% \item The quite nice \package{dcolumn} package.
% \item The more useful \package{delarray} package.
% \item The rather spiffy \package{hhline} package.
% \item The truly wonderful \package{tabularx} package.
% \item The utterly magnificent \package{longtable} package.
% \end{itemize}
%
% Note that I've looked at \package{supertabular} as well: it won't work, so
% use \package{longtable} instead, 'cos it's much better.
%
%
% \subsection{The downside}
%
% There's no such thing as a free lunch.  The \package{mdwtab} environment
% is not 100\% compatible with the \env{tabular} environment found in
% \LaTeXe\ or the \package{array} package.
%
% The differences between \package{mdwtab} and \LaTeXe's \env{tabular}
% environment are as follows:
%
% \begin{itemize} \synshorts \let\`=\lq
%
% \item The vertical spacing in \env{array} environments is different to
%       that in \env{tabular} environments.  This produces more attractive
%       results in most mathematical uses of \env{array}s, in the author's
%       opinion.  The spacing can be modified by playing with length
%       parameters.
%
% \item The presence of horizontal and vertical rules will alter the spacing
%       of the table (so a pair of columns separated by a `|' is wider than
%       a pair with no separation by "\\arrayrulewidth".  This does mean that
%       horizontal and vertical rules match up properly -- the usual \LaTeX\
%       environment makes the horizontal rules stop just short of the edge
%       of the table, making an ugly mess (check out the \textit{\LaTeX\
%       book} if you don't believe me -- page~62 provides a good example).
%       The \package{array} package handles rules in the same way as
%       \package{mdwtab}.
%
% \setbox0=\hbox{\footnotesize`\\def\\xcs{\\tabskip=\\fill}'}
% \setbox2=\hbox{\footnotesize`...@{\\span\\xcs}...'}
% \item In common with the \package{array} package, there are some
%       restrictions on the use of the "\\extracolsep" command in preambles:
%       you may use at most one "\\extracolsep" command in each `@' or `!'
%       expression.  Also, you can't say
%       \begin{listing}
%\newcommand{\xcs}{\extracolsep{\fill}}
%       \end{listing}
%       and then expect something like `...@{\\xcs}...' to actually work --
%       the "\\extracolsep" mustn't be hidden inside any other
%       commands.  Because things like `@' expressions aren't expanded at
%       the time, "\\extracolsep" has to be searched and processed
%       \`by hand'.\footnote{^^A
%         All \cs{extracolsep} does is modify the \cs{tabskip} glue, so
%         if you were an evil \TeX\ hacker like me, you could just say
%         \unhbox0\ and put \unhbox2\ in your preamble.  That'd work nicely.
%         It also works with the \package{array} package.}
%
% \item Control sequences (commands) in a table's preamble aren't expanded
%       before the preamble is read.  In fact, commands in the preamble are
%       considered to be column types, and their names are entirely
%       independent of normal \LaTeX\ commands.  No column types of this
%       nature have yet been defined\footnote{^^A
%         There used to be an internal \cs{@magic} type used by
%         \env{eqnarray}, but you're not supposed to know about that.
%         Besides, it's not there any more.}
%       but the possibility's always there.  Use the "\\newcolumntype" or
%       "\\coldef" commands to define new column types.
%
% \item The preamble parsing works in a completely different way.  There is
%       a certain amount of compatibility provided, although it's heavily
%       geared towards keeping \package{longtable} happy and probably won't
%       work with other packages.
%
% \item Obscure constructs which were allowed by the old preamble parser but
%       violate the syntax shown in the next section (e.g., `|@{}|' to
%       suppress the "\\doublerulesep" space between two vertical rules,
%       described in \textit{The \LaTeX\ Companion} as \`a misuse of the
%       `@{...}' qualifier') are now properly outlawed.  You will be given
%       an error message if you attempt to use such a construction.
%
% \item The `*' forms (which repeat column types) are now expanded at a
%       different time.  Previously, preambles like `c@*{4}{{:}@}{--}c'
%       were considered valid (the example would expand to
%       `c@{:}@{:}@{:}@{:}@{--}c'), because `*'s were expanded before the
%       preamble was actually parsed.  In the new system, `*' is treated
%       just like any other preamble character (it just has a rather odd
%       action), and preambles like this will result in an error (and
%       probably a rather confusing one).
%
% \end{itemize}
%
% There are also several incompatibilities between \package{mdwtab} and
% \package{array}:
%
% \begin{itemize} \synshorts \let\`=\lq
%
% \item Because of the way "\\newcolumntype" works in the \package{array}
%       package, a horrid construction like
%       \begin{listing}
%\newcolumntype{x}{{:}}
%\begin{tabular}{|c!xc|}
%       \end{listing}
%       is considered to be valid, and is interpreted as `|c!{:}c|'.  My
%       reading of pages~54 and~55 of the \textit{\LaTeX\ book} tells me
%       that this sort of thing is forbidden in normal \LaTeX\ commands.
%       The \package{mdwtab} preamble parser now treats column type letters
%       much more like commands with the result that the hacking above won't
%       work any more.  The construction above would actually be interpreted
%       as `|c!{x}c|' (i.e., the `x' column type wouldn't be expanded to
%       `{:}' because the parser noticed that it was the argument to the
%       `!' modifier\footnote{^^A
%         This is a direct result of the way \TeX\ treats undelimited
%         arguments.  See chapters~5 and~20 of \textit{The \TeX book} for
%         more information about how grouping affects argument reading.}).
%
% \item Most of the points above, particularly those relating to the
%       handling of the preamble, also apply to the \package{array} package.
%       it's not such an advance over the \LaTeXe\ version as everyone said
%       it was.
%
% \end{itemize}
%
%
% \subsection{Syntax}
%
% \DescribeEnv{tabular}
% \DescribeEnv{tabular*}
% \DescribeEnv{array}
% So that everyone knows where I stand, here's a complete syntax for my
% version of the \env{tabular} environment, and friends
%
% \begin{grammar}
%
% <tabular-env>	::= \[[
%   "\\begin"
%   \begin{stack}
%     "{tabular}" \\ "{tabular*}" "{" <length> "}" \\
%     "{array}" \\ "{smarray}"
%   \end{stack}
%   \[ "[" <position-arg> "]" \]
%   "{" <preamble> "}" <text>
%   "\\end"
%   \( "{tabular}" \\ "{tabular*}" \\ "{array}" \\ "{smarray}" \)
% \]]
%
% <position-arg> ::= (see below)
%
% <preamble>	::= \[[
%   <first-column>
%   \[ \< <column> \> \]
% \]]
%
% <first-column> ::= \[[ \[ <rule> \] <column> \]]
%
% <column>	::= \[[
%   \[ <spacing> \] \[ \< <user-pre-text> \> \] <column-type>
%   \[ \< <user-post-text> \> \] \[ <spacing> \] \[ <rule> \]
% \]]
%
% <spacing>	::= \[[ "@" "{" <text> "}" \]]
%
% <user-pre-text> ::= \[[ ">" "{" <text> "}" \]]
%
% <column-type>	::= \[[
%   \begin{stack}
%     \[ "T" \\ "M" \] \( "l" \\ "c" \\ "r" \) \\
%     \( "p" \\ "m" \\ "b" \) "{" <length> "}" \\
%     "#" "{" <raw-pre-text> "}" "{" <raw-post-text> "}"
%   \end{stack}
% \]]
%
% <user-post-text> ::= \[[ "<" "{" <text> "}" \]]
%
% <rule>	::= \[[ \( "|" \\ "!" "{" <text> "}" \) \]]
%
% \end{grammar}
%
% If you examine the above very carefully, you'll notice a slight deviation
% from the original -- an |@|-expression \emph{following} a rule is
% considered to be part of the \emph{next} column, not the current one.  This
% is, I think, an almost insignificant change, and essential for some of the
% new features.  You'll also notice the new |#| column type form, which
% allows you to define new real column types instead of just modifying
% existing ones.  It's not intended for direct use in preambles -- it's
% there mainly for the benefit of people who know what they're doing and
% insist on using |\newcolumntype| anyway.
%%
% The actual column types are shown in table~\ref{tbl:columns}.
%
% \begin{table}
% \begin{tabular}[C]{| >{\synshorts} c | m{3in} |}	   \hlx{hv[1]}
%
% \multicolumn{2}{|c|}{\bf Column types}		\\ \hlx{v[1]hv}
% \bf Name	& \bf Meaning				\\ \hlx{vhv.}
% "l"		& Left aligned text (\env{tabular}) or
%		  equation (\env{array}).		\\ \hlx{.}
% "c"		& Centred text (\env{tabular}) or
%		  equation (\env{array}).		\\ \hlx{.}
% "r"		& Right aligned text (\env{tabular}) or
%		  equation (\env{array}).		\\ \hlx{vhv.}
% "Ml", "Mc" and "Mr" & Left, centre and right aligned
%		  equations.*				\\ \hlx{.}
% "Tl", "Tc" and "Tr" & Left, centre and right aligned
%		  text.*				\\ \hlx{vhv.}
% "p{"<width>"}" & Top aligned paragraph with the given
%		  width.				\\ \hlx{.}
% "m{"<width>"}" & Vertically centred paragraph with
%		  the given width.			\\ \hlx{.}
% "b{"<width>"}" & Bottom aligned paragraph with the
%		  given width.				\\ \hlx{vhv.}
% "#{"<pre>"}{"<post>"}" & User defined column type:
%		  \<pre> is inserted before the
%		  cell entry, \<post> is inserted
%		  afterwards.*				\\ \hlx{vhhv[1]}
%
% \multicolumn{2}{|c|}{\bf Other modifier characters}	\\ \hlx{v[1]hv}
% \bf Name	& \bf Meaning				\\ \hlx{vhv.}
% "|"		& Inserts a vertical rule between
%		  columns.				\\ \hlx{.}
% "!{"<text>"}"	& Inserts \<text> between columns,
%		  treating it as a vertical rule.	\\ \hlx{vhv.}
% "@{"<text>"}"	& Inserts \<text> instead of the
%		  usual intercolumn space.		\\ \hlx{vhv.}
% ">{"<text>"}"	& Inserts \<text> just before the
%		  actual column entry.			\\ \hlx{.}
% "<{"<text>"}"	& Inserts \<text> just after the
%		  actual column entry.			\\ \hlx{vhv.}
% "*{"<count>"}{"<chars>"}" & Inserts \<count>
%		  copies of the \<chars> into the
%		  preamble.				\\ \hlx{vhs}
%
% \multicolumn{2}{@{}l}{* This column type is a new feature}
% \end{tabular}
%
% \caption{\package{array} and \package{tabular} column types and modifiers}
% \label{tbl:columns}
% \end{table}
%
% Now that's sorted everything out, there shouldn't be any arguments at all
% about what a column means.
%
% The lowercase \<position-arg>s \lit{t}, \lit{c} and \lit{b} do exactly
% what they did before: control the vertical positioning of the table.  The
% uppercase ones control the \emph{horizontal} positioning -- this is how you
% create \emph{unboxed} tables.  You can only create unboxed tables in
% paragraph mode.
%
% Note that unboxed tables still can't be broken across pages.  Use
% the \package{longtable} package for this, because it already does an
% excellent job.
%
% \DescribeMacro{\tabpause}
% One thing you can to with unboxed tables, however, is to `interrupt' them,
% do some normal typesetting, and then continue.  This is achieved by the
% |\tabpause| command: its argument is written out in paragraph mode, and
% the table is continued after the argument finishes.
% Note that it isn't a real argument as far as commands like |\verb| are
% concerned -- they'll work inside |\tabpause| without any problems.
%
% \DescribeMacro{\vline}
% The |\vline| command draws a vertical rule the height of the current table
% cell (unless the current cell is being typeset in paragraph mode -- it
% only works in the simple LR-mode table cells, or in \lit{@} or \lit{!}
% modifiers).  It's now been given an optional argument which gives the
% width of the rule to draw:
%
% { \let\tabstyle=\relax
% \begin{demo}{An example of \cmd\vline}
%\large
%\begin{tabular}
%  {| c !{\vline[2pt]} c | c |}
%  \hlx{hv}
%  \bf A & \it B & \sf C \\
%  \hlx{vhv}
%  \bf D & \it E & \sf F \\
%  \hlx{vh}
%\end{tabular}
% \end{demo}
% }
%
% \DescribeMacro{smarray}
% You've probably noticed that there's an unfamiliar environment mentioned
% in the syntax shown above.  The \env{smarray} environment produces a
% `small' array, with script size cells rather than the normal full text
% size cells.  I've seen examples of this sort of construction\footnote{^^A
%   There's a nasty use of \env{smallmatrix} in the |testmath.tex| file which
%   comes with the \package{amslatex} distribution.  It's actually there to
%   simulate a `smallcases' environment, which the \package{mathenv} package
%   includes, based around \env{smarray}.}
% being implemented by totally unsuitable commands.  Someone may find it
% handy.
%
%
% \subsection{An updated \cs{cline} command}
%
% \DescribeMacro{\cline}
% The standard \LaTeX\ |\cline| command has been updated.  As well as just
% passing a range of columns to draw lines through, you can now pass a comma
% separated list of column numbers and ranges:
%
% \begin{grammar}
% <cline-cmd>	::= \[[
%   "\\cline" "{" \< <number> \[ "-" <number> \] \\ "," \> "}"
% \]]
% \end{grammar}
%
% The positioning of the horizontal lines has also been improved a bit, so
% that they meet up with the vertical lines properly.  Displays like the one
% in the example below don't look good unless this has been done properly.
%
% {\let\tabstyle\relax
% \begin{demo}[w]{A \cs{cline} example}
%\newcommand{\mc}{\multicolumn{1}}
%\begin{tabular}[C]{|c|c|c|c|}            \cline{2,4}
%  \mc{c|}{one} & two & three & four   \\ \hline
%  five & six & seven & \mc{c}{eight}  \\ \cline{1,3}
%\end{tabular}
% \end{demo}
% }
%
% \subsection{Spacing control}
%
% One of the most irritating things about \LaTeX's tables is that there isn't
% enough space around horizontal rules.  Donald Knuth, in \textit{The
% \TeX book}, describes addition of some extra vertical space here as `a mark
% of quality', and since \TeX\ was designed to produce `beautiful documents'
% it seems a shame that \LaTeX\ doesn't allow this to be done nicely.  Well,
% it does now.
%
% \DescribeMacro{\vgap}
% The extra vertical space is added using a command |\vgap|, with the
% following syntax:
%
% \begin{grammar}
%
% <vgap-cmd>	::= \[[
%   "\\vgap" \[ "[" <which-cols> "]" \] "{" <length> "}"
% \]]
%
% <which-cols>	::= \[[ \< <number> \[ "-" <number> \] \\ "," \> \]]
%
% \end{grammar}
%
% This command must appear either immediately after the beginning of the
% table or immediately after the |\\| which ends a row.  (Actually, there are
% other commands which also have this requirement -- you can specify a
% collection of them wherever you're allowed to give any one.)  It adds some
% vertical space (the amount is given by the \<length>) to the table,
% making sure that the vertical rules of the table are extended correctly.
%
% The |\vgap| command relies on information stored while your table preamble
% is being examined.  However, it's possible that you might not want some
% of the rules drawn (e.g., if you've used |\multicolumn|).  The optional
% \<which-cols> argument allows you to specify which rules are \emph{not}
% to be drawn.  You can specify either single column numbers or ranges.  The
% rule at the very left hand side is given the number~0; the rules at the
% end of column~$n$ are numbered~$n$.  It's easy really.
%
% \DescribeMacro{\hlx}
% Using |\vgap| is all very well, but it's a bit cumbersome, and takes up a
% lot of typing, especially when combined with |\hline| commands.  The |\hlx|
% command tries to tidy things.
%
% The syntax is simple:
% \begin{grammar}
%
% <hlx-cmd>	::= \[[
%   "\\hlx" "{"
%   \begin{rep}
%     \begin{stack}
%       "h" \\
%       \tok{"v["<which-cols>"]["<length>"]"} \\
%       \tok{"s["<length>"]"} \\
%       \tok{"c{"<which-cols>"}"} \\
%       "b" \\
%       \tok{"/["<number>"]"} \\
%       "."
%     \end{stack}
%   \end{rep}
%   "}"
% \]]
%
% \end{grammar}
% The argument works a bit like a table preamble, really.  Each letter is a
% command.  The following are supported:
%
% \begin{description}
%
% \item [\lit*{h}] Works just like |\hline|.  If you put two adjacent to each
%       other, a gap will be put between them.
%
% \item [\lit*{v[}\<which-cols>\lit*{][}\<length>\lit*{]}]  Works
%       like \syntax{"\\vgap["<which-cols>"]{"<length>"}"}.  If the
%       \<length> is omitted, the value of |\doublerulesep| is used.
%       This usually looks right.
%
% \item [\lit*{s[}\<length>\lit*{]}]  Leaves a vertical gap with the
%       given size.  If you omit the \<length> then |\doublerulesep| is
%       used.  This is usually right.
%
% \item [\lit*{c\char`\{}\<which-cols>\lit*{\char`\}}]  Works just like
%       |\cline|.
%
% \item [\lit*{b}] Inserts a backspace the width of a rule.  This is useful
%       when doing \package{longtable}s.
%
% \item [\lit*{/[}\<number>\lit*{]}]  Allows a page break in a table.  Don't
%       use this except in a \env{longtable} environment.  The \<number>
%       works exactly the same as it does in the |\pagebreak| command,
%       except that the default is 0, which just permits a break without
%       forcing it.
%
% \item [\lit*{.}]  (That's a dot)  Starts the next row of the table.  No
%       more characters may follow the dot, and no |\hline|, |\hlx|, |\vgap|
%       or |\multicolumn| commands may be used after it.  You don't have to
%       include it, and most of the time it's totally useless.  It can be
%       handy for some macros, though.  I used it in (and in fact added it
%       especially for) the table of column types.
%
% \end{description}
%
% An example of the use of |\hlx| is given, so you can see what's going on.
%
% \begin{figure}
% \let\tabstyle\relax
% \begin{demo}[w]{Beautiful table example}
%\newcommand{\zerowidth}[1]{\hbox to 0pt{\hss#1\hss}}
%\setlength{\tabcolsep}{1.5em}
%\begin{tabular}[C]{| r | c | r |}                  \hlx{hv[1,2]}
%  \multicolumn{3}{|c|}{\bf AT\&T Common Stock}  \\ \hlx{v[1,2]hv}
%  \multicolumn{1}{|c|}{\zerowidth{\bf Year}} &
%  \multicolumn{1}{c|}{\zerowidth{\bf Price}} &
%  \multicolumn{1}{c|}{\zerowidth{\bf Dividend}} \\ \hlx{vhv}
%  1971 & 41--54 & \$2.60                        \\
%     2 & 41--54 &   2.70                        \\
%     3 & 46--55 &   2.87                        \\
%     4 & 40--53 &   3.24                        \\
%     5 & 45--52 &   3.40                        \\
%     6 & 51--59 &    .95\rlap{*}                \\ \hlx{vhs}
%  \multicolumn{3}{@{}l}{* (first quarter only)}
%\end{tabular}
% \end{demo}
% \end{figure}
%
%
% \subsection{Creating beautiful long tables}
%
% You can use the |\vgap| and |\hlx| commands with David Carlisle's
% stunning \package{longtable} package.  However, there are some things you
% should be away of to ensure that your tables always come out looking
% lovely.
%
% The \package{longtable} package will break a table at an |\hline| command,
% leaving a rule at the bottom of the page and another at the top of the
% next page.  This means that a constructions like |\hlx{vhv}| will be
% broken into something like |\hlx{vh}| at the bottom of the page and
% |\hlx{hv}| at the top of the next.  You need to design the table headers
% and footers with this in mind.
%
% However, there appears to be a slight problem:\footnote
%   {You might very well call it a bug.  I couldn't possibly comment.}
% if the footer starts with an |\hline|, and a page is broken at an |\hline|,
% then you get an extra thick rule at the bottom of the page.  This is a bit
% of a problem, because if the rule isn't there in the footer and you get
% a break between two rows \emph{without} a rule between them, then the page
% looks very odd.  
%
% If you want to do ruled longtables, I'd recommend that you proceed as
% follows:
% \begin{itemize}
% \item End header sections with an |\hlx{vh}|.
% \item Begin footer sections with an |\hlx{bh}|.
% \item Begin the main table with |\hlx{v}|.
% \item Insert |\hlx{vhv}| commands in the main table body as usual.
% \end{itemize}
% If \package{longtable} gets modified appropriately, the use of the \lit{b}
% command won't be necessary.
%
% Here's an example of the sort of thing you'd type.
%
% \begin{listinglist} \listingsize
% \verb"\begin{longtable}[c]{|c|l|}           \hlx{hv}" \\
% \verb"\bf Heading & \bf Also heading     \\ \hlx{vh}" \\
% \verb"\endhead" \\
% \verb"\hlx{bh}" \\
% \verb"\endfoot" \\
% \verb"\hlx{v}" \\
% \verb"First main & table line            \\ \hlx{vhv}" \\
% \verb"Lots of text & like this           \\ \hlx{vhv}" \\
% \null\quad\vdots \\
% \verb"Lots of text & like this           \\ \hlx{vhv}" \\
% \verb"Last main & table line             \\ \hlx{vh}" \\
% \verb"\end{longtable}"
% \end{listinglist}
%
%
% \subsection{Rules and vertical positioning}
%
% In the \LaTeXe\ and \package{array.sty} versions of \env{tabular}, you run
% into problems if you try to use ruled tables together with the \lit{[t]} or
% \lit{[b]} position specifiers -- the top or bottom rule ends up being
% nicely lined up with the text baseline, giving you an effect which is
% nothing like the one you expected.  The \textit{\LaTeX\ Companion} gives
% two commands |\firsthline| and |\lasthline| which are supposed to help with
% this problem.  (These commands have since migrated into the \package{array}
% package.)  Unfortunately, |\firsthline| doesn't do its job properly --
% it gets the text position wrong by exactly the width of the table rules.
%
% The \package{mdwtab} package makes all of this automatic.  It gets the
% baseline positions exactly right, whether or not you use rules.  Earlier
% versions of this package required that you play with a length parameter
% called |\rulefudge|; this is no longer necessary (or even possible -- the
% length parameter no longer exists).  The package now correctly compensates
% for all sorts of rules and |\vgap|s at the top and bottom of a table and
% it gets the positioning right all by itself.  You've never had it so good.
%
%
% \subsection{User serviceable parts}
%
% There are a lot of parameters which you can modify in order to make arrays
% and tables look nicer.  They are all listed in table~\ref{tbl:config}.
%
% \begin{table}
% \begin{tabular}[C]{| l | m{3in} |}				   \hlx{hv}
% \bf Parameter		& \bf Meaning				\\ \hlx{vhv}
% |\tabstyle|		& A command executed at the beginning of
%			  a \env{tabular} or \env{tabular$*$}
%			  environment.  By default does nothing.
%			  Change using |\renewcommand|.		\\ \hlx{vhv}
% |\extrarowheight|	& A length added to the height of every
%			  row, used to stop table rules
%			  overprinting ascenders.  Default 0\,pt.
%			  Usage is deprecated now: use |\hlx|
%			  instead.				\\ \hlx{vhv}
% |\tabextrasep|	& Extra space added between rows in a
%			  \env{tabular} or \env{tabular$*$}
%			  environment (added \emph{before} any
%			  following |\hline|).  Default 0\,pt.	\\
% |\arrayextrasep|	& Analogous to |\tabextrasep|, but for
%			  \env{array} environments.  Default
%			  1\,jot (3\,pt).			\\
% |\smarrayextrasep|	& Analogous to |\tabextrasep|, but for
%			  \env{smarray} environments.  Default
%			  1\,pt.				\\ \hlx{vhv}
% |\tabcolsep|		& Space added by default on each side of
%			  a table cell (unless suppressed by an
%			  \lit{@}-expression) in \env{tabular}
%			  environments.  Default is defined by
%			  your document class.			\\
% |\arraycolsep|	& Analogous to |\tabcolsep|, but for
%			  \env{array} environments.  Default is
%			  defined by your document class.	\\
% |\smarraycolsep|	& Analogous to |\tabcolsep|, but for
%			  \env{smarray} environments.  Default
%			  is 3\,pt.				\\ \hlx{vhv}
% |\arrayrulewidth|	& The width of horizontal and vertical
%			  rules in tables.			\\
% |\doublerulesep|	& Space added between two adjacent
%			  vertical or horizontal rules.  Also
%			  used by |\hlx{v}|.			\\ \hlx{vhv}
% |\arraystretch|	& Command containing a factor to
%			  multiply the default row height.
%			  Default is defined by your document
%			  class (usually 1).			\\ \hlx{vh}
% \end{tabular}
%
% \caption{Parameters for configuring table environments}
% \label{tbl:config}
%
% \end{table}
%
%
% \subsection{Defining column types}
%
% \DescribeMacro{\newcolumntype}
% The easy way to define new column types is using |\newcolumntype|.  It
% works in more or less the same way as |\newcommand|:
%
% \begin{grammar}
%
% <new-col-type-cmd> ::= \[[
%   "\\newcolumntype"
%   "{" <column-name> "}"
%   \[ "[" <num-args> "]" \]
%   \[ "[" <default-arg> "]" \]
%   "{" <first-column> \[ \< <column> \> \] "}"
% \]]
%
% \end{grammar}
%
% (The \env{array.sty} implementation doesn't accept the \<default-arg>
% argument.  I've no idea why not, 'cos it was very easy to implement.)
%
% \DescribeMacro{\colset}
% This implementation allows you to define lots of different sets of columns.
% You can change the current set using the |\colset| declaration:
% \begin{grammar}
% <colset-cmd>	::= \[[ "\\colset" "{" <set-name> "}" \]]
% \end{grammar}
% This leaves a problem, though: at any particular moment, the current
% column set could be anything, since other macros and packages can change
% it.
%
% \DescribeMacro{\colpush}
% \DescribeMacro{\colpop}
% What actually happens is that a stack of column sets is maintained.  The
% |\colset| command just replaces the item at the top of the stack.  The
% command |\colpush| pushes its argument onto the top of the stack, making
% it the new current set.  The corresponding |\colpop| macro (which doesn't
% take any arguments) removes the top item from the stack, reinstating the
% previous current column set.
%
% \begin{grammar}
% <colpush-cmd>	::= \[[ "\\colpush" "{" <set-name> "}" \]]
% <colpop-cmd>	::= \[[ "\\colpop" \]]
% \end{grammar}
%
% The macros which manipulate the column set stack work \emph{locally}.
% The contents of the stack are saved when you open a new group.
%
% To make sure everyone behaves themselves properly, these are the rules for
% using the column set stack:
%
% \begin{itemize}
%
% \item Packages defining column types must ensure that they preserve the
%       current column set.  Either they must push their own column type
%       and pop it off when they're finished defining columns, or they must
%       avoid changing the stack at all, and use the optional arguments to
%       |\coldef| and |\collet|.
%
% \item Packages must not assume that any particular column set is current
%       unless they have made sure of it themselves.
%
% \item Packages must ensure that they pop exactly as much as they push.
%       There isn't much policing of this (perhaps there should be more),
%       so authors are encouraged to behave responsibly.
%
% \item Packages must change the current column set (using |\colset|) when
%       they start up their table environment.  This will be restored when
%       the environment closes.
%
% \end{itemize}
%
% \DescribeMacro{\coldef}
% |\newcolumntype| is probably enough for most purposes.  However, Real
% \TeX nicians, and people writing new table-generating environments, require
% something lower-level.
%
% \begin{grammar}
% <coldef-cmd>	::= \[[
%   "\\coldef"
%     \[ "[" <set-name> "]" \]
%     <col-name> <arg-template> "{" <replacement-text> "}"
% \]]
% \end{grammar}
%
% Note that this defines a column type in the current colset.  It works
% almost exactly the same way as \TeX's primitive |\def|.  There is a
% potential gotcha here: a |\tab@mkpream| token is inserted at the end of
% your replacement text.  If you need to read an optional argument or
% something, you'll need to gobble this token before you carry on.  The
% |\@firstoftwo| macro could be handy here:
% \begin{listing}
%\coldef x{\@firstoftwo{\@ifnextchar[\@xcolumn@i\@xcolumn@ii}}}
% \end{listing}
% This isn't a terribly pretty state of affairs, and I ought to do something
% about it.  I've not seen any use for an optional argument yet, though.
% Note that if you do gobble the |\tab@mkpream|, it's your responsibility to
% insert another one at the very end of your macro's expansion (so that
% further preamble characters can be read).
% 
% The replacement text is inserted directly.  It's normal to insert preamble
% elements here.  There are several to choose from:
%
% \begin{description}
%
% \item [Column items] provide the main `meat' of a column.  You insert a
%       column element by saying
%       \syntax{"\\tabcoltype{"<pre-text>"}{"<post-text>"}"}.
%       The user's text gets inserted between these two.  (So do user pre-
%       and post-texts.  Bear this in mind.)
%
% \item [User pre-text items] work like the \lit{>} preamble command.  You
%       use the \syntax{"\\tabuserpretype{"<text>"}"} command to insert it.
%       User pre-texts are written in \emph{reverse} order between the
%       pre-text of the column item and the text from the table cell.
%
% \item [User post-text items] work like the \lit{<} preamble command.  You
%       use the \syntax{"\\tabuserposttype{"<text>"}"} command to insert it.
%       Like user pre-texts, user post-texts are written in reverse order,
%       between the table cell text and the column item post-text.
%
% \item [Space items] work like the \lit{@} preamble command.  They're
%       inserted with the \syntax{"\\tabspctype{"<text>"}"} command.
%
% \item [Rule items] work like the `\verb"|"' and \lit{!} commands.  You
%       insert them with the \syntax{"\\tabruletype{"<text>"}"} command.
%       Note that the text is inserted by |\vgap| too, so it should contain
%       things which adjust their vertical size nicely.  If you really need
%       to, you can test |\iftab@vgap| to see if you're in a |\vgap|.
%
% \end{description}
%
% \DescribeMacro{\collet}
% As well as defining columns, you can copy definitions (rather like |\let|
% allows you to copy macros).  The syntax is like this:
%
% \begin{grammar}
%
% <collet-cmd>	::= \[[
%   \[ "[" <set-name> "]" \] <col-name> \[ "=" \] \[ "[" <set-name> "]" \]
%   <col-name>
% \]]
%
% \end{grammar}
%
% (In other words, you can copy defintions from other column sets.)
%
%
% \subsection{Defining new table-generating environments}
%
% Quite a few routines are provided specifically to help you to define new
% environments which do alignment in a nice way.
%
% \subsubsection{Reading preambles}
%
% The main tricky bit in doing table-like environments is parsing preambles.
% No longer.
%
% \DescribeMacro{\tab@readpreamble}
% \DescribeMacro{\tab@doreadpream}
% The main parser routine is called |\tab@doreadpream|.  Given a user
% preamble string as an argument, it will build an |\halign| preamble to
% return to you.  However, the preamble produced won't be complete.  This is
% because you can actually make multiple calls to |\tab@doreadpream| with
% bits of user preambles.  The |\newcolumntype| system uses this mechanism,
% as does the \lit{*} (repeating) modifier.  When there really is no more
% preamble to read, you need to \emph{commit} the heldover tokens to the
% output.  The |\tab@readpreamble| routine will do this for you -- given a
% user preamble, it builds a complete output from it.
%
% A token register |\tab@preamble| is used to store the generated preamble.
% Before starting, you must iniitialise this token list to whatever you want.
% There's another token register, |\tab@shortline|, which is used to store
% tokens used by |\vgap|.  For each column in the table, the list contains
% an |\omit| (to override the standard preamble) and an |\hfil| space taking
% up most of the column.  Finally, for each rule item in the user preamble,
% the shortline list contains an entry of the form:
% \begin{quote} \synshorts
%   "\\tab@ckr{"<column-number>"}{"<rule-text>"}"
% \end{quote}
% This is used to decide whether to print the rule or an empty thing of the
% same width.  You probably ought to know that the very first column does
% \emph{not} have a leading |\omit| -- this is supplied by |\vgap| so that
% it can then look for optional arguments.
%
% \DescribeMacro{\tab@initread}
% As well as initialising |\tab@preamble| and emptying |\tab@shortline|,
% there are several other operations required to initialise a preamble read.
% These are all performed by the |\tab@initread| macro, although you may want
% to change some of the values for your specific application.  For reference,
% the actions performed are:
% \begin{itemize}
% \item initialising the parser state by setting $|\tab@state| =
%       |\tab@startstate|$;
% \item clearing the token lists |\tab@preamble| and |\tab@shortlist|;
% \item initialising the macros |\tab@tabtext|, |\tab@midtext|, and
%       |\tab@multicol| to their default values of `|&|',
%       `|\ignorespaces#\unskip|' and the empty token list respectively.^^A
%       \footnote{^^A
%         These are macros rather than token lists to avoid hogging all
%         the token list registers.  Actually, the package only allocates
%         two, although it does use almost all of the temporary registers as
%         well.  Also, there's a lie: \cs{unskip} is too hamfisted to remove
%         trailing spaces properly;  I really use a macro called
%         \cs{@maybe@unskip}}
% \item clearing the internal token list registers |\tab@pretext|,
%       |tab@userpretext| and |\tab@posttext|;
% \item clearing the column counter |\tab@columns| to zero;
% \item clearing the action performed when a new column is started (by making
%       the |\tab@looped| macro equal to |\relax|; this is used to make
%       |\multicolumn| macro raise an error if you try to do more than one
%       column); and
% \item setting up some other switches used by the parser (|\iftab@rule|,
%       |\iftab@initrule| and |\iftab@firstcol|, all of which are set to be
%       |true|).
% \end{itemize}
%
% The macro |\tab@multicol| is used by the |\multicolumn| command to insert
% any necessary items (e.g., struts) before the actual column text.  If you
% set this to something non-empty, you should probably consider adding a
% call to the macro to the beginning of |\tab@preamble|.
%
% When parsing is finally done, the count register |\tab@columns| contains
% the number of columns in the alignment.  Don't corrupt this value, because
% it's used for handling |\hline| commands.
%
% \subsubsection{Starting new lines}
%
% The other messy bit required by table environments is the newline command
% |\\|.  There are nasty complications involved with starting new lines, some
% of which can be handled by this package, and some on which I can only give
% advice.
%
% \DescribeMacro{\tab@cr}
% The optional arguments and star-forms etc. can be read fairly painlessly
% using the |\tab@cr| command:
%
% \begin{grammar}
% <tabcr-cmd>	::= \[[
%   "\\tab@cr" <command> "{" <non-star-text> "}" "{" <star-text> "}"
% \]]
% \end{grammar}
%
% This will call your \<command> with two arguments.  The first is the
% contents of the optional argument, or `|\z@|' if there wasn't one.  The
% second is either \<star-text> or \<non-star-text> depending on
% whether the user wrote the $*$-form or not.
%
% Somewhere in your \<command>, you'll have to use the |\cr| primitive to
% end the table row.  After you've done this, you \emph{must} ensure that you
% don't do anything that gets past \TeX's mouth without protecting it --
% otherwise |\hline| and co.\ won't work.  I usually wrap things up in a
% |\noalign| to protect them, although there are other methods.  Maybe.
%
% You might like to have a look at the \env{eqnarray} implementation provided
% to see how all this gets put into practice.
%
%
% \subsection{The \env{mathenv} package alignment environments}
%
% The \env{mathenv} package provides several environments for aligning
% equations in various ways.  They're mainly provided as a demonstration of
% the table handling macros in \package{mdwtab}, so don't expect great
% things.  If you want truly beautiful mathematics, use
% \package{amsmath}.\footnote{^^A
%   Particularly since nice commands like \cmd\over\ are being reactivated
%   in a later release of \package{amsmath}.}
% However, the various environments do nest in an approximately useful way.
% I also think that the \env{matrix} and \env{script} environments provided
% here give better results than their \package{amsmath} equivalents, and
% they are certainly more versatile.
%
% \subsubsection{The new \env{eqnarray} environment}
%
% \DescribeEnv{eqnarray}
% \DescribeEnv{eqnarray*}
% As an example of the new column defining features, and because the original
% isn't terribly good, I've included a rewritten version of the
% \env{eqnarray} environment.  The new implementation closes the gap between
% \env{eqnarray} and \AmSTeX\ alignment features.  It's in a separate,
% package called \package{mathenv}, to avoid wasting your memory.
%
% \begin{grammar}
%
% <eqnarray-env> ::= \[[
%   <begin-eqnarray> \< <row> \\ "\\\\" \> <end-eqnarray>
% \]]
%
% <begin-eqnarray> ::= \[[
%   "\\begin" \( "{eqnarray}" \\ "{eqnarray*}" \)
%   \[ "[" \< <eqa-column> \> "]" \]
% \]]
%
% <eqa-column>	::= \[[
%   \[ "q" \\ ":" \]
%   \[ \< ">" "{" <pre-text> "}" \> \]
%   \begin{stack}
%     \[ "T" \] \( "r" \\ "c" \\ "l" \) \\
%     "L" \\
%     "x"
%   \end{stack}
%   \[ \< "<" "{" <post-text> "}" \> \]
% \]]
%
% <end-eqnarray> ::= \[[
%   "\\end" \begin{stack} "{eqnarray}" \\ "{eqnarray*}" \end{stack}
% \]]
%
% \end{grammar}
%
% Descriptions of the various column types are given in
% table~\ref{tbl:eqnarray}.
%
% \begin{table}
% \begin{tabular}[C]{| >{\synshorts} c | m{3in} |}	   \hlx{hv[1]}
%
% \multicolumn{2}{|c|}{\bf Column types}		\\ \hlx{v[1]hv}
% \bf Name	& \bf Meaning				\\ \hlx{vhv.}
% "l"		& Left aligned piece of equation.	\\ \hlx{.}
% "c"		& Centred piece of equation.		\\ \hlx{.}
% "x"		& Centred or flush-left whole equation
%		  (depending on \textsf{fleqn} option).	\\ \hlx{.}
% "r"		& Right aligned piece of equation.	\\ \hlx{vhv.}
% "L"		& Left aligned piece of equation whose
%		  width is considered to be 2\,em.	\\ \hlx{vhv.}
% "Tl", "Tc" and "Tr" & Left, centre and right aligned
%		  text.					\\ \hlx{vhhv[1]}
%
% \multicolumn{2}{|c|}{\bf Other modifier characters}	\\ \hlx{v[1]hv}
% \bf Name	& \bf Meaning				\\ \hlx{vhv.}
% ":"		& Leaves a big gap between equations.
%		  By default, the `chunks' separated by
%		  \lit{:}s are equally spaced on the
%		  line.					\\ \hlx{.}
% "q"		& Inserts 1\,em of space		\\ \hlx{vhv.}
% ">{"<text>"}"	& Inserts \<text> just before the
%		  actual column entry.			\\ \hlx{.}
% "<{"<text>"}"	& Inserts \<text> just after the
%		  actual column entry.			\\ \hlx{vhv.}
% "*{"<count>"}{"<chars>"}" & Inserts \<count>
%		  copies of the \<chars> into the
%		  preamble.				\\ \hlx{vh}
% \end{tabular}
%
% \caption{\package{eqnarray} column types and modifiers}
% \label{tbl:eqnarray}
% \end{table}
%
% The default preamble, if you don't supply one of your own, is \lit{rcl}.
% Most of the time, \lit{rl} is sufficient, although compatibility is more
% important to me.
%
% By default, there is no space between columns, which makes formul\ae\ in an
% \env{eqnarray} environment look just like formul\ae\ typeset on their own,
% except that things get aligned in columns.  This is where the default
% \env{eqnarray} falls down: it leaves |\arraycolsep| space between each
% column making the thing look horrible.
%
% An example would be good here, I think.  This one's from exercise 22.9 of
% the \textit{\TeX book}.
%
% \begin{demo}[w]{Simultaneous equations}
%\begin{eqnarray}[*3{rc}rl]
%  10w & + &  3x & + & 3y & + & 18z & = 1 \\
%   6w & - & 17x &   &    & - &  5z & = 2
%\end{eqnarray}
% \end{demo}
%
% Choosing a more up-to-date example, here's some examples from  the
% \textit{\LaTeX\ Companion}.
%
% \begin{demo}[w]{Lots of equations}
%\begin{eqnarray}[rl:rl:lq]
% V_i &= v_i - q_i v_j,	& X_i &= x_i - q_i x_j,	&
%       U_i = u_i, \qquad \mbox{for $i \ne j$}  \\
% V_j &= v_j,           & X_j &= x_j            &
%       U_j u_j + \sum_{i \ne j} q_i u_i. \label{eq:A}
%\end{eqnarray}
% \end{demo}
%
% \begin{figure}
% \begin{demo}[w]{Plain text column and \cs{tabpause}}
%\begin{eqnarray}[rlqqTl]
%     x  &= y           & by (\ref{eq:A}) \\
%     x' &= y'          & by definition \\
%\tabpause{and}
% x + x' &= y + y'      & by Axiom~1
%\end{eqnarray}
% \end{demo}
% \end{figure}
%
% The new features also mean that you don't need to mess about with
% |\lefteqn| any more.  This is handled by the \lit{L} column type:
%
% \begin{demo}{Splitting example}
%\begin{eqnarray*}[Ll]
%   w+x+y+z = \\
%    & a+b+c+d+e+ \\
%    & f+g+h+i+j
%\end{eqnarray*}
% \end{demo}
%
% Finally, just to prove that the spacing's right at last, here's another one
% from the \textit{Companion}.
%
% \begin{demo}{Spacing demonstration}
%\begin{equation}
%  x^2 + y^2 = z^2
%\end{equation}
%\begin{eqnarray}[rl]
%  x^2 + y^2 &= z^2 \\
%        y^2 &< z^2
%\end{eqnarray}
% \end{demo}
%
% Well, that was easy enough.  Now on to numbering.  As you've noticed, the
% equations above are numbered.  You can use the \env{eqnarray$*$}
% environment to turn off the numbering in the whole environment, or say
% |\nonumber| on a line to suppress numbering of that one in particular.
%
% \DescribeMacro{\eqnumber}
% More excitingly, you can say |\eqnumber| to enable numbering for a
% particular equation, or \syntax{"\\eqnumber["<text>"]"} to choose what to
% show instead of the line number.  This works for both starred and unstarred
% versions of the environment.  Now |\nonumber| becomes merely a synonym for
% `|\eqnumber[]|'.
%
% A note for cheats: you can use the sparkly new \env{eqnarray} for simple
% equations by specifying \lit{x} as the column description.  Who needs
% \AmSTeX?\ |;-)|
%
% \DescribeEnv{eqlines}
% \DescribeEnv{eqlines*}
% In fact, there's a separate environment \env{eqlines}, which is equivalent
% to \env{eqnarray} with a single \lit{x} column; the result is that you can
% insert a collection of displayed equations separated by |\\| commands.  If
% you don't like numbering, use \env{eqlines$*$} insead.
%
% \subsubsection{The \env{eqnalign} environment}
%
% \DescribeEnv{eqnalign}
% There's a new environment, \env{eqnalign}, which does almost the same
% thing as \env{eqnarray} but not quite.  It doesn't do equation numbers,
% and it wraps its contents up in a box.  The result of this is that:
%
% \begin{itemize}
%
% \item You can use \env{eqnalign} for just a part of a formula.
%       The \env{eqnarray} environment must take up the whole display.
%
% \item You can use \env{eqnalign} within \env{eqnarray} for extra fine
%       alignment of subsidiary bits.
%
% \item You can break off from doing an \env{eqnarray} using the |\tabpause|
%       command.  You can't use |\tabpause| inside
%       \env{eqnalign}.\footnote{^^A
%         Well, technically speaking there's nothing to stop you.  However,
%         the results won't be pretty.}
%
% \end{itemize}
%
% The \env{eqnalign} environment works like this:
%
% \begin{grammar}
%
% <eqnalign-env> ::= \[[
%   <begin-eqnalign> <contents> <end-eqnalign>
% \]]
%
% <begin-eqnalign> ::= \[[
%   "\\begin" "{eqnalign}"
%   \[ "[" \< <eqa-column> \> "]" \]
%   \[ "[" \( "t" \\ "c" \\ "b" \) "]" \]
% \]]
%
% <end-eqnalign> ::= \[[ "\\end" "{eqnalign}" \]]
%
% \end{grammar}
%
% As the syntax suggests, the preamble for the \env{eqnalign} environment
% works exactly the same way as for \env{eqnarray}.  Example time: another
% one from the \textit{\TeX book}.
%
% \begin{figure}
% \begin{demo}[w]{Example of \env{eqnalign}}
%\[
%  \left\{ \begin{eqnalign}[rl]
%    \alpha &= f(z) \\ \beta  &= f(z^2) \\
%    \gamma &= f(z^3)
%  \end{eqnalign} \right\}
%  \qquad
%  \left\{ \begin{eqnalign}[rl]
%    x &= \alpha^2 - \beta \\ y &= 2\gamma
%  \end{eqnalign} \right\}.
%\]
% \end{demo}
% \end{figure}
%
% \DescribeMacro{\multicolumn}
% The |\multicolumn| command works correctly in both the \env{eqnarray} and
% \env{eqnalign} environments, although you should bear in mind that you
% should give \env{eqnarray} column types, not \env{array} ones.
%
% \subsubsection{A note on spacing in alignment environments}
%
% Most of the time, equations in \env{eqnarray} and \env{eqnalign}
% environments will be beautiful.  However, there are some things you should
% bear in mind when you produce beautiful equations.
%
% The main problem with spacing is making sure that binary relations and
% binary operators have the correct amount of space on each side of them.
% The alignment environments insert `hidden' objects at the ends of table
% cells to assist with the spacing: \lit{l} column types have a hidden object
% on the left, \lit{r} types have a hidden object on the right, and \lit{c}
% types have a hidden object on \emph{both} ends.  These hidden objects add
% the correct space when there's a binary operator or relation next to them.
% If some other sort of object is lurking there, no space is added.  So far,
% so good.
%
% The only problem comes when you have something like this:
%
% \begin{demo}{How not to do an \env{eqnarray}}
%\begin{eqnarray*}[rcl]
%  x +  y & = & 12 \\
% 2x - 5y & = & -6
%\end{eqnarray*}
% \end{demo}
%
% The `$-$' sign in the second equation has been treated as a binary operator
% when really it should be a unary prefix operator, but \TeX\ isn't clever
% enough to know the difference.  (Can you see the difference in the spacing
% between $-6$~and~${}-6$?)  There are two possible solutions to the
% problem.  You could wrap the `|-6|' up in a group (`|{-6}|'), or just the
% $-$ sign (`|{-}6|').  A better plan, though, is to get rid of the middle
% column altogether:
%
% \begin{demo}{How to do an \env{eqnarray}}
%\begin{eqnarray*}[rl]
%  x +  y & = 12 \\
% 2x - 5y & = -6
%\end{eqnarray*}
% \end{demo}
%
% Since the things in the middle column were the same width, it's not
% actually doing any good.  Also, now that \TeX\ can see that the thing on
% the left of the `$-$' sign is a relation (the `$=$' sign), it will space
% the formula correctly.
%
% In this case, it might be even better to add some extra columns, and line
% up the $x$ and $y$ terms in the left hand side:
%
% \begin{demo}{Extra beautiful \env{eqnarray}}
%\begin{eqnarray*}[rrl]
%  x + &  y & = 12 \\
% 2x - & 5y & = -6
%\end{eqnarray*}
% \end{demo}
%
% ^^A Some hacking now to display box sizes.
%
% {
% \catcode`p=12 \catcode`t=12
% \gdef\magni#1pt{#1}
% }
%
% \newcommand{\widthof}[1]{^^A
%   \settowidth{\dimen0 }{#1}^^A
%   \expandafter\magni\the\dimen0\,pt^^A
% }
%
% ^^A The text below makes an assumption which looks correct to me (I asked
% ^^A TeX, and it agreed with me), although in case anything changes, I want
% ^^A to be informed.
%
% \sbox0{$+$} \sbox2{$-$} \ifdim\wd0=\wd2\else%
%   \errmessage{Assertion failed: `+' and `-' are different widths!}
% \fi
%
% There's no need to put the `$+$' and `$-$' operators in their own column
% here, because they're both \widthof{$+$} wide, even though they don't
% look it.
%
% \subsubsection{Configuring the alignment environments}
%
% There are a collection of parameters you can use to make the equation
% alignment environments (\env{eqnarray} and \env{eqnalign}) look the way
% you like them.  These are all shown in table~\ref{tbl:eqnparms}.
%
% \begin{table}
% \begin{tabular}[C]{| l | p{3in} |}				   \hlx{hv}
% \bf Parameter		& \bf Use				\\ \hlx{vhv}
% |\eqaopenskip|	& Length put on the left of an
%			  \env{eqnarray} environment.  By
%			  default, this is |\@centering| (to
%			  centre the alignment) or |\mathindent|
%			  (to left align) depending on whether
%			  you're using the \textsf{fleqn}
%			  document class option.		\\
% |\eqacloseskip|	& Length put on the right of an
%			  \env{eqnarray} environment.  By
%			  default, this is |\@centering|, to
%			  align the environment correctly.	\\ \hlx{vhv}
% |\eqacolskip|		& Space added by the \lit{:} column
%			  modifier.  This should be a rubber
%			  length, although it only stretches in
%			  \env{eqnarray}, not in \env{eqnalign}.
%			  The default value is 1\smallf1/2\,em
%			  with 1000\,pt of stretch.		\\
% |\eqainskip|		& Space added at each side of a normal
%			  column.  By default this is 0\,pt.	\\ \hlx{vhv}
% |\eqastyle|		& The maths style used in the alignment.
%			  By default, this is |\textstyle|,
%			  and you probably won't want to change
%			  it.					\\ \hlx{vh}
% \end{tabular}
%
% \caption{Parameters for the \env{eqnarray} and \env{eqnalign} environments}
% \label{tbl:eqnparms}
% \end{table}
%
%
% \subsection{Other multiline equations}
%
% Sometimes there's no sensible alignment point for splitting equations.  The
% normal thing to do under these circumstances is to put the first line way
% over to the left of the page, and the last line over to the right.  (If
% there are more lines, I imagine we put them in the middle.)
%
% \DescribeEnv{spliteqn}
% \DescribeEnv{spliteqn*}
% The \env{spliteqn} environment allows you to do such splitting of
% equations.  Rather than tediously describe it, I'll just give an example,
% because it's really easy.  The $*$-version works the same, except it
% doesn't put an equation number in.
%
% \begin{figure}
% \begin{demo}[w]{A split equation}
%\begin{spliteqn}
%  \sum_{1\le j\le n}
%  \frac {1} { (x_j - x_1) \ldots (x_j - x_{j-1})
%              (x - x_j) (x_j - x_{j+1}) \ldots (x_j - x_n) }
%  \\
%  = \frac {1} { (x - x_1) \ldots (x - x_n) }.
%\end{spliteqn}
% \end{demo}
% \end{figure}
%
% \DescribeEnv{subsplit}
% If you have a very badly behaved equation, you might want to split a part
% of it (say, a bit of a fraction), particularly if you're doing things in
% narrow columns.
%
% \begin{figure}
% \begin{demo}[w]{A \env{subsplit} environment}
%\begin{equation}
%  \frac{
%    \begin{subsplit}
%      q^{\frac{1}{2} n(n+1)}(ea; q^2)_\infty (eq/a; q^2)_\infty \\
%                 (caq/e; q^2)_\infty (cq^2/ae; q^2)_\infty
%    \end{subsplit}
%  }{
%    (e; q)_\infty (cq/e; q)_\infty
%  }
%\end{equation}
% \end{demo}
% \end{figure}
%
% \subsection{Matrices}
%
% Also included in the \package{mathenv} package is a collection of things
% for typesetting matrices.  The standard \env{array} doesn't (in my opinion)
% provide the right sort of spacing for matrices.  \PlainTeX\ provides some
% quite nice matrix handling macros, but they don't work in the appropriate
% \LaTeX\ way.
%
% \textbf{Warning:} These definitions will make old versions of
% \package{plain.sty} unhappy; newer versions correctly restore the
% Plain~\TeX\ macros |\matrix| and |\pmatrix|.
%
% \DescribeEnv{matrix}
% The simple way to do matrices is with the \env{matrix} environment.
%
% \begin{grammar}
%
% <matrix-env>	::= \[[ <begin-matrix> <contents> <end-matrix> \]]
%
% <begin-matrix> ::= \[[ "\\begin{matrix}" \[ "[" <matrix-cols> "]" \] \]]
%
% <matrix-cols>	::= \[[
%   \< \[ "[" \] \[ "T" \] \( "l" \\ "c" \\ "r" \) \>
% \]]
%
% <end-matrix>	::= \[[ "\\end{stack}" \]]
%
% \end{grammar}
%
% The \lit{l}, \lit{c} and \lit{r} columns are fairly obvious -- they align
% their contents in the appropriate way.  The \lit{[} character is more
% complicated.  It means `repeat the remaining column types forever', so a
% preamble of \lit{cc[lr} means `two centred columns, then alternating left-
% and right-aligned columns for as often as needed'.  The default preamble,
% if you don't specify one, is \lit{[c} -- `any number of centred columns'.
%
% \DescribeMacro{\multicolumn}
% The |\multicolumn| command works correctly in matrices, although you should
% bear in mind that you should give \env{matrix} column types, not
% \env{array} ones.
%
% \DescribeEnv{pmatrix}
% The standard \env{matrix} environment doesn't put any delimiters around the
% matrix.  You can use the standard |\left| and |\right| commands, although
% this is a bit nasty.  The \env{pmatrix} environment will put parentheses
% around the matrix it creates; it's otherwise exactly the same as
% \env{matrix}.
%
% \DescribeEnv{dmatrix}
% A \env{dmatrix} environment is also provided.  It takes two extra
% arguments: the left and right delimiter characters (without |\left| or
% |\right|).
%
% \begin{figure}
% \begin{demo}[w]{Various \env{matrix} environments}
%\[ \begin{matrix} 1 & 0 \\ 0 & -1 \end{matrix} \quad
%   \begin{pmatrix}
%     \cos\theta & \sin\theta \\
%    -\sin\theta & \cos\theta
%   \end{pmatrix} \quad
%   \begin{dmatrix}[] 0 & -i \\ i & 0 \end{dmatrix}
%\]
% \end{demo}
% \end{figure}
%
% \DescribeEnv{smatrix}
% Normal matrices always come out the same size; they don't change size
% according to the surrounding context (unfortunately).  However, it can be
% occasionally useful to put matrices in running text, so you can talk about
% $A$ being $\bigl( \begin{smatrix} a & b \\ b & c \end{smatrix} \bigr)$
% being its own transpose (i.e., $A = A^T$).  This is accomplished using the
% \env{smatrix} (the `s' stands for `small' -- I thought that `smallmatrix'
% was too big to type inline).  As well as inline text, the \env{smatrix}
% can be useful in displays, if the matrix is deep in a subformula.  I can't
% think of any examples offhand, though.
%
% \DescribeEnv{spmatrix}
% \DescribeEnv{sdmatrix}
% The \env{smatrix} environment doesn't supply any delimiters, like
% \env{matrix}.  There are \env{spmatrix} and \env{sdmatrix} environments
% which do, though.  Note that delimiters have a tendency to get too big and
% mess up the line spacing -- I had to use explicitly |\big| delimiters
% in the above example.
%
% \DescribeEnv{pmatrix*}
% \DescribeEnv{spmatrix*}
% \DescribeEnv{sdmatrix*}
% All the small matrix environments have starred versions, which are more
% suitable for use in displays, since they have more space between the rows.
% They're intended for typesetting really big matrices in displays.
%
% \DescribeMacro{\ddots}
% \DescribeMacro{\vdots}
% The standard |\vdots| and |\ddots| commands don't produce anything at all
% nice in small matrices, so this package redefines them so that they scale
% properly to smaller sizes.
%
% \DescribeEnv{genmatrix}
% Actually, all these environments are special cases of one: \env{genmatrix}.
% This takes oodles of arguments:
% \begin{quote} \synshorts
% "\\begin{genmatrix}{"<matrix-style>"}{"<outer-style>"}" \\
% \null \qquad "{"<spacing>"}{"<left-delim>"}{"<right-delim>"}" \\
% \null \quad\vdots \\
% "\\end{genmatrix}"
% \end{quote}
% The two `style' arguments should be things like |\textstyle| or
% |\scriptstyle|; the first, \<matrix-style>, is the style to use for the
% matrix elements, and the second, \<outer-style>, is the style to assume
% for the surrounding text (this affects the spacing within the matrix; it
% should usually be the same as \<matrix-style>).  The \<spacing> is inserted
% between the matrix and the delimiters, on each side of the matrix.  It's
% usually `|\,|' in full-size matrices, and blank for small ones.  The
% delimiters are inserted around the matrices, and sized appropriately.
%
% \DescribeEnv{newmatrix}
% You can create your own matrix environments if you like, using the
% |\newmatrix| command.  It takes two arguments, although they're a bit
% odd.  The first is the name of the environment, and the second contains
% the arguments to pass to \env{genmatrix}.  For example, the \env{pmatrix}
% environment was defined by saying
%
% \begin{listing}
%\newmatrix{pmatrix}{{\textstyle}{\textstyle}{\,}{(}{)}}
% \end{listing}
%
% If you don't pass all three arguments, then you end up requiring the
% user to specify the remaining ones.  This is how \env{dmatrix} works.
%
% \DescribeEnv{script}
% Finally, although it's not really a matrix, stacked super- and subscripts
% follow much the same sorts of spacing rules.  The \env{script} environment
% allows you to do this sort of thing very easily.  It essentially provides
% a `matrix' with the right sort of spacing.  The default preamble string is
% \lit{c}, giving you centred scripts, although you can say
% |\begin{script}[l]| for left-aligned scripts, which is better if the
% script is being placed to the right of its operator.  If you're really
% odd, you can have more than one column.
%
% \begin{demo}{Example of \env{script}}
%\[ \mathop{{\sum}'}_{x \in A}
%   f(x)
%   \stackrel{\mathrm{def}}{=}
%   \sum_{\begin{script}
%     x \in A \\ x \ne 0
%   \end{script}} f(x)
%\]
% \end{demo}
%
%
% \subsection{Other \package{mathenv} environments}
%
% The \package{mathenv} package contains some other environments which may
% be useful, based on the enhanced \env{tabular} and \env{array}
% environments.
%
% \DescribeEnv{cases}
% The \env{cases} environment lets you say things like the following:
%
% \begin{demo}[w]{Example of \env{cases}}
%\[ P_{r-j} = \begin{cases}
%               0 & if $r-j$ is odd \\
%               r!\,(-1)^{(r-j)/2} & if $r-j$ is even
%             \end{cases}
%\]
% \end{demo}
%
% The spacing required for this is a bit messy, so providing an environment
% for it is quite handy.
%
% \DescribeEnv{smcases}
% The \env{smcases} environment works the same way as \env{cases}, but with
% scriptsize lettering.
%
% \implementation
%
%
%^^A-------------------------------------------------------------------------
% \section{Implementation of table handling}
%
%
% Here we go.  It starts horrid and gets worse.  However, it does stay nicer
% than the original, IMHO.
%
%    \begin{macrocode}
%<*mdwtab>
%    \end{macrocode}
%
%
% \subsection{Registers, switches and things}
%
% We need lots of these.  It's great fun.
%
% The two count registers are simple enough:
%
% \begin{description}
% \item [\cs{tab@state}] contains the current parser state.  Since we
%       probably won't be parsing preambles recursively, this is a global
%       variable.
% \item [\cs{tab@columns}] contains the number of the current column.
% \item [\cs{tab@hlstate}] contains the state required for hline management.
% \end{description}
%
%    \begin{macrocode}
\newcount\tab@state
\newcount\tab@columns
%    \end{macrocode}
%
% We need \emph{lots} of token registers.  Fortunately, most of them are only
% used during parsing.  We'll use \PlainTeX's scratch tokens for this.  Note
% that |\toks\tw@| isn't used here.  It, and |\toks@|, are free for use by
% column commands.
%
%    \begin{macrocode}
\newtoks\tab@preamble
\newtoks\tab@shortline
\toksdef\tab@pretext 4
\toksdef\tab@posttext 6
\toksdef\tab@userpretext 8
%    \end{macrocode}
%
% The dimens are fairly straightforward.  The inclusion of |\col@sep| is a
% sacrifice to compatibility -- judicious use of |\let| in \package{array}
% would have saved a register.
%
%    \begin{macrocode}
\newdimen\extrarowheight
\newdimen\tabextrasep
\newdimen\arrayextrasep
\newdimen\smarraycolsep
\newdimen\smarrayextrasep
\newdimen\tab@width
\newdimen\col@sep
\newdimen\tab@endheight
%    \end{macrocode}
%
% Some skip registers too.  Phew.
%
%    \begin{macrocode}
\newskip\tab@leftskip
\newskip\tab@rightskip
%    \end{macrocode}
%
% And some switches.  The first three are for the parser.
%
%    \begin{macrocode}
\newif\iftab@firstcol
\newif\iftab@initrule
\newif\iftab@rule
\newif\iftab@vgap
%    \end{macrocode}
%
% Now assign some default values to new dimen parameters.  These definitions
% are essentially the equivalent of an |\openup 1\jot| in \env{array}, but
% not in \env{tabular}.  This looks nice, I think.
%
%    \begin{macrocode}
\tabextrasep\z@
\arrayextrasep\jot
\smarraycolsep\thr@@\p@
\smarrayextrasep\z@
%    \end{macrocode}
%
% Set some things up for alien table environments.
%
%    \begin{macrocode}
\let\tab@extrasep\tabextrasep
\let\tab@penalty\relax
%    \end{macrocode}
%
%
% \subsection{Some little details}
%
% \begin{macro}{\@maybe@unskip}
%
% This macro solves a little problem.  In an alignment (and in other places)
% it's desirable to suppress trailing space.  The usual method, to say
% |\unskip|, is a little hamfisted, because it removes perfectly reasonable
% aligning spaces like |\hfil|s.  While as a package writer I can deal with
% this sort of thing by saying |\kern\z@| in appropriate places, it can
% annoy users who are trying to use |\hfill| to override alignment in funny
% places.
%
% My current solution seems to be acceptable.  I'll remove the natural width
% of the last glue item, so that it can still stretch and shrink if
% necessary.  The implementation makes use of the fact that multiplying
% a \<skip> by a \<number> kills off the stretch.  (Bug fix: don't do this
% when we're in vertical mode.)
%
%    \begin{macrocode}
\def\@maybe@unskip{\ifhmode\hskip\m@ne\lastskip\relax\fi}
%    \end{macrocode}
%
% \end{macro}
%
% \begin{macro}{\q@delim}
%
% Finally, for the sake of niceness, here's a delimiter token I can use
% for various things.  It's a `quark', for what it's worth (i.e., it expands
% to itself) although I'm not really sure why this is a good thing.  As far
% as I'm concerned, it's important that it has a unique meaning (i.e., that
% it won't be |\ifx|-equal to other things, or something undefined) and that
% it won't be used where I don't expect it to be used.  \TeX\ will loop
% horridly if it tries to expand this, so I don't think that quarks are
% wonderfully clever thing to use.  (Maybe it should really expand to
% something like `\syntax{<quark>"."}', which will rapdly fill \TeX's memory
% if it gets accidentally expanded.  Still, I'll leave it as it is until
% such time as I understand the idea more.)
%
%    \begin{macrocode}
\def\q@delim{\q@delim}
%    \end{macrocode}
%
% \end{macro}
%
%
% \subsection{Parser states}
% 
% Now we start on the parser.  It's really simple, deep down.  We progress
% from state to state, extracing tokens from the preamble and building
% command names from them.  Each command calls one of the element-building
% routines, which works out which state it should be in.  We go through each
% of the states in between (see later) doing default things for the ones we
% missed out.
% 
% Anyway, here's some symbolic names for the states.  It makes my life
% easier.
%
%    \begin{macrocode}
\chardef\tab@startstate 0
\chardef\tab@loopstate 1
\chardef\tab@rulestate 1
\chardef\tab@prespcstate 2
\chardef\tab@prestate 3
\chardef\tab@colstate 4
\chardef\tab@poststate 5
\chardef\tab@postspcstate 6
\chardef\tab@limitstate 7
%    \end{macrocode}
%
%
% \subsection{Adding things to token lists}
%
% Define some macros for adding stuff to the beginning and end of token
% lists.  This is really easy, actually.  Here we go.
%
%    \begin{macrocode}
\def\tab@append#1#2{#1\expandafter{\the#1#2}}
\def\tab@prepend#1#2{%
  \toks@{#2}#1\expandafter{\the\expandafter\toks@\the#1}%
}
%    \end{macrocode}%
%
%
% \subsection{Committing a column to the preamble}
%
% Each time we pass the `rule' state, we `commit' the tokens we've gathered
% so far to the main preamble token list.  This is how we do it.  Note the
% icky use of |\expandafter|.
%
%    \begin{macrocode}
\def\tab@commit{%
%    \end{macrocode}
%
% If this isn't the first column, then we need to put in a column separator.
%
%    \begin{macrocode}
  \iftab@firstcol\else%
    \expandafter\tab@append\expandafter\tab@preamble%
      \expandafter{\tab@tabtext}%
  \fi%
%    \end{macrocode}
%
% Now we spill the token registers into the main list in a funny order (which
% is why we're doing it in this strange way in the first place.
%
%    \begin{macrocode}
  \toks@\expandafter{\tab@midtext}%
  \tab@preamble\expandafter{%
    \the\expandafter\tab@preamble%
    \the\expandafter\tab@pretext%
    \the\expandafter\tab@userpretext%
    \the\expandafter\toks@%
    \the\tab@posttext%
  }%
%    \end{macrocode}
%
% Now reset token lists and things for the next go round.
%
%    \begin{macrocode}
  \tab@firstcolfalse%
  \tab@pretext{}%
  \tab@userpretext{}%
  \tab@posttext{}%
}
%    \end{macrocode}
%
%
% \subsection{Playing with parser states}
%
% \begin{macro}{\tab@setstate}
%
% This is how we set new states.  The algorithm is fairly simple, really.
%
% ^^A Let's see how good my TeX really is... ;-)
% ^^A Actually, it doesn't seem to have worked out too badly.  Maybe I should
% ^^A write a package to do this automatically.  It's rather tricky, though.
%
% \def\qq{\mbox{\quad}}
% \sbox{0}{\itshape\textunderscore}\def\_{\usebox{0}}
%
% \begin{quote}
% {\bf while} $\it tab\_state \ne s$ {\bf do} \\
% \qq $\mathit{tab\_state = tab\_state}+1$; \\
% \qq {\bf if} $\it tab\_state = tab\_limitState$ {\bf then}
%				$\it tab\_state=tab\_loopState$; \\
% \qq {\bf if} $\it tab\_state = tab\_preSpcState$ {\bf then} \\
% \qq \qq {\bf if} $\it tab\_initRule$ {\bf then} \\
% \qq \qq \qq $\it tab\_initRule = {\bf false}$; \\
% \qq \qq {\bf else} \\
% \qq \qq \qq {\bf if} $\it tab\_inMultiCol$ {\bf then moan}; \\
% \qq \qq \qq $\it commit$; \\
% \qq \qq \qq $\it append(tab\_shortLine,\hbox{`|&\omit|')}$; \\
% \qq \qq {\bf end\,if}; \\
% \qq {\bf end\,if}; \\
% \qq {\bf if} $\it tab\_state \ne s$ {\bf then}
%				$\it do\_default(tab\_state)$; \\
% {\bf end\,while};
% \end{quote}
%
% First we decide if there's anything to do.  If so, we call another macro to
% do it for us.
%
%    \begin{macrocode}
\def\tab@setstate#1{%
  \ifnum#1=\tab@state\else%
    \def\@tempa{\tab@setstate@i{#1}}%
    \@tempa%
  \fi%
}
%    \end{macrocode}
%
% This is where the fun is.  First we bump the state by one, and loop back
% if we fall off the end.
%
%    \begin{macrocode}
\def\tab@setstate@i#1{%
  \global\advance\tab@state\@ne%
  \ifnum\tab@state>\tab@limitstate%
    \global\tab@state\tab@loopstate%
  \fi%
%    \end{macrocode}
%
% Now, if we've just passed the ruleoff state, we commit the current text
% \emph{unless} this was the strange initial rule at the very beginning.  We
% provide a little hook here so that |\multicolumn| can moan if you try and
% give more than one column there.  We also add another tab/omit pair to the
% list we use for |\vgap|.
%
%    \begin{macrocode}
  \ifnum\tab@state=\tab@prespcstate%
    \iftab@initrule%
      \tab@initrulefalse%
    \else%
      \tab@looped%
      \tab@commit%
      \tab@append\tab@shortline{&\omit}%
    \fi%
  \fi%
%    \end{macrocode}
%
% Now we decide whether to go round again.  If not, we do the default thing
% for this state.  This is mainly here so that we can put the |\tabcolsep| or
% whatever in if the user didn't give an \lit{@} expression.
%
%    \begin{macrocode}
  \ifnum#1=\tab@state%
    \let\@tempa\relax%
  \else%
    \csname tab@default@\number\tab@state\endcsname%
  \fi%
  \@tempa%
}
%    \end{macrocode}
%
% \end{macro}
%
% Now we set up the default actions for the various states.
%
% In state~2 (pre-space) we add in the default gap if either we didn't have
% an \lit{@} expression in the post-space state or there was an explicit
% intervening rule.
%
%    \begin{macrocode}
\@namedef{tab@default@2}{%
  \iftab@rule%
    \tab@append\tab@pretext{\hskip\col@sep}%
  \fi%
}
%    \end{macrocode}
%
% If the user omits the column type, we insert an `l'-type column and moan
% a lot.
%
%    \begin{macrocode}
\@namedef{tab@default@4}{%
  \tab@err@misscol%
  \tab@append\tab@pretext{\tab@bgroup\relax}%
  \tab@append\tab@posttext{\relax\tab@egroup\hfil}%
  \tab@append\tab@shortline{\hfil}%
  \advance\tab@columns\@ne%
}
%    \end{macrocode}
%
% Finally we deal with the post-space state.  We set a marker so that we
% put in the default space in the pre-space state later too.
%
%    \begin{macrocode}
\@namedef{tab@default@6}{%
  \tab@append\tab@posttext{\hskip\col@sep}%
  \tab@ruletrue%
}
%    \end{macrocode}
%
%
% \subsection{Declaring token types}
%
% \begin{macro}{\tab@extracol}
%
% Before we start, we need to handle |\extracolsep|.  This is a right pain,
% because the original version of \env{tabular} worked on total expansion,
% which is a Bad Thing.  On the other hand, turning |\extracolsep| into a
% |\tabskip| is also a major pain.
%
%    \begin{macrocode}
\def\tab@extracol#1#2{\tab@extracol@i#1#2\extracolsep{}\extracolsep\end}
\def\tab@extracol@i#1#2\extracolsep#3#4\extracolsep#5\end{%
  \ifx @#3@%
    \def\@tempa{#1{#2}}%
  \else%
    \def\@tempa{#1{#2\tabskip#3\relax#4}}%
  \fi%
  \@tempa%
}
%    \end{macrocode}
%
% \end{macro}
%
% This is where we do the work for inserting preamble elements.
%
% \begin{macro}{\tabruletype}
%
% Inserting rules is interesting, because we have to decide where to put
% them.  If this is the funny initial rule, it goes in the pre-text list,
% otherwise it goes in the post-text list.  We work out what to do first
% thing:
%
%    \begin{macrocode}
\def\tabruletype#1{\tab@extracol\tabruletype@i{#1}}%
\def\tabruletype@i#1{%
  \iftab@initrule%
    \let\tab@tok\tab@pretext%
  \else%
    \let\tab@tok\tab@posttext%
  \fi%
%    \end{macrocode}
%
% Now if we're already in the rule state, we must have just done a rule.
% This means we must put in the |\doublerulesep| space, both here and in the
% shortline list.  Otherwise we just stick the rule in.
%
% This is complicated, because |\vgap| needs to be able to remove some bits
% of rule.  We pass each one to a macro |\tab@ckr|, together with the column
% number, which is carefully bumped at the right times, and this macro will
% vet the rules and output the appropriate ones.  There's lots of extreme
% |\expandafter| nastiness as a result.  Amazingly, this actually works.
%
%    \begin{macrocode}
  \ifnum\tab@state=\tab@rulestate%
    \tab@append\tab@tok{\hskip\doublerulesep\begingroup#1\endgroup}%
    \expandafter\tab@append\expandafter\tab@shortline\expandafter{%
      \expandafter\hskip\expandafter\doublerulesep%
      \expandafter\tab@ckr\expandafter{\the\tab@columns}%
        {\begingroup#1\endgroup}%
    }%
  \else%
    \tab@setstate\tab@rulestate%
    \tab@append\tab@tok{\begingroup#1\endgroup}%
    \expandafter\tab@append\expandafter\tab@shortline\expandafter{%
      \expandafter\tab@ckr\expandafter{\the\tab@columns}%
        {\begingroup#1\endgroup}%
    }%
  \fi%
%    \end{macrocode}
%
% Finally, we say there was a rule here, so that default space gets put in
% after this.  Otherwise we lose lots of generality.
%
%    \begin{macrocode}
  \tab@ruletrue%
}
%    \end{macrocode}
%
% \end{macro}
%
% \begin{macro}{\tabspctype}
%
% We need to work out which space-state we should be in.  Then we just put
% the text in.  Easy, really.
%
%    \begin{macrocode}
\def\tabspctype#1{\tab@extracol\tabspctype@i{#1}}%
\def\tabspctype@i#1{%
  \tab@rulefalse%
  \ifnum\tab@state>\tab@prespcstate%
    \tab@setstate\tab@postspcstate%
    \let\tab@tok\tab@posttext%
  \else%
    \tab@setstate\tab@prespcstate%
    \let\tab@tok\tab@pretext%
  \fi%
  \tab@append\tab@tok{\begingroup#1\endgroup}%
}
%    \end{macrocode}
%
% \end{macro}
%
% \begin{macro}{\tabcoltype}
%
% If we're already in the column state, we bump the state and loop round
% again, to get all the appropriate default behaviour.  We bump the column
% counter, and add the bits of text we were given to appropriate token lists.
% We also add the |\hfil| glue to the shortline list, to space out the rules
% properly.
%
%    \begin{macrocode}
\def\tabcoltype#1#2{%
  \ifnum\tab@state=\tab@colstate%
    \global\advance\tab@state\@ne%
  \fi%
  \advance\tab@columns\@ne%
  \tab@setstate\tab@colstate%
  \tab@append\tab@pretext{#1}%
  \tab@append\tab@posttext{#2}%
  \tab@append\tab@shortline{\hfil}%
}
%    \end{macrocode}
%
% \end{macro}
%
% \begin{macro}{\tabuserpretype}
% \begin{macro}{\tabuserposttype}
%
% These are both utterly trivial.
%
%    \begin{macrocode}
\def\tabuserpretype#1{%
  \tab@setstate\tab@prestate%
  \tab@prepend\tab@userpretext{#1}%
}
%    \end{macrocode}
%
%    \begin{macrocode}
\def\tabuserposttype#1{%
  \tab@setstate\tab@poststate%
  \tab@prepend\tab@posttext{#1}%
}
%    \end{macrocode}
%
% \end{macro}
% \end{macro}
%
%
% \subsection{The colset stack}
%
% Let's start with something fairly easy.  We'll keep a stack of column sets
% so that users don't get confused by package authors changing the current
% column set.  This is fairly easy, really.
%
% \begin{macro}{\tab@push}
% \begin{macro}{\tab@pop}
% \begin{macro}{\tab@head}
%
% These are the stack management routines.  The only important thing to note
% is that |\tab@head| must take place \emph{only} in \TeX's mouth, so we can
% use it in |\csname|\dots|\endcsname| constructions.
%
%    \begin{macrocode}
\def\tab@push#1#2{%
  \toks@{{#2}}%
  \expandafter\def\expandafter#1\expandafter{\the\expandafter\toks@#1}%
}
\def\tab@pop#1{\expandafter\def\expandafter#1\expandafter{\@gobble#1}}
\def\tab@head#1{\expandafter\tab@head@i#1\relax}
\def\tab@head@i#1#2\relax{#1}
%    \end{macrocode}
%
% \end{macro}
% \end{macro}
% \end{macro}
%
% \begin{macro}{\colset}
% \begin{macro}{\colpush}
% \begin{macro}{\colpop}
%
% Now we can define the user macros.
%
%    \begin{macrocode}
\def\tab@colstack{{tabular}}
\def\colset{\colpop\colpush}
\def\colpush{\tab@push\tab@colstack}
\def\colpop{\tab@pop\tab@colstack}
%    \end{macrocode}
%
% \end{macro}
% \end{macro}
% \end{macro}
%
% \begin{macro}{\tab@colset}
%
% Now we define a shortcut for reading the top item off the stack.
%
%    \begin{macrocode}
\def\tab@colset{\tab@head\tab@colstack}
%    \end{macrocode}
%
% \end{macro}
%
%
% \subsection{The main parser routine}
%
% \begin{macro}{\tab@initread}
%
% This macro sets up lots of variables to their normal states prior to
% parsing a preamble.  Some things may need changing, but not many.
%
%    \begin{macrocode}
\def\tab@initread{%
%    \end{macrocode}
%
% First, reset the parser state to the start state.
%
%    \begin{macrocode}
  \global\tab@state\tab@startstate%
%    \end{macrocode}
%
% We clear the token lists to sensible values, mostly.  The midtext macro
% contains what to put in the very middle of each template -- |\multicolumn|
% will insert its argument here.
%
%    \begin{macrocode}
  \tab@preamble{}%
  \tab@shortline{}%
  \def\tab@tabtext{&}%
  \def\tab@midtext{\ignorespaces####\@maybe@unskip}%
  \tab@pretext{}%
  \tab@userpretext{}%
  \tab@posttext{}%
  \let\tab@multicol\@empty%
  \def\tab@startpause{\penalty\postdisplaypenalty\medskip}%
  \def\tab@endpause{\penalty\predisplaypenalty\medskip}%
%    \end{macrocode}
%
% Finally, reset the column counter, don't raise errors when we loop, and set
% some parser flags to their appropriate values.
%
%    \begin{macrocode}
  \tab@columns\z@%
  \let\tab@looped\relax%
  \tab@ruletrue%
  \tab@initruletrue%
  \tab@firstcoltrue%
}
%    \end{macrocode}
%
% \end{macro}
%
% \begin{macro}{\tab@readpreamble}
%
% This is the main macro for preamble handling.  Actually, all it does is
% gobble its argument's leading brace and call another macro, but it does it
% with style.
%
%    \begin{macrocode}
\def\tab@readpreamble#1{%
  \tab@doreadpream{#1}%
  \iftab@initrule\global\tab@state\tab@prespcstate\fi%
  \tab@setstate\tab@rulestate%
  \tab@commit%
}
%    \end{macrocode}
%
% \end{macro}
%
% \begin{macro}{\tab@doreadpream}
%
% The preamble is in an argument.  Previous versions used a nasty trick using
% |\let| and |\afterassignment|.  Now we use an explicit end token, to allow
% dodgy column type handlers to scoop up the remaining preamble tokens
% and process them.  Not that anyone would want to do that, oh no (see
% the \lit{[} type in the \env{eqnarray} environment |;-)|).
%
%    \begin{macrocode}
\def\tab@doreadpream#1{\tab@mkpreamble#1\q@delim}
%    \end{macrocode}
%
% \end{macro}
%
% \begin{macro}{\tab@mkpreamble}
%
% This is the main parser routine.  It takes each token in turn, scrutinises
% it carefully, and does the appropriate thing with it.
%
% The preamble was given as an argument to |\tab@doreadpream|, and that has
% helpfully stripped off the initial |{| character.  We need to pick off the
% next token (whatever it is) so we can examine it.  We'll use |\futurelet|
% so we can detect groups and things in funny places.
%
%    \begin{macrocode}
\def\tab@mkpreamble{\futurelet\@let@token\tab@mkpreamble@i}
%    \end{macrocode}
%
% If we find a space token, we'll go off and do something a bit special,
% since spaces are sort of hard to handle.  Otherwise we'll do it in the old
% fashioned way.
%
%    \begin{macrocode}
\def\tab@mkpreamble@i{%
  \ifx\@let@token\@sptoken%
    \expandafter\tab@mkpreamble@spc%
  \else%
    \expandafter\tab@mkpreamble@ii%
  \fi%
}
%    \end{macrocode}
%
% If we find a |\@@endpreamble| token, that's it and we're finished.  We just
% gobble it and return.  Otherwise, if it's an open group character, we'll
% complain because someone's probably tried to put an argument in the wrong
% place.  Finally, if none of the other things apply, we'll deal with the
% character below.
%
%    \begin{macrocode}
\def\tab@mkpreamble@ii{%
  \ifx\@let@token\q@delim%
    \def\@tempa{\let\@let@token}%
  \else%
    \ifcat\bgroup\noexpand\@let@token%
      \tab@err@oddgroup%
      \def\@tempa##1{\tab@mkpreamble}%
    \else%
      \let\@tempa\tab@mkpreamble@iii%
    \fi%
  \fi%
  \@tempa%
}
%    \end{macrocode}
%
% Handle a character.  This involves checking to see if it's actually
% defined, and then doing it.  Doing things this way means we won't get
% stranded in mid-preamble unless a package author has blown it.
%
%    \begin{macrocode}
\def\tab@mkpreamble@iii#1{%
  \@ifundefined{\tab@colset!col.\string#1}{%
    \tab@err@undef{#1}\tab@mkpreamble%
  }{%
    \@nameuse{\tab@colset!col.\string#1}%
  }%
}
%    \end{macrocode}
%
% If we get given a space character, we'll look up the command name as
% before.  If no-one's defined the column type we'll just skip it silently,
% which lets users do pretty formatting if they like.
%
%    \begin{macrocode}
\@namedef{tab@mkpreamble@spc} {%
  \@ifundefined{\tab@colset!col. }{%
    \tab@mkpreamble%
  }{%
    \@nameuse{\tab@colset!col. }%
  }%
}
%    \end{macrocode}
%
% \end{macro}
%
% \begin{macro}{\coldef}
%
% Here's how to define column types the nice way.  Some dexterity is required
% to make everything work right, but it's simple really.
%
%    \begin{macrocode}
\def\coldef{\@ifnextchar[\coldef@i{\coldef@i[\tab@colset]}}
\def\coldef@i[#1]#2#3#{\coldef@ii[#1]{#2}{#3}}
\def\coldef@ii[#1]#2#3#4{%
  \expandafter\def\csname#1!col.\string#2\endcsname#3{%
    #4\tab@mkpreamble%
  }%
}
%    \end{macrocode}
%
% \end{macro}
%
% \begin{macro}{\collet}
%
% We'd like to let people copy column types from other places.  This is how
% to do it.
%
%    \begin{macrocode}
\def\collet{\@ifnextchar[\collet@i{\collet@i[\tab@colset]}}
\def\collet@i[#1]#2{%
  \@ifnextchar=%
    {\collet@ii[#1]{#2}}%
    {\collet@ii[#1]{#2}=}%
}
\def\collet@ii[#1]#2={%
  \@ifnextchar[%
    {\collet@iii[#1]{#2}}%
    {\collet@iii[#1]{#2}[\tab@colset]}%
}
\def\collet@iii[#1]#2[#3]#4{%
  \expandafter\let\csname#1!col.\string#2\expandafter\endcsname%
                  \csname#3!col.\string#4\endcsname%
}
%    \end{macrocode}
%
% \end{macro}
%
% \begin{macro}{\newcolumntype}
%
% We just bundle the text off to |\newcommand| and expect it to cope.  It
% ought to.  The column type code inserts the user's tokens directly, rather
% than calling |\tab@doreadpream| recursively.  The magic control sequence
% is the one looked up by the parser.
%
% There's some additional magic here for compatiblity with the obscure way
% that \package{array} works.
%
%    \begin{macrocode}
\def\newcolumntype#1{\@ifnextchar[{\nct@i{#1}}{\nct@i#1[0]}}
\def\nct@i#1[#2]{\@ifnextchar[{\nct@ii{#1}[#2]}{\nct@iii{#1}{[#2]}}}
\def\nct@ii#1[#2][#3]{\nct@iii{#1}{[#2][#3]}}
\def\nct@iii#1#2#3{%
  \expandafter\let\csname\tab@colset!col.\string#1\endcsname\relax%
  \expandafter\newcommand\csname\tab@colset!col.\string#1\endcsname#2{%
    \tab@deepmagic{#1}%
    \tab@mkpreamble%
    #3%
  }%
}
%    \end{macrocode}
%
% Now for some hacking for compatibility with \package{tabularx}.
%
%    \begin{macrocode}
\def\newcol@#1[#2]{\nct@iii{#1}{[#2]}}
%    \end{macrocode}
%
% And now some more.  This is seriously deep magic.  Hence the name.
%
%    \begin{macrocode}
\def\tab@deepmagic#1{%
  \csname NC@rewrite@\string#1\endcsname\NC@find\tab@@magic@@%
}
\def\NC@find#1\tab@@magic@@{}
%    \end{macrocode}
%
% \end{macro}
%
%
% \subsection{Standard column types}
%
% First, make sure we're setting up the right columns.  This also sets the
% default for the user.  Other packages must not use the |\colset| command
% for defining columns -- they should use the stack operations defined above.
%
%    \begin{macrocode}
\colset{tabular}
%    \end{macrocode}
%
% Now do the simple alignment types.  These are fairly simple.  The
% mysterious kern in the \lit{l} type is to stop the |\col@sep| glue from
% vanishing due to the |\unskip| inserted by the standard |\tab@midtext| if
% the column contains no text.  (Thanks for spotting this bug go to that
% nice Mr~Carlisle.)
%
%    \begin{macrocode}
\coldef l{\tabcoltype{\kern\z@\tab@bgroup}{\tab@egroup\hfil}}
\coldef c{\tabcoltype{\hfil\tab@bgroup}{\tab@egroup\hfil}}
\coldef r{\tabcoltype{\hfil\tab@bgroup}{\tab@egroup}}
%    \end{macrocode}
%
% Some extensions now.  These are explicitly teextual or mathematical
% columns.  Can be useful if you're providing column types for other people.
% I've inserted a kern here for exactly the same reason as for the \lit{l}
% column type above.
%
%    \begin{macrocode}
\coldef T#1{\tab@aligncol{#1}{\tab@btext}{\tab@etext}}
\coldef M#1{\tab@aligncol{#1}{\tab@bmaths}{\tab@emaths}}
\def\tab@aligncol#1#2#3{%
  \if#1l\tabcoltype{\kern\z@#2}{#3\hfil}\fi%
  \if#1c\tabcoltype{\hfil#2}{#3\hfil}\fi%
  \if#1r\tabcoltype{\hfil#2}{#3}\fi%
}
%    \end{macrocode}
%
% Now for the default rules.
%
%    \begin{macrocode}
\coldef |{\tabruletype{\vrule\@width\arrayrulewidth}}
\coldef !#1{\tabruletype{#1}}
%    \end{macrocode}
%
% Deal with \lit{@} expressions.
%
%    \begin{macrocode}
\coldef @#1{\tabspctype{#1}}
%    \end{macrocode}
%
% And the paragraph types.  I've added things to handle footnotes here.
%
%    \begin{macrocode}
\coldef p#1{\tabcoltype%
             {\savenotes\vtop\tab@bpar{#1}}%
             {\tab@epar\spewnotes\hfil}}
\coldef m#1{\tabcoltype%
             {\savenotes$\vcenter\tab@bpar{#1}}%
             {\tab@epar$\spewnotes\hfil}}
\coldef b#1{\tabcoltype%
             {\savenotes\vbox\tab@bpar{#1}}%
             {\tab@epar\spewnotes\hfil}}
%    \end{macrocode}
%
% Phew.  Only a few more left now.  The user text ones.
%
%    \begin{macrocode}
\coldef >#1{\tabuserpretype{#1}}
\coldef <#1{\tabuserposttype{#1}}
%    \end{macrocode}
%
% The strange column type.
%
%    \begin{macrocode}
\coldef ##1#2{\tabcoltype{#1}{#2}}
%    \end{macrocode}
%
% And \lit{*}, which repeats a preamble spec.  This is really easy, and not
% at all like the original one.
%
%    \begin{macrocode}
\coldef *#1#2{%
  \count@#1%
  \loop\ifnum\count@>0\relax%
    \tab@doreadpream{#2}%
    \advance\count@\m@ne%
  \repeat%
}
%    \end{macrocode}
%
%
% \subsection{Paragraph handling}
%
% First of all, starting new paragraphs: the vbox token is already there, and
% we have the width as an argument.
%
% \begin{macro}{\tab@bpar}
%
% There are some gymnastics to do here to support lists which form the
% complete text of the parbox.  One of the odd things I'll do here is to
% not insert a strut on the first line: instead, I'll put the text into a
% box register so that I can inspect it later.  So that I have access to
% the height of the first line, I'll use a |\vtop| -- I can get at the
% final depth by using |\prevdepth|, so this seems to be the most general
% solution.
%
%    \begin{macrocode}
\def\tab@bpar#1{%
  \bgroup%
  \hsize#1\relax%
  \@arrayparboxrestore%
  \setbox\z@\vtop\bgroup
  \global\@minipagetrue%
  \everypar{%
    \global\@minipagefalse%
    \everypar{}%
  }%
}
%    \end{macrocode}
%
% \end{macro}
%
% \begin{macro}{\tab@epar}
%
% To end the paragraph, close the box.  That sounds easy, doesn't it?
% I need to space out the top and bottom of the box so that it looks as if
% struts have been applied.
%
%    \begin{macrocode}
\def\tab@epar{%
%    \end{macrocode}
%
% Anyway, I should end the current paragraph if I'm still in horizontal
% mode.  A simple |\par| will do this nicely.  I'll also remove any trailing
% vertical glue (which may be left there by a list environment), because
% things will look very strange otherwise.
%
%    \begin{macrocode}
  \ifhmode\@maybe@unskip\par\fi%
  \unskip%
%    \end{macrocode}
%
% Now I'll look at the depth of the last box: if it's less deep than my
% special strut, I'll cunningly backpedal by a bit, and add a box with the
% appropriate depth.  Since this will lie on the previous baseline, it won't
% alter the effective height of the box.  There's a snag here.  |\prevdepth|
% may be wrong for example if the last thing inserted was a rule, or the
% box is just empty.  Check for this specially.  (Thanks to Rowland for
% spotting this.)
%
%    \begin{macrocode}
  \ifdim\prevdepth>-\@m\p@\ifdim\prevdepth<\dp\@arstrutbox%
    \kern-\prevdepth%
    \nointerlineskip%
    \vtop to\dp\@arstrutbox{}%
  \fi\fi%
%    \end{macrocode}
%
% I've finished the bottom of the box now: I'll close it, and start work on
% the top again.
%
%    \begin{macrocode}
  \egroup%
%    \end{macrocode}
%
% For top-alignment to work, the first item in the box must be another box.
% (This is why I couldn't just set |\prevdepth| at the beginning.)  If the
% box isn't high enough, I'll add a box of the right height and then kern
% backwards so that the `real' first box ends up in the right place.
%
%    \begin{macrocode}
  \ifdim\ht\z@<\ht\@arstrutbox%
    \vbox to\ht\@arstrutbox{}%
    \kern-\ht\z@%
  \fi%
  \unvbox\z@%
  \egroup%
}
%    \end{macrocode}
%
% \end{macro}
%
%
% \subsection{Gentle persuasion}
%
% To persuade \package{longtable} to work, we emulate some features of
% the \package{array} way of doing things.  It's a shame, but we have to do
% it, because \package{longtable} came first.
%
% Note the horribleness with the grouping here.  In order to get everything
% expanded at the right time, |\@preamble| just replaces itself with the (not
% expanded!) preamble string, using |\the|.  This means that the preamble
% string must be visible in the group just above us.  Now,
% \package{longtable} (and \package{array} for that matter) does
% |\@mkpreamble| immediately after opening a new group.  So all we need to do
% is close that group, do our stuff, and reopen the group again.  (Evil
% laughter\dots)
%
%    \begin{macrocode}
\def\@mkpream#1{%
  \endgroup%
  \colset{tabular}%
  \tab@initread%
  \def\tab@multicol{\@arstrut}%
  \tab@preamble{\tab@multicol}%
  \def\tab@midtext{\ignorespaces\@sharp\@sharp\@maybe@unskip}%
  \tab@readpreamble{#1}%
  \gdef\@preamble{\the\tab@preamble}%
  \let\tab@bgroup\begingroup%
  \let\tab@egroup\endgroup%
  \begingroup%
}
%    \end{macrocode}
%
%
% \subsection{Debugging}
%
% This macro just parses a preamble and displays it on the terminal.  It
% means I can see whether the thing's working.
%
%    \begin{macrocode}
\def\showpream#1{%
  \tab@initread%
  \tab@readpreamble{#1}%
  \showthe\tab@preamble%
  \showthe\tab@shortline%
}
%    \end{macrocode}
%
% A quick macro for showing column types.
%
%    \begin{macrocode}
\def\showcol#1{%
  \expandafter\show\csname\tab@colset!col.\string#1\endcsname%
}
%    \end{macrocode}
%
%
% \subsection{The \env{tabular} and \env{array} environments}
%
% This is where we define the actual environments which users play with.
%
% \subsubsection{The environment routines}
%
% The real work is done in the |\@array| macro later.  We just set up lots
% (and I mean \emph{lots}) of parameters first, and then call |\@array|.
%
% \begin{macro}{\tab@array}
%
% The |\tab@array| macro does most of the common array things.
%
%    \begin{macrocode}
\def\tab@array{%
  \tab@width\z@%
  \let\tab@bgroup\tab@bmaths%
  \let\tab@egroup\tab@emaths%
  \@tabarray%
}
%    \end{macrocode}
%
% \end{macro}
%
% \begin{macro}{\tab@btext}
% \begin{macro}{\tab@bmaths}
% \begin{macro}{\tab@etext}
% \begin{macro}{\tab@emaths}
%
% These macros contain appropriate things to use when typesetting
% text or maths macros.  They're all trivial.  They're here only for
% later modification by funny things like the \env{smarray} environment.
%
%    \begin{macrocode}
\def\tab@btext{\begingroup}
\def\tab@bmaths{$}
\def\tab@etext{\endgroup}
\def\tab@emaths{\m@th$}
%    \end{macrocode}
%
% \end{macro}
% \end{macro}
% \end{macro}
% \end{macro}
%
% \begin{environment}{array}
%
% Now for the \env{array} environment.  The `|$|' signs act as a group, so we
% don't need to do extra grouping this time.  Closing the environment is
% easy.
%
%    \begin{macrocode}
\def\array{%
  \col@sep\arraycolsep%
  \let\tab@extrasep\arrayextrasep%
  \tab@normalstrut%
  \tab@array%
}
\def\endarray{%
  \crcr%
  \egroup%
  \tab@right%
  \tab@restorehlstate%
}
%    \end{macrocode}
%
% \end{environment}
%
% \begin{environment}{smarray}
%
% Now for something a little different.  The \env{smarray} environment
% gives you an array with lots of small text.
%
%    \begin{macrocode}
\def\smarray{%
  \extrarowheight\z@%
  \col@sep\smarraycolsep%
  \let\tab@extrasep\smarrayextrasep%
  \def\tab@bmaths{$\scriptstyle}%
  \def\tab@btext{\begingroup\scriptsize}%
  \setbox\z@\hbox{\scriptsize\strut}%
  \dimen@\ht\z@\dimen\tw@\dp\z@\tab@setstrut%
  \tab@array%
}
\let\endsmarray\endarray
%    \end{macrocode}
%
% \end{environment}
%
% \begin{macro}{\tabstyle}
%
% This is a little hook that document designers can use to modify the
% appearance of tables throughout a document.  For example, I've set it to
% make the text size |\small| in all tables in this document.  Macro writers
% shouldn't try to use it as a hook for their own evilness, though.  I've
% used |\providecommand| to avoid nobbling an existing definition.
%
%    \begin{macrocode}
\providecommand\tabstyle{}
%    \end{macrocode}
%
% \end{macro}
%
% \begin{macro}{\@tabular}
%
% The two \env{tabular} environments share lots of common code, so we
% separate that out.  (This needs to be done better.)  All we really do here
% is set up the |\tab@bgroup| and |\tab@egroup| to localise things properly,
% and then go.
%
%    \begin{macrocode}
\def\@tabular#1{%
  \tabstyle%
  \tab@width#1%
  \let\tab@bgroup\tab@btext%
  \let\tab@egroup\tab@etext%
  \col@sep\tabcolsep%
  \let\tab@extrasep\tabextrasep%
  \tab@normalstrut%
  \@tabarray%
}
%    \end{macrocode}
%
% \end{macro}
%
% \begin{environment}{tabular}
% \begin{environment}{tabular*}
%
% These environments just call a macro which does all the common stuff.
%
%    \begin{macrocode}
\def\tabular{\@tabular\z@}
\expandafter\let\csname tabular*\endcsname\@tabular
\let\endtabular\endarray
\expandafter\let\csname endtabular*\endcsname\endarray
%    \end{macrocode}
%
% \end{environment}
% \end{environment}
%
% \subsubsection{Setting the strut height}
%
% \begin{macro}{\tab@setstrut}
%
% We use a magical strut, called |\@arstrut|, which keeps the table from
% collapsing around our heads.  This is where we set it up.
%
% It bases the array strut size on the given values of |\dimen@| and
% |\dimen\tw@|, amended by various appropriate fiddle values added in by
% various people.
%
%    \begin{macrocode}
\def\tab@setstrut{%
  \setbox\@arstrutbox\hbox{%
    \vrule%
      \@height\arraystretch\dimen@%
      \@depth\arraystretch\dimen\tw@%
      \@width\z@%
  }%
}
%    \end{macrocode}
%
% \end{macro}
%
% \begin{macro}{\tab@normalstrut}
%
% This sets the strut the normal way, from the size of |\strutbox|.
%
%    \begin{macrocode}
\def\tab@normalstrut{%
  \dimen@\ht\strutbox\advance\dimen@\extrarowheight%
  \dimen\tw@\dp\strutbox%
  \tab@setstrut%
}
%    \end{macrocode}
%
% \end{macro}
%
% \subsubsection{Setting up the alignment}
%
% The following bits are mainly for other packages to hook themselves onto.
%
%    \begin{macrocode}
\let\@arrayleft\relax%
\let\@arrayright\relax%
%    \end{macrocode}
%
%    \begin{macrocode}
\def\@tabarray{%
  \let\@arrayleft\relax%
  \let\@arrayright\relax%
  \@ifnextchar[\@array{\@array[c]}%
}
%    \end{macrocode}
%
% \begin{macro}{\@array}
%
% The |\@array| macro does most of the real work for the environments.  The
% first job is to set up the row strut, which keeps the table rows at the
% right height.  We just take the normal strut box, and extend its height by
% the |\extrarowheight| length parameter.
%
%    \begin{macrocode}
\def\@array[#1]#2{%
%    \end{macrocode}
%
% Sort out the hline state variable.  We'll store the old value in a
% control sequence to avoid wasting any more count registers.
%
%    \begin{macrocode}
  \edef\tab@restorehlstate{%
    \global\tab@endheight\the\tab@endheight%
    \gdef\noexpand\tab@hlstate{\tab@hlstate}%
  }%
  \def\tab@hlstate{n}%
%    \end{macrocode}
%
% Now we read the preamble.  All the clever things we've already done are
% terribly useful here.
%
% The |\tab@setcr| sets up |\\| to be a newline even if users have changed it
% using something like |\raggedright|.
%
%    \begin{macrocode}
  \colset{tabular}%
  \tab@initread%
  \def\tab@midtext{\tab@setcr\ignorespaces####\@maybe@unskip}%
  \def\tab@multicol{\@arstrut\tab@startrow}%
  \tab@preamble{\tab@multicol\tabskip\z@skip}%
  \tab@readpreamble{#2}%
%    \end{macrocode}
%
% Set up the default tabskip glue.  This is easy: there isn't any.
%
%    \begin{macrocode}
  \tab@leftskip\z@skip%
  \tab@rightskip\z@skip%
%    \end{macrocode}
%
% Now set up the positioning of the table.  This is put into a separate macro
% because it's rather complicated.
%
%    \begin{macrocode}
  \tab@setposn{#1}%
%    \end{macrocode}
%
% Now work out how to start the alignment.
%
%    \begin{macrocode}
  \ifdim\tab@width=\z@%
    \def\tab@halign{}%
  \else%
    \def\tab@halign{to\tab@width}%
  \fi%
%    \end{macrocode}
%
% Finally, do all the normal things we need to do before an alignment.  Note
% that we define |\tabularnewline| first, then set |\\| from that (using
% |\tab@setcr|).  Since |\\| is reset in the |\tab@midtext| of every table
% cell, it becomes secondary to |\tabularnewline|.  Doing things this way
% avoids the problems with declarations like |\raggedright| which redefine
% |\\| in their own (usually rather strange) way, so you don't need to mess
% about with things like the |\PreserveBackslash| command given in the
% \textit{\LaTeX\ Companion}.
%
%    \begin{macrocode}
  \lineskip\z@\baselineskip\z@%
  \m@th%
  \def\tabularnewline{\tab@arraycr\tab@penalty}%
  \tab@setcr%
  \let\par\@empty%
  \everycr{}\tabskip\tab@leftskip%
  \tab@left\halign\tab@halign\expandafter\bgroup%
    \the\tab@preamble\tabskip\tab@rightskip\cr%
}
%    \end{macrocode}
%
% \end{macro}
%
% You've no doubt noticed the |\tab@left| and |\tab@right| macros above.
% These are set up here and elsewhere to allow other things to gain control
% at various points of the table (they include and take the place of the
% |\@arrayleft| and |\@arrayright| hooks in \package{array}, put in for
% \package{delarray}'s use.
%
% \subsubsection{Positioning the table}
%
% \begin{macro}{\tab@setposn}
%
% This macro sets everything up for the table's positioning.  It's rather
% long, but not all that complicated.  Honest.
%
% First, we set up some defaults (for centring).  If anything goes wrong, we
% just do the centring things.
%
%    \begin{macrocode}
\def\tab@setposn#1{%
  \def\tab@left{%
    \savenotes%
    \leavevmode\hbox\bgroup$\@arrayleft\vcenter\bgroup%
  }%
  \def\tab@right{%
    \egroup%
    \m@th\@arrayright$\egroup%
    \spewnotes%
  }%
  \global\tab@endheight\z@%
%    \end{macrocode}
%
% For the standard positioning things, we just do appropriate boxing things.
% Note that the dollar signs are important, since \package{delarray} might
% want to put its delimiters in here.
%
% The |\if@tempswa| switch it used to decide if we're doing an unboxed
% tabular.  We'll set it if we find an unbox-type position code, and then
% check that everything's OK for this.
%
%    \begin{macrocode}
  \@tempswafalse%
  \let\tab@penalty\relax%
  \if#1t%
    \def\tab@left{%
      \savenotes%
      \leavevmode\setbox\z@\hbox\bgroup$\@arrayleft\vtop\bgroup%
    }%
    \def\tab@right{%
      \egroup%
      \m@th\@arrayright$\egroup%
      \tab@raisebase%
      \spewnotes%
    }%
    \gdef\tab@hlstate{t}%
    \global\tab@endheight\ht\@arstrutbox%
  \else\if#1b%
    \def\tab@left{%
      \savenotes%
      \leavevmode\setbox\z@\hbox\bgroup$\@arrayleft\vbox\bgroup%
    }%
    \def\tab@right{%
      \egroup%
      \m@th\@arrayright$\egroup%
      \tab@lowerbase%
      \spewnotes%
    }%
    \gdef\tab@hlstate{b}%
  \else%
    \if#1L\@tempswatrue\fi%
    \if#1C\@tempswatrue\fi%
    \if#1R\@tempswatrue\fi%
  \fi\fi%
%    \end{macrocode}
%
% Now for some tests to make sure we're allowed to do the unboxing.  We text
% for |\@arrayleft| being defined, because people trying to hook us won't
% understand unboxed tabulars.
%
%    \begin{macrocode}
  \if@tempswa\ifhmode%
    \ifinner\tab@err@unbrh\@tempswafalse\else\par\fi%
  \fi\fi%
  \if@tempswa\ifmmode\tab@err@unbmm\@tempswafalse\fi\fi%
  \if@tempswa\ifx\@arrayleft\relax\else%
    \tab@err@unbext\@tempswafalse%
  \fi\fi%
%    \end{macrocode}
%
% Finally, if we're still doing an unboxed alignment, we need to sort out the
% spacing.  We know that no-one's tried to hook on to the environment, so we
% clear |\tab@left| and |\tab@right|.
%
%    \begin{macrocode}
  \if@tempswa%
    \def\tab@left{\vskip\parskip\medskip}%
    \def\tab@right{\par\@endpetrue\global\@ignoretrue}%
%    \end{macrocode}
%
% Now we need to sort out the alignment.  The only way we can do this is by
% playing with tabskip glue.  There are two possiblities:
%
% \begin{itemize}
%
% \item If this is a straight \env{tabular} or an \env{array}, we just use
%       infinite glue.  This is reasonable, I think.
%
% \item If we have a width for the table, we calculate the fixed values of
%       glue on either side.  This is fairly easy, and forces the table to
%       the required width.
%
% \end{itemize}
%
% First, set up the left and right glues to represent the prevailing
% margins set up by \env{list} environments.  I think this is the right
% thing to do.
%
%    \begin{macrocode}
    \tab@leftskip\@totalleftmargin%
    \tab@rightskip\hsize%
    \advance\tab@rightskip-\linewidth%
    \advance\tab@rightskip-\@totalleftmargin%
%    \end{macrocode}
%
% First of all, deal with the simple case.  I'm using 10000\,fill glue here,
% in an attempt to suppress |\extracolsep| glue from making the table the
% wrong width.  It can always use filll glue if it really needs to, though.
%
%    \begin{macrocode}
    \ifdim\tab@width=\z@%
      \if#1L\else\advance\tab@leftskip\z@\@plus10000fill\fi%
      \if#1R\else\advance\tab@rightskip\z@\@plus10000fill\fi%
%    \end{macrocode}
%
% Now for the fun bit.  This isn't too hard really.  The extra space I must
% add around the table adds up to $|\linewidth| - |\tab@width|$.  I just
% need to add this onto the appropriate sides of the table.
%
%    \begin{macrocode}
    \else%
      \dimen@\linewidth%
      \advance\dimen@-\tab@width%
      \if#1L\advance\tab@rightskip\dimen@\fi%
      \if#1R\advance\tab@leftskip\dimen@\fi%
      \if#1C%
        \advance\tab@leftskip.5\dimen@%
        \advance\tab@rightskip.5\dimen@%
      \fi%
    \fi%
%    \end{macrocode}
%
% Don't allow page breaks.  David Carlisle's wonderful \env{longtable}
% package does page breaks far better than I could possibly do here, and
% we're compatible with it (wahey!).
%
%    \begin{macrocode}
    \def\tab@penalty{\penalty\@M}%
%    \end{macrocode}
%
% Finally, set the new width of the table, and leave.
%
%    \begin{macrocode}
    \tab@width\hsize%
  \fi%
}
%    \end{macrocode}
%
% \end{macro}
%
% \subsubsection{Handling tops and bottoms}
%
% This is how the tops and bottoms of tables are made to line up with the
% text on the same line, in the presence of arbitrary rules and space.  The
% old method, based on the way the \package{array} package worked, wasn't
% terribly good.  This new version copes much better with almost anything
% that gets thrown at it.
%
% I'll keep a state in a macro (|\tab@hlstate|), which tells me what I'm
% meant to be doing.  The possible values are \lit{n}, which means I don't
% have to do anything, \lit{t}, which means that I'm meant to be handling
% top-aligned tables, and \lit{b}, which means that I'm meant to be lining
% up the bottom.  There are several other `substates' which have various
% magic meanings.
%
%    \begin{macrocode}
\def\tab@hlstate{n}
%    \end{macrocode}
%
% When all's said and done, I extract the box containing the table, and
% play with the height and depth to try and make it correct.
%
% \begin{macro}{\tab@addruleheight}
%
% This macro is called by `inter-row' things to add their height to our
% dimen register.
%
% Only do this if the state indicates that it's sensible.
%
%    \begin{macrocode}
\def\tab@addruleheight#1{%
  \if\tab@hlstate n\else%
    \global\advance\tab@endheight#1\relax%
  \fi%
}
%    \end{macrocode}
%
% \end{macro}
%
% \begin{macro}{\tab@startrow}
%
% This is called at the start of a row, from within the array preamble.
% Currently, this assumes that the rows aren't bigger than their struts:
% this is reasonable, although slightly limiting, and it could be done better
% if I was willing to rip the alignment apart and put it back together
% again.
%
%    \begin{macrocode}
\def\tab@startrow{%
  \if\tab@hlstate t%
    \gdef\tab@hlstate{n}%
  \else\if\tab@hlstate b%
    \global\tab@endheight\dp\@arstrutbox%
  \fi\fi%
}
%    \end{macrocode}
%
% \end{macro}
%
% \begin{macro}{\tab@raisebase}
%
% This macro is called at the end of it all, to set the height and depth
% of the box correctly.  It sets the height to |\tab@endheight|, and the
% depth to everything else.  The box is in |\box|~0 currently.
%
%    \begin{macrocode}
\def\tab@raisebase{%
  \global\advance\tab@endheight-\ht\z@%
  \raise\tab@endheight\box\z@%
}
%    \end{macrocode}
%
% \end{macro}
%
% \begin{macro}{\tab@lowerbase}
%
% And, for symmetry's sake, here's how to set the bottom properly instead.
%
%    \begin{macrocode}
\def\tab@lowerbase{%
  \global\advance\tab@endheight-\dp\z@%
  \lower\tab@endheight\box\z@%
}
%    \end{macrocode}
%
% \end{macro}
%
%
% \subsection{Breaking tables into bits}
%
% Unboxed tables have a wonderful advantage over boxed ones: you can stop
% halfway through and do something else for a bit.  Here's how:
%
% \begin{macro}{\tabpause}
%
% I'd like to avoid forbidding catcode changes here.  I'll use |\doafter|
% now I've got it, to ensure that colour handling and things occur
% \emph{inside} the |\noalign| (otherwise they'll mess up the alignment
% very seriously).
%
% We have to be careful here to ensure that everything works correctly within
% lists.  (The \package{amsmath} package had this problem in its
% |\intertext| macro, so I'm not alone here.)
%
%    \begin{macrocode}
\def\tabpause#{%
  \noalign{\ifnum0=`}\fi%
  \@parboxrestore%
  \tab@startpause%
  \vskip-\parskip%
  \parshape\@ne\@totalleftmargin\linewidth%
  \noindent%
  \doafter\tabpause@i%
}
\def\tabpause@i{%
  \nobreak%
  \tab@endpause%
  \ifnum0=`{\fi}%
}
%    \end{macrocode}
%
% \end{macro}
%
%
% \subsection{The wonderful world of \cmd\multicolumn}
%
% \begin{macro}{\multicolumn}
%
% This is actually fantasitcally easy.  Watch and learn.  Make sure you
% notice the |\long|s here: remember that some table cells can contain
% paragraphs, so it seems sensible to allow |\par| into the argument.
% (As far as I know, most other |\multicolumn| commands don't do this,
% which seems a little silly.  Then again, I forgot to do it the first
% time around.)
%
%    \begin{macrocode}
\long\def\multicolumn#1#2#3{%
  \multispan{#1}%
  \begingroup%
    \tab@multicol%
    \tab@initread%
    \tab@preamble{}%
    \long\def\tab@midtext{#3}%
    \let\tab@looped\tab@err@multi%
    \tab@readpreamble{#2}%
    \the\tab@preamble%
  \endgroup%
  \ignorespaces%
}
%    \end{macrocode}
%
% \end{macro}
%
%
% \subsection{Interlude: range lists}
%
% For processing arguments to |\vgap| and |\cline|, we need to be able to
% do things with lists of column ranges.  To save space, and to make my
% fingers do less typing, here's some routines which do range handling.
%
% \begin{macro}{\ranges}
%
% Given a macro name and a comma separated list of ranges and simple numbers,
% this macro will call the macro giving it each range in the list in turn.
% Single numbers~$n$ will be turned into ranges $n$--$n$.
%
% The first job is to read the macro to do (which may already have some
% arguments attached to it).  We'll also start a group to make sure that
% our changes to temp registers don't affect anyone else.
%
% There's a space before the delimiting |\q@delim| to stop numbers being
% parsed to far and expanding our quark (which will stop \TeX\ dead in its
% tracks).  Since we use |\@ifnextchar| to look ahead, spaces in range lists
% are perfectly all right.
%
%    \begin{macrocode}
\def\ranges#1#2{%
  \gdef\ranges@temp{#1}%
  \begingroup%
  \ranges@i#2 \q@delim%
}
%    \end{macrocode}
%
%
% We're at the beginning of the list.  We expect either the closing marker
% (if this is an empty list) or a number, which we can scoop up into a
% scratch register.
%
%    \begin{macrocode}
\def\ranges@i{%
  \@ifnextchar\q@delim\ranges@done{\afterassignment\ranges@ii\count@}%
}
%    \end{macrocode}
%
% We've read the first number in the range.  If there's another number, we'll
% expect a `|-|' sign to be next.  If there is no `|-|', call the user's code
% with the number duplicated and then do the rest of the list.
%
%    \begin{macrocode}
\def\ranges@ii{%
  \@ifnextchar-\ranges@iii{\ranges@do\count@\count@\ranges@v}%
}
%    \end{macrocode}
%
% Now we strip the `|-|' off and read the other number into a temporary
% register.
%
%    \begin{macrocode}
\def\ranges@iii-{\afterassignment\ranges@iv\@tempcnta}
%    \end{macrocode}
%
% We have both ends of the range now, so call the user's code, passing it
% both ends of the range.
%
%    \begin{macrocode}
\def\ranges@iv{\ranges@do\count@\@tempcnta\ranges@v}
%    \end{macrocode}
%
% We've finished doing an item now.  If we have a `|,|' next, then start
% over with the next item.  Otherwise, if we're at the end of the list,
% we can end happily.  Finally, if we're totally confused, raise an
% error.
%
%    \begin{macrocode}
\def\ranges@v{%
  \@ifnextchar,%
    \ranges@vi%
    {%
      \@ifnextchar\q@delim%
        \ranges@done%
        {\tab@err@range\ranges@vi,}%
    }%
}
%    \end{macrocode}
%
% We had a comma, so gobble it, read the next number, and go round again.
%
%    \begin{macrocode}
\def\ranges@vi,{\afterassignment\ranges@ii\count@}
%    \end{macrocode}
%
% Here's how we call the user's code, now.  We close the group, so that the
% user's code doesn't have to do global things to remember its results, and
% we expand the two range ends from their count registers.  We also ensure
% that the range is the right way round.
%
%    \begin{macrocode}
\def\ranges@do#1#2{%
  \ifnum#1>#2\else%
    \expandafter\endgroup%
    \expandafter\ranges@temp%
    \expandafter{%
    \the\expandafter#1%
    \expandafter}%
    \expandafter{%
    \the#2%
    }%
    \begingroup%
  \fi%
}
%    \end{macrocode}
%
% And finishing the scan is really easy.  We close the group after gobbling
% the close token.
%
%    \begin{macrocode}
\def\ranges@done\q@delim{\endgroup}
%    \end{macrocode}
%
% \end{macro}
%
% \begin{macro}{\ifinrange}
%
% Something a little more useful, now.  |\ifinrange| takes four arguments:
% a number, a range list (as above), and two token lists which I'll call
% \emph{then} and \emph{else}.  If the number is in the list, I'll do
% \emph{then}, otherwise I'll do \emph{else}.
%
%    \begin{macrocode}
\def\ifinrange#1#2{%
  \@tempswafalse%
  \count@#1%
  \ranges\ifinrange@i{#2}%
  \if@tempswa%
    \expandafter\@firstoftwo%
  \else%
    \expandafter\@secondoftwo%
  \fi%
}
\def\ifinrange@i#1#2{%
  \ifnum\count@<#1 \else\ifnum\count@>#2 \else\@tempswatrue\fi\fi%
}
%    \end{macrocode}
%
% \end{macro}
%
%
% \subsection{Horizontal rules OK}
%
% This is where all the gubbins for |\vgap| and friends is kept, lest it
% contaminate fairly clean bits of code found elsewhere.
%
% \subsubsection{Drawing horizontal rules}
%
% \begin{macro}{\hline}
%
% Note the funny use of |\noalign| to allow \TeX\ stomach ops like
% |\futurelet| without starting a new table row.  This lets us see if there's
% another |\hline| coming up, so we can see if we need to insert extra
% vertical space.
%
%    \begin{macrocode}
\def\hline{%
  \tab@dohline%
  \noalign{\ifnum0=`}\fi%
  \tab@penalty%
  \futurelet\@let@token\hline@i%
}
%    \end{macrocode}
%
% We check here for another |\hline| command, and insert glue if there is.
% This looks terrible, though, and |\hlx{hvh}| is much nicer.  Still\dots
%
%    \begin{macrocode}
\def\hline@i{%
  \ifx\@let@token\hline%
    \vskip\doublerulesep%
    \tab@addruleheight\doublerulesep%
  \fi%
  \ifnum0=`{\fi}%
}
%    \end{macrocode}
%
% \end{macro}
%
% \begin{macro}{\tab@dohline}
%
% This is where hlines actually get drawn.  
% Drawing lines is more awkward than it used to be, particularly in unboxed
% tables.  It used to be a case simply of saying |\noalign{\hrule}|.
% However, since unboxed tables are actually much wider than they look, this
% would make the rules stretch right across the page and look generally
% horrible.
%
% The solution is simple: we basically do a dirty big |\cline|.
%
%    \begin{macrocode}
\def\tab@dohline{%
  \multispan{\tab@columns}%
  \leaders\hrule\@height\arrayrulewidth\hfil%
  \tab@addruleheight\arrayrulewidth%  
  \cr%
}
%    \end{macrocode}
%
% \end{macro}
%
% \subsubsection{Vertical rules}
%
% I couldn't fit these in anywhere else, so they'll have to go here.  I'll
% provide a new optional argument which specifies the width of the rule; this
% gets rid of the problem described in the \emph{Companion}, where to get
% an unusually wide vertical rule, you have to play with things like
% \syntax{"\\vrule width" <dimen>} which really isn't too nice.
%
% \begin{macro}{\vline}
%
% The new |\vline| has an optional argument which gives the width of the
% rule.  The |\relax| stops \TeX\ trying to parse a \<rule-specification> for
% too long, in case someone says something like `|\vline depthcharges|' or
% something equally unlikely.
%
%    \begin{macrocode}
\renewcommand\vline[1][\arrayrulewidth]{\vrule\@width#1\relax}
%    \end{macrocode}
%
% \end{macro}
%
% \subsubsection{Drawing bits of lines}
%
% Just for a bit of fun, here's an extended version of |\cline| which takes
% a list of columns to draw lines under, rather than just a single range.
%
% \begin{macro}{\cline}
%
% Not a single line of code written yet, and we already have a dilemma on
% our hands.  Multiple consecutive |\cline| commands are meant to draw
% on the same vertical bit of table.  But horizontal lines are meant to have
% thickness now.  Oh, well [sigh], we'll skip back on it after all.
%
% Now the problem remains how best to do the job.  The way I see it, there
% are three possibilities:
%
% \begin{itemize}
%
% \item We can start a table row, and then for each column of the table
%       (as recorded in |\tab@columns|) we look to see if that column is
%       listed in the range list and if so draw the rule.  This requires
%       lots of scanning of the range list.
%
% \item We can take each range in the list, and draw rules appropriately,
%       just like the old |\cline| used to do, and starting a new table row
%       for each.
%
% \item We can start a table row, and then for each range remember where we
%       stopped drawing the last row, move to the start of the new one, and
%       draw it.  If we start moving backwards, we close the current row
%       and open a new one.
%
% \end{itemize}
%
% The last option looks the most efficient, and the most difficult.  This
% is therefore what I shall do |;-)|.
%
% The first thing to do is to add in a little negative space, and start a
% table row (omitting the first item).  Then scan the range list, and finally
% close the table row and add some negative space again.
%
% We need a global count register to keep track of where we are.  Mixing
% local and global assignments causes all sorts of tragedy, so I shall hijack
% |\tab@state|.
%
%    \begin{macrocode}
\def\cline#1{%
  \noalign{\kern-.5\arrayrulewidth\tab@penalty}%
  \omit%
  \global\tab@state\@ne%
  \ranges\cline@i{#1}%
  \cr%
  \noalign{\kern-.5\arrayrulewidth\tab@penalty}%
}
%    \end{macrocode}
%
% Now for the tricky bit.  When we're given a range, we look to see if the
% first number is less than |\tab@state|.  If so, we quickly close the
% current row, kern backwards and start again with an |\omit| and reset
% |\tab@state| to 1, and try again.
%
%    \begin{macrocode}
\def\cline@i#1#2{%
  \ifnum#1<\tab@state\relax%
    \tab@@cr%
    \noalign{\kern-\arrayrulewidth\tab@penalty}%
    \omit%
    \global\tab@state\@ne%
  \fi%
%    \end{macrocode}
%
% We are now either at or in front of the column position required.  If
% we're too far back, we must |\hfil&\omit| our way over to the correct%
% column.
%
%    \begin{macrocode}
  \@whilenum\tab@state<#1\do{%
    \hfil\tab@@tab@omit%
    \global\advance\tab@state\@ne%
  }%
%    \end{macrocode}
%
% We've found the start correctly.  We must deal with a tiny problem now:
% if this is not the first table cell, the left hand vertical rule is in the
% column to the left, so our horizontal rule won't match up properly.  So
% we skip back by a bit to compensate.  If there isn't actually a vertical
% rule to line up with, no-one will notice, because the rules are so thin.
% This adds a little touch of quality to the whole thing, which is after all
% the point of this whole exercise.
%
%    \begin{macrocode}
  \ifnum\tab@state>\@ne%
    \kern-\arrayrulewidth%
  \fi%
%    \end{macrocode}
%
% Now we must stretch this table cell to the correct width.
%
%    \begin{macrocode}
  \@whilenum\tab@state<#2\do{%
    \tab@@span@omit%
    \global\advance\tab@state\@ne%
  }%
%    \end{macrocode}
%
% We're ready.  Draw the rule.  Note that this is |\hfill| glue, just in case
% we start putting in |\hfil| glue when we step onto the next cell.
%
%    \begin{macrocode}
  \leaders\hrule\@height\arrayrulewidth\hfill%
}
%    \end{macrocode}
%
% Some alignment primitives are hidden inside macros so they don't get seen
% at the wrong time.  This is what they look like:
%
%    \begin{macrocode}
\def\tab@@cr{\cr}
\def\tab@@tab@omit{&\omit}
\def\tab@@span@omit{\span\omit}
%    \end{macrocode}
%
% \end{macro}
%
% \subsubsection{Drawing short table rows}
%
% Before I start on a description of more code, I think I'll briefly discuss
% my reasons for leaving the |\vgap| command in its current state.  There's a
% reasonable case for introducing an interface between |\vgap| and
% |\multicolumn|, to avoid all the tedious messing about with column
% ranges.  There are good reasons why I'm not going to do this:
%
% \begin{itemize}
%
% \item It's very difficult to do: it requires either postprocessing of
%       the table or delaying processing of each row until I know exactly
%       what's in it; a |\multicolumn| in a row should be able to affect
%       a |\vgap| before the row, which gets very nasty.  This package is
%       probably far too large already, and adding more complexity and
%       running the risk of exhausting \TeX's frustratingly finite capacity
%       for the sake of relieving the user of a fairly trivial job doesn't
%       seem worthwhile.
%
% \item Perhaps more importantly, there are perfectly valid occasions when
%       it's useful to have the current vgap behaviour.  For example, the
%       \texttt{MIX} word layout diagrams found in \emph{The Art of
%       Computer Programming} use the little `stub lines' to show where
%       data items cross byte boundaries:
%
%	^^A This actually looks terrifyingly similar to the original.
%	^^A The leading @{} is there to stop the table looking off-centre,
%	^^A because there's no left hand rule telling you where the table
%	^^A starts, like there is on the right, just the \tabcolsep glue.
%
%	\begingroup
%	\newcommand{\wide}[2]{\multicolumn{#1}{c|}{\ttfamily #2}}
%	\begin{tabular}[C]{@{} r @{\qquad} | Mc | *{5}{c|}} \hlx{c{2-7} v}
%          empty & - & 1 & 0 & 0 & 0 & 0		\\ \hlx{v c{2-7} v}
%	occupied & + & \wide{2}{LINK} & \wide{3}{KEY}	\\ \hlx{v c{2-7}}
%	\end{tabular}
%	\endgroup
%
% \end{itemize}
%
% That's my excuses out of the way; now I'll press on with the actual
% programming.
%
% \begin{macro}{\tab@checkrule}
%
% We have a range list in |\tab@xcols| and a number as an argument.  If we
% find the number in the list, wejust space out the following group,
% otherwise we let it be.
%
%    \begin{macrocode}
\def\tab@checkrule#1{%
  \count@#1\relax%
  \expandafter\ifinrange%
  \expandafter\count@%
  \expandafter{\tab@xcols}%
    {\tab@checkrule@i}%
    {}%
}
\def\tab@checkrule@i#1{\setbox\z@\hbox{#1}\hb@xt@\wd\z@{}}
%    \end{macrocode}
%
% \end{macro}
%
% \begin{macro}{\vgap}
%
% We must tread carefully here.  A single misplaced stomach operation can
% cause error messages.  We therefore start with an |\omit| so we can search
% for optional arguments.
%
% So that |\hlx| can get control after |\vgap| has finished, we provide a
% hook called |\vgap@after| which is expanded after |\vgap| has finished.
% Here we make it work like |\@empty|, which expands to nothing.  (Note that
% |\relax| will start a new table row, so we can't use that.)  There are
% some penalty items here to stick the |\vgap| row to the text row and
% |\hline| that are adjacent to it.  The \package{longtable} package will
% split an |\hline| in half, so this is the correct thing to do.
%
%    \begin{macrocode}
\def\vgap{%
  \noalign{\nobreak}%
  \omit%
  \global\let\vgap@after\@empty%
  \iffalse{\fi\ifnum0=`}\fi%
  \@ifnextchar[\vgap@i\vgap@simple%
}
%    \end{macrocode}
%
% We set up two different sorts of |\vgap| -- a simple one which allows all
% rules to be passed through, and a specific one which carefully vets each
% one (and is therefore slower).  We decide which to so based on the presence
% of an optional argument.
%
% The optional argument handler just passes its argument to an interface
% routine which is used by |\hlx|.
%
%    \begin{macrocode}
\def\vgap@i[#1]{\vgap@spec{#1}}
%    \end{macrocode}
%
% Now we handle specified columns.  Since we're in an omitted table cell, we
% must set things up globally.  Assign the column spec to a macro, and set up
% vetting by the routine above.  Then just go and do the job.
%
%    \begin{macrocode}
\def\vgap@spec#1#2{%
  \gdef\tab@xcols{#1}%
  \global\let\tab@ckr\tab@checkrule%
  \vgap@do{#2}%
}
%    \end{macrocode}
%
% Handle all columns.  Just gobble the column number for each rule, and let
% the drawing pass unharmed.  Easy.
%
%    \begin{macrocode}
\def\vgap@simple#1{%
  \global\let\tab@ckr\@gobble%
  \vgap@do{#1}%
}
%    \end{macrocode}
%
% This is where stuff actually gets done.  We set the |\vgap| flag on while
% we do the short row.  Then just expand the token list we built while
% scanning the preamble.
%
% Note that the flag is cleared at the end of the last column, to allow other
% funny things like |\noalign| and |\omit| before a new row is started.
%
%    \begin{macrocode}
\def\vgap@do#1{%
  \ifnum0=`{}\fi%
  \global\tab@vgaptrue%
  \the\tab@shortline%
    \vrule\@height#1\@width\z@%
    \global\tab@vgapfalse
    \tab@addruleheight{#1}%
    \cr%
  \noalign{\nobreak}%
  \vgap@after%
}
%    \end{macrocode}
%
% \end{macro}
%
% \subsubsection{Prettifying syntax}
%
% \begin{macro}{\hlx}
%
% This is like a poor cousin to the preamble parser.  The whole loop is
% carefully written to take place \emph{only} in \TeX's mouth, so the
% alignment handling bits half way down the gullet don't see any of this.
%
% First, pass the string to another routine.
%
%    \begin{macrocode}
\def\hlx#1{\hlx@loop#1\q@delim}
%    \end{macrocode}
%
% Now peel off a token, and dispatch using |\csname|.  We handle
% undefinedness of the command in a fairly messy way, although it probably
% works.  Maybe.
%
%    \begin{macrocode}
\def\hlx@loop#1{%
  \ifx#1\q@delim\else%
    \@ifundefined{hlx@cmd@\string#1}{%
      \expandafter\hlx@loop%
    }{%
      \csname hlx@cmd@\string#1\expandafter\endcsname%
    }%
  \fi%
}
%    \end{macrocode}
%
% \end{macro}
%
% \begin{macro}{\hlxdef}
%
% New |\hlx| commands can be defined using |\hlxdef|.  This is a simple
% abbreviation.
%
%    \begin{macrocode}
\def\hlxdef#1{\@namedef{hlx@cmd@#1}}
%    \end{macrocode}
%
% \end{macro}
%
% \begin{macro}{\hlx h}
%
% Handle an \lit{h} character.  Just do an |\hline| and return to the loop.
% We look ahead to see if there's another \lit{h} coming up, and if so
% insert two |\hline| commands.  This strange (and inefficient) behaviour
% keeps packages which redefine |\hline| happy.
%
%    \begin{macrocode}
\hlxdef h#1{%
  \noalign{%
    \ifx#1h%
      \def\@tempa{\hline\hline\hlx@loop}%
    \else%
      \def\@tempa{\hline\hlx@loop#1}%
    \fi%
    \expandafter
  }%
  \@tempa%
}
%    \end{macrocode}
%
% \end{macro}
%
% \begin{macro}{\hlx b}
%
% The \lit{b} character does a nifty backspace, for \package{longtable}'s
% benefit.
%
%    \begin{macrocode}
\hlxdef b{\noalign{\kern-\arrayrulewidth}\hlx@loop}
%    \end{macrocode}
%
% \end{macro}
%
% \begin{macro}{\hlx /}
%
% The `"/"' character allows a page break at the current position.
%
%    \begin{macrocode}
\hlxdef /{%
  \noalign{\ifnum0=`}\fi%
  \@ifnextchar[\hlx@cmd@break@i{\hlx@cmd@break@i[0]}%
}
\def\hlx@cmd@break@i[#1]{\ifnum0=`{\fi}\pagebreak[0]\hlx@loop}
%    \end{macrocode}
%
% \end{macro}
%
% \begin{macro}{\hlx v}
%
% Handle a \lit{v} character.  This is rather like the |\vgap| code above,
% although there are syntactic differences.
%
%    \begin{macrocode}
\hlxdef v{%
  \noalign{\nobreak}%
  \omit%
  \iffalse{\fi\ifnum0=`}\fi%
  \global\let\vgap@after\hlx@loop%
  \@ifnextchar[\hlx@vgap@i{\hlx@vgap@ii\vgap@simple}%
}
\def\hlx@vgap@i[#1]{%
  \ifx!#1!%
    \def\@tempa{\hlx@vgap@ii\vgap@simple}%
  \else%
    \def\@tempa{\hlx@vgap@ii{\vgap@spec{#1}}}%
  \fi%
  \@tempa%
}
\def\hlx@vgap@ii#1{%
  \@ifnextchar[{\hlx@vgap@iii{#1}}{\hlx@vgap@iii{#1}[\doublerulesep]}%
}
\def\hlx@vgap@iii#1[#2]{#1{#2}}
%    \end{macrocode}
%
% \end{macro}
%
% \begin{macro}{\hlx s}
%
% Allow the user to leave a small gap using the \lit{s} command.
%
%    \begin{macrocode}
\hlxdef s{%
  \noalign{\ifnum0=`}\fi%
  \nobreak%
  \@ifnextchar[\hlx@space@i{\hlx@space@i[\doublerulesep]}%
}
\def\hlx@space@i[#1]{%
  \vskip#1%
  \tab@addruleheight{#1}%
  \ifnum0=`{\fi}%
  \hlx@loop%
}
%    \end{macrocode}
%
% \end{macro}
%
% \begin{macro}{\hlx c}
%
% We might as well allow a \lit{c} command to do a |\cline|.
%
%    \begin{macrocode}
\hlxdef c#1{\cline{#1}\hlx@loop}
%    \end{macrocode}
%
% \end{macro}
%
% \begin{macro}{\hlx .}
%
% The \lit{.} character forces a start of the new column.  There's a little
% problem here.  Since the \lit{.} character starts the next column, we need
% to gobble any spaces following the |\hlx| command before the cell contents
% actually starts.  Unfortunately, |\ignorespaces| will start the column for
% us, so we can't put it in always.  We'll handle it here, then.  We'll take
% the rest of the `preamble' string, and warn if it's not empty.  Then we'll
% |\ignorespaces| -- this will start the column for us, so we don't need to
% |\relax| any more.
%
%    \begin{macrocode}
\hlxdef .#1\q@delim{%
  \ifx @#1@\else%
    \PackageWarning{mdwtab}{%
      Ignoring \protect\hlx\space command characters following a
      `.'\MessageBreak command%
    }%
  \fi%
  \ignorespaces%
}
%    \end{macrocode}
%
% \end{macro}
%
%
% \subsection{Starting new table rows}
%
% We take a break from careful mouthery at last, and start playing with
% newlines.  The standard one allows pagebreaks in unboxed tables, which
% isn't really too desirable.
%
% Anyway, we'll try to make this macro rather more reusable than the standard
% one.  Here goes.
%
% \begin{macro}{\@arraycr}
%
% We pass lots of information to a main parser macro, and expect it to cope.
%
%    \begin{macrocode}
\def\@arraycr{\tab@arraycr{}}
\def\tab@arraycr#1{\tab@cr{\tab@tabcr{#1}}{}{}}
%    \end{macrocode}
%
% Now to actually do the work.  |\tab@cr| passes us the skip size, and the
% appropriate one of the two arguments given above (both of which are empty)
% depending on the presence of the $*$.
%
%    \begin{macrocode}
\def\tab@tabcr#1#2{%
%    \end{macrocode}
%
% If the total height I need to add between rows (from the optional argument
% and the `extrasep' parameter) is greater than zero, I'll handle this by
% extending the strut slightly.  I'm not actually sure whether this is the
% right thing to do, to be honest, although it's easier than trying to
% to an automatic |\vgap|, because I need to know which columns to skip.
% If the space is less than zero, I'll just insert the vertical space with
% in a |\noalign|.
%
% First, to calculate how much space needs adding.
%
%    \begin{macrocode}
  \dimen@#2%
  \advance\dimen@\tab@extrasep%
%    \end{macrocode}
%
% If the height is greater than zero, I need to play with the strut.  I must
% bear in mind that the current table cell (which I'm still in, remember)
% may be in vertical mode, and I may or may not be in a paragraph.
%
% If I am in vertical mode, I'll backpedal to the previous box and put the
% strut in an hbox superimposed on the previous baseline.  Otherwise, I can
% just put the strut at the end of the text.  (This works in either LR
% or paragraph mode as long as I'm not between paragraphs.)  Again, Rowland's
% empty cell bug strikes.  (See |\tab@epar| for details.)
%
%    \begin{macrocode}
  \ifdim\dimen@>\z@%
    \ifvmode%
      \unskip\ifdim\prevdepth>-\@m\p@\kern-\prevdepth\fi%
      \nointerlineskip\expandafter\hbox%
    \else%
      \@maybe@unskip\expandafter\@firstofone%
    \fi%
    {\advance\dimen@\dp\@arstrutbox\vrule\@depth\dimen@\@width\z@}%
  \fi%
%    \end{macrocode}
%
% This table cell works as a group (which is annoying here).  I'll copy the
% interrow gap into a global register so that I can use it in the |\noalign|.
%
%    \begin{macrocode}
  \global\dimen\@ne\dimen@%
  \cr%
  \noalign{%
    #1%
    \ifdim\dimen\@ne<\z@\vskip\dimen\@ne\relax\fi%
  }%
  \@gobble%
}
%    \end{macrocode}
%
% \end{macro}
%
% \begin{macro}{\tab@setcr}
%
% To set the |\\| command correctly in each table cell, we make it a part of
% the preamble (in |\tab@midtext|) to call this routine.  It's easy -- just
% saves the preamble from being huge.
%
%    \begin{macrocode}
\def\tab@setcr{\let\\\tabularnewline}
%    \end{macrocode}
%
% \end{macro}
%
% \begin{macro}{\tab@cr}
%
% Now we do the parsing work.  This is fun.  Note the revenge of the funny
% braces here.  Nothing to worry about, honest.  The tricky bit is to keep
% track of which arguments are which.  (Thanks to David Carlisle for pointing
% out that I'd missed out the |\relax| here.)
%
%    \begin{macrocode}
\def\tab@cr#1#2#3{%
  \relax%
  \iffalse{\fi\ifnum0=`}\fi%
  \@ifstar{\tab@cr@i{#1}{#3}}{\tab@cr@i{#1}{#2}}%
}
\def\tab@cr@i#1#2{%
  \@ifnextchar[{\tab@cr@ii{#1}{#2}}{\tab@cr@ii{#1}{#2}[\z@]}%
}
\def\tab@cr@ii#1#2[#3]{%
  \ifnum0=`{}\fi%
  #1{#3}{#2}%
}
%    \end{macrocode}
%
% \end{macro}
%
%
% \subsection{Gratuitous grotesquery}
%
% So far we've had an easy-ish ride (or should that be \emph{queasy}?).  Now
% for something unexplainably evil.  We convince \LaTeX\ that it's loaded the
% \package{array} package, so that packages which need it think they've got
% it.
%
% The bogus date is the same as the date for the \package{array} package I've
% got here -- this will raise a warning if Frank updates his package which
% should filter back to me telling me that there's something I need to
% know about.
%
% The messing with |\xdef| and the funny parsing ought to insert the current
% \package{mdwtab} version and date into the fake \package{array} version
% string, giving a visible clue to the user that this isn't the real
% \package{array} package.
%
%    \begin{macrocode}
\begingroup
\catcode`.=11
\def\@tempa#1 #2 #3\@@{#1 #2}
\xdef\ver@array.sty
  {1995/11/19 [mdwtab.sty \expandafter\@tempa\ver@mdwtab.sty\@@]}
\endgroup
%    \end{macrocode}
%
%
% \subsection{Error messages}
%
% I've put all the error messages together, where I can find them, translate
% them or whatever.
%
% First, some token-space saving (which also saves my fingers):
%
%    \begin{macrocode}
\def\tab@error{\PackageError{mdwtab}}
%    \end{macrocode}
%
% Now do the error messages.
%
%    \begin{macrocode}
\def\tab@err@misscol{%
  \tab@error{Missing column type}{%
    I'm lost.  I was expecting something describing^^J%
    the type of the current column, but you seem to^^J%
    have missed it out.  I've inserted a type `l'^^J%
    column here in the hope that this makes sense.%
  }%
}
%    \end{macrocode}
%
%    \begin{macrocode}
\def\tab@err@oddgroup{%
  \tab@error{Misplaced group in table preamble}{%
    I've found an open brace character in your preamble^^J%
    when I was expecting a specifier character.  I'm^^J%
    going to gobble the whole group and carry on as if^^J%
    I'd never seen it.%
  }%
}
%    \end{macrocode}
%
%    \begin{macrocode}
\def\tab@err@undef#1{%
  \tab@error{Unknown `\tab@colset' preamble character `\string#1'}{%
    I don't understand what you meant by typing this^^J%
    character.  Anyway, I'll ignore it this time around.^^J%
    Just don't you do it again.%
  }%
}
%    \end{macrocode}
%
%    \begin{macrocode}
\def\tab@err@unbrh{%
  \tab@error{Can't use unboxed tabular in LR mode}{%
    You've asked for a tabular or array environment with^^J%
    `L', `C' or `R' as the position specifier, but you're^^J%
    in LR (restricted horizontal) mode, so it won't work.^^J%
    I'll assume you really meant `c' and soldier on.%
  }%
}
%    \end{macrocode}
%
%    \begin{macrocode}
\def\tab@err@unbmm{%
  \tab@error{Can't use unboxed tabular in maths mode}{%
    You've asked for a tabular or array environment with^^J%
    `L', `C' or `R' as the position specifier, but you're^^J%
    in maths mode, so it won't work.  I'll pretend that^^J%
    you really typed `c', and that this is all a bad dream.%
  }%
}
%    \end{macrocode}
%
%    \begin{macrocode}
\def\tab@err@unbext{%
  \tab@error{Can't extend unboxed tabulars}{%
    You're trying to use kludgy extensions (e.g.,^^J%
    `delarray') on an array or tabular with `L', `C'^^J%
    or `R' as the position specifier.  I'll assume you^^J%
    subconsciously wanted a `c' type all along.%
  }%
}
%    \end{macrocode}
%
%    \begin{macrocode}
\def\tab@err@multi{%
  \tab@error{More than one column in a \protect\multicolumn}{%
    You've put more than one column into a \string\multicolumn^^J%
    descriptor.  It won't work.  I have no idea what^^J%
    will happen, although it won't be pleasant.  Hold^^J%
    on tight now...%
  }%
}
%    \end{macrocode}
%
%    \begin{macrocode}
\def\tab@err@range{%
  \tab@error{Expected `,' or `<end>' in range list}{%
    I was expecting either the end of the range list,^^J%
    or a comma, followed by another range.  I've^^J%
    inserted a comma to try and get me back on track.^^J%
    Good luck.%
  }%
}
%    \end{macrocode}
%
% That's it.  No more.  Move along please.
%
%    \begin{macrocode}
%</mdwtab>
%    \end{macrocode}
%
%
%^^A-------------------------------------------------------------------------
% \section{Implementation of \package{mathenv}}
%
%
% This is in a separate package, mainly to avoid wasting people's memory.
%
%    \begin{macrocode}
%<*mathenv>
%    \end{macrocode}
%
%
% \subsection{Options handling}
%
% We need to be able to cope with \textsf{fleqn} and \textsf{leqno} options.
% This will adjust our magic modified \env{eqnarray} environment
% appropriately.
%
%    \begin{macrocode}
\newif\if@fleqn
\newif\if@leqno
\DeclareOption{fleqn}{\@fleqntrue}
\DeclareOption{leqno}{\@leqnotrue}
\ProcessOptions
%    \end{macrocode}
%
% We use the \package{mdwtab} package for all its nice table handling things.
% (Oh, and to inflict it on users who want to do nice equations and don't
% care about our tables.)
%
%    \begin{macrocode}
\RequirePackage{mdwtab}
%    \end{macrocode}
%
%
% \subsection{Some useful registers}
%
% The old \LaTeX\ version puts the equation numbers in by keeping a count of
% where it is in the alignment.  Since I don't know how may columns there are
% going to be, I'll just use a switch in the preamble to tell me to stop
% tabbing.
%
%    \begin{macrocode}
\newif\if@eqalast
%    \end{macrocode}
%
% Now define some useful length parameters.  First allocate them:
%
%    \begin{macrocode}
\newskip\eqaopenskip
\newskip\eqacloseskip
\newskip\eqacolskip
\newskip\eqainskip
\newskip\splitleft
\newskip\splitright
%    \end{macrocode}
%
% Now assign some default values.  Users can play with these if they really
% want although I can't see the point myself.
%
%    \begin{macrocode}
\AtBeginDocument{%
  \eqacloseskip\@centering%
  \eqacolskip1.5em\@plus\@m\p@
  \eqainskip\z@%
  \if@fleqn%
    \eqaopenskip\mathindent%
    \splitleft\mathindent\relax%
    \splitright\mathindent\@minus\mathindent\relax%  
  \else%
    \eqaopenskip\@centering%
    \splitleft2.5em\@minus2.5em%
    \splitright\splitleft%
  \fi%
  \relax%
}
%    \end{macrocode}
%
%
% \subsection{A little display handling}
%
% I'm probably going a little far here, and invading territory already
% claimed by the \package{amsmath} stuff (and done a good deal better than
% I can be bothered to do), but just for completeness, this is how we handle
% attempts to put displays inside other displays without screwing up the
% spacing.
%
% \begin{macro}{\dsp@startouter}
%
% This is how we start an outermost display.  It's fairly easy really.  We
% make |\dsp@start| start an inner display, and make |\dsp@end| close the
% outer display.
%
%    \begin{macrocode}
\def\dsp@startouter{%
  \let\dsp@end\dsp@endouter%
  $$%
}
%    \end{macrocode}
%
% \end{macro}
%
% \begin{macro}{\dsp@endouter}
%
% Ending the outer display is utterly trivial.
%
%    \begin{macrocode}
\def\dsp@endouter{$$}
%    \end{macrocode}
%
% \end{macro}
%
% \begin{macro}{\dsp@startinner}
%
% Starting inner displays is done in a vbox (actually I choose |\vbox| or
% |\vtop| depending on the setting of \textsf{leqno} to put the equation
% number the right way round).
%
%    \begin{macrocode}
\def\dsp@startinner{%
  \let\dsp@end\dsp@endinner%
  \if@fleqn\kern-\mathindent\fi%
  \if@leqno\vtop\else\vtop\fi\bgroup%
}
%    \end{macrocode}
%
% \end{macro}
%
% \begin{macro}{\dsp@endinner}
%
% Ending an inner display is also really easy.
%
%    \begin{macrocode}
\def\dsp@endinner{\egroup}
%    \end{macrocode}
%
% \end{macro}
%
% \begin{macro}{\dsp@start}
%
% This is what other bits of code uses to start displays.  It's one of the
% start macros up above, and outer by default.
%
%    \begin{macrocode}
\def\dsp@start{%
  \ifmmode%
    \ifinner\mth@err@mdsp\fi%
    \expandafter\dsp@startinner%
  \else%
    \ifhmode\ifinner\mth@err@hdsp\fi\fi%
    \expandafter\dsp@startouter%
  \fi%
}
%    \end{macrocode}
%
% \end{macro}
%
% \begin{macro}{\dsp@tabpause}
%
% This sets up the correct pre- and postambles for the |\tabpause| macro in
% maths displays.  This is fairly simple stuff.
%
%    \begin{macrocode}
\def\dsp@tabpause{%
  \def\tab@startpause%
    {\penalty\postdisplaypenalty\vskip\belowdisplayskip}%
  \def\tab@endpause%
    {\penalty\predisplaypenalty\vskip\abovedisplayskip}%
}
%    \end{macrocode}
%
% \end{macro}
%
%
% \subsection{The \env{eqnarray} environment}
%
% We allow the user to play with the style if this is really wanted.  I dunno
% why, really.  Maybe someone wants very small alignments.
%
%    \begin{macrocode}
\let\eqastyle\displaystyle
%    \end{macrocode}
%
% \subsubsection{The main environments}
%
% \begin{environment}{eqnarray}
% \begin{environment}{eqnarray*}
%
% We define the toplevel commands here.  They just add in default arguments
% and then call |\@eqnarray| with a preamble string.  We handle equation
% numbers by setting up a default (|\eqa@defnumber|) which is put into
% the final column.  At the beginning of each row, we globally |\let|
% |\eqa@number| equal to |\eqa@defnumber|.  The |\eqnumber| macro just
% changes |\eqa@number| as required.  Since |\eqa@number| is changed globally
% we must save it in this environment.
%
% First, we must sort out the optional arguments and things.  This is really
% easy.  The only difference between the starred and non-starred environments
% is the default definition of |\eqa@defnumber|.
%
%    \begin{macrocode}
\def\eqnarray{%
  \eqnarray@i\eqa@eqcount%
}
\@namedef{eqnarray*}{\eqnarray@i{}}
\def\eqnarray@i#1{\@ifnextchar[{\eqnarray@ii{#1}}{\eqnarray@ii{#1}[rcl]}}
%    \end{macrocode}
%
% Right.  Now for the real work.  The first argument is the default numbering
% tokens; the second is the preamble string.
%
%    \begin{macrocode}
\def\eqnarray@ii#1[#2]{%
%    \end{macrocode}
%
% Set up the equation counter and labels correctly.
%
% \medskip\par\noindent|\begin{rant}|\par
% The hacking with |\@currentlabel| is here because (in the author's opinion)
% \LaTeX's |\refstepcounter| macro is broken.  It's currently defined as
% \begin{listing}
%\def\refstepcounter#1{%
%  \stepcounter{#1}%
%  \protected@edef\@currentlabel%
%    {\csname p@#1\endcsname\csname the#1\endcsname}%
%}
% \end{listing}
% which means that the current label gets `frozen' as soon as you do the
% counter step.  By redefining the macro as
% \begin{listing}
%\def\refstepcounter#1{%
%  \stepcounter{#1}%
%  \edef\@currentlabel{%
%    \expandafter\noexpand\csname p@#1\endcsname%
%    \expandafter\noexpand\csname the#1\endcsname%
%  }%
%}
% \end{listing}
% these sorts of problems would be avoided, without any loss of functionality
% or compatibility that I can see.
% \par\noindent|\end{rant}|\par
%
%    \begin{macrocode}
  \stepcounter{equation}%
  \def\@currentlabel{\p@equation\theequation}%
%    \end{macrocode}
%
% The next step is to set up the numbering.  I must save the old numbering
% so I can restore it later (once in the alignment, I must assign these
% things globally).
%
%    \begin{macrocode}
  \let\eqa@oldnumber\eqa@number%
  \def\eqa@defnumber{#1}%
  \global\let\eqa@number\eqa@defnumber%
%    \end{macrocode}
%
% The |\if@eqalastfalse| switch is false everywhere except when we're in the
% final column.
%
%    \begin{macrocode}
  \@eqalastfalse%
%    \end{macrocode}
%
% Remove the |\mathsurround| kerning, since it will look very odd inside
% the display.  We have our own spacing parameters for configuring these
% things, so |\mathsurround| is unnecessary.
%
%    \begin{macrocode}
  \m@th%
%    \end{macrocode}
%
% Time to parse the preamble string now.  I must choose the correct column
% set, initialise the preamble parser and set up the various macros.  The%
% extra `|@{\tabskip\eqacloseskip}|' item sets up the tabskip glue to centre
% the alignment properly.
%
%    \begin{macrocode}
  \colset{eqnarray}%
  \tab@initread%
  \def\tab@tabtext{&\tabskip\z@skip}%
  \tab@preamble{\tabskip\z@skip}%
  \tab@readpreamble{#2@{\tabskip\eqacloseskip}}%
  \dsp@tabpause%
%    \end{macrocode}
%
% Now for some final setting up.  The column separation is set from the
% user's parameter, the |\everycr| tokens are cleared, and I set up the
% newline command appropriately.
%
%    \begin{macrocode}
  \col@sep.5\eqainskip%
  \everycr{}%
  \let\\\@eqncr%
%    \end{macrocode}
%
% Now start a maths display and do the alignment.  Set up the left hand
% tabskip glue to centre the alignment, and do the actual alignment.
% The preamble used is mainly that generated from the user's string, although
% the stuff at the end is how we set up the equation number -- it repeats
% appropriately so we can always find it.
%
%    \begin{macrocode}
  \dsp@start%
  \tabskip\eqaopenskip%
  \halign to\displaywidth\expandafter\bgroup%
    \the\tab@preamble%
    &&\eqa@lastcol\hb@xt@\z@{\hss##}\tabskip\z@\cr%
}
%    \end{macrocode}
%
% Now for the end of the environment.  This is really easy.  Set the final
% equation number, close the |\halign|, tidy up the equation counter (it's
% been stepped once too many times) and close the display.
%
%    \begin{macrocode}
\def\endeqnarray{%
  \eqa@eqnum%
  \egroup%
  \dsp@end%
  \global\let\eqa@number\eqa@oldnumber%
  \global\@ignoretrue%
  \global\advance\c@equation\m@ne%
}
\expandafter\let\csname endeqnarray*\endcsname\endeqnarray
%    \end{macrocode}
%
% \end{environment}
% \end{environment}
%
% Now we can define the column types. 
%
%    \begin{macrocode}
\colpush{eqnarray}
%    \end{macrocode}
%
% Note the positioning of ord atoms in the stuff below.  This will space out
% relations and binops correctly when they occur at the edges of columns, and
% won't affect ord atoms at the edges, because ords pack closely.
%
% First the easy onces.  Just stick |\hfil| in the right places and
% everything will be all right.
%
%    \begin{macrocode}
\coldef r{\tabcoltype{\hfil$\eqastyle}{{}$}}
\coldef c{\tabcoltype{\hfil$\eqastyle{}}{{}$\hfil}}
\coldef l{\tabcoltype{$\eqastyle{}}{$\hfil}}
\coldef x{\tabcoltype{\if@fleqn\else\hfil\fi$\eqastyle}{$\hfil}}
%    \end{macrocode}
%
% Now for the textual ones.  This is also fairly easy.
%
%    \begin{macrocode}
\collet T [tabular]T
%    \end{macrocode}
%
% Sort of split types of equations.  I mustn't use |\rlap| here, or
% everything goes wrong -- |\\| doesn't get noticed by \TeX\ in the same way
% as |\cr| does.
%
%    \begin{macrocode}
\coldef L{\tabcoltype{\hb@xt@2em\bgroup$\eqastyle}{$\hss\egroup}}
%    \end{macrocode}
%
% The \lit{:} column type is fairly simple.
%
%    \begin{macrocode}
\coldef :{\tabspctype{\tabskip\eqacolskip}}
\coldef q{\tabspctype{\quad}}
%    \end{macrocode}
%
% The other column types just insert given text in an appropriate way.
%
%    \begin{macrocode}
\collet > [tabular]>
\collet < [tabular]<
\collet * [tabular]*
\collet @ [tabular]@
%    \end{macrocode}
%
% Finally, the magical `|\magic|' column type, which sets the equation
% number.  We set up the |\tabskip| glue properly, tab on, and set the flag
% which marks the final column.  The |\eqa@lastcol| command is there to
% raise an error if the user tabs over to this column.  I'll temporarily
% redefine it to |\@eqalasttrue| when I enter this column legitimately.
% The extra magical bits here will make the final column repeat, so that we
% can find it if necessary.  Well is this column type named.
%
% That's it.  We can return to normal now.
%
%    \begin{macrocode}
\colpop
%    \end{macrocode}
%
% \subsubsection{Newline codes}
%
% Newline sequences (|\\|) get turned into calls of |\@eqncr|.  The job is
% fairly simple, really.
%
%    \begin{macrocode}
\def\@eqncr{\tab@cr\eqacr@i\interdisplaylinepenalty\@M}%
\def\eqacr@i#1#2{%
  \eqa@eqnum%
  \noalign{\penalty#2\vskip\jot\vskip#1}%
}
%    \end{macrocode}
%
% \subsubsection{Setting equation numbers}
%
% \begin{macro}{\eqa@eqpos}
%
% Before we start, we need to generalise the flush-left number handling bits.
% The macro |\eqa@eqpos| will put its argument in the right place.
%
%    \begin{macrocode}
\if@leqno
  \def\eqa@eqpos#1{%
    \hb@xt@.01\p@{}\rlap{\normalfont\normalcolor\hskip-\displaywidth#1}%
  }
\else
  \def\eqa@eqpos#1{\normalfont\normalcolor#1}
\fi
%    \end{macrocode}
%
% \end{macro}
%
% \begin{macro}{\eqa@eqnum}
%
% Here we typeset an equation number in roughly the right place.  First I'll
% redefine |\eqa@lastcol| so that it tells me I'm in the right place, and
% start a loop to find that place.
%
%    \begin{macrocode}
\def\eqa@eqnum{%
  \global\let\eqa@lastcol\@eqalasttrue%
  \eqa@eqnum@i%
}
%    \end{macrocode}
%
% Now for the loop.  The |\relax| here is absolutely vital -- it starts the
% table column, inserting useful tokens like `|\eqa@lastcol|' which tell
% me where I am in the alignment.  Then, if I've reached the end, I can
% typeset the equation number; otherwise I go off into another macro and
% step on to the next column.
%
%    \begin{macrocode}
\def\eqa@eqnum@i{%
  \relax%
  \if@eqalast%
    \expandafter\eqa@eqnum@ii%
  \else%
    \expandafter\eqa@eqnum@iii%
  \fi%
}
\def\eqa@eqnum@ii{%
  \eqa@eqpos\eqa@number%
  \global\let\eqa@number\eqa@defnumber%
  \global\let\eqa@lastcol\eqa@@lastcol%
  \cr%
}
\def\eqa@eqnum@iii{&\eqa@eqnum@i}
%    \end{macrocode}
%
% \end{macro}
%
% \begin{macro}{\eqa@lastcol}
%
% This is used as a marker for the final column in an \env{eqnarray}
% environment.  By default it informs the user that they've been very
% silly and swallows the contents of the column.  I'll redefine it to
% something more useful at appropriate times, and then turn it back again.
%
%    \begin{macrocode}
\def\eqa@@lastcol{\mth@err@number\setbox\z@}
\let\eqa@lastcol\eqa@@lastcol
%    \end{macrocode}
%
% \end{macro}
%
% \subsubsection{Numbering control}
%
% \begin{macro}{\eqnumber}
%
% The |\eqnumber| command sets the equation number on the current equation.
% This is really easy, actually.
%
%    \begin{macrocode}
\newcommand\eqnumber[1][\eqa@eqcount]{\gdef\eqa@number{#1}}
%    \end{macrocode}
%
% \end{macro}
%
% \begin{macro}{\eqa@eqcount}
%
% This is how a standard equation number is set, stepping the counter and
% all.  It's really easy and obvious.
%
%    \begin{macrocode}
\def\eqa@eqcount{(\theequation)\global\advance\c@equation\@ne}
%    \end{macrocode}
%
% \end{macro}
%
% \begin{macro}{\nonumber}
%
% The \LaTeX\ |\nonumber| command could be defined by saying
% \begin{listing}
%\renewcommand{\nonumber}{\eqnumber[]}
% \end{listing}
% but I'll be slightly more efficient and redefine |\eqa@number| directly.
%
%    \begin{macrocode}
\def\nonumber{\global\let\eqa@number\@empty}
%    \end{macrocode}
%
% \end{macro}
%
% \subsubsection{The \env{eqnalign} environment}
%
% As a sort of companion to \env{eqnarray}, here's an environment which does
% similar things inside a box, rather than taking up the whole display width.
% It uses the same column types that we've already created, so there should
% be no problems.
%
% \begin{environment}{eqnalign}
%
% First, sort out some simple things like optional arguments.
%
%    \begin{macrocode}
\def\eqnalign{\@ifnextchar[\eqnalign@i{\eqnalign@i[rcl]}}
\def\eqnalign@i[#1]{%
  \@ifnextchar[{\eqnalign@ii{#1}}{\eqnalign@ii{#1}[c]}%
}
%    \end{macrocode}
%
% Now we actually do the environment.  This is fairly easy, actually.
%
%    \begin{macrocode}
\def\eqnalign@ii#1[#2]{%
  \let\\\eqn@cr%
  \colset{eqnarray}%
  \tab@initread%
  \def\tab@tabtext{&\tabskip\z@skip}%
  \tabskip\z@skip%
  \col@sep.5\eqainskip%
  \tab@readpreamble{#1}%
  \everycr{}%
  \if#2t\vtop\else%
    \if#2b\vbox\else%
      \vcenter%
    \fi%
  \fi%
  \bgroup%
  \halign\expandafter\bgroup\the\tab@preamble\cr%
}
%    \end{macrocode}
%
% Finishing the environment is even simpler.
%
%    \begin{macrocode}
\def\endeqnalign{%
  \crcr%
  \egroup%
  \egroup%
}
%    \end{macrocode}
%
% \end{environment}
%
% \begin{macro}{\eqn@cr}
%
% Newlines are really easy here.
%
%    \begin{macrocode}
\def\eqn@cr{\tab@cr\eqn@cr@i{}{}}
\def\eqn@cr@i#1{\cr\noalign{\vskip\jot\vskip#1}\@gobble}
%    \end{macrocode}
%
% \end{macro}
%
%
% \subsection{Simple multiline equations}
%
% As a sort of example and abbreviation, here's a multiline display
% environment which just centres everything.
%
% \begin{environment}{eqlines}
%
% We just get |\eqnarray| to do everything for us.  This is really easy.
%
%    \begin{macrocode}
\def\eqlines{\eqnarray[x]}
\let\endeqlines\endeqnarray
%    \end{macrocode}
%
% \end{environment}
%
% \begin{environment}{eqlines*}
%
% There's a $*$ version which omits numbers.  This is easy too.  Lots of
% hacking with expansion here to try and reduce the number of tokens being
% used.  Is it worth it?
%
%    \begin{macrocode}
\expandafter\edef\csname eqlines*\endcsname{%
  \expandafter\noexpand\csname eqnarray*\endcsname[x]%
}
\expandafter\let\csname endeqlines*\expandafter\endcsname
                \csname endeqnarray*\endcsname
%    \end{macrocode}
%
% \end{environment}
%
%
% \subsection{Split equations}
%
% Based on an idea from \textit{The \TeX book}, we provide some simple
% environments for doing split equations.  These's plenty of scope for
% improvement here, though.
%
% \begin{environment}{spliteqn}
% \begin{environment}{spliteqn*}
%
% The only difference between these two is that the $*$-version doesn't put
% in an equation number by default (although this behaviour can be
% changed by |\eqnumber|).
%
% The fun here mainly concerns putting in the equation number at the right
% place -- for |leqno| users, we need to put the number on the first line;
% otherwise we put it on the last line.
%
% The way we handle this is to have two macros, |\\| (which clearly does
% all the user line breaks) and |\seq@lastcr| which is used at the end of
% the environment to wrap everything up.  The |\seq@eqnocr| macro puts an
% equation number on the current line and then does a normal |\\|.  It also
% resets |\\| and |\seq@lastcr| so that they don't try to put another
% equation number in.  This must be done globally, although anyone who tries
% to nest maths displays will get what they deserve.
%
% For the non-$*$ environment, then, we need to step the equation counter,
% and set |\\| to |\seq@cr| or |\seq@eqnocr| as appropriate for the setting
% of the |leqno| flag -- |\seq@lastcr| always gets set to put an equation
% number in (because it will be reset if the number actually gets done
% earlier -- this catches stupid users trying to put a single row into
% a split environment).
%
%    \begin{macrocode}
\def\spliteqn{%
  \let\eqa@oldnumber\eqa@number%
  \global\let\eqa@number\eqa@eqcount%
  \spliteqn@i%
}
%    \end{macrocode}
%
% For the $*$ variant, we don't need to bother with equation numbering, so
% this is really easy.
%
%    \begin{macrocode}
\@namedef{spliteqn*}{%
  \let\eqa@oldnumber\eqa@number%
  \gdef\eqa@number{}%
  \spliteqn@i%
}
%    \end{macrocode}
%
% Ending the environments is easy.  Most of the stuff here will be described
% later.
%
%    \begin{macrocode}
\def\endspliteqn{%
  \hfilneg\seq@lastcr%
  \egroup%
  \dsp@end%
  \global\let\eqa@number\eqa@oldnumber%
  \global\advance\c@equation\m@ne%
}
\expandafter\let\csname endspliteqn*\endcsname\endspliteqn
%    \end{macrocode}
%
% \end{environment}
% \end{environment}
%
% \begin{macro}{\spliteqn@i}
%
% Here we handle the full display splits.  Start a maths display, and make
% each row of the alignment take up the full display width.
%
% The macro |\seq@dosplit| does most of the real work for us -- setting up
% the alignment and so forth.  The template column is interesting.  There
% are two items glue on both sides of the actual text:
%
% \begin{itemize}
%
% \item Some glue which can shrink.  This keeps the display from the edges
%       of the page unless we get a really wide item.
%
% \item An |\hfil| to do the alignment.  By default, this centres the
%       equations.  On the first line, however, we put a leading |\hfilneg|
%       which cancels the first |\hfil|, making the first row left aligned.
%       Similarly, at the end, we put an |\hfilneg| after the last equation
%       to right align the last line.
%
% \end{itemize}
%
% We pass this information on as an argument.  It's easy really.
%
%    \begin{macrocode}
\def\spliteqn@i{%
%    \end{macrocode}
%
% First, set up equation numbering properly.  See my rant about
% |\refstepcounter| above.
%
%    \begin{macrocode}
  \stepcounter{equation}%
  \def\@currentlabel{\p@equation\theequation}%
%    \end{macrocode}
%
% Right; now to sort out the numbering and newline handling.  If the number's
% meant to be on the first line (for \textsf{leqno} users), then it gets
% typeset on the first like; otherwise we just do a normal newline on
% all lines except the first.  Once |\seq@eqnocr| has done its stuff, it
% redefines all the newline handling not to insert another number.
%
%    \begin{macrocode}
  \if@leqno%
    \global\let\seq@docr\seq@eqnocr%
  \else%
    \global\let\seq@docr\seq@cr%
  \fi%
  \global\let\seq@lastcr\seq@eqnocr%
%    \end{macrocode}
%
% For my next trick, I'll do some display handling -- start a (possibly
% nested) maths display, set up the |\tabpause| macro appropriately, and
% set the newline command to do the right thing.
%
%    \begin{macrocode}
  \dsp@start%
  \dsp@tabpause%
  \def\\{\seq@docr}%
%    \end{macrocode}
%
% Finally, call another macro to do the remaining bits of setting up.
%
%    \begin{macrocode}
  \seq@dosplit%
    {\hb@xt@\displaywidth{%
      \hskip\splitleft\hfil$\displaystyle##$%
      \hfil\hskip\splitright}}%
    {\hfilneg}%
}
%    \end{macrocode}
%
% \end{macro}
%
% \begin{environment}{subsplit}
%
% For doing splits in the middle of equations, we provide a similar
% environment.  Here, we make |\\| just start a new line.  We also use
% a |\vcenter| rather than a full maths display.  The glue items are also
% a bit different: we use plain double-quads on each side of the item, and
% we need to remove them by hand at the extremeties of the environment.
%
%    \begin{macrocode}
\def\subsplit{%
  \let\\\seq@cr%
  \vcenter\bgroup%
  \seq@dosplit{\hfil\qquad$##$\qquad\hfil}{\hfilneg\hskip-2em}%
}
%    \end{macrocode}
%
% Ending the environment is fairly easy.  We remove the final glue item,
% and close the alignment and the vbox.
%
%    \begin{macrocode}
\def\endsubsplit{%
  \hfilneg\hskip-2em\cr%
  \egroup\egroup%
}
%    \end{macrocode}
%
% \end{environment}
%
% \begin{macro}{\seq@dosplit}
%
% Here we do most of the real work.  Actually, since the preamble is passed
% in as an argument, most of the work is already done.  The only thing to
% really note is the template for subsequent columns.  To stop users putting
% in extra columns (which is where we put the equation number) we raise an
% error and discard the input in a scratch box register.  This template is
% repeated infinitely so as to allow us to put the equation number in nicely.
% However, the final negative glue item won't work properly, so the equation
% will look awful.
%
%    \begin{macrocode}
\def\seq@dosplit#1#2{%
  \halign\bgroup%
    #1&&\mth@err@number\setbox\z@\hbox{##}\cr%
  #2\relax%
}
%    \end{macrocode}
%
% \end{macro}
%
% \begin{macro}{\seq@eqnocr}
%
% Here's how we set equation numbers.  Since the column provided raises
% errors as soon as a token finds its way into it, we start with a |&\omit|.
% Then we just put the equation number in a zero-width box.  Finally, we
% reset the newline commands to avoid putting in more than one equation
% number, and do normal newline things.
%
%    \begin{macrocode}
\def\seq@eqnocr{%
  &\omit%
  \hb@xt@\z@{\hss\eqa@eqpos\eqa@number}%
  \global\let\seq@docr\seq@cr%
  \global\let\seq@lastcr\seq@cr%
  \seq@cr%
}
%    \end{macrocode}
%
% \end{macro}
%
% \begin{macro}{\seq@cr}
%
% Newlines are very easy.  We add a |\jot| of extra space, since this is
% a nice thing to do.
%
%    \begin{macrocode}
\def\seq@cr{\tab@cr\seq@cr@i\interdisplaylinepenalty\@M}
\def\seq@cr@i#1#2{\cr\noalign{\penalty#2\vskip\jot\vskip#1}}
%    \end{macrocode}
%
% \end{macro}
%
%
% \subsection{Matrix handling}
%
% There's been a complete and total overhaul of the spacing calculations
% for matrices here.  The vertical spacing now bears an uncanny similarity
% to the rules \TeX\ uses to space out |\atop|-like fractions, the difference
% being that you can have more than one column in a matrix.  This has the
% interesting side-effect that we get an \package{amsmath}-style
% sub/superscript environment almost free of charge with the matrix handling
% (it just ends up being a script-size single-column matrix).
%
% What is rather gratifying is that our \env{matrix} environment looks
% rather nicer than \package{amsmath}'s (which is based directly on
% \env{array}, giving it nasty restrictions on the numbers of columns and
% so on); in particular, the version here gives the `correct' result for
% Knuth's exercise~18.42 (which states categorically that a |\smallskip|
% should be placed between the rows of the big matrix).
%
% The reason the interrow space doesn't come out in the AMS version is
% that \env{array} inserts extra vertical space by extending the depth of
% the final row using a strut: the big matrix already extends deeper than
% this, so the strut doesn't make any difference.  If the space was added
% by |\hlx{s[\smallskipamount]}| instead of the |\\| command, things would
% be different.
%
% \begin{figure}
%
% ^^A This is essentially what amsmath (version 1.2b) does.  The real
% ^^A implementation requires a counter MaxMatrixCols, and has fewer braces:
% ^^A that's all the difference.  Oh, and I turn off \arrayextrasep here,
% ^^A since amsmath doesn't expect it to be there (accurate emulation, see?)
% ^^A and I've used \hspace instead of \hskip since everything else is
% ^^A `proper' LaTeX stuff.
%
% \newenvironment{ams-pmatrix}{^^A
%   \setlength{\arrayextrasep}{0pt}^^A
%   \left(^^A
%   \hspace{-\arraycolsep}^^A
%   \begin{array}{*{10}{c}}^^A
% }{^^A
%  \end{array}^^A
%  \hspace{-\arraycolsep}^^A
%  \right)^^A
% }
%
% \begin{demo}{Exercise 18.42 from \emph{The \TeX book}}
%\newcommand{\domatrix}[1]{
%  \def\mat##1
%    {\begin{#1}##1\end{#1}}
%  \[ \begin{#1}
%     \mat{a & b \\ c & d} &
%     \mat{e & f \\ g & h}
%     \\[\smallskipamount]
%     0 &
%     \mat{i & j \\ k & l}
%     \end{#1}
%  \]
%}
%\domatrix{pmatrix}
%\domatrix{ams-pmatrix}
% \end{demo}
%
% \end{figure}
%
% \begin{environment}{genmatrix}
%
% The first job is to store my maths style and font away, because I'll be
% needing it lots later.
%
%    \begin{macrocode}
\def\genmatrix#1#2#3#4#5{%
  \let\mat@style#1%
  \ifx#2\scriptstyle%
    \let\mat@font\scriptfont%
  \else\ifx#2\scriptscriptstyle%
    \let\mat@font\scriptscriptfont%
  \else%
    \let\mat@font\textfont%
  \fi\fi%
%    \end{macrocode}
%
% Now to cope with inserted text.  This is easy.
%
%    \begin{macrocode}
  \ifx\mat@style\scriptstyle%
    \let\mat@textsize\scriptsize%
  \else\ifx\mat@style\scriptscriptstyle%
    \let\mat@textsize\scriptscriptsize%
  \else%
    \let\mat@textsize\relax%
  \fi\fi%
%    \end{macrocode}
%
% Now for some fun.  I'll remember how to start and end the matrix in a
% couple of macros |\mat@left| and |\mat@right|.  I haven't yet worked out
% exactly what needs to be in |\mat@right| yet, though, so I'll build that
% up in a scratch token list while I'm making my mind up.
%
% Initially, I want to open a group (to trap the style changes), set the
% maths style (to get the right spacing), insert the left delimiter, insert
% some spacing around the matrix, and start a centred box.  The ending just
% closes all the groups and delimiters I opened.
%
%    \begin{macrocode}
  \def\mat@left{\bgroup\mat@style\left#4#3\vcenter\bgroup}%
  \toks@{\egroup#3\right#5\egroup}%
%    \end{macrocode}
%
% Now comes a slightly trickier bit.  If the maths style is script or
% scriptscript, then I need to raise the box by a little bit to make it look
% really good.  The right amount is somewhere around \smallf 3/4\,pt, I
% think, so that's what I'll use.
%
%    \begin{macrocode}
  \@tempswatrue%
  \ifx\mat@style\displaystyle\else\ifx\mat@style\textstyle\else%
    \@tempswafalse%
    \setbox\z@\hbox\bgroup$%
    \toks@\expandafter{\the\toks@$\m@th\egroup\raise.75\p@\box\z@}%
  \fi\fi%
%    \end{macrocode}
%
% If I'm not in maths mode right now, then I should enter maths mode, and
% remember to leave it later.
%
%    \begin{macrocode}
  \if@tempswa\ifmmode\else%
    $\m@th%
    \toks@\expandafter{\the\toks@$}%
  \fi\fi%
%    \end{macrocode}
%
% Now I've sorted out how to end the environment properly, so I can set up
% the macro, using |\edef|.
%
%    \begin{macrocode}
  \edef\mat@right{\the\toks@}%
%    \end{macrocode}
%
% Now see if there's an optional argument.  If not, create lots of centred
% columns.
%
%    \begin{macrocode}
  \@ifnextchar[\genmatrix@i{\genmatrix@i[[c]}%
}
%    \end{macrocode}
%
% Now to sort out everything else.
%
%    \begin{macrocode}
\def\genmatrix@i[#1]{%
%    \end{macrocode}
%
% Some initial setting up: choose the correct column set, and set up some
% variables for reading the preamble.
%
%    \begin{macrocode}
  \colset{matrix}%
  \tab@initread%
%    \end{macrocode}
%
% Now comes some of the tricky stuff.  The space between columns should be
% 12\,mu (by trial and error).  We put the space in a box so we can measure
% it in the correct mathstyle.
%
%    \begin{macrocode}
  \setbox\z@\hbox{$\mat@style\mskip12mu$}%
  \edef\tab@tabtext{&\kern\the\wd\z@}%
  \tab@readpreamble{#1}%
%    \end{macrocode}
%
% Now we need to decide how to space out the rows.  The code here is based
% on the information in appendix~G of \emph{The \TeX book}: I think it'd be
% nice if my matrices were spaced out in the same way as normal fractions
% (particularly |\choose|y things).  The standard |\baselineskip| and
% |\lineskip| parameters come in really handy here.
%
% The parameters vary according to the size of the text, so I need to see
% if we have scriptsize or less, or not.  The tricky |\if| sorts this out.
%
%    \begin{macrocode}
  \if1\ifx\mat@style\scriptstyle1\else%
      \ifx\mat@style\scriptscriptstyle1\else0\fi\fi%
    \baselineskip\fontdimen10\mat@font\tw@%
    \advance\baselineskip\fontdimen12\mat@font\tw@%
    \lineskip\thr@@\fontdimen8\mat@font\thr@@%
  \else%
    \baselineskip\fontdimen8\mat@font\tw@%
    \advance\baselineskip\fontdimen11\mat@font\tw@%
    \lineskip7\fontdimen8\mat@font\thr@@%
  \fi%
  \lineskiplimit\lineskip%
%    \end{macrocode}
%
% Now actually set up for the alignment.  Assign |\\| to the correct value.
% Set up the |\tabskip|.  Do the appropriate |\mat@left| thing set up above.
% And then start the alignment.
%
%    \begin{macrocode}
  \let\\\mat@cr%
  \tabskip\z@skip%
  \col@sep\z@%
  \mat@left%
  \halign\expandafter\bgroup\the\tab@preamble\tabskip\z@skip\cr%
%    \end{macrocode}
%
% Now for a little hack to make the spacing consistent between matrices of
% the same height.  This comes directly from \PlainTeX.  This appears to
% make the spacing \emph{exactly} the same as the \TeX\ primites, oddly
% enough.
%
%    \begin{macrocode}
  \ifx\mat@font\textfont%
    \omit$\mat@style\mathstrut$\cr\noalign{\kern-\baselineskip}%
  \fi%
}
%    \end{macrocode}
%
% Finishing the environment is really easy.  We do the spacing hack again
% at the bottom, close the alignment and then tidy whatever we started in
% |\mat@left|.
%
%    \begin{macrocode}
\def\endgenmatrix{%
  \crcr%
  \ifx\mat@font\textfont%
    \omit$\mat@style\mathstrut$\cr\noalign{\kern-\baselineskip}%
  \fi%
  \egroup%
  \mat@right%
}
%    \end{macrocode}
%
% \end{environment}
%
% \begin{macro}{\mat@cr}
%
% Newlines are really easy.  The $*$-form means nothing here, so we ignore
% it.
%
%    \begin{macrocode}
\def\mat@cr{\tab@cr\mat@cr@i{}{}}
\def\mat@cr@i#1{\cr\noalign{\vskip#1}\@gobble}
%    \end{macrocode}
%
% \end{macro}
%
% \begin{macro}{\newmatrix}
%
% This is how we define new matrix environments.  It's simple fun with
% |\csname| and |\expandafter|.
%
%    \begin{macrocode}
\def\newmatrix#1#2{%
  \@namedef{#1}{\genmatrix#2}%
  \expandafter\let\csname end#1\endcsname\endgenmatrix%
}
%    \end{macrocode}
%
% \end{macro}
%
% \begin{environment}{matrix}
% \begin{environment}{pmatrix}
% \begin{environment}{dmatrix}
% \begin{environment}{smatrix}
% \begin{environment}{spmatrix}
% \begin{environment}{sdmatrix}
% \begin{environment}{smatrix*}
% \begin{environment}{spmatrix*}
% \begin{environment}{sdmatrix*}
%
% Now we define all the other environments we promised.  This is easy.
%
%    \begin{macrocode}
\newmatrix{matrix}{{\textstyle}{\textstyle}{\,}{.}{.}}
\newmatrix{pmatrix}{{\textstyle}{\textstyle}{\,}{(}{)}}
\newmatrix{dmatrix}{{\textstyle}{\textstyle}{\,}}
\newmatrix{smatrix}{{\scriptstyle}{\scriptstyle}{}{.}{.}}
\newmatrix{spmatrix}{{\scriptstyle}{\scriptstyle}{}{(}{)}}
\newmatrix{sdmatrix}{{\scriptstyle}{\scriptstyle}{}}
\newmatrix{smatrix*}{{\scriptstyle}{\textstyle}{}{.}{.}}
\newmatrix{spmatrix*}{{\scriptstyle}{\textstyle}{}{(}{)}}
\newmatrix{sdmatrix*}{{\scriptstyle}{\textstyle}{}}
%    \end{macrocode}
%
% \end{environment}
% \end{environment}
% \end{environment}
% \end{environment}
% \end{environment}
% \end{environment}
% \end{environment}
% \end{environment}
% \end{environment}
%
% \begin{environment}{script}
%
% Now for superscripts and subscripts.  This is fairly easy, because I
% took so much care over the matrix handling.
%
%    \begin{macrocode}
\def\script{%
  \let\mat@style\scriptstyle%
  \def\mat@left{\vcenter\bgroup}%
  \def\mat@right{\egroup}%
  \let\mat@font\scriptfont%
  \let\mat@textsize\scriptsize%
  \@ifnextchar[\genmatrix@i{\genmatrix@i[c]}%
}
\let\endscript\endgenmatrix
%    \end{macrocode}
%
% \end{environment}
%
% Now define the column types.
%
%    \begin{macrocode}
\colpush{matrix}
\coldef l{\tabcoltype{\kern\z@$\mat@style}{\m@th$\hfil}}
\coldef c{\tabcoltype{\hfil$\mat@style}{\m@th$\hfil}}
\coldef r{\tabcoltype{\hfil$\mat@style}{\m@th$}}
\coldef T#1{\tab@aligncol{#1}{\begingroup\mat@textsize}{\endgroup}}
%    \end{macrocode}
%
% The repeating type is more awkward.  Things will go wrong if this is
% given before the first column, so we must do a whole repeat by hand.  We
% can tell if we haven't contributed a column yet, since |\tab@column| will
% be zero.  Otherwise, we fiddle the parser state to start a new column, and
% insert the |&| character to make \TeX\ repeat the preamble.
%
%    \begin{macrocode}
\coldef {[}{%
  \@firstoftwo{%
    \ifnum\tab@columns=\z@%
      \def\@tempa##1\q@delim{%
        \tab@mkpreamble##1[##1\q@delim%
      }%
      \expandafter\@tempa%
    \else%
      \tab@setstate\tab@prestate%
      \tab@append\tab@preamble{&}%
      \expandafter\tab@mkpreamble%
    \fi%
  }%
}
%    \end{macrocode}
%
% We're done defining columns now.
%
%    \begin{macrocode}
\colpop
%    \end{macrocode}
%
%
% \subsection{Dots\dots}
%
% Nothing whatsoever to do with alignments, although vertical and diagonal
% dots in small matrices look really silly.  The following hacky definitions
% work rather better.
%
% \begin{macro}{\mdw@dots}
%
% First of all, here's some definitions common to both of the dots macros.
% The macro takes as an argument the actual code to draw the dots, passing
% it the scaled size of a point in the scratch register |\dimen@|; the
% register |\box 0| is set to contain a dot of the appropriate size.
%
%    \begin{macrocode}
\def\mdw@dots#1{\ensuremath{\mathpalette\mdw@dots@i{#1}}}
\def\mdw@dots@i#1#2{%
  \setbox\z@\hbox{$#1\mskip1.8mu$}%
  \dimen@\wd\z@%
  \setbox\z@\hbox{$#1.$}%
  #2%
}
%    \end{macrocode}
%
% \end{macro}
%
% \begin{macro}{\vdots}
%
% I'll start with the easy one.  This is a simple translation of the original
% implementation.
%
%    \begin{macrocode}
\def\vdots{%
  \mdw@dots{\vbox{%
    \baselineskip4\dimen@%
    \lineskiplimit\z@%
    \kern6\dimen@%
    \copy\z@\copy\z@\box\z@%
  }}%
}
%    \end{macrocode}
%
% \end{macro}
%
% \begin{macro}{\ddots}
%
% And I'll end with the other easy one\dots
%
%    \begin{macrocode}
\def\ddots{%
  \mdw@dots{\mathinner{%
    \mkern1mu%
    \raise7\dimen@\vbox{\kern7\dimen@\copy\z@}%
    \mkern2mu%
    \raise4\dimen@\copy\z@%
    \mkern2mu%
    \raise\dimen@\box\z@%
    \mkern1mu%
  }}%
}
%    \end{macrocode}
%
% \end{macro}
%
%
% \subsection{Lucky dip}
%
% Time to round off with some trivial environments, just to show how easy
% this stuff is.
%
% \begin{environment}{cases}
% \begin{environment}{smcases}
%
% These are totally and utterly trivial.
%
%    \begin{macrocode}
\def\cases{\left\{\,\array{@{}lTl@{}}}
\def\endcases{\endarray\,\right.}
\def\smcases{\left\{\smarray{@{}lTl@{}}}
\def\endsmcases{\endsmarray\,\right.}
%    \end{macrocode}
%
% \end{environment}
% \end{environment}
%
% \subsection{Error messages}
%
% Some token saving:
%
%    \begin{macrocode}
\def\mth@error{\PackageError{mathenv}}
%    \end{macrocode}
%
% Now for the error messages.
%
%    \begin{macrocode}
\def\mth@err@number{%
  \mth@error{Too many `&' characters found}{%
    You've put too many `&' characters in an alignment^^J%
    environment (like `eqnarray' or `spliteqn') and wandered^^J%
    into trouble.  I've gobbled the contents of that column^^J%
    and hopefully I can recover fairly easily.%
  }%
}
%    \end{macrocode}
%
%    \begin{macrocode}
\def\mth@err@mdsp{%
  \mth@error{Can't do displays in nondisplay maths mode}{%
    You're trying to start a display environment, but you're^^J%
    in nondisplay maths mode.  The display will appear but^^J%
    don't blame me when it looks horrible.%
  }%
}
%    \end{macrocode}
%
%    \begin{macrocode}
\def\mth@err@hdsp{%
  \mth@error{Can't do displays in LR mode}{%
    You're trying to start a display environment, but you're^^J%
    in LR (restricted horizontal) mode.  Everything will go^^J%
    totally wrong, so your best bet is to type `X', fix the^^J%
    mistake and start again.%
  }%
}
%    \end{macrocode}
%
% \vskip\parskip\vbox{ ^^A The best way I could find of keeping this lot
%                      ^^A together, I'm afraid.
% That's all there is.  Byebye.
%
%    \begin{macrocode}
%</mathenv>
%    \end{macrocode}
% \nopagebreak
%
% \hfill Mark Wooding, \today
% }
%
% \Finale
%
\endinput
